%package list
\documentclass{article}
\usepackage[top=3cm, bottom=3cm, outer=3cm, inner=3cm]{geometry}
\usepackage{multicol}
\usepackage{graphicx}
\usepackage{url}
%\usepackage{cite}
\usepackage{hyperref}
\usepackage{array}
%\usepackage{multicol}
\newcolumntype{x}[1]{>{\centering\arraybackslash\hspace{0pt}}p{#1}}
\usepackage{natbib}
\usepackage{pdfpages}
\usepackage{multirow}
\usepackage[normalem]{ulem}
\useunder{\uline}{\ul}{}
\usepackage{svg}
\usepackage{xcolor}
\usepackage{listings}
\lstdefinestyle{ascii-tree}{
    literate={├}{|}1 {─}{--}1 {└}{+}1 
  }
\lstset{basicstyle=\ttfamily,
  showstringspaces=false,
  commentstyle=\color{red},
  keywordstyle=\color{blue}
}
%\usepackage{booktabs}
\usepackage{caption}
\usepackage{subcaption}
\usepackage{float}
\usepackage{array}

\newcolumntype{M}[1]{>{\centering\arraybackslash}m{#1}}
\newcolumntype{N}{@{}m{0pt}@{}}


%%%%%%%%%%%%%%%%%%%%%%%%%%%%%%%%%%%%%%%%%%%%%%%%%%%%%%%%%%%%%%%%%%%%%%%%%%%%
%%%%%%%%%%%%%%%%%%%%%%%%%%%%%%%%%%%%%%%%%%%%%%%%%%%%%%%%%%%%%%%%%%%%%%%%%%%%
\newcommand{\itemEmail}{vmamanian@unsa.edu.pe}
\newcommand{\itemStudent}{Victor Mamani Anahua}
\newcommand{\itemCourse}{Fundamentos de la Programación II}
\newcommand{\itemCourseCode}{20230489}
\newcommand{\itemSemester}{II}
\newcommand{\itemUniversity}{Universidad Nacional de San Agustín de Arequipa}
\newcommand{\itemFaculty}{Facultad de Ingeniería de Producción y Servicios}
\newcommand{\itemDepartment}{Departamento Académico de Ingeniería de Sistemas e Informática}
\newcommand{\itemSchool}{Escuela Profesional de Ingeniería de Sistemas}
\newcommand{\itemAcademic}{2023 - B}
\newcommand{\itemInput}{Del 15 Enero 2024}
\newcommand{\itemOutput}{Al 22 Enero 2024}
\newcommand{\itemPracticeNumber}{22}
\newcommand{\itemTheme}{Laboratorio 22}
%%%%%%%%%%%%%%%%%%%%%%%%%%%%%%%%%%%%%%%%%%%%%%%%%%%%%%%%%%%%%%%%%%%%%%%%%%%%
%%%%%%%%%%%%%%%%%%%%%%%%%%%%%%%%%%%%%%%%%%%%%%%%%%%%%%%%%%%%%%%%%%%%%%%%%%%%

\usepackage[english,spanish]{babel}
\usepackage[utf8]{inputenc}
\AtBeginDocument{\selectlanguage{spanish}}
\renewcommand{\figurename}{Figura}
\renewcommand{\refname}{Referencias}
\renewcommand{\tablename}{Tabla} %esto no funciona cuando se usa babel
\AtBeginDocument{%
	\renewcommand\tablename{Tabla}
}

\usepackage{fancyhdr}
\pagestyle{fancy}
\fancyhf{}
\setlength{\headheight}{30pt}
\renewcommand{\headrulewidth}{1pt}
\renewcommand{\footrulewidth}{1pt}
\fancyhead[L]{\raisebox{-0.2\height}{\includegraphics[width=3cm]{img/logo_episunsa.png}}}
\fancyhead[C]{\fontsize{7}{7}\selectfont	\itemUniversity \\ \itemFaculty \\ \itemDepartment \\ \itemSchool \\ \textbf{\itemCourse}}
\fancyhead[R]{\raisebox{-0.2\height}{\includegraphics[width=1.2cm]{img/logo_abet}}}
\fancyfoot[L]{Estudiante Victor Mamani A.}
\fancyfoot[C]{\itemCourse}
\fancyfoot[R]{Página \thepage}

% para el codigo fuente
\usepackage{listings}
\usepackage{color, colortbl}
\definecolor{dkgreen}{rgb}{0,0.6,0}
\definecolor{gray}{rgb}{0.5,0.5,0.5}
\definecolor{mauve}{rgb}{0.58,0,0.82}
\definecolor{codebackground}{rgb}{0.95, 0.95, 0.92}
\definecolor{tablebackground}{rgb}{0.8, 0, 0}

\lstdefinestyle{java}{frame=tb,
	language=Java,
	showstringspaces=false,
	columns=flexible,
	basicstyle={\footnotesize\ttfamily\color[RGB]{255,255,255}},
	numberstyle=\color{mygray},
	numbers=left, 
	keywordstyle=\color{myblue},
	morekeywords={String, System},
	commentstyle=\color{mygray},
	stringstyle=\color{mygreen},
	breaklines=true,
	breakatwhitespace=true,
	tabsize=2,
	backgroundcolor= \color{codebackgroundCode},
	showspaces=false,
	showtabs=false,
	showlines=false,
}

\lstset{frame=tb,
	language=bash,
	aboveskip=3mm,
	belowskip=3mm,
	showstringspaces=false,
	columns=flexible,
	basicstyle={\small\ttfamily},
	numbers=none,
	numberstyle=\tiny\color{gray},
	keywordstyle=\color{blue},
	commentstyle=\color{dkgreen},
	stringstyle=\color{mauve},
	breaklines=true,
	breakatwhitespace=true,
	tabsize=3,
	backgroundcolor= \color{codebackground},
}

\begin{document}
	
	\vspace*{10px}
	
	\begin{center}	
		\fontsize{17}{17} \textbf{ Informe de Laboratorio \itemPracticeNumber}
	\end{center}
	\centerline{\textbf{\Large Tema: \itemTheme}}
	%\vspace*{0.5cm}	

	\begin{flushright}
		\begin{tabular}{|M{2.5cm}|N|}
			\hline 
			\rowcolor{tablebackground}
			\color{white} \textbf{Nota}  \\
			\hline 
			     \\[30pt]
			\hline 			
		\end{tabular}
	\end{flushright}	

	\begin{table}[H]
		\begin{tabular}{|x{4.7cm}|x{4.8cm}|x{4.8cm}|}
			\hline 
			\rowcolor{tablebackground}
			\color{white} \textbf{Estudiante} & \color{white}\textbf{Escuela}  & \color{white}\textbf{Asignatura}   \\
			\hline 
			{\itemStudent \par \itemEmail} & \itemSchool & {\itemCourse \par Semestre: \itemSemester \par Código: \itemCourseCode}     \\
			\hline 			
		\end{tabular}
	\end{table}		
	
	\begin{table}[H]
		\begin{tabular}{|x{4.7cm}|x{4.8cm}|x{4.8cm}|}
			\hline 
			\rowcolor{tablebackground}
			\color{white}\textbf{Laboratorio} & \color{white}\textbf{Tema}  & \color{white}\textbf{Duración}   \\
			\hline 
			\itemPracticeNumber & \itemTheme & 04 horas   \\
			\hline 
		\end{tabular}
	\end{table}
	
	\begin{table}[H]
		\begin{tabular}{|x{4.7cm}|x{4.8cm}|x{4.8cm}|}
			\hline 
			\rowcolor{tablebackground}
			\color{white}\textbf{Semestre académico} & \color{white}\textbf{Fecha de inicio}  & \color{white}\textbf{Fecha de entrega}   \\
			\hline 
			\itemAcademic & \itemInput &  \itemOutput  \\
			\hline 
		\end{tabular}
	\end{table}
	
	\section{Tarea}
	\begin{itemize}		
		\item Cree una versión del videojuego de estrategia usando componentes básicos GUI: Etiquetas, botones,
		cuadros de texto, JOptionPane, Color.
		\item Además, utilizar componentes avanzados GUI: Layouts, JPanel, áreas de texto, checkbox, botones de
		radio y combobox.
		\item Considerar nivel estratégico y táctico.
		\item Considerar hasta las unidades especiales de los reinos.
		\item Hacerlo iterativo.
	\end{itemize}

	\section{Equipos, materiales y temas utilizados}
	\begin{itemize}
		\item Sistema Operativo Ubuntu GNU Linux 23 lunar 64 bits Kernell 6.2.v
		\item Visual Studio Code.
		\item VIM 9.0.
		\item OpenJDK 64-Bits 19.0.7.
		\item Git 2.39.2.
		\item Cuenta en GitHub con el correo institucional.
		\item Programación Orientada a Objetos.
		\item Actividades del Laboratorio 22.	
	\end{itemize}
	
	\section{URL de Repositorio Github}
	\begin{itemize}
		\item URL del Repositorio GitHub para clonar o recuperar.
		\item \url{https://github.com/VictorMA18/fp2-23b.git}
		\item URL para el laboratorio 22 en el Repositorio GitHub.
		\item \url{https://github.com/VictorMA18/fp2-23b/tree/main/Fase03/Lab22}
	\end{itemize}
	
	\section{Actividades del Laboratorio 22}

	\subsection{Ejercicio Soldado(Herencias)}
	\begin{itemize}	
		\item En esta seccion solo reutilizamos las Herencias de la clase Soldado.
		\item El codigo y el commit seria el siguiente:
	\end{itemize}	
	\begin{lstlisting}[language=bash,caption={Commit}][H]
		$ git commit -m "Agregando la clase Soldado y Espadachin para poder hacer el juego bueno solo en la clase espadachin usamos la herencia que nos deja la clase Soldado y tambien creamos la funcion muroEscudo() la cual devuelve como mensaje el uso de esta habilidad defensiva y los getters y setters"
	\end{lstlisting}	
	\begin{lstlisting}[language=java,caption={Las lineas de codigos de la clase Espadachin creada:}][H]
		public class Espadachin extends Soldado{
			private int swordlth;
			public Espadachin(String name , int attacklevel, int defenselevel, int lifelevel, int speed, String attitude ,boolean lives, int row, String column, int swordlth){
				super(name, attacklevel, defenselevel, lifelevel, speed, attitude, lives, row, column);
				this.swordlth = swordlth;
			}
			public void muroEscudo(){
				System.out.println("Usted uso la habilidad muro de Escudos");
			}
			public int getSwordlth(){
				return swordlth;
			}
			public void setSwordlth(int n){
				this.swordlth = n;
			}
		}
	\end{lstlisting}
	\begin{lstlisting}[language=java,caption={Las lineas de codigos de la clase Caballero creada:}][H]
		public class Caballero extends Soldado{
			private boolean montar;
			private String arma;
			public Caballero(){
			}
			public Caballero(String name , int attacklevel, int defenselevel, int lifelevel, int speed, String attitude ,boolean lives, int row, String column,boolean montar){
				super(name, attacklevel, defenselevel, lifelevel, speed, attitude, lives, row, column);
				this.montar = montar;
			}
			public void montar(){
				if(!this.montar){
					this.arma = "Lanza";
					this.embestir();
				}
			}
			public void desmontar(){
				if (this.montar) {
					this.arma = "Espada";
				}
			}
			public void embestir(){
				if(!montar){
					this.atacar();
					this.atacar();
				}else{
					this.atacar();
					this.atacar();
					this.atacar();
				}
			}
			public String getArma(){
				return arma;
			}
		}
	\end{lstlisting}
	\begin{lstlisting}[language=java,caption={Las lineas de codigos de la clase Arquero creada:}][H]
		public class Arquero extends Soldado{
			private int flechas;
			public Arquero(){
				
			}
			public Arquero(String name , int attacklevel, int defenselevel, int lifelevel, int speed, String attitude ,boolean lives, int row, String column, int flechas){
				super(name, attacklevel, defenselevel, lifelevel, speed, attitude, lives, row, column);
				this.flechas = flechas;
			}
			public void disparar(){
				if(this.flechas == 0){
					System.out.println("El arquero ya tiene flechas para poder disparar");
				}else{
					this.flechas = flechas - 1;
					this.atacar();
				}
			}
			public void setFlechas(int n){
				this.flechas = n;
			}
			public int getFlechas(){
				return flechas;
			}
		}
	\end{lstlisting}
	\begin{lstlisting}[language=java,caption={Las lineas de codigos de la clase Lancero creada:}][H]
		public class Lancero extends Soldado{
			private int lancelth;
			public Lancero(){
			}
			public Lancero(String name , int attacklevel, int defenselevel, int lifelevel, int speed, String attitude ,boolean lives, int row, String column, int lancelth){
				super(name, attacklevel, defenselevel, lifelevel, speed, attitude, lives, row, column);
				this.lancelth = lancelth;
			}
			public void schiltrom(){
				this.setDefenseLevel(this.getDefenseLevel() + 1);
				System.out.println("El lancero uso el schiltrom su nivel de defensa subio 1 punto");
			}
			public void setLancelth(int n){
				this.lancelth = n;
			}
			public int getLancelth(){
				return lancelth;
			}
		}
	\end{lstlisting}
	\begin{lstlisting}[language=java,caption={Las lineas de codigos de la clase Soldado creada:}][H]
		// Laboratorio Nro 22 - Ejercicio Soldado
		// Autor: Mamani Anahua Victor Narciso
		// Colaboro:
		// Tiempo:
		import java.util.*;
		public class Soldado { //CREAMOS LA CLASE SOLDODADO PARA PODER USAR UN ARREGLO BIDIMENSIONAL DONDE NECESITAMOS LA VIDA , EL NOMBRE DEL SOLDADO Y TAMBIEN SU POSICION COMO LA FILA Y LA COLUMNA   
		
			private String name;
			private int lifeactual;
			private int row;
			private String column;
			private int attacklevel;
			private int defenselevel;
			private int lifelevel;
			private int speed;
			private String attitude;
			private boolean lives;
		
			Random rdm = new Random();
		
			//Anadiendo metodo que nos permita que un arreglo tenga datos nulos si este esta vacio
			public Soldado(){
				this.name = "";
				this.row = 0;
				this.column  = "";
				this.attacklevel = 0;
				this.defenselevel = 0;
				this.lifelevel = 0;
				this.lifeactual = 0;
				this.speed = 0;
				this.attitude = "";
				this.lives = false;
			}
		
			//Constructor
			public Soldado(String name, int health, int row, String column){
				this.name = name;
				this.lifeactual = health;
				this.lifelevel = health;
				this.lifeactual = health;
				this.row = row;
				this.column = column;
				this.lives = true;
				
				//YA QUE ESTOS DATOS SERIAN ALEATORIOS YA QUE SE ESTARIA CREANDO EL SOLDADO TENDRIAMOS DATOS QUE SERIAN COMO ATTACKLEVEL DEFENSELEVEL EL CUAL TENDRIAN QUE SER ALEATORIOS    
				this.attacklevel = rdm.nextInt(5) + 1;
				this.defenselevel = rdm.nextInt(5) + 1;
		
			}
			
			//Constructor para los diferentes niveles como de vidad defensa ataque velocidad
			public Soldado(String name , int attacklevel, int defenselevel, int lifelevel, int speed, String attitude ,boolean lives, int row, String column) {
				this.name = name;
				this.attacklevel = attacklevel;
				this.defenselevel = defenselevel;
				this.lifeactual = lifelevel;
				this.lifelevel = lifelevel;
				this.speed = speed;
				this.lives = lives;
				this.row = row;
				this.column = column;
				this.attitude = attitude;
			}
		
			//Metodos necesarios como avanzar defender huir al seratacado al retroceder
			public void advance(){
				this.speed = getSpeed() + 1;
				System.out.println("El soldado " + this.name + "avanzo");
			}
			public void defense(){
				this.speed = 0;
				this.attitude = "DEFENSIVA";
				System.out.println("El soldado " + this.name + "esta defendiendo");
			}
			public void flee(){
				this.speed = getSpeed() + 2;
				this.attitude = "HUYE";
				System.out.println("El soldado " + this.name + "esta huyendo");
			}
			public void back(){
				System.out.println("El soldado " + this.name + "esta retrocediendo");
				if(this.speed == 0){
					this.speed = rdm.nextInt(5) - 5;
				}else{
					if(this.speed > 0){
						this.speed = 0;
						this.attitude = "DEFENSIVA";
					}
				}
			}
			public void atacar(){
				this.speed += 1;
				this.attitude = "Atacar";
				this.lifeactual += 1;
				if(this.lifeactual == 0){
					morir();
				}
			}
			public void attack(Soldado soldier){
				if(this.getLifeActual() > soldier.getLifeActual()){
					int life = this.getLifeActual() - soldier.getLifeActual();
					this.setLifeActual(life);
					this.setLifeLevel(life);
					soldier.lives = false;
					soldier.morir();
					System.out.println(this.name + " asesino al soldado " + soldier.name);
				}else if(soldier.getLifeActual() > this.getLifeActual()){
					int life = soldier.getLifeActual() - this.getLifeActual();
					this.lives = false;
					this.morir();
					soldier.setLifeActual(life);
					soldier.setLifeLevel(life);
					System.out.println(soldier.name + " asesino al soldado " + this.name);
				}else{
					this.lives = false;
					this.morir();   
					soldier.lives = false;
					soldier.morir();
					System.out.println("los 2 soldados se asesinaron");
				}
			}
			public void morir(){
				this.lives = false;
				this.attitude = "SOLDADO MUERTO";
			}
		
			// Metodos mutadores
			public void setName(String n){
				name = n;
			}
			public void setLifeActual(int p){
				lifeactual = p;
			}
			public void setRow(int b){
				row = b;
			}
			public void setColumn(String c){
				column = c; 
			}
			public void setAttackLevel(int attacklevel) {
				this.attacklevel = attacklevel;
			}
			public void setDefenseLevel(int defenselevel) {
				this.defenselevel = defenselevel;
			}
			public void setLifeLevel(int lifelevel){
				this.lifelevel = lifelevel;
			}
			public void setSpeed(int speed) {
				this.speed = speed;
			}
			public void setAttitude(String attitude) {
				this.attitude = attitude;
			}
			public void setLives(boolean lives) {
				this.lives = lives;
			}
		
			// Metodos accesores
			public String getName(){
				return name;
			}
			public int getLifeActual(){
				return lifeactual;
			}
			public int getRow(){
				return row;
			}
			public String getColumn(){
				return column;
			}
			public int getAttackLevel() {
				return attacklevel;
			}
			public int getDefenseLevel() {
				return defenselevel;
			}
			public int getLifeLevel(){
				return lifelevel;
			}
			public int getSpeed() {
				return speed;
			}
			public String getAttitude() {
				return attitude;
			}
			public boolean getLives() {
				return lives;
			}
		
			// Completar con otros metodos necesarios
			public String toString(){ //CREAMOS ESTE METODO PARA IMPRIMIR LOS DATOS DEl OBJETO
				String join = "\nNombre: " + getName() + "\nVida: " + getLifeActual() + "\nFila: " + getRow() + "\nColumna: " + getColumn() + "\nNivel de ataque: " + getAttackLevel() + "\nNivel de Defensa: " + getDefenseLevel() + "\nNivel de vida: " + getLifeLevel() + "\nVelocidad: " + getSpeed() + "\nActitud: " + getAttitude() + "\nEstado: " + getLives(); //Agregamos un espaciador para poder separar
				return join;
			}
			
		}
	\end{lstlisting}
	\subsection{Ejercicio Ejercito y Mapa}
	\begin{itemize}	
		\item En esta seccion En la clase Ejercito creamos un contructor null y constructor el cual reciba parametros para reconocer a este le agregamos los metodos viewsoldiers() , longerlife() , averagelife(),  rankingInsercionLife() y tambien sus getters y setters esto con el fin que nos ayude a simplificar algunas funciones las cuales sean representadas mejor con una clase Ejercito.
		\item Por parte de clase Mapa integramos 2 objetos de la clase Ejercito las cuales nos ayudan con sus metodos para ver la informacion de la batalla entre los 2 ejercitos.
		\item El codigo y el commit seria el siguiente:
	\end{itemize}	
	\begin{lstlisting}[language=bash,caption={Commit}][H]
		$ git commit -m "Creamos una clase Ejercito la cual nos va a poder ayudar al momento de querer datos de este como los soldados , el soldado de mayor vida promedio y tambien tenenemos en la clase Mapa la cual creamos objetos de esta clase Ejercito las cuales vamos a poder usar sus metodos"
	\end{lstlisting}	
	\begin{lstlisting}[language=java,caption={Las lineas de codigos de la clase Ejercito creada:}][H]
		import java.util.*;
		class Ejercito {   
			private ArrayList<ArrayList<Soldado>> army;
			private String kingdom;
			public Ejercito(){
			}
			public Ejercito(ArrayList<ArrayList<Soldado>> army, String kingdom){
				this.army = army;
				this.kingdom = kingdom;
			}
			public void viewSoldiers(String armyespe, int num, ArrayList<ArrayList<Soldado>> army){
				int numbersoldiers = 0;
				System.out.println("El Ejercito " + armyespe + " del " + num + " ejercito sus soldados son :");
				for(int i = 0; i < 10; i++){ //ITERACION
					for(int j = 0; j < 10 ; j++){//ITERACION
						if(army.get(i).get(j) != null){
							System.out.println("\n*********************************");
							System.out.println("El " + (numbersoldiers + 1) + " soldado es: ");
							System.out.println(army.get(i).get(j).toString());
							numbersoldiers++;
						}
					}
				}
			}
			public void longerLife(ArrayList<ArrayList<Soldado>> army, String kingdom){
				int mayor = 0;//METODO CREADO PARA PODER PERMITIRNOS A CONOCER EL SOLDADO CON MAYOR VIDA DE CADA EJERCITO 
				Soldado soldier = null;
				for(int i = 0; i < army.size(); i++){
					for(int j = 0; j < army.get(i).size(); j++){
						if(army.get(i).get(j) != null){ //COMPROBACION QUE HACEMOS PARA PODER DECIR QUE EL CASILLERO DONDE ESTAMOS ES UN SOLDADO QUE EXISTE
							if(army.get(i).get(j).getLifeActual() > mayor){ //COMPARAMOS PUNTOS DE VIDA DE CADA SOLDADO PARA VER QUIEN ES EL MAYOR 
								mayor = army.get(i).get(j).getLifeActual();
								soldier = army.get(i).get(j);
							}
						}
					}
				}
				System.out.println("El soldado con mayor vida del Ejercito " + kingdom + " es: ");
				System.out.println(soldier.toString());//IMPRIMIMOS SUS DATOS PARA PODER VER DE QUE SOLDADO SE TRATA 
				System.out.println("*********************************");
			}
			public double averageLife(ArrayList<ArrayList<Soldado>> army , String kingdom){
				int sum = 0;
				int count = 0;
				System.out.println("El promedio de puntos de vida del Ejercito " + kingdom + " es: ");
				for(int i = 0; i < 10; i++){
					for(int j = 0; j < 10; j++){ //ITERACION LA CUAL NOS AYUDA A PASAR POR TODOS LOS SOLDADOS DE CADA EJERCITO
						if(army.get(i).get(j) != null){ 
							sum += army.get(i).get(j).getLifeActual();
							count++;
						}
					}
				}
				if(sum != 0){
					double avg = sum / (count * 1.0);
					System.out.println(avg); // DAMOS A CONOCER EL PROMEDIO DE VIDA DE CADA EJERCITO 
					System.out.println("*********************************");
					return avg;
				}else{
					double avg = 0;
					System.out.println(avg); // DAMOS A CONOCER EL PROMEDIO DE VIDA DE CADA EJERCITO 
					System.out.println("*********************************");
					return avg;
				}
			}
			public void rankingInsercionLife(ArrayList<ArrayList<Soldado>> army , String kingdom){
				System.out.println("\nEl Ejercito " + kingdom + " ordenando por metodo insercion: ");
				int count = 0;
				for(int i = 0; i < 10; i++){
					for(int j = 0; j < 10; j++){ //ITERACION LA CUAL NOS AYUDA A PASAR POR TODOS LOS SOLDADOS DE CADA EJERCITO
						if(army.get(i).get(j) != null){ 
						   count++;
						}
					}
				}
				System.out.println("------------------------------------------");
				System.out.println("Mostrando Ranking del Ejercito " + kingdom + " ..... ////// --->");
				Soldado[] soldados = new Soldado[count];
				int x = 0;
				for(int i = 0; i < 10; i++){
					for(int j = 0; j < 10; j++){ //ITERACION LA CUAL NOS AYUDA A PASAR POR TODOS LOS SOLDADOS AL ARRAY SOLDADO PARA PODER USAR EL USO DEL METODO DE ORDENACION INSERCION
						if(army.get(i).get(j) != null){ 
							if(count - count + x == count){
								break;
							}else{
								soldados[count - count + x] = army.get(i).get(j); //LA MISMA LOGICA QUE EL ANTERIOR METODO SOLO QUE EN ESTE LO USARIAMOS DE MANERA DIFERENTE YA QUE ESTE SERIA DE FORMA DE INSERCION
							}
							x++;   
						}
					}
				}
				int n = soldados.length;
				for (int i = 1; i < n; i++) {
					Soldado actual = soldados[i];
					int j = i - 1;
					while (j >= 0 && soldados[j].getLifeActual() < actual.getLifeActual()) { //ORDENAMOS EL EJERCITO RESPECTIVAMENTE MEDIANTE EL METODO QUE NOS OFRECE INSERCION EL CUAL ES ESTE CODIGO
						soldados[j + 1] = soldados[j];
						j--;
					}
					soldados[j + 1] = actual;
				}
				for(int i = 0; i < soldados.length; i++){
					System.out.print("\n" + "Puesto " + (i + 1));
					System.out.println(soldados[i].toString()); //PUBLICAMOS RESULTADOS
					System.out.println("------------------");
				}
				System.out.println("*********************************");
			}
			public void setArmy(ArrayList<ArrayList<Soldado>> army){
				this.army = army;
			}
			public void setKingdom(String kingdom){
				this.kingdom = kingdom;
			}
			public String getKingdom(){
				return kingdom;
			}
		}
	\end{lstlisting}
	\begin{lstlisting}[language=java,caption={Las lineas de codigos de la clase Mapa creada:}][H]
		import java.text.DecimalFormat;
		import java.util.*;
		
		public class Mapa {
			Scanner sc = new Scanner(System.in);
			Random rdm = new Random();
			private String territory;
			private ArrayList<ArrayList<Soldado>> board;
			private Ejercito army1e;
			private Ejercito army2e;
			private String[] typesterritory = {"bosque", "campo abierto", "montana", "desierto", "playa"};
			private String[] kingdoms = {"Inglaterra", "Francia", "Sacro", "Castilla", "Aragon", "Moros"};
			public Mapa(){
				this.board = fillboard();
			}
			public void iniciarJuego() {
				menuBatalla();
				int n = sc.nextInt();
				while (n == 1) {
					String kingdom1 = kingdoms[rdm.nextInt(6)];
					String kingdom2 = kingdoms[rdm.nextInt(6)];
					ArrayList<ArrayList<Soldado>> army1 = fillarray(kingdom1, 1);
					ArrayList<ArrayList<Soldado>> army2 = fillarray(kingdom2, 2);
					army1e = new Ejercito(army1,kingdom1);
					army2e = new Ejercito(army2,kingdom2);
					territory = typesterritory[rdm.nextInt(5)];
					System.out.println("\n*********************************");
					System.out.println("El tipo de territorio es: " + territory);
					System.out.println("\n*********************************");
					bonificacion(army1, territory, kingdom1);
					bonificacion(army2, territory, kingdom2);
					army1e.viewSoldiers(kingdom1, 1, army1);
					army2e.viewSoldiers(kingdom2, 2, army2);
					viewBoard(army1, army2);
					army1e.longerLife(army1, kingdom1);
					army2e.longerLife(army2, kingdom2);
					double avg1 = army1e.averageLife(army1, kingdom1);
					double avg2 = army2e.averageLife(army2, kingdom2);
					army1e.rankingInsercionLife(army1, kingdom1);
					army2e.rankingInsercionLife(army2, kingdom2);
					viewBoard(army1, army2);
					resultBattleInfo(army1, kingdom1 , 1);
					resultBattleInfo(army2, kingdom2 , 2);
					int sum1 = resultBattleSum(army1, kingdom1, 1);
					int sum2 = resultBattleSum(army2, kingdom2, 2);
					double sumtotal = (sum1 * 1.0) + (sum2 * 1.0);
					resbattle(sumtotal, sum1, sum2, 1, 2, kingdom1, kingdom2);
					volveraJugar();
					n = sc.nextInt();
				}
			}
			public static ArrayList<ArrayList<Soldado>> fillboard(){
				ArrayList<ArrayList<Soldado>> army = new ArrayList<ArrayList<Soldado>>();
				for(int i = 0; i < 10; i++){ //ITERACION
					army.add(new ArrayList<Soldado>()); //LLENAMOS NUESTROS ARRAYLIST BIDIMENSIONAL CON CADA FILA PARA QUE CUMPLAN CON ESTRUCTURA DEL TABLERO
					for(int j = 0; j < 10 ; j++){//ITERACION
						army.get(i).add(null); // LLENAMOS CADA FILA DEL ARRAYLIST CON UN OBJETO SOLDADO CON TAL QUE ESTE SEA NULL PARA QUE SEPA QUE ESTE TIENE UNA CASILLA PERO NO HAY NADIE TODAVIA SE PUEDE LLENAR 
					}
				}
				return army;
			}
			public static ArrayList<ArrayList<Soldado>> fillarray(String armyespe, int num){
				Random rdm = new Random();
				ArrayList<ArrayList<Soldado>> army = new ArrayList<ArrayList<Soldado>>();
				int numbersoldiers = rdm.nextInt(10) + 1; //NUMERO DE SOLDADOS ALEATORIOS ENTRE 1 A 10 SOLDADOS 
				for(int i = 0; i < 10; i++){ //ITERACION
					army.add(new ArrayList<Soldado>()); //LLENAMOS NUESTROS ARRAYLIST BIDIMENSIONAL CON CADA FILA PARA QUE CUMPLAN CON ESTRUCTURA DEL TABLERO
					for(int j = 0; j < 10 ; j++){//ITERACION
						army.get(i).add(null); // LLENAMOS CADA FILA DEL ARRAYLIST CON UN OBJETO SOLDADO CON TAL QUE ESTE SEA NULL PARA QUE SEPA QUE ESTE TIENE UNA CASILLA PERO NO HAY NADIE TODAVIA SE PUEDE LLENAR 
					}
				}
				System.out.println("El Ejercito " + armyespe + " tiene " + numbersoldiers + " soldados : " ); 
				System.out.println("");
				for(int i = 0; i < numbersoldiers; i++){ //LLENAMOS CASILLAS CON CADA SOLDADO CREADO ALEATORIAMENTE
					Soldado soldado = getRandomSoldado();
					String name = "Soldado" + i + "X" + num;
					//System.out.println(name); PRUEBA QUE SE HIZO PARA VER LOS NOMBRES
					int health = rdm.nextInt(5) + 1;
					int row = rdm.nextInt(10) + 1;
					int speed = rdm.nextInt(5) + 1;
					String column = String.valueOf((char)(rdm.nextInt(10) + 65)); //REUTILIZAMOS CODIGO DEL ANTERIOR ARCHIVO VIDEOJUEGO2.JAVA YA QUE TENDRIAN LA MISMA FUNCIONALIDAD
					//System.out.println(army.get(row - 1).get((int)column.charAt(0) - 65)); PRUEBA QUE SE HIZO PARA COMPROBAR SI EL OBJETO SE ESTABA DANDO O NO CAPAZ NI EXISTIA  
					int lifelevel, defenselevel = 0;
					int attacklevel = 0;
					if (soldado instanceof Espadachin) {
						name = "Espadachin" + i + "X" + num; 
						lifelevel = rdm.nextInt(3) + 8; 
						attacklevel = 10;
						defenselevel = 8;
						soldado.setName(name);                  
						soldado.setAttackLevel(attacklevel);
						soldado.setDefenseLevel(defenselevel);                    
						soldado.setLifeLevel(lifelevel);
						soldado.setRow(row);
						soldado.setColumn(column);
						if(army.get(row - 1).get((int)column.charAt(0) - 65) == null){
							System.out.println("Registrando al " + (i + 1) + " soldado del Ejercito " + armyespe + "");
							army.get(row - 1).set((int)column.charAt(0) - 65, new Espadachin(name, attacklevel, defenselevel, lifelevel, speed, "Espadachin", true, row, column, attacklevel));
							army.get(row - 1).get((int)column.charAt(0) - 65).setSpeed(speed);
							System.out.println(army.get(row - 1).get((int)column.charAt(0) - 65).toString());
							System.out.println("---------------------------------");
						}else{
							i -= 1; //NOS AYUDARIA CON LOS SOLDADOS QUE SE REPITEN EN EL MISMO CASILLERO CON TAL QUE NO DEBERIA CONTAR 
						}
					} else if (soldado instanceof Arquero) {
						name = "Arquero" + i + "X" + num; 
						attacklevel = 7;
						defenselevel = 3;
						lifelevel = rdm.nextInt(3) + 3; 
						soldado.setName(name);                  
						soldado.setAttackLevel(attacklevel);
						soldado.setDefenseLevel(defenselevel);                    
						soldado.setLifeLevel(lifelevel);
						soldado.setRow(row);
						soldado.setColumn(column);
						if(army.get(row - 1).get((int)column.charAt(0) - 65) == null){
							System.out.println("Registrando al " + (i + 1) + " soldado del Ejercito " + armyespe + "");
							army.get(row - 1).set((int)column.charAt(0) - 65, new Arquero(name, attacklevel, defenselevel, lifelevel, speed, "Arquero", true, row, column, attacklevel));
							army.get(row - 1).get((int)column.charAt(0) - 65).setSpeed(speed);
							System.out.println(army.get(row - 1).get((int)column.charAt(0) - 65).toString());
							System.out.println("---------------------------------");
						}else{
							i -= 1; //NOS AYUDARIA CON LOS SOLDADOS QUE SE REPITEN EN EL MISMO CASILLERO CON TAL QUE NO DEBERIA CONTAR 
						}
					} else if (soldado instanceof Caballero) {
						name = "Caballero" + i + "X" + num; 
						attacklevel = 13;
						defenselevel = 7;
						lifelevel = rdm.nextInt(3) + 10; 
						soldado.setName(name);                  
						soldado.setAttackLevel(attacklevel);
						soldado.setDefenseLevel(defenselevel);                    
						soldado.setLifeLevel(lifelevel);
						soldado.setRow(row);
						soldado.setColumn(column);
						if(army.get(row - 1).get((int)column.charAt(0) - 65) == null){
							System.out.println("Registrando al " + (i + 1) + " soldado del Ejercito " + armyespe + "");
							army.get(row - 1).set((int)column.charAt(0) - 65, new Caballero(name, attacklevel, defenselevel, lifelevel, speed, "Caballero", true, row, column, false));
							army.get(row - 1).get((int)column.charAt(0) - 65).setSpeed(speed);
							System.out.println(army.get(row - 1).get((int)column.charAt(0) - 65).toString());
							System.out.println("---------------------------------");
						}else{
							i -= 1; //NOS AYUDARIA CON LOS SOLDADOS QUE SE REPITEN EN EL MISMO CASILLERO CON TAL QUE NO DEBERIA CONTAR 
						}
					} else if (soldado instanceof Lancero) {
						name = "Lancero" + i + "X" + num; 
						attacklevel = 5;
						defenselevel = 10;
						lifelevel = rdm.nextInt(3) + 5;
						soldado.setName(name);                  
						soldado.setAttackLevel(attacklevel);
						soldado.setDefenseLevel(defenselevel);                    
						soldado.setLifeLevel(lifelevel);
						soldado.setRow(row);
						soldado.setColumn(column);
						if(army.get(row - 1).get((int)column.charAt(0) - 65) == null){
							System.out.println("Registrando al " + (i + 1) + " soldado del Ejercito " + armyespe + "");
							army.get(row - 1).set((int)column.charAt(0) - 65, new Lancero(name, attacklevel, defenselevel, lifelevel, speed, "Lancero", true, row, column, attacklevel));
							army.get(row - 1).get((int)column.charAt(0) - 65).setSpeed(speed);
							System.out.println(army.get(row - 1).get((int)column.charAt(0) - 65).toString());
							System.out.println("---------------------------------");
						}else{
							i -= 1; //NOS AYUDARIA CON LOS SOLDADOS QUE SE REPITEN EN EL MISMO CASILLERO CON TAL QUE NO DEBERIA CONTAR 
						}
					}
				}
				System.out.println("*********************************");
				return army;
			}
			public static void menuBatalla(){
				System.out.println("-------------------------------------------");
				System.out.println("--                 MENU                  --"); 
				System.out.println("-------------------------------------------");
				System.out.println(" SELECCIONE UN NUMERO PARA PODER EMPEZAR O TERMINAR");
				System.out.println(" 1 : JUGAR");
				System.out.println(" 2 : NO JUGAR");
			}
			public static void viewBoard(ArrayList<ArrayList<Soldado>> army1, ArrayList<ArrayList<Soldado>> army2){ //EN ESTE METODO DEMOSTRAREMOS LA TABLA REUTILIZAREMOS CODIGOS DE ANTERIORES LABORATORIOS PARA PODER HACER LA BASE DE ESTE TABLERO
				System.out.println("\nMostrando tabla de posicion ... --");
				System.out.println("Leyenda: Ejercito1 --> 1#1 | Ejercito2 --> 2#2"); //RECONOCIMIENTO PARA LOS EJERCITOS Y POSICION DE SUS SOLDADOS
				System.out.println("\n \t   A\t   B\t   C\t   D\t   E\t   F\t   G\t   H\t   I\t   J"); // RECONOCIMIENTO PARA CADA UBICACION DE CADA SOLDADO EN EL TABLERO POR PARTE DE LAS COLUMNAS
				System.out.println("\t_________________________________________________________________________________");
				for(int i = 0; i < 10; i++ ){
					System.out.print((i + 1) + "\t"); // RECONOCIMIENTO PARA CADA UBICACION DE CADA SOLDADO EN EL TABLERO POR PARTE DE LAS FILAS
						for(int j = 0; j < 10; j++){
								if(army1.get(i).get(j) != null && army2.get(i).get(j) != null){ //CREAMOS UN IF PARA QUE ESTE NOS AYUDE A SABER QUIEN DE ESTOS SOLDADOS SE OCUPARA DEL CASILLERO EL CUAL DONDE ESTAN PELEANDO
									if(army1.get(i).get(j).getLifeActual() > army2.get(i).get(j).getLifeActual()){
										army1.get(i).get(j).setLifeActual(army1.get(i).get(j).getLifeActual() - army2.get(i).get(j).getLifeActual()); //Cambiamos 
										army2.get(i).set(j, null); 
										System.out.print("|  1" + obtenerInicial(army1.get(i).get(j)) + "1  ");
									}else if(army2.get(i).get(j).getLifeActual() > army1.get(i).get(j).getLifeActual()){
										army2.get(i).get(j).setLifeActual(army2.get(i).get(j).getLifeActual() - army1.get(i).get(j).getLifeActual());
										army1.get(i).set(j, null);;
										System.out.print("|  2" + obtenerInicial(army2.get(i).get(j)) + "2  ");
									}else{
										army2.get(i).set(j, null);
										army1.get(i).set(j, null);
										System.out.print("|   " + "" + "   ");
									}
								}else if(army1.get(i).get(j) != null){
									System.out.print("|  1" + obtenerInicial(army1.get(i).get(j)) + "1  ");
								}else if(army2.get(i).get(j) != null){
									System.out.print("|  2" + obtenerInicial(army2.get(i).get(j)) + "2  ");
								}else{
									System.out.print("|   " + " " + "   ");
								}
						}
						System.out.println("|");
						System.out.println("\t|_______|_______|_______|_______|_______|_______|_______|_______|_______|_______|");
				}
				System.out.println("\n*********************************");
			}
			public static Soldado getRandomSoldado() {
				Random rdm = new Random();
				int tipoSoldado = rdm.nextInt(4);
				switch (tipoSoldado) {
					case 0:
						return new Espadachin();
					case 1:
						return new Arquero();
					case 2:
						return new Lancero();
					case 3:
						return new Caballero();
					default:
						return new Espadachin();
				}
			}
			public static String obtenerInicial(Soldado soldado) {
				if (soldado instanceof Espadachin) {
					return "E";
				} else if (soldado instanceof Arquero) {
					return "A";
				} else if (soldado instanceof Caballero) {
					return "C";
				} else if (soldado instanceof Lancero) {
					return "L";
				} else {
					return "S";
				}
			}
			public void bonificacion(ArrayList<ArrayList<Soldado>> army, String territory , String kingdom) {
				for(int i = 0; i < 10; i++){ //ITERACION
					for(int j = 0; j < 10 ; j++){//ITERACION
						if(army.get(i).get(j) != null){
							if(kingdom.equals("Inglaterra") && territory.equals("bosque")){
								army.get(i).get(j).setLifeLevel(army.get(i).get(j).getLifeLevel() + 1);
							}else if(kingdom.equals("Francia") && territory.equals("campo abierto")){
								army.get(i).get(j).setLifeLevel(army.get(i).get(j).getLifeLevel() + 1);
							}else if((kingdom.equals("Castilla") || kingdom.equals("Aragon")) && territory.equals("montana")){
								army.get(i).get(j).setLifeLevel(army.get(i).get(j).getLifeLevel() + 1);
							}else if(kingdom.equals("Moros") && territory.equals("desierto")){
								army.get(i).get(j).setLifeLevel(army.get(i).get(j).getLifeLevel() + 1);
							}else if(kingdom.equals("Sacro") && (territory.equals("desierto") || territory.equals("playa") || territory.equals("campo abierto"))){
								army.get(i).get(j).setLifeLevel(army.get(i).get(j).getLifeLevel() + 1);
							}
						}
					}
				}
			} 
			public static void resultBattleInfo(ArrayList<ArrayList<Soldado>> army, String kingdom, int n){
				System.out.println("Ejercito " + n + " : " + kingdom);
				int numbersoldiers = 0;
				int numberespadachines = 0;
				int numbercaballeros = 0;
				int numberlanceros = 0;
				int numberarqueros = 0;
				for(int i = 0; i < 10; i++){ //ITERACION
					for(int j = 0; j < 10 ; j++){//ITERACION
						if(army.get(i).get(j) != null){
							numbersoldiers++;
							if(army.get(i).get(j) instanceof Espadachin){
								numberespadachines++;
							}else if(army.get(i).get(j) instanceof Caballero){
								numbercaballeros++;
							}else if(army.get(i).get(j) instanceof Lancero){
								numberlanceros++;
							}else if(army.get(i).get(j) instanceof Arquero){
								numberarqueros++;
							}
						}
					}
				}
				System.out.println("Cantidad total de soldados creados: " + numbersoldiers);
				System.out.println("Espadachines: " + numberespadachines);
				System.out.println("Arqueros: " + numberarqueros);
				System.out.println("Caballeros: " + numbercaballeros);
				System.out.println("Lanceros: " + numberlanceros + "\n");
			}
			public static int resultBattleSum(ArrayList<ArrayList<Soldado>> army, String kingdom, int n){
				int sum = 0;
				for(int i = 0; i < 10; i++){ //ITERACION
					for(int j = 0; j < 10 ; j++){//ITERACION
						if(army.get(i).get(j) != null){
							sum += army.get(i).get(j).getLifeLevel();
						}
					}
				}
				System.out.println("Ejercito " + n + ": " + kingdom + ": " + sum);
				return sum;
			}
			public static void resbattle(double sumtotal , int sum1, int sum2 , int n1, int n2 , String kingdom1, String kingdom2){
				DecimalFormat df = new DecimalFormat("#.##");
				String numero1 = df.format(((sum1 / sumtotal) * 100));
				String numero2 = df.format(((sum2 / sumtotal) * 100));
				if(sum1 > sum2){
					System.out.println("El ganador es el ejercito " + n1 + " de: " + kingdom1 +". Ya que al generar los porcentajes de probabilidad de victoria basada en los niveles de vida de sus soldados y aplicando un experimento aleatorio salio vencedor. (Aleatorio generado : "+ numero1 + ")");
				}else if (sum2 > sum1){
					System.out.println("El ganador es el ejercito " + n2 + " de: " + kingdom2 +". Ya que al generar los porcentajes de probabilidad de victoria basada en los niveles de vida de sus soldados y aplicando un experimento aleatorio salio vencedor. (Aleatorio generado : "+ numero2 + ")");
				}else{
					System.out.println("El resultado de la batalla es Empate");
				}
			}
			public static void volveraJugar(){
				System.out.println("\n*********************************");
				System.out.println(" DESEA VOLVER A JUGAR");
				System.out.println(" 1 : JUGAR");
				System.out.println(" 2 : NO JUGAR");
			}
		}
	\end{lstlisting}
	\subsection{Ejercicio Mapa}
	\begin{itemize}	
		\item En esta seccion solo mejoramos cuando se imprime el tablero con la informacion del soldado el cual esta en esta cuadrilla .
		\item El codigo y el commit seria el siguiente:
	\end{itemize}	
	\begin{lstlisting}[language=bash,caption={Commit}][H]
		$ git commit -m "Mejorando el tablero su informacion con el ejercito donde pertenecen , el tipo de soldado y su nivel de vida"
	\end{lstlisting}	
	\begin{lstlisting}[language=java,caption={Las lineas de codigos de la clase Mapa:}][H]
		public static void viewBoard(ArrayList<ArrayList<Soldado>> army1, ArrayList<ArrayList<Soldado>> army2){ //EN ESTE METODO DEMOSTRAREMOS LA TABLA REUTILIZAREMOS CODIGOS DE ANTERIORES LABORATORIOS PARA PODER HACER LA BASE DE ESTE TABLERO
        System.out.println("\nMostrando tabla de posicion ... --");
        System.out.println("Leyenda: Ejercito1 --> 1#- | Ejercito2 --> 2#-"); //RECONOCIMIENTO PARA LOS EJERCITOS Y POSICION DE SUS SOLDADOS
        System.out.println("\n \t   A\t   B\t   C\t   D\t   E\t   F\t   G\t   H\t   I\t   J"); // RECONOCIMIENTO PARA CADA UBICACION DE CADA SOLDADO EN EL TABLERO POR PARTE DE LAS COLUMNAS
        System.out.println("\t_________________________________________________________________________________");
        for(int i = 0; i < 10; i++ ){
            System.out.print((i + 1) + "\t"); // RECONOCIMIENTO PARA CADA UBICACION DE CADA SOLDADO EN EL TABLERO POR PARTE DE LAS FILAS
                for(int j = 0; j < 10; j++){
                        if(army1.get(i).get(j) != null && army2.get(i).get(j) != null){ //CREAMOS UN IF PARA QUE ESTE NOS AYUDE A SABER QUIEN DE ESTOS SOLDADOS SE OCUPARA DEL CASILLERO EL CUAL DONDE ESTAN PELEANDO
                            if(army1.get(i).get(j).getLifeActual() > army2.get(i).get(j).getLifeActual()){
                                army1.get(i).get(j).setLifeActual(army1.get(i).get(j).getLifeActual() - army2.get(i).get(j).getLifeActual()); //Cambiamos 
                                army2.get(i).set(j, null); 
                                System.out.printf("|  1%s%2d ", obtenerInicial(army1.get(i).get(j)), army1.get(i).get(j).getLifeActual());
                            }else if(army2.get(i).get(j).getLifeActual() > army1.get(i).get(j).getLifeActual()){
                                army2.get(i).get(j).setLifeActual(army2.get(i).get(j).getLifeActual() - army1.get(i).get(j).getLifeActual());
                                army1.get(i).set(j, null);;
                                System.out.printf("|  2%s%2d ", obtenerInicial(army2.get(i).get(j)), army2.get(i).get(j).getLifeActual());
                            }else{
                                army2.get(i).set(j, null);
                                army1.get(i).set(j, null);
                                System.out.print("|   " + "" + "   ");
                            }
                        }else if(army1.get(i).get(j) != null){
                            System.out.printf("|  1%s%2d ", obtenerInicial(army1.get(i).get(j)), army1.get(i).get(j).getLifeActual());
                        }else if(army2.get(i).get(j) != null){
                            System.out.printf("|  2%s%2d ", obtenerInicial(army2.get(i).get(j)), army2.get(i).get(j).getLifeActual());
                        }else{
                            System.out.print("|   " + " " + "   ");
                        }
                }
                System.out.println("|");
                System.out.println("\t|_______|_______|_______|_______|_______|_______|_______|_______|_______|_______|");
        }
        System.out.println("\n*********************************");
    }
	\end{lstlisting}
	\begin{lstlisting}[language=bash,caption={Ejecucion:}][H]
		-------------------------------------------
		--                 MENU                  --
		-------------------------------------------
		 SELECCIONE UN NUMERO PARA PODER EMPEZAR O TERMINAR
		 1 : JUGAR
		 2 : NO JUGAR
		1
		El Ejercito Inglaterra tiene 9 soldados : 
		
		Registrando al 1 soldado del Ejercito Inglaterra
		
		Nombre: Espadachin0X1
		Vida: 10
		Fila: 2
		Columna: E
		Nivel de ataque: 10
		Nivel de Defensa: 8
		Nivel de vida: 10
		Velocidad: 1
		Actitud: Espadachin
		Estado: true
		---------------------------------
		Registrando al 2 soldado del Ejercito Inglaterra
		
		Nombre: Caballero1X1
		Vida: 12
		Fila: 9
		Columna: E
		Nivel de ataque: 13
		Nivel de Defensa: 7
		Nivel de vida: 12
		Velocidad: 5
		Actitud: Caballero
		Estado: true
		---------------------------------
		Registrando al 3 soldado del Ejercito Inglaterra
		
		Nombre: Lancero2X1
		Vida: 7
		Fila: 2
		Columna: G
		Nivel de ataque: 5
		Nivel de Defensa: 10
		Nivel de vida: 7
		Velocidad: 1
		Actitud: Lancero
		Estado: true
		---------------------------------
		Registrando al 4 soldado del Ejercito Inglaterra
		
		Nombre: Caballero3X1
		Vida: 11
		Fila: 10
		Columna: J
		Nivel de ataque: 13
		Nivel de Defensa: 7
		Nivel de vida: 11
		Velocidad: 3
		Actitud: Caballero
		Estado: true
		---------------------------------
		Registrando al 5 soldado del Ejercito Inglaterra
		
		Nombre: Espadachin4X1
		Vida: 10
		Fila: 7
		Columna: D
		Nivel de ataque: 10
		Nivel de Defensa: 8
		Nivel de vida: 10
		Velocidad: 1
		Actitud: Espadachin
		Estado: true
		---------------------------------
		Registrando al 6 soldado del Ejercito Inglaterra
		
		Nombre: Espadachin5X1
		Vida: 8
		Fila: 6
		Columna: J
		Nivel de ataque: 10
		Nivel de Defensa: 8
		Nivel de vida: 8
		Velocidad: 4
		Actitud: Espadachin
		Estado: true
		---------------------------------
		Registrando al 7 soldado del Ejercito Inglaterra
		
		Nombre: Lancero6X1
		Vida: 7
		Fila: 8
		Columna: F
		Nivel de ataque: 5
		Nivel de Defensa: 10
		Nivel de vida: 7
		Velocidad: 3
		Actitud: Lancero
		Estado: true
		---------------------------------
		Registrando al 8 soldado del Ejercito Inglaterra
		
		Nombre: Espadachin7X1
		Vida: 8
		Fila: 10
		Columna: E
		Nivel de ataque: 10
		Nivel de Defensa: 8
		Nivel de vida: 8
		Velocidad: 2
		Actitud: Espadachin
		Estado: true
		---------------------------------
		Registrando al 9 soldado del Ejercito Inglaterra
		
		Nombre: Lancero8X1
		Vida: 6
		Fila: 1
		Columna: D
		Nivel de ataque: 5
		Nivel de Defensa: 10
		Nivel de vida: 6
		Velocidad: 2
		Actitud: Lancero
		Estado: true
		---------------------------------
		*********************************
		El Ejercito Sacro tiene 3 soldados : 
		
		Registrando al 1 soldado del Ejercito Sacro
		
		Nombre: Espadachin0X2
		Vida: 10
		Fila: 4
		Columna: I
		Nivel de ataque: 10
		Nivel de Defensa: 8
		Nivel de vida: 10
		Velocidad: 4
		Actitud: Espadachin
		Estado: true
		---------------------------------
		Registrando al 2 soldado del Ejercito Sacro
		
		Nombre: Lancero1X2
		Vida: 5
		Fila: 3
		Columna: G
		Nivel de ataque: 5
		Nivel de Defensa: 10
		Nivel de vida: 5
		Velocidad: 1
		Actitud: Lancero
		Estado: true
		---------------------------------
		Registrando al 3 soldado del Ejercito Sacro
		
		Nombre: Lancero2X2
		Vida: 7
		Fila: 7
		Columna: B
		Nivel de ataque: 5
		Nivel de Defensa: 10
		Nivel de vida: 7
		Velocidad: 3
		Actitud: Lancero
		Estado: true
		---------------------------------
		*********************************
		
		*********************************
		El tipo de territorio es: campo abierto
		
		*********************************
		El Ejercito Inglaterra del 1 ejercito sus soldados son :
		
		*********************************
		El 1 soldado es: 
		
		Nombre: Lancero8X1
		Vida: 6
		Fila: 1
		Columna: D
		Nivel de ataque: 5
		Nivel de Defensa: 10
		Nivel de vida: 6
		Velocidad: 2
		Actitud: Lancero
		Estado: true
		
		*********************************
		El 2 soldado es: 
		
		Nombre: Espadachin0X1
		Vida: 10
		Fila: 2
		Columna: E
		Nivel de ataque: 10
		Nivel de Defensa: 8
		Nivel de vida: 10
		Velocidad: 1
		Actitud: Espadachin
		Estado: true
		
		*********************************
		El 3 soldado es: 
		
		Nombre: Lancero2X1
		Vida: 7
		Fila: 2
		Columna: G
		Nivel de ataque: 5
		Nivel de Defensa: 10
		Nivel de vida: 7
		Velocidad: 1
		Actitud: Lancero
		Estado: true
		
		*********************************
		El 4 soldado es: 
		
		Nombre: Espadachin5X1
		Vida: 8
		Fila: 6
		Columna: J
		Nivel de ataque: 10
		Nivel de Defensa: 8
		Nivel de vida: 8
		Velocidad: 4
		Actitud: Espadachin
		Estado: true
		
		*********************************
		El 5 soldado es: 
		
		Nombre: Espadachin4X1
		Vida: 10
		Fila: 7
		Columna: D
		Nivel de ataque: 10
		Nivel de Defensa: 8
		Nivel de vida: 10
		Velocidad: 1
		Actitud: Espadachin
		Estado: true
		
		*********************************
		El 6 soldado es: 
		
		Nombre: Lancero6X1
		Vida: 7
		Fila: 8
		Columna: F
		Nivel de ataque: 5
		Nivel de Defensa: 10
		Nivel de vida: 7
		Velocidad: 3
		Actitud: Lancero
		Estado: true
		
		*********************************
		El 7 soldado es: 
		
		Nombre: Caballero1X1
		Vida: 12
		Fila: 9
		Columna: E
		Nivel de ataque: 13
		Nivel de Defensa: 7
		Nivel de vida: 12
		Velocidad: 5
		Actitud: Caballero
		Estado: true
		
		*********************************
		El 8 soldado es: 
		
		Nombre: Espadachin7X1
		Vida: 8
		Fila: 10
		Columna: E
		Nivel de ataque: 10
		Nivel de Defensa: 8
		Nivel de vida: 8
		Velocidad: 2
		Actitud: Espadachin
		Estado: true
		
		*********************************
		El 9 soldado es: 
		
		Nombre: Caballero3X1
		Vida: 11
		Fila: 10
		Columna: J
		Nivel de ataque: 13
		Nivel de Defensa: 7
		Nivel de vida: 11
		Velocidad: 3
		Actitud: Caballero
		Estado: true
		El Ejercito Sacro del 2 ejercito sus soldados son :
		
		*********************************
		El 1 soldado es: 
		
		Nombre: Lancero1X2
		Vida: 5
		Fila: 3
		Columna: G
		Nivel de ataque: 5
		Nivel de Defensa: 10
		Nivel de vida: 6
		Velocidad: 1
		Actitud: Lancero
		Estado: true
		
		*********************************
		El 2 soldado es: 
		
		Nombre: Espadachin0X2
		Vida: 10
		Fila: 4
		Columna: I
		Nivel de ataque: 10
		Nivel de Defensa: 8
		Nivel de vida: 11
		Velocidad: 4
		Actitud: Espadachin
		Estado: true
		
		*********************************
		El 3 soldado es: 
		
		Nombre: Lancero2X2
		Vida: 7
		Fila: 7
		Columna: B
		Nivel de ataque: 5
		Nivel de Defensa: 10
		Nivel de vida: 8
		Velocidad: 3
		Actitud: Lancero
		Estado: true
		
	\end{lstlisting}
	\begin{figure}[H]
		\centering
		\includegraphics[width=1.0\textwidth,keepaspectratio]{img/Commit3.png}
		%\includesvg{img/automata.svg}
		%\label{img:mot2}
		%\caption{Product backlog.}
	\end{figure}
	\subsection{Ejercicio Soldado(Herencia)}
	\begin{itemize}	
		\item En esta seccion publicamos los otros tipos de soldados especiales por cada reino los cuales son CaballeroFranco, CaballeroMoro , EspadachinReal , EspadachinConquistador, EspadachinTeutonico.
		\item El codigo y el commit seria el siguiente:
	\end{itemize}	
	\begin{lstlisting}[language=bash,caption={Commit}][H]
		$ git commit -m "Agregando la clase EspadachinReal el cual es una unidad especial para el reino de inglaterra creamos sus metodos y sus atributos necesarios"
		$ git commit -m "Agregando la clase CaballeroFranco la cual tiene sus metodos , getters y setters, y sus debidos atributos"
		$ git commit -m "Agregando la clase EspadachinConquistador la cual tiene sus metodos,getters y setters, y sus debidos atributos"
		$ git commit -m "Agregando la clase EspadachinTeutonico la cual tiene sus metodos como el modoTortuga() y otros, getters y setters, y sus debidos atributos"
		$ git commit -m "Agregando la clase CaballeroMoro la cual tiene sus metodos, getters y setters, y sus debidos atributos"
	\end{lstlisting}	
	\begin{lstlisting}[language=java,caption={Las lineas de codigos de la clase CaballeroFranco:}][H]
		public class CaballeroFranco extends Soldado{
			private int numspears;
			private int longspears;
			private int evolve = 0;
			public CaballeroFranco(){
			}
			public CaballeroFranco(String name , int attacklevel, int defenselevel, int lifelevel, int speed, String attitude ,boolean lives, int row, String column,int numspears, int longspears){
				super(name, attacklevel, defenselevel, lifelevel, speed, attitude, lives, row, column);
				this.numspears = numspears;
				this.longspears = longspears;
			}
			public void throwSpears(){
				if(numspears == 0){
					System.out.println("Ya no quedan lanzas");
				}else{
					numspears--;
				}
			}
			public void evolveSoldier(){
				if(evolve < 4){
					evolve++;
					numspears += evolve;
					numspears += evolve;
					System.out.println("Caballero Franco evoluciono a nivel " + evolve);
				}else{
					System.out.println("Caballero Franco ya esta en su nivel maximo de evolucion.");
				}
			}
			public int getnumSpears(){
				return numspears;
			}
			public int getlongSpears(){
				return longspears;
			}
			public void setnumSpears(int n){
				this.numspears = n;
			}
		}
	\end{lstlisting}
	\begin{lstlisting}[language=java,caption={Las lineas de codigos de la clase CaballeroMoro:}][H]
		public class CaballeroMoro extends Soldado{
			private int flechas;
			private int longflechas;
			private int evolve = 0;
			public CaballeroMoro(){
			}
			public CaballeroMoro(String name , int attacklevel, int defenselevel, int lifelevel, int speed, String attitude ,boolean lives, int row, String column, int flechas, int longflechas){
				super(name, attacklevel, defenselevel, lifelevel, speed, attitude, lives, row, column);
				this.flechas = flechas;
				this.longflechas = longflechas;
			}
			public void disparar(){
				if(this.flechas == 0){
					System.out.println("El arquero ya no tiene flechas para poder disparar");
				}else{
					this.flechas = flechas - 1;
					this.atacar();
				}
			}
			public void evolveSoldier(){
				if(evolve < 4){
					evolve++;
					flechas += evolve;
					longflechas += evolve;
					System.out.println("Espadachin Conquistador evoluciono a nivel " + evolve);
				}else{
					System.out.println("Espadachin Conquistador ya esta en su nivel maximo de evolucion.");
				}
			}
			public void setFlechas(int n){
				this.flechas = n;
			}
			public int getFlechas(){
				return flechas;
			}
			public int getLongFlechas(){
				return longflechas;
			}
		}
	\end{lstlisting}
	\begin{lstlisting}[language=java,caption={Las lineas de codigos de la clase EspadachinReal:}][H]
		public class EspadachinReal extends Soldado{
			private int numknifes;
			private int longknife;
			private int evolve = 0;
			public EspadachinReal(){
			}
			public EspadachinReal(String name , int attacklevel, int defenselevel, int lifelevel, int speed, String attitude ,boolean lives, int row, String column, int numknifes, int longknife){
				super(name, attacklevel, defenselevel, lifelevel, speed, attitude, lives, row, column);
				this.numknifes = numknifes;
				this.longknife = longknife;
			}
			public int getnumKnifes(){
				return numknifes;
			}
			public int getlongKnifes(){
				return longknife;
			}
			public void throwKnifes(){
				if(numknifes == 0){
					System.out.println("Ya no quedan cuchillos");
				}else{
					numknifes--;
				}
			}
			public void evolveSoldier(){
				if(evolve < 4){
					evolve++;
					numknifes += evolve;
					longknife += evolve;
					System.out.println("Espadachin Real evoluciono a nivel " + evolve);
				}else{
					System.out.println("Espadachin Real ya esta en su nivel maximo de evolucion.");
				}
			}
			public void setnumKnifes(int n){
				this.numknifes = n;
			}
		}
	\end{lstlisting}
	\begin{lstlisting}[language=java,caption={Las lineas de codigos de la clase EspadachinConquistador:}][H]
		public class EspadachinConquistador extends Soldado{
			private int numaxes;
			private int longaxes;
			private int evolve;
			public EspadachinConquistador(){
			}
			public EspadachinConquistador(String name , int attacklevel, int defenselevel, int lifelevel, int speed, String attitude ,boolean lives, int row, String column, int numaxes, int longaxes){
				super(name, attacklevel, defenselevel, lifelevel, speed, attitude, lives, row, column);
				this.numaxes = numaxes;
				this.longaxes = longaxes;
			}
			public void throwAxes(){
				if(numaxes == 0){
					System.out.println("No es posible que el espadachin lance hachas");
				} else {
					numaxes--;
				}
			}
			public void evolveSoldier(){
				if(evolve < 4){
					evolve++;
					numaxes += evolve;
					longaxes += evolve;
					System.out.println("Espadachin Conquistador evoluciono a nivel " + evolve);
				}else{
					System.out.println("Espadachin Conquistador ya esta en su nivel maximo de evolucion.");
				}
			}
			public void setnumAxes(int n){
				this.numaxes = n;
			}
			public int getnumAxes(){
				return numaxes;
			}
			public int getlongAxes(){
				return longaxes;
			}
		}
	\end{lstlisting}
	\begin{lstlisting}[language=java,caption={Las lineas de codigos de la clase EspadachinTeutonico:}][H]
		public class EspadachinTeutonico extends Soldado{
			private int numjavelin;
			private int longjavelin;
			private int evolve;
			public EspadachinTeutonico(){
			}
			public EspadachinTeutonico(String name , int attacklevel, int defenselevel, int lifelevel, int speed, String attitude ,boolean lives, int row, String column, int numjavelin, int longjavelin){
				super(name, attacklevel, defenselevel, lifelevel, speed, attitude, lives, row, column);
				this.numjavelin = numjavelin;
				this.longjavelin = longjavelin;
			}
			public void throwJavelin(){
				if(numjavelin == 0){
					System.out.println("No es posible que el espadachin lance jabalinas");
				} else {
					numjavelin--;
				}
			}
			public void modoTortuga(){
				System.out.println("Usted uso la habilidad modo Tortuga, Defensa Especial");
			}
			public void evolveSoldier(){
				if(evolve < 4){
					evolve++;
					numjavelin += evolve;
					longjavelin += evolve;
					System.out.println("Espadachin Teutonico evoluciono a nivel " + evolve);
				}else{
					System.out.println("Espadachin Teutonico ya esta en su nivel maximo de evolucion.");
				}
			}
			public void setnumJavelin(int n){
				this.numjavelin = n;
			}
			public int getnumJavelin(){
				return numjavelin;
			}
			public int getlongJavelin(){
				return longjavelin;
			}
		}
	\end{lstlisting}
	\subsection{Ejercicio Mapa}
	\begin{itemize}	
		\item En esta seccion publicamos los otros tipos de soldados especiales por cada reino los cuales son CaballeroFranco, CaballeroMoro , EspadachinReal , EspadachinConquistador, EspadachinTeutonico.
		\item El codigo y el commit seria el siguiente:
	\end{itemize}	
	\begin{lstlisting}[language=bash,caption={Commit}][H]
		$ git commit -m "Agregando opciones en getRandomSoldado() el cual serian estas clases heredadas anteriormente respetando si estas pertenecen a cierto reino especifico tambien modificamos el metodo obtenerIncial() el cual tambien agregamos las inciales de cada tipo de Soldado y tambien en el metodo fillarray() el cual creara a cada soldado dependiendo el tipo que sea con un constructor seria solo agregar mas clases heredadas en resumen "
	\end{lstlisting}	
	\begin{lstlisting}[language=java,caption={Las lineas de codigos de la clase Mapa}][H]
		public static Soldado getRandomSoldado(String armyespe) {
			Random rdm = new Random();
			int tipoSoldado = rdm.nextInt(5);
			switch (tipoSoldado) {
				case 0:
					return new Espadachin();
				case 1:
					return new Arquero();
				case 2:
					return new Lancero();
				case 3:
					return new Caballero();
				case 4:
					if(armyespe == "Inglaterra"){
						return new EspadachinReal();
					}else if(armyespe == "Francia"){
						return new CaballeroFranco();
					}else if(armyespe == "Sacro"){
						return new EspadachinTeutonico();
					}else if(armyespe == "Aragon" || armyespe == "Castilla"){
						return new EspadachinConquistador();
					}else if(armyespe == "Moros"){
						return new CaballeroMoro();
					}
				default:
					return new Espadachin();
			}
		}
		public static String obtenerInicial(Soldado soldado) {
			if (soldado instanceof Espadachin) {
				return "E";
			} else if (soldado instanceof Arquero) {
				return "A";
			} else if (soldado instanceof Caballero) {
				return "C";
			} else if (soldado instanceof Lancero) {
				return "L";
			} else if (soldado instanceof EspadachinReal){
				return "ER";
			} else if (soldado instanceof CaballeroFranco){
				return "CF";
			} else if (soldado instanceof EspadachinTeutonico){
				return "ET";
			} else if (soldado instanceof EspadachinConquistador){
				return "EC";
			} else if (soldado instanceof CaballeroMoro){
				return "CM";
			}else{
				return " ";
			}
		}
		public static ArrayList<ArrayList<Soldado>> fillarray(String armyespe, int num){
			Random rdm = new Random();
			ArrayList<ArrayList<Soldado>> army = new ArrayList<ArrayList<Soldado>>();
			int numbersoldiers = rdm.nextInt(10) + 1; //NUMERO DE SOLDADOS ALEATORIOS ENTRE 1 A 10 SOLDADOS 
			for(int i = 0; i < 10; i++){ //ITERACION
				army.add(new ArrayList<Soldado>()); //LLENAMOS NUESTROS ARRAYLIST BIDIMENSIONAL CON CADA FILA PARA QUE CUMPLAN CON ESTRUCTURA DEL TABLERO
				for(int j = 0; j < 10 ; j++){//ITERACION
					army.get(i).add(null); // LLENAMOS CADA FILA DEL ARRAYLIST CON UN OBJETO SOLDADO CON TAL QUE ESTE SEA NULL PARA QUE SEPA QUE ESTE TIENE UNA CASILLA PERO NO HAY NADIE TODAVIA SE PUEDE LLENAR 
				}
			}
			System.out.println("El Ejercito " + armyespe + " tiene " + numbersoldiers + " soldados : " ); 
			System.out.println("");
			for(int i = 0; i < numbersoldiers; i++){ //LLENAMOS CASILLAS CON CADA SOLDADO CREADO ALEATORIAMENTE
				Soldado soldado = getRandomSoldado(armyespe);
				String name = "Soldado" + i + "X" + num;
				//System.out.println(name); PRUEBA QUE SE HIZO PARA VER LOS NOMBRES
				int health = rdm.nextInt(5) + 1;
				int row = rdm.nextInt(10) + 1;
				int speed = rdm.nextInt(5) + 1;
				String column = String.valueOf((char)(rdm.nextInt(10) + 65)); //REUTILIZAMOS CODIGO DEL ANTERIOR ARCHIVO VIDEOJUEGO2.JAVA YA QUE TENDRIAN LA MISMA FUNCIONALIDAD
				//System.out.println(army.get(row - 1).get((int)column.charAt(0) - 65)); PRUEBA QUE SE HIZO PARA COMPROBAR SI EL OBJETO SE ESTABA DANDO O NO CAPAZ NI EXISTIA  
				int lifelevel, defenselevel = 0;
				int attacklevel = 0;
				if (soldado instanceof Espadachin) {
					name = "Espadachin" + i + "X" + num; 
					lifelevel = rdm.nextInt(3) + 8; 
					attacklevel = 10;
					defenselevel = 8;
					soldado.setName(name);                  
					soldado.setAttackLevel(attacklevel);
					soldado.setDefenseLevel(defenselevel);                    
					soldado.setLifeLevel(lifelevel);
					soldado.setRow(row);
					soldado.setColumn(column);
					if(army.get(row - 1).get((int)column.charAt(0) - 65) == null){
						System.out.println("Registrando al " + (i + 1) + " soldado del Ejercito " + armyespe + "");
						army.get(row - 1).set((int)column.charAt(0) - 65, new Espadachin(name, attacklevel, defenselevel, lifelevel, speed, "Espadachin", true, row, column, attacklevel));
						army.get(row - 1).get((int)column.charAt(0) - 65).setSpeed(speed);
						System.out.println(army.get(row - 1).get((int)column.charAt(0) - 65).toString());
						System.out.println("---------------------------------");
					}else{
						i -= 1; //NOS AYUDARIA CON LOS SOLDADOS QUE SE REPITEN EN EL MISMO CASILLERO CON TAL QUE NO DEBERIA CONTAR 
					}
				} else if (soldado instanceof Arquero) {
					name = "Arquero" + i + "X" + num; 
					attacklevel = 7;
					defenselevel = 3;
					lifelevel = rdm.nextInt(3) + 3; 
					soldado.setName(name);                  
					soldado.setAttackLevel(attacklevel);
					soldado.setDefenseLevel(defenselevel);                    
					soldado.setLifeLevel(lifelevel);
					soldado.setRow(row);
					soldado.setColumn(column);
					if(army.get(row - 1).get((int)column.charAt(0) - 65) == null){
						System.out.println("Registrando al " + (i + 1) + " soldado del Ejercito " + armyespe + "");
						army.get(row - 1).set((int)column.charAt(0) - 65, new Arquero(name, attacklevel, defenselevel, lifelevel, speed, "Arquero", true, row, column, attacklevel));
						army.get(row - 1).get((int)column.charAt(0) - 65).setSpeed(speed);
						System.out.println(army.get(row - 1).get((int)column.charAt(0) - 65).toString());
						System.out.println("---------------------------------");
					}else{
						i -= 1; //NOS AYUDARIA CON LOS SOLDADOS QUE SE REPITEN EN EL MISMO CASILLERO CON TAL QUE NO DEBERIA CONTAR 
					}
				} else if (soldado instanceof Caballero) {
					name = "Caballero" + i + "X" + num; 
					attacklevel = 13;
					defenselevel = 7;
					lifelevel = rdm.nextInt(3) + 10; 
					soldado.setName(name);                  
					soldado.setAttackLevel(attacklevel);
					soldado.setDefenseLevel(defenselevel);                    
					soldado.setLifeLevel(lifelevel);
					soldado.setRow(row);
					soldado.setColumn(column);
					if(army.get(row - 1).get((int)column.charAt(0) - 65) == null){
						System.out.println("Registrando al " + (i + 1) + " soldado del Ejercito " + armyespe + "");
						army.get(row - 1).set((int)column.charAt(0) - 65, new Caballero(name, attacklevel, defenselevel, lifelevel, speed, "Caballero", true, row, column, false));
						army.get(row - 1).get((int)column.charAt(0) - 65).setSpeed(speed);
						System.out.println(army.get(row - 1).get((int)column.charAt(0) - 65).toString());
						System.out.println("---------------------------------");
					}else{
						i -= 1; //NOS AYUDARIA CON LOS SOLDADOS QUE SE REPITEN EN EL MISMO CASILLERO CON TAL QUE NO DEBERIA CONTAR 
					}
				} else if (soldado instanceof Lancero) {
					name = "Lancero" + i + "X" + num; 
					attacklevel = 5;
					defenselevel = 10;
					lifelevel = rdm.nextInt(3) + 5;
					soldado.setName(name);                  
					soldado.setAttackLevel(attacklevel);
					soldado.setDefenseLevel(defenselevel);                    
					soldado.setLifeLevel(lifelevel);
					soldado.setRow(row);
					soldado.setColumn(column);
					if(army.get(row - 1).get((int)column.charAt(0) - 65) == null){
						System.out.println("Registrando al " + (i + 1) + " soldado del Ejercito " + armyespe + "");
						army.get(row - 1).set((int)column.charAt(0) - 65, new Lancero(name, attacklevel, defenselevel, lifelevel, speed, "Lancero", true, row, column, attacklevel));
						army.get(row - 1).get((int)column.charAt(0) - 65).setSpeed(speed);
						System.out.println(army.get(row - 1).get((int)column.charAt(0) - 65).toString());
						System.out.println("---------------------------------");
					}else{
						i -= 1; //NOS AYUDARIA CON LOS SOLDADOS QUE SE REPITEN EN EL MISMO CASILLERO CON TAL QUE NO DEBERIA CONTAR 
					}
				} else if (soldado instanceof EspadachinReal){
					name = "Espadachin Real" + i + "X" + num;
					attacklevel = 10;
					defenselevel = 8;
					lifelevel = 12;
					soldado.setName(name);                  
					soldado.setAttackLevel(attacklevel);
					soldado.setDefenseLevel(defenselevel);                    
					soldado.setLifeLevel(lifelevel);
					soldado.setRow(row);
					soldado.setColumn(column);
					if(army.get(row - 1).get((int)column.charAt(0) - 65) == null){
						System.out.println("Registrando al " + (i + 1) + " soldado del Ejercito " + armyespe + "");
						army.get(row - 1).set((int)column.charAt(0) - 65, new EspadachinReal(name, attacklevel, defenselevel, lifelevel, speed, "Espadachin Real", true, row, column, attacklevel, attacklevel));
						army.get(row - 1).get((int)column.charAt(0) - 65).setSpeed(speed);
						System.out.println(army.get(row - 1).get((int)column.charAt(0) - 65).toString());
						System.out.println("---------------------------------");
					}else{
						i -= 1; //NOS AYUDARIA CON LOS SOLDADOS QUE SE REPITEN EN EL MISMO CASILLERO CON TAL QUE NO DEBERIA CONTAR 
					}
				} else if (soldado instanceof CaballeroFranco){
					name = "Caballero Franco" + i + "X" + num;
					attacklevel = 13;
					defenselevel = 7;
					lifelevel = 15;
					soldado.setName(name);                  
					soldado.setAttackLevel(attacklevel);
					soldado.setDefenseLevel(defenselevel);                    
					soldado.setLifeLevel(lifelevel);
					soldado.setRow(row);
					soldado.setColumn(column);
					if(army.get(row - 1).get((int)column.charAt(0) - 65) == null){
						System.out.println("Registrando al " + (i + 1) + " soldado del Ejercito " + armyespe + "");
						army.get(row - 1).set((int)column.charAt(0) - 65, new CaballeroFranco(name, attacklevel, defenselevel, lifelevel, speed, "Caballero Franco", true, row, column, attacklevel, attacklevel));
						army.get(row - 1).get((int)column.charAt(0) - 65).setSpeed(speed);
						System.out.println(army.get(row - 1).get((int)column.charAt(0) - 65).toString());
						System.out.println("---------------------------------");
					}else{
						i -= 1; //NOS AYUDARIA CON LOS SOLDADOS QUE SE REPITEN EN EL MISMO CASILLERO CON TAL QUE NO DEBERIA CONTAR 
					}
				} else if (soldado instanceof EspadachinTeutonico){
					name = "Espadachin Teutonico" + i + "X" + num;
					attacklevel = 10;
					defenselevel = 8;
					lifelevel = 13;
					soldado.setName(name);                  
					soldado.setAttackLevel(attacklevel);
					soldado.setDefenseLevel(defenselevel);                    
					soldado.setLifeLevel(lifelevel);
					soldado.setRow(row);
					soldado.setColumn(column);
					if(army.get(row - 1).get((int)column.charAt(0) - 65) == null){
						System.out.println("Registrando al " + (i + 1) + " soldado del Ejercito " + armyespe + "");
						army.get(row - 1).set((int)column.charAt(0) - 65, new EspadachinTeutonico(name, attacklevel, defenselevel, lifelevel, speed, "Espadachin Teutonico", true, row, column, attacklevel, attacklevel));
						army.get(row - 1).get((int)column.charAt(0) - 65).setSpeed(speed);
						System.out.println(army.get(row - 1).get((int)column.charAt(0) - 65).toString());
						System.out.println("---------------------------------");
					}else{
						i -= 1; //NOS AYUDARIA CON LOS SOLDADOS QUE SE REPITEN EN EL MISMO CASILLERO CON TAL QUE NO DEBERIA CONTAR 
					}
				} else if (soldado instanceof EspadachinConquistador){
					name = "Espadachin Conquistador" + i + "X" + num;
					attacklevel = 10;
					defenselevel = 8;
					lifelevel = 14;
					soldado.setName(name);                  
					soldado.setAttackLevel(attacklevel);
					soldado.setDefenseLevel(defenselevel);                    
					soldado.setLifeLevel(lifelevel);
					soldado.setRow(row);
					soldado.setColumn(column);
					if(army.get(row - 1).get((int)column.charAt(0) - 65) == null){
						System.out.println("Registrando al " + (i + 1) + " soldado del Ejercito " + armyespe + "");
						army.get(row - 1).set((int)column.charAt(0) - 65, new EspadachinConquistador(name, attacklevel, defenselevel, lifelevel, speed, "Espadachin Conquistador", true, row, column, attacklevel, attacklevel));
						army.get(row - 1).get((int)column.charAt(0) - 65).setSpeed(speed);
						System.out.println(army.get(row - 1).get((int)column.charAt(0) - 65).toString());
						System.out.println("---------------------------------");
					}else{
						i -= 1; //NOS AYUDARIA CON LOS SOLDADOS QUE SE REPITEN EN EL MISMO CASILLERO CON TAL QUE NO DEBERIA CONTAR 
					}
				} else if (soldado instanceof CaballeroMoro){
					name = "Caballero Moro" + i + "X" + num;
					attacklevel = 13;
					defenselevel = 7;
					lifelevel = 13;
					soldado.setName(name);                  
					soldado.setAttackLevel(attacklevel);
					soldado.setDefenseLevel(defenselevel);                    
					soldado.setLifeLevel(lifelevel);
					soldado.setRow(row);
					soldado.setColumn(column);
					if(army.get(row - 1).get((int)column.charAt(0) - 65) == null){
						System.out.println("Registrando al " + (i + 1) + " soldado del Ejercito " + armyespe + "");
						army.get(row - 1).set((int)column.charAt(0) - 65, new CaballeroMoro(name, attacklevel, defenselevel, lifelevel, speed, "Caballero Moro", true, row, column, attacklevel, attacklevel));
						army.get(row - 1).get((int)column.charAt(0) - 65).setSpeed(speed);
						System.out.println(army.get(row - 1).get((int)column.charAt(0) - 65).toString());
						System.out.println("---------------------------------");
					}else{
						i -= 1; //NOS AYUDARIA CON LOS SOLDADOS QUE SE REPITEN EN EL MISMO CASILLERO CON TAL QUE NO DEBERIA CONTAR 
					}
				}
			}
			System.out.println("*********************************");
			return army;
		}
	\end{lstlisting}
	\begin{lstlisting}[language=bash,caption={Ejecucion:}][H]
		-------------------------------------------
		--                 MENU                  --
		-------------------------------------------
		 SELECCIONE UN NUMERO PARA PODER EMPEZAR O TERMINAR
		 1 : JUGAR
		 2 : NO JUGAR
		1
		El Ejercito Francia tiene 3 soldados : 
		
		Registrando al 1 soldado del Ejercito Francia
		
		Nombre: Espadachin0X1
		Vida: 8
		Fila: 6
		Columna: H
		Nivel de ataque: 10
		Nivel de Defensa: 8
		Nivel de vida: 8
		Velocidad: 2
		Actitud: Espadachin
		Estado: true
		---------------------------------
		Registrando al 2 soldado del Ejercito Francia
		
		Nombre: Espadachin1X1
		Vida: 10
		Fila: 2
		Columna: A
		Nivel de ataque: 10
		Nivel de Defensa: 8
		Nivel de vida: 10
		Velocidad: 5
		Actitud: Espadachin
		Estado: true
		---------------------------------
		Registrando al 3 soldado del Ejercito Francia
		
		Nombre: Arquero2X1
		Vida: 3
		Fila: 4
		Columna: F
		Nivel de ataque: 7
		Nivel de Defensa: 3
		Nivel de vida: 3
		Velocidad: 2
		Actitud: Arquero
		Estado: true
		---------------------------------
		*********************************
		El Ejercito Aragon tiene 3 soldados : 
		
		Registrando al 1 soldado del Ejercito Aragon
		
		Nombre: Espadachin0X2
		Vida: 10
		Fila: 7
		Columna: G
		Nivel de ataque: 10
		Nivel de Defensa: 8
		Nivel de vida: 10
		Velocidad: 5
		Actitud: Espadachin
		Estado: true
		---------------------------------
		Registrando al 2 soldado del Ejercito Aragon
		
		Nombre: Arquero1X2
		Vida: 3
		Fila: 5
		Columna: C
		Nivel de ataque: 7
		Nivel de Defensa: 3
		Nivel de vida: 3
		Velocidad: 2
		Actitud: Arquero
		Estado: true
		---------------------------------
		Registrando al 3 soldado del Ejercito Aragon
		
		Nombre: Lancero2X2
		Vida: 6
		Fila: 8
		Columna: E
		Nivel de ataque: 5
		Nivel de Defensa: 10
		Nivel de vida: 6
		Velocidad: 5
		Actitud: Lancero
		Estado: true
		---------------------------------
		*********************************
		
		*********************************
		El tipo de territorio es: campo abierto
		
		*********************************
		El Ejercito Francia del 1 ejercito sus soldados son :
		
		*********************************
		El 1 soldado es: 
		
		Nombre: Espadachin1X1
		Vida: 10
		Fila: 2
		Columna: A
		Nivel de ataque: 10
		Nivel de Defensa: 8
		Nivel de vida: 11
		Velocidad: 5
		Actitud: Espadachin
		Estado: true
		
		*********************************
		El 2 soldado es: 
		
		Nombre: Arquero2X1
		Vida: 3
		Fila: 4
		Columna: F
		Nivel de ataque: 7
		Nivel de Defensa: 3
		Nivel de vida: 4
		Velocidad: 2
		Actitud: Arquero
		Estado: true
		
		*********************************
		El 3 soldado es: 
		
		Nombre: Espadachin0X1
		Vida: 8
		Fila: 6
		Columna: H
		Nivel de ataque: 10
		Nivel de Defensa: 8
		Nivel de vida: 9
		Velocidad: 2
		Actitud: Espadachin
		Estado: true
		El Ejercito Aragon del 2 ejercito sus soldados son :
		
		*********************************
		El 1 soldado es: 
		
		Nombre: Arquero1X2
		Vida: 3
		Fila: 5
		Columna: C
		Nivel de ataque: 7
		Nivel de Defensa: 3
		Nivel de vida: 3
		Velocidad: 2
		Actitud: Arquero
		Estado: true
		
		*********************************
		El 2 soldado es: 
		
		Nombre: Espadachin0X2
		Vida: 10
		Fila: 7
		Columna: G
		Nivel de ataque: 10
		Nivel de Defensa: 8
		Nivel de vida: 10
		Velocidad: 5
		Actitud: Espadachin
		Estado: true
		
		*********************************
		El 3 soldado es: 
		
		Nombre: Lancero2X2
		Vida: 6
		Fila: 8
		Columna: E
		Nivel de ataque: 5
		Nivel de Defensa: 10
		Nivel de vida: 6
		Velocidad: 5
		Actitud: Lancero
		Estado: true
		
		*********************************
		El tipo de territorio es: campo abierto
		
		*********************************
				
	\end{lstlisting}
	\begin{figure}[H]
		\centering
		\includegraphics[width=1.0\textwidth,keepaspectratio]{img/Commit9.png}
		%\includesvg{img/automata.svg}
		%\label{img:mot2}
		%\caption{Product backlog.}
	\end{figure}
	\begin{lstlisting}[language=bash,caption={Ejecucion:}][H]
		*********************************
		El soldado con mayor vida del Ejercito Francia es: 
		
		Nombre: Espadachin1X1
		Vida: 10
		Fila: 2
		Columna: A
		Nivel de ataque: 10
		Nivel de Defensa: 8
		Nivel de vida: 11
		Velocidad: 5
		Actitud: Espadachin
		Estado: true
		*********************************
		El soldado con mayor vida del Ejercito Aragon es: 
		
		Nombre: Espadachin0X2
		Vida: 10
		Fila: 7
		Columna: G
		Nivel de ataque: 10
		Nivel de Defensa: 8
		Nivel de vida: 10
		Velocidad: 5
		Actitud: Espadachin
		Estado: true
		*********************************
		El promedio de puntos de vida del Ejercito Francia es: 
		7.0
		*********************************
		El promedio de puntos de vida del Ejercito Aragon es: 
		6.333333333333333
		*********************************
		
		El Ejercito Francia ordenando por metodo insercion: 
		------------------------------------------
		Mostrando Ranking del Ejercito Francia ..... ////// --->
		
		Puesto 1
		Nombre: Espadachin1X1
		Vida: 10
		Fila: 2
		Columna: A
		Nivel de ataque: 10
		Nivel de Defensa: 8
		Nivel de vida: 11
		Velocidad: 5
		Actitud: Espadachin
		Estado: true
		------------------
		
		Puesto 2
		Nombre: Espadachin0X1
		Vida: 8
		Fila: 6
		Columna: H
		Nivel de ataque: 10
		Nivel de Defensa: 8
		Nivel de vida: 9
		Velocidad: 2
		Actitud: Espadachin
		Estado: true
		------------------
		
		Puesto 3
		Nombre: Arquero2X1
		Vida: 3
		Fila: 4
		Columna: F
		Nivel de ataque: 7
		Nivel de Defensa: 3
		Nivel de vida: 4
		Velocidad: 2
		Actitud: Arquero
		Estado: true
		------------------
		*********************************
		
		El Ejercito Aragon ordenando por metodo insercion: 
		------------------------------------------
		Mostrando Ranking del Ejercito Aragon ..... ////// --->
		
		Puesto 1
		Nombre: Espadachin0X2
		Vida: 10
		Fila: 7
		Columna: G
		Nivel de ataque: 10
		Nivel de Defensa: 8
		Nivel de vida: 10
		Velocidad: 5
		Actitud: Espadachin
		Estado: true
		------------------
		
		Puesto 2
		Nombre: Lancero2X2
		Vida: 6
		Fila: 8
		Columna: E
		Nivel de ataque: 5
		Nivel de Defensa: 10
		Nivel de vida: 6
		Velocidad: 5
		Actitud: Lancero
		Estado: true
		------------------
		
		Puesto 3
		Nombre: Arquero1X2
		Vida: 3
		Fila: 5
		Columna: C
		Nivel de ataque: 7
		Nivel de Defensa: 3
		Nivel de vida: 3
		Velocidad: 2
		Actitud: Arquero
		Estado: true
		------------------
		*********************************
		
		*********************************
		El tipo de territorio es: campo abierto
		
		*********************************		
	\end{lstlisting}
	\begin{figure}[H]
		\centering
		\includegraphics[width=1.0\textwidth,keepaspectratio]{img/Commit9.png}
		%\includesvg{img/automata.svg}
		%\label{img:mot2}
		%\caption{Product backlog.}
	\end{figure}
	\begin{lstlisting}[language=bash,caption={Ejecucion:}][H]
		*********************************
		Ejercito 1 : Francia
		Cantidad total de soldados creados: 3
		Espadachines: 2
		Arqueros: 1
		Caballeros: 0
		Lanceros: 0
		
		Ejercito 2 : Aragon
		Cantidad total de soldados creados: 3
		Espadachines: 1
		Arqueros: 1
		Caballeros: 0
		Lanceros: 1
		
		Ejercito 1: Francia: 24
		Ejercito 2: Aragon: 19
		El ganador es el ejercito 1 de: Francia. Ya que al generar los porcentajes de probabilidad de victoria basada en los niveles de vida de sus soldados y aplicando un experimento aleatorio salio vencedor. (Aleatorio generado : 55.81)
		
		*********************************
		 DESEA VOLVER A JUGAR
		 1 : JUGAR
		 2 : NO JUGAR
			
	\end{lstlisting}
	\subsection{Ejercicio Mapa}
	\begin{itemize}	
		\item En esta seccion mi funcion resultBattleInfo, proporciono informacion detallada sobre la composicion del ejercito representado por la matriz de objetos ArrayList bidimensional army. La funcion recorre la matriz y cuenta la cantidad de soldados totales, asi como la distribucion por tipo, que incluye Espadachines, Arqueros, Caballeros, Lanceros, Espadachines Reales, Caballeros Francos, Caballeros Moros, Espadachines Teutonicos y Espadachines Conquistadores. La informacion se presenta de manera organizada, especificando la cantidad de cada tipo de soldado en el ejercito del reino correspondiente (identificado por el parametro kingdom). Los resultados se imprimen en la consola, proporcionando una vision general clara de la fuerza militar de cada reino en terminos de composicion de tropas.
		\item El codigo y el commit seria el siguiente:
	\end{itemize}	
	\begin{lstlisting}[language=bash,caption={Commit}][H]
		$ git commit -m "Agregando opciones en getRandomSoldado() el cual serian estas clases heredadas anteriormente respetando si estas pertenecen a cierto reino especifico tambien modificamos el metodo obtenerIncial() el cual tambien agregamos las inciales de cada tipo de Soldado y tambien en el metodo fillarray() el cual creara a cada soldado dependiendo el tipo que sea con un constructor seria solo agregar mas clases heredadas en resumen "
	\end{lstlisting}	
	\begin{lstlisting}[language=java,caption={Las lineas de codigos de la clase Mapa}][H]
		public static void resultBattleInfo(ArrayList<ArrayList<Soldado>> army, String kingdom, int n){
			System.out.println("Ejercito " + n + " : " + kingdom);
			int numbersoldiers = 0;
			int numberespadachines = 0;
			int numbercaballeros = 0;
			int numberlanceros = 0;
			int numberarqueros = 0;
			int numberespadachinesreales = 0;
			int numbercaballerosfrancos = 0;
			int numbercaballerosmoros = 0;
			int numberespadachinesteutonicos = 0;
			int numberespadachinesconquistadores = 0;
			for(int i = 0; i < 10; i++){ //ITERACION
				for(int j = 0; j < 10 ; j++){//ITERACION
					if(army.get(i).get(j) != null){
						numbersoldiers++;
						if(army.get(i).get(j) instanceof Espadachin){
							numberespadachines++;
						}else if(army.get(i).get(j) instanceof Caballero){
							numbercaballeros++;
						}else if(army.get(i).get(j) instanceof Lancero){
							numberlanceros++;
						}else if(army.get(i).get(j) instanceof Arquero){
							numberarqueros++;
						}else if(army.get(i).get(j) instanceof EspadachinReal){
							numberespadachinesreales++;
						}else if(army.get(i).get(j) instanceof CaballeroFranco){
							numbercaballerosfrancos++;
						}else if(army.get(i).get(j) instanceof CaballeroMoro){
							numbercaballerosmoros++;
						}else if(army.get(i).get(j) instanceof EspadachinTeutonico){
							numberespadachinesteutonicos++;
						}else if(army.get(i).get(j) instanceof EspadachinConquistador){
							numberespadachinesconquistadores++;
						}
					}
				}
			}
			System.out.println("Cantidad total de soldados creados: " + numbersoldiers);
			System.out.println("Espadachines: " + numberespadachines);
			System.out.println("Arqueros: " + numberarqueros);
			System.out.println("Caballeros: " + numbercaballeros);
			System.out.println("Lanceros: " + numberlanceros);
			System.out.println("Espadachin Real: " + numberespadachinesreales);
			System.out.println("Caballero Moro: " + numbercaballerosmoros);
			System.out.println("Caballero Franco: " + numbercaballerosfrancos);
			System.out.println("Espadachin Teutonico: " + numberespadachinesteutonicos);
			System.out.println("Espadachin Conquistador: " + numberespadachinesconquistadores + "\n");
	
		}
	\end{lstlisting}
	\begin{lstlisting}[language=bash,caption={Ejecucion:}][H]
		-------------------------------------------
		--                 MENU                  --
		-------------------------------------------
		 SELECCIONE UN NUMERO PARA PODER EMPEZAR O TERMINAR
		 1 : JUGAR
		 2 : NO JUGAR
		1
		El Ejercito Castilla tiene 6 soldados : 
		
		Registrando al 1 soldado del Ejercito Castilla
		
		Nombre: Lancero0X1
		Vida: 6
		Fila: 6
		Columna: G
		Nivel de ataque: 5
		Nivel de Defensa: 10
		Nivel de vida: 6
		Velocidad: 1
		Actitud: Lancero
		Estado: true
		---------------------------------
		Registrando al 2 soldado del Ejercito Castilla
		
		Nombre: Lancero1X1
		Vida: 6
		Fila: 9
		Columna: D
		Nivel de ataque: 5
		Nivel de Defensa: 10
		Nivel de vida: 6
		Velocidad: 5
		Actitud: Lancero
		Estado: true
		---------------------------------
		Registrando al 3 soldado del Ejercito Castilla
		
		Nombre: Espadachin Conquistador2X1
		Vida: 14
		Fila: 9
		Columna: J
		Nivel de ataque: 10
		Nivel de Defensa: 8
		Nivel de vida: 14
		Velocidad: 3
		Actitud: Espadachin Conquistador
		Estado: true
		---------------------------------
		Registrando al 4 soldado del Ejercito Castilla
		
		Nombre: Espadachin Conquistador3X1
		Vida: 14
		Fila: 2
		Columna: G
		Nivel de ataque: 10
		Nivel de Defensa: 8
		Nivel de vida: 14
		Velocidad: 3
		Actitud: Espadachin Conquistador
		Estado: true
		---------------------------------
		Registrando al 5 soldado del Ejercito Castilla
		
		Nombre: Caballero4X1
		Vida: 10
		Fila: 10
		Columna: F
		Nivel de ataque: 13
		Nivel de Defensa: 7
		Nivel de vida: 10
		Velocidad: 2
		Actitud: Caballero
		Estado: true
		---------------------------------
		Registrando al 6 soldado del Ejercito Castilla
		
		Nombre: Espadachin Conquistador5X1
		Vida: 14
		Fila: 9
		Columna: F
		Nivel de ataque: 10
		Nivel de Defensa: 8
		Nivel de vida: 14
		Velocidad: 1
		Actitud: Espadachin Conquistador
		Estado: true
		---------------------------------
		*********************************
		El Ejercito Aragon tiene 8 soldados : 
		
		Registrando al 1 soldado del Ejercito Aragon
		
		Nombre: Espadachin0X2
		Vida: 8
		Fila: 1
		Columna: F
		Nivel de ataque: 10
		Nivel de Defensa: 8
		Nivel de vida: 8
		Velocidad: 4
		Actitud: Espadachin
		Estado: true
		---------------------------------
		Registrando al 2 soldado del Ejercito Aragon
		
		Nombre: Espadachin Conquistador1X2
		Vida: 14
		Fila: 4
		Columna: I
		Nivel de ataque: 10
		Nivel de Defensa: 8
		Nivel de vida: 14
		Velocidad: 1
		Actitud: Espadachin Conquistador
		Estado: true
		---------------------------------
		Registrando al 3 soldado del Ejercito Aragon
		
		Nombre: Espadachin Conquistador2X2
		Vida: 14
		Fila: 6
		Columna: B
		Nivel de ataque: 10
		Nivel de Defensa: 8
		Nivel de vida: 14
		Velocidad: 2
		Actitud: Espadachin Conquistador
		Estado: true
		---------------------------------
		Registrando al 4 soldado del Ejercito Aragon
		
		Nombre: Espadachin Conquistador3X2
		Vida: 14
		Fila: 4
		Columna: G
		Nivel de ataque: 10
		Nivel de Defensa: 8
		Nivel de vida: 14
		Velocidad: 4
		Actitud: Espadachin Conquistador
		Estado: true
		---------------------------------
		Registrando al 5 soldado del Ejercito Aragon
		
		Nombre: Espadachin Conquistador4X2
		Vida: 14
		Fila: 6
		Columna: D
		Nivel de ataque: 10
		Nivel de Defensa: 8
		Nivel de vida: 14
		Velocidad: 3
		Actitud: Espadachin Conquistador
		Estado: true
		---------------------------------
		Registrando al 6 soldado del Ejercito Aragon
		
		Nombre: Espadachin5X2
		Vida: 9
		Fila: 8
		Columna: A
		Nivel de ataque: 10
		Nivel de Defensa: 8
		Nivel de vida: 9
		Velocidad: 5
		Actitud: Espadachin
		Estado: true
		---------------------------------
		Registrando al 7 soldado del Ejercito Aragon
		
		Nombre: Caballero6X2
		Vida: 10
		Fila: 10
		Columna: E
		Nivel de ataque: 13
		Nivel de Defensa: 7
		Nivel de vida: 10
		Velocidad: 2
		Actitud: Caballero
		Estado: true
		---------------------------------
		Registrando al 8 soldado del Ejercito Aragon
		
		Nombre: Caballero7X2
		Vida: 10
		Fila: 5
		Columna: D
		Nivel de ataque: 13
		Nivel de Defensa: 7
		Nivel de vida: 10
		Velocidad: 5
		Actitud: Caballero
		Estado: true
		---------------------------------
		*********************************
		
		*********************************
		El tipo de territorio es: montana
		
		*********************************
		El Ejercito Castilla del 1 ejercito sus soldados son :
		
		*********************************
		El 1 soldado es: 
		
		Nombre: Espadachin Conquistador3X1
		Vida: 14
		Fila: 2
		Columna: G
		Nivel de ataque: 10
		Nivel de Defensa: 8
		Nivel de vida: 15
		Velocidad: 3
		Actitud: Espadachin Conquistador
		Estado: true
		
		*********************************
		El 2 soldado es: 
		
		Nombre: Lancero0X1
		Vida: 6
		Fila: 6
		Columna: G
		Nivel de ataque: 5
		Nivel de Defensa: 10
		Nivel de vida: 7
		Velocidad: 1
		Actitud: Lancero
		Estado: true
		
		*********************************
		El 3 soldado es: 
		
		Nombre: Lancero1X1
		Vida: 6
		Fila: 9
		Columna: D
		Nivel de ataque: 5
		Nivel de Defensa: 10
		Nivel de vida: 7
		Velocidad: 5
		Actitud: Lancero
		Estado: true
		
		*********************************
		El 4 soldado es: 
		
		Nombre: Espadachin Conquistador5X1
		Vida: 14
		Fila: 9
		Columna: F
		Nivel de ataque: 10
		Nivel de Defensa: 8
		Nivel de vida: 15
		Velocidad: 1
		Actitud: Espadachin Conquistador
		Estado: true
		
		*********************************
		El 5 soldado es: 
		
		Nombre: Espadachin Conquistador2X1
		Vida: 14
		Fila: 9
		Columna: J
		Nivel de ataque: 10
		Nivel de Defensa: 8
		Nivel de vida: 15
		Velocidad: 3
		Actitud: Espadachin Conquistador
		Estado: true
		
		*********************************
		El 6 soldado es: 
		
		Nombre: Caballero4X1
		Vida: 10
		Fila: 10
		Columna: F
		Nivel de ataque: 13
		Nivel de Defensa: 7
		Nivel de vida: 11
		Velocidad: 2
		Actitud: Caballero
		Estado: true
		El Ejercito Aragon del 2 ejercito sus soldados son :
		
		*********************************
		El 1 soldado es: 
		
		Nombre: Espadachin0X2
		Vida: 8
		Fila: 1
		Columna: F
		Nivel de ataque: 10
		Nivel de Defensa: 8
		Nivel de vida: 9
		Velocidad: 4
		Actitud: Espadachin
		Estado: true
		
		*********************************
		El 2 soldado es: 
		
		Nombre: Espadachin Conquistador3X2
		Vida: 14
		Fila: 4
		Columna: G
		Nivel de ataque: 10
		Nivel de Defensa: 8
		Nivel de vida: 15
		Velocidad: 4
		Actitud: Espadachin Conquistador
		Estado: true
		
		*********************************
		El 3 soldado es: 
		
		Nombre: Espadachin Conquistador1X2
		Vida: 14
		Fila: 4
		Columna: I
		Nivel de ataque: 10
		Nivel de Defensa: 8
		Nivel de vida: 15
		Velocidad: 1
		Actitud: Espadachin Conquistador
		Estado: true
		
		*********************************
		El 4 soldado es: 
		
		Nombre: Caballero7X2
		Vida: 10
		Fila: 5
		Columna: D
		Nivel de ataque: 13
		Nivel de Defensa: 7
		Nivel de vida: 11
		Velocidad: 5
		Actitud: Caballero
		Estado: true
		
		*********************************
		El 5 soldado es: 
		
		Nombre: Espadachin Conquistador2X2
		Vida: 14
		Fila: 6
		Columna: B
		Nivel de ataque: 10
		Nivel de Defensa: 8
		Nivel de vida: 15
		Velocidad: 2
		Actitud: Espadachin Conquistador
		Estado: true
		
		*********************************
		El 6 soldado es: 
		
		Nombre: Espadachin Conquistador4X2
		Vida: 14
		Fila: 6
		Columna: D
		Nivel de ataque: 10
		Nivel de Defensa: 8
		Nivel de vida: 15
		Velocidad: 3
		Actitud: Espadachin Conquistador
		Estado: true
		
		*********************************
		El 7 soldado es: 
		
		Nombre: Espadachin5X2
		Vida: 9
		Fila: 8
		Columna: A
		Nivel de ataque: 10
		Nivel de Defensa: 8
		Nivel de vida: 10
		Velocidad: 5
		Actitud: Espadachin
		Estado: true
		
		*********************************
		El 8 soldado es: 
		
		Nombre: Caballero6X2
		Vida: 10
		Fila: 10
		Columna: E
		Nivel de ataque: 13
		Nivel de Defensa: 7
		Nivel de vida: 11
		Velocidad: 2
		Actitud: Caballero
		Estado: true
		
		*********************************
		El tipo de territorio es: montana
		
		*********************************
		
	\end{lstlisting}
	\begin{figure}[H]
		\centering
		\includegraphics[width=1.0\textwidth,keepaspectratio]{img/Commit10.png}
		%\includesvg{img/automata.svg}
		%\label{img:mot2}
		%\caption{Product backlog.}
	\end{figure}
	\begin{lstlisting}[language=bash,caption={Ejecucion:}][H]
		El soldado con mayor vida del Ejercito Castilla es: 

		Nombre: Espadachin Conquistador3X1
		Vida: 14
		Fila: 2
		Columna: G
		Nivel de ataque: 10
		Nivel de Defensa: 8
		Nivel de vida: 15
		Velocidad: 3
		Actitud: Espadachin Conquistador
		Estado: true
		*********************************
		El soldado con mayor vida del Ejercito Aragon es: 
		
		Nombre: Espadachin Conquistador3X2
		Vida: 14
		Fila: 4
		Columna: G
		Nivel de ataque: 10
		Nivel de Defensa: 8
		Nivel de vida: 15
		Velocidad: 4
		Actitud: Espadachin Conquistador
		Estado: true
		*********************************
		El promedio de puntos de vida del Ejercito Castilla es: 
		10.666666666666666
		*********************************
		El promedio de puntos de vida del Ejercito Aragon es: 
		11.625
		*********************************
		
		El Ejercito Castilla ordenando por metodo insercion: 
		------------------------------------------
		Mostrando Ranking del Ejercito Castilla ..... ////// --->
		
		Puesto 1
		Nombre: Espadachin Conquistador3X1
		Vida: 14
		Fila: 2
		Columna: G
		Nivel de ataque: 10
		Nivel de Defensa: 8
		Nivel de vida: 15
		Velocidad: 3
		Actitud: Espadachin Conquistador
		Estado: true
		------------------
		
		Puesto 2
		Nombre: Espadachin Conquistador5X1
		Vida: 14
		Fila: 9
		Columna: F
		Nivel de ataque: 10
		Nivel de Defensa: 8
		Nivel de vida: 15
		Velocidad: 1
		Actitud: Espadachin Conquistador
		Estado: true
		------------------
		
		Puesto 3
		Nombre: Espadachin Conquistador2X1
		Vida: 14
		Fila: 9
		Columna: J
		Nivel de ataque: 10
		Nivel de Defensa: 8
		Nivel de vida: 15
		Velocidad: 3
		Actitud: Espadachin Conquistador
		Estado: true
		------------------
		
		Puesto 4
		Nombre: Caballero4X1
		Vida: 10
		Fila: 10
		Columna: F
		Nivel de ataque: 13
		Nivel de Defensa: 7
		Nivel de vida: 11
		Velocidad: 2
		Actitud: Caballero
		Estado: true
		------------------
		
		Puesto 5
		Nombre: Lancero0X1
		Vida: 6
		Fila: 6
		Columna: G
		Nivel de ataque: 5
		Nivel de Defensa: 10
		Nivel de vida: 7
		Velocidad: 1
		Actitud: Lancero
		Estado: true
		------------------
		
		Puesto 6
		Nombre: Lancero1X1
		Vida: 6
		Fila: 9
		Columna: D
		Nivel de ataque: 5
		Nivel de Defensa: 10
		Nivel de vida: 7
		Velocidad: 5
		Actitud: Lancero
		Estado: true
		------------------
		*********************************
		
		El Ejercito Aragon ordenando por metodo insercion: 
		------------------------------------------
		Mostrando Ranking del Ejercito Aragon ..... ////// --->
		
		Puesto 1
		Nombre: Espadachin Conquistador3X2
		Vida: 14
		Fila: 4
		Columna: G
		Nivel de ataque: 10
		Nivel de Defensa: 8
		Nivel de vida: 15
		Velocidad: 4
		Actitud: Espadachin Conquistador
		Estado: true
		------------------
		
		Puesto 2
		Nombre: Espadachin Conquistador1X2
		Vida: 14
		Fila: 4
		Columna: I
		Nivel de ataque: 10
		Nivel de Defensa: 8
		Nivel de vida: 15
		Velocidad: 1
		Actitud: Espadachin Conquistador
		Estado: true
		------------------
		
		Puesto 3
		Nombre: Espadachin Conquistador2X2
		Vida: 14
		Fila: 6
		Columna: B
		Nivel de ataque: 10
		Nivel de Defensa: 8
		Nivel de vida: 15
		Velocidad: 2
		Actitud: Espadachin Conquistador
		Estado: true
		------------------
		
		Puesto 4
		Nombre: Espadachin Conquistador4X2
		Vida: 14
		Fila: 6
		Columna: D
		Nivel de ataque: 10
		Nivel de Defensa: 8
		Nivel de vida: 15
		Velocidad: 3
		Actitud: Espadachin Conquistador
		Estado: true
		------------------
		
		Puesto 5
		Nombre: Caballero7X2
		Vida: 10
		Fila: 5
		Columna: D
		Nivel de ataque: 13
		Nivel de Defensa: 7
		Nivel de vida: 11
		Velocidad: 5
		Actitud: Caballero
		Estado: true
		------------------
		
		Puesto 6
		Nombre: Caballero6X2
		Vida: 10
		Fila: 10
		Columna: E
		Nivel de ataque: 13
		Nivel de Defensa: 7
		Nivel de vida: 11
		Velocidad: 2
		Actitud: Caballero
		Estado: true
		------------------
		
		Puesto 7
		Nombre: Espadachin5X2
		Vida: 9
		Fila: 8
		Columna: A
		Nivel de ataque: 10
		Nivel de Defensa: 8
		Nivel de vida: 10
		Velocidad: 5
		Actitud: Espadachin
		Estado: true
		------------------
		
		Puesto 8
		Nombre: Espadachin0X2
		Vida: 8
		Fila: 1
		Columna: F
		Nivel de ataque: 10
		Nivel de Defensa: 8
		Nivel de vida: 9
		Velocidad: 4
		Actitud: Espadachin
		Estado: true
		------------------
		*********************************
		
		*********************************
		El tipo de territorio es: montana
		
		*********************************
			
	\end{lstlisting}
	\begin{figure}[H]
		\centering
		\includegraphics[width=1.0\textwidth,keepaspectratio]{img/Commit10.png}
		%\includesvg{img/automata.svg}
		%\label{img:mot2}
		%\caption{Product backlog.}
	\end{figure}
	\begin{lstlisting}[language=bash,caption={Ejecucion:}][H]
		*********************************
		Ejercito 1 : Castilla
		Cantidad total de soldados creados: 6
		Espadachines: 0
		Arqueros: 0
		Caballeros: 1
		Lanceros: 2
		Espadachin Real: 0
		Caballero Moro: 0
		Caballero Franco: 0
		Espadachin Teutonico: 0
		Espadachin Conquistador: 3
		
		Ejercito 2 : Aragon
		Cantidad total de soldados creados: 8
		Espadachines: 2
		Arqueros: 0
		Caballeros: 2
		Lanceros: 0
		Espadachin Real: 0
		Caballero Moro: 0
		Caballero Franco: 0
		Espadachin Teutonico: 0
		Espadachin Conquistador: 4
		
		Ejercito 1: Castilla: 70
		Ejercito 2: Aragon: 101
		El ganador es el ejercito 2 de: Aragon. Ya que al generar los porcentajes de probabilidad de victoria basada en los niveles de vida de sus soldados y aplicando un experimento aleatorio salio vencedor. (Aleatorio generado : 59.06)
		
		*********************************
		 DESEA VOLVER A JUGAR
		 1 : JUGAR
		 2 : NO JUGAR
		
	\end{lstlisting}
	\subsection{Ejercicio JuegoPrincipal}
	\begin{itemize}	
		\item En esta seccion en el método main, he creado una interfaz gráfica de usuario (GUI) utilizando la biblioteca Swing de Java. La interfaz consta de un JFrame llamado ventanaBienvenida que representa el menú de inicio del juego. Este marco tiene un tamaño predeterminado de 350x145 píxeles y se cerrará cuando el usuario haga clic en el botón de cierre. El diseño del contenido se gestiona mediante un JPanel llamado panelPrincipal, que utiliza un diseño de flujo (FlowLayout) para centrar y espaciar los elementos.
		\item Dentro del panel, he añadido un JLabel llamado mensajeLabel con el texto "¿Desea iniciar el juego?" y dos botones (JButton), botonSi y botonNo, con los textos "Sí" y "No" respectivamente. Los botones tienen colores de fondo distintivos (verde para "Sí" y rojo para "No"). Se ha agregado un ActionListener a cada botón para manejar los eventos de clic.
		\item Si el usuario hace clic en el botón "Sí", se cierra la ventana de bienvenida (ventanaBienvenida.dispose()) y se crea un objeto Mapa para iniciar el juego llamando al método iniciarJuego() de la clase Mapa. Si el usuario hace clic en el botón "No", se cierra la ventana de bienvenida y se muestra un cuadro de diálogo (JOptionPane) con el mensaje "Programa finalizado".
		\item Finalmente, se configuran propiedades adicionales de la ventana, como su posición (ventanaBienvenida.setLocationRelativeTo(null)) para que aparezca en el centro de la pantalla, y se establece la visibilidad de la ventana en true para mostrarla al usuario. 
		\item El codigo y el commit seria el siguiente:
	\end{itemize}	
	\begin{lstlisting}[language=bash,caption={Commit}][H]
		$ git commit -m "Agregandole GUI al main el cual importamos java swing y java awt para el menu de inicio del juego y los diferentes metodos que aplicamos para su buena dispocion graficamente"
	\end{lstlisting}	
	\begin{lstlisting}[language=java,caption={Las lineas de codigos de la clase JuegoPrincipal}][H]
		import javax.swing.*;
		import java.awt.*;
		import java.awt.event.ActionEvent;
		import java.awt.event.ActionListener;
		public class Juegoprincipal{
			public static void main(String[] args) {
				JFrame ventanaBienvenida = new JFrame("MENU DE BATALLA");
				ventanaBienvenida.setSize(350, 145);
				ventanaBienvenida.setDefaultCloseOperation(JFrame.EXIT_ON_CLOSE);
				JPanel panelPrincipal = new JPanel(new FlowLayout(FlowLayout.CENTER, 10, 30));
				JLabel mensajeLabel = new JLabel("-Desea iniciar el juego-");
				JButton botonSi = new JButton("Si");
				JButton botonNo = new JButton("No");
				botonSi.setBackground(Color.GREEN);
				botonNo.setBackground(Color.RED);
				panelPrincipal.add(mensajeLabel);
				panelPrincipal.add(botonSi);
				panelPrincipal.add(botonNo);
				botonSi.addActionListener(new ActionListener() {
					public void actionPerformed(ActionEvent e) {             
						ventanaBienvenida.dispose();
						Mapa mapa = new Mapa();
						mapa.iniciarJuego();
					}
				});
				botonNo.addActionListener(new ActionListener() {              
					public void actionPerformed(ActionEvent e) {
						ventanaBienvenida.dispose();
						JOptionPane.showMessageDialog(null, "Programa finalizado.");
					}
				});
				ventanaBienvenida.add(panelPrincipal);
				ventanaBienvenida.setLocationRelativeTo(null); 
				ventanaBienvenida.setVisible(true);      
			}
		}
	\end{lstlisting}
	\begin{figure}[H]
		\centering
		\includegraphics[width=1.0\textwidth,keepaspectratio]{img/Commit11-1.png}
		%\includesvg{img/automata.svg}
		%\label{img:mot2}
		%\caption{Product backlog.}
	\end{figure}
	\begin{figure}[H]
		\centering
		\includegraphics[width=1.0\textwidth,keepaspectratio]{img/Commit11-2.png}
		%\includesvg{img/automata.svg}
		%\label{img:mot2}
		%\caption{Product backlog.}
	\end{figure}
	\begin{lstlisting}[language=bash,caption={Ejecucion:}][H]
		El Ejercito Castilla tiene 9 soldados : 

		Registrando al 1 soldado del Ejercito Castilla
		
		Nombre: Arquero0X1
		Vida: 4
		Fila: 1
		Columna: J
		Nivel de ataque: 7
		Nivel de Defensa: 3
		Nivel de vida: 4
		Velocidad: 5
		Actitud: Arquero
		Estado: true
		---------------------------------
		Registrando al 2 soldado del Ejercito Castilla
		
		Nombre: Caballero1X1
		Vida: 11
		Fila: 5
		Columna: J
		Nivel de ataque: 13
		Nivel de Defensa: 7
		Nivel de vida: 11
		Velocidad: 4
		Actitud: Caballero
		Estado: true
		---------------------------------
		Registrando al 3 soldado del Ejercito Castilla
		
		Nombre: Arquero2X1
		Vida: 5
		Fila: 10
		Columna: J
		Nivel de ataque: 7
		Nivel de Defensa: 3
		Nivel de vida: 5
		Velocidad: 5
		Actitud: Arquero
		Estado: true
		---------------------------------
		Registrando al 4 soldado del Ejercito Castilla
		
		Nombre: Caballero3X1
		Vida: 11
		Fila: 3
		Columna: C
		Nivel de ataque: 13
		Nivel de Defensa: 7
		Nivel de vida: 11
		Velocidad: 2
		Actitud: Caballero
		Estado: true
		---------------------------------
		Registrando al 5 soldado del Ejercito Castilla
		
		Nombre: Arquero4X1
		Vida: 4
		Fila: 5
		Columna: B
		Nivel de ataque: 7
		Nivel de Defensa: 3
		Nivel de vida: 4
		Velocidad: 3
		Actitud: Arquero
		Estado: true
		---------------------------------
		Registrando al 6 soldado del Ejercito Castilla
		
		Nombre: Caballero5X1
		Vida: 10
		Fila: 9
		Columna: D
		Nivel de ataque: 13
		Nivel de Defensa: 7
		Nivel de vida: 10
		Velocidad: 4
		Actitud: Caballero
		Estado: true
		---------------------------------
		Registrando al 7 soldado del Ejercito Castilla
		
		Nombre: Caballero6X1
		Vida: 12
		Fila: 5
		Columna: H
		Nivel de ataque: 13
		Nivel de Defensa: 7
		Nivel de vida: 12
		Velocidad: 2
		Actitud: Caballero
		Estado: true
		---------------------------------
		Registrando al 8 soldado del Ejercito Castilla
		
		Nombre: Caballero7X1
		Vida: 10
		Fila: 2
		Columna: I
		Nivel de ataque: 13
		Nivel de Defensa: 7
		Nivel de vida: 10
		Velocidad: 3
		Actitud: Caballero
		Estado: true
		---------------------------------
		Registrando al 9 soldado del Ejercito Castilla
		
		Nombre: Espadachin Conquistador8X1
		Vida: 14
		Fila: 9
		Columna: H
		Nivel de ataque: 10
		Nivel de Defensa: 8
		Nivel de vida: 14
		Velocidad: 3
		Actitud: Espadachin Conquistador
		Estado: true
		---------------------------------
		*********************************
		El Ejercito Moros tiene 8 soldados : 
		
		Registrando al 1 soldado del Ejercito Moros
		
		Nombre: Caballero0X2
		Vida: 12
		Fila: 9
		Columna: D
		Nivel de ataque: 13
		Nivel de Defensa: 7
		Nivel de vida: 12
		Velocidad: 1
		Actitud: Caballero
		Estado: true
		---------------------------------
		Registrando al 2 soldado del Ejercito Moros
		
		Nombre: Caballero1X2
		Vida: 12
		Fila: 10
		Columna: A
		Nivel de ataque: 13
		Nivel de Defensa: 7
		Nivel de vida: 12
		Velocidad: 3
		Actitud: Caballero
		Estado: true
		---------------------------------
		Registrando al 3 soldado del Ejercito Moros
		
		Nombre: Espadachin2X2
		Vida: 10
		Fila: 6
		Columna: B
		Nivel de ataque: 10
		Nivel de Defensa: 8
		Nivel de vida: 10
		Velocidad: 2
		Actitud: Espadachin
		Estado: true
		---------------------------------
		Registrando al 4 soldado del Ejercito Moros
		
		Nombre: Espadachin3X2
		Vida: 8
		Fila: 4
		Columna: J
		Nivel de ataque: 10
		Nivel de Defensa: 8
		Nivel de vida: 8
		Velocidad: 3
		Actitud: Espadachin
		Estado: true
		---------------------------------
		Registrando al 5 soldado del Ejercito Moros
		
		Nombre: Caballero Moro4X2
		Vida: 13
		Fila: 8
		Columna: A
		Nivel de ataque: 13
		Nivel de Defensa: 7
		Nivel de vida: 13
		Velocidad: 5
		Actitud: Caballero Moro
		Estado: true
		---------------------------------
		Registrando al 6 soldado del Ejercito Moros
		
		Nombre: Caballero Moro5X2
		Vida: 13
		Fila: 1
		Columna: A
		Nivel de ataque: 13
		Nivel de Defensa: 7
		Nivel de vida: 13
		Velocidad: 2
		Actitud: Caballero Moro
		Estado: true
		---------------------------------
		Registrando al 7 soldado del Ejercito Moros
		
		Nombre: Lancero6X2
		Vida: 7
		Fila: 9
		Columna: H
		Nivel de ataque: 5
		Nivel de Defensa: 10
		Nivel de vida: 7
		Velocidad: 2
		Actitud: Lancero
		Estado: true
		---------------------------------
		Registrando al 8 soldado del Ejercito Moros
		
		Nombre: Arquero7X2
		Vida: 5
		Fila: 1
		Columna: H
		Nivel de ataque: 7
		Nivel de Defensa: 3
		Nivel de vida: 5
		Velocidad: 5
		Actitud: Arquero
		Estado: true
		---------------------------------
		*********************************
		
		*********************************
		El tipo de territorio es: playa
		
		*********************************
		El Ejercito Castilla del 1 ejercito sus soldados son :
		
		*********************************
		El 1 soldado es: 
		
		Nombre: Arquero0X1
		Vida: 4
		Fila: 1
		Columna: J
		Nivel de ataque: 7
		Nivel de Defensa: 3
		Nivel de vida: 4
		Velocidad: 5
		Actitud: Arquero
		Estado: true
		
		*********************************
		El 2 soldado es: 
		
		Nombre: Caballero7X1
		Vida: 10
		Fila: 2
		Columna: I
		Nivel de ataque: 13
		Nivel de Defensa: 7
		Nivel de vida: 10
		Velocidad: 3
		Actitud: Caballero
		Estado: true
		
		*********************************
		El 3 soldado es: 
		
		Nombre: Caballero3X1
		Vida: 11
		Fila: 3
		Columna: C
		Nivel de ataque: 13
		Nivel de Defensa: 7
		Nivel de vida: 11
		Velocidad: 2
		Actitud: Caballero
		Estado: true
		
		*********************************
		El 4 soldado es: 
		
		Nombre: Arquero4X1
		Vida: 4
		Fila: 5
		Columna: B
		Nivel de ataque: 7
		Nivel de Defensa: 3
		Nivel de vida: 4
		Velocidad: 3
		Actitud: Arquero
		Estado: true
		
		*********************************
		El 5 soldado es: 
		
		Nombre: Caballero6X1
		Vida: 12
		Fila: 5
		Columna: H
		Nivel de ataque: 13
		Nivel de Defensa: 7
		Nivel de vida: 12
		Velocidad: 2
		Actitud: Caballero
		Estado: true
		
		*********************************
		El 6 soldado es: 
		
		Nombre: Caballero1X1
		Vida: 11
		Fila: 5
		Columna: J
		Nivel de ataque: 13
		Nivel de Defensa: 7
		Nivel de vida: 11
		Velocidad: 4
		Actitud: Caballero
		Estado: true
		
		*********************************
		El 7 soldado es: 
		
		Nombre: Caballero5X1
		Vida: 10
		Fila: 9
		Columna: D
		Nivel de ataque: 13
		Nivel de Defensa: 7
		Nivel de vida: 10
		Velocidad: 4
		Actitud: Caballero
		Estado: true
		
		*********************************
		El 8 soldado es: 
		
		Nombre: Espadachin Conquistador8X1
		Vida: 14
		Fila: 9
		Columna: H
		Nivel de ataque: 10
		Nivel de Defensa: 8
		Nivel de vida: 14
		Velocidad: 3
		Actitud: Espadachin Conquistador
		Estado: true
		
		*********************************
		El 9 soldado es: 
		
		Nombre: Arquero2X1
		Vida: 5
		Fila: 10
		Columna: J
		Nivel de ataque: 7
		Nivel de Defensa: 3
		Nivel de vida: 5
		Velocidad: 5
		Actitud: Arquero
		Estado: true
		El Ejercito Moros del 2 ejercito sus soldados son :
		
		*********************************
		El 1 soldado es: 
		
		Nombre: Caballero Moro5X2
		Vida: 13
		Fila: 1
		Columna: A
		Nivel de ataque: 13
		Nivel de Defensa: 7
		Nivel de vida: 13
		Velocidad: 2
		Actitud: Caballero Moro
		Estado: true
		
		*********************************
		El 2 soldado es: 
		
		Nombre: Arquero7X2
		Vida: 5
		Fila: 1
		Columna: H
		Nivel de ataque: 7
		Nivel de Defensa: 3
		Nivel de vida: 5
		Velocidad: 5
		Actitud: Arquero
		Estado: true
		
		*********************************
		El 3 soldado es: 
		
		Nombre: Espadachin3X2
		Vida: 8
		Fila: 4
		Columna: J
		Nivel de ataque: 10
		Nivel de Defensa: 8
		Nivel de vida: 8
		Velocidad: 3
		Actitud: Espadachin
		Estado: true
		
		*********************************
		El 4 soldado es: 
		
		Nombre: Espadachin2X2
		Vida: 10
		Fila: 6
		Columna: B
		Nivel de ataque: 10
		Nivel de Defensa: 8
		Nivel de vida: 10
		Velocidad: 2
		Actitud: Espadachin
		Estado: true
		
		*********************************
		El 5 soldado es: 
		
		Nombre: Caballero Moro4X2
		Vida: 13
		Fila: 8
		Columna: A
		Nivel de ataque: 13
		Nivel de Defensa: 7
		Nivel de vida: 13
		Velocidad: 5
		Actitud: Caballero Moro
		Estado: true
		
		*********************************
		El 6 soldado es: 
		
		Nombre: Caballero0X2
		Vida: 12
		Fila: 9
		Columna: D
		Nivel de ataque: 13
		Nivel de Defensa: 7
		Nivel de vida: 12
		Velocidad: 1
		Actitud: Caballero
		Estado: true
		
		*********************************
		El 7 soldado es: 
		
		Nombre: Lancero6X2
		Vida: 7
		Fila: 9
		Columna: H
		Nivel de ataque: 5
		Nivel de Defensa: 10
		Nivel de vida: 7
		Velocidad: 2
		Actitud: Lancero
		Estado: true
		
		*********************************
		El 8 soldado es: 
		
		Nombre: Caballero1X2
		Vida: 12
		Fila: 10
		Columna: A
		Nivel de ataque: 13
		Nivel de Defensa: 7
		Nivel de vida: 12
		Velocidad: 3
		Actitud: Caballero
		Estado: true
		
		*********************************
		El tipo de territorio es: playa
		
		*********************************
		
	\end{lstlisting}
	\begin{figure}[H]
		\centering
		\includegraphics[width=1.0\textwidth,keepaspectratio]{img/Commit11-3.png}
		%\includesvg{img/automata.svg}
		%\label{img:mot2}
		%\caption{Product backlog.}
	\end{figure}
	\begin{lstlisting}[language=bash,caption={Ejecucion:}][H]
		*********************************
		El soldado con mayor vida del Ejercito Castilla es: 
		
		Nombre: Caballero6X1
		Vida: 12
		Fila: 5
		Columna: H
		Nivel de ataque: 13
		Nivel de Defensa: 7
		Nivel de vida: 12
		Velocidad: 2
		Actitud: Caballero
		Estado: true
		*********************************
		El soldado con mayor vida del Ejercito Moros es: 
		
		Nombre: Caballero Moro5X2
		Vida: 13
		Fila: 1
		Columna: A
		Nivel de ataque: 13
		Nivel de Defensa: 7
		Nivel de vida: 13
		Velocidad: 2
		Actitud: Caballero Moro
		Estado: true
		*********************************
		El promedio de puntos de vida del Ejercito Castilla es: 
		8.0
		*********************************
		El promedio de puntos de vida del Ejercito Moros es: 
		9.0
		*********************************
		
		El Ejercito Castilla ordenando por metodo insercion: 
		------------------------------------------
		Mostrando Ranking del Ejercito Castilla ..... ////// --->
		
		Puesto 1
		Nombre: Caballero6X1
		Vida: 12
		Fila: 5
		Columna: H
		Nivel de ataque: 13
		Nivel de Defensa: 7
		Nivel de vida: 12
		Velocidad: 2
		Actitud: Caballero
		Estado: true
		------------------
		
		Puesto 2
		Nombre: Caballero3X1
		Vida: 11
		Fila: 3
		Columna: C
		Nivel de ataque: 13
		Nivel de Defensa: 7
		Nivel de vida: 11
		Velocidad: 2
		Actitud: Caballero
		Estado: true
		------------------
		
		Puesto 3
		Nombre: Caballero1X1
		Vida: 11
		Fila: 5
		Columna: J
		Nivel de ataque: 13
		Nivel de Defensa: 7
		Nivel de vida: 11
		Velocidad: 4
		Actitud: Caballero
		Estado: true
		------------------
		
		Puesto 4
		Nombre: Caballero7X1
		Vida: 10
		Fila: 2
		Columna: I
		Nivel de ataque: 13
		Nivel de Defensa: 7
		Nivel de vida: 10
		Velocidad: 3
		Actitud: Caballero
		Estado: true
		------------------
		
		Puesto 5
		Nombre: Espadachin Conquistador8X1
		Vida: 7
		Fila: 9
		Columna: H
		Nivel de ataque: 10
		Nivel de Defensa: 8
		Nivel de vida: 14
		Velocidad: 3
		Actitud: Espadachin Conquistador
		Estado: true
		------------------
		
		Puesto 6
		Nombre: Arquero2X1
		Vida: 5
		Fila: 10
		Columna: J
		Nivel de ataque: 7
		Nivel de Defensa: 3
		Nivel de vida: 5
		Velocidad: 5
		Actitud: Arquero
		Estado: true
		------------------
		
		Puesto 7
		Nombre: Arquero0X1
		Vida: 4
		Fila: 1
		Columna: J
		Nivel de ataque: 7
		Nivel de Defensa: 3
		Nivel de vida: 4
		Velocidad: 5
		Actitud: Arquero
		Estado: true
		------------------
		
		Puesto 8
		Nombre: Arquero4X1
		Vida: 4
		Fila: 5
		Columna: B
		Nivel de ataque: 7
		Nivel de Defensa: 3
		Nivel de vida: 4
		Velocidad: 3
		Actitud: Arquero
		Estado: true
		------------------
		*********************************
		
		El Ejercito Moros ordenando por metodo insercion: 
		------------------------------------------
		Mostrando Ranking del Ejercito Moros ..... ////// --->
		
		Puesto 1
		Nombre: Caballero Moro5X2
		Vida: 13
		Fila: 1
		Columna: A
		Nivel de ataque: 13
		Nivel de Defensa: 7
		Nivel de vida: 13
		Velocidad: 2
		Actitud: Caballero Moro
		Estado: true
		------------------
		
		Puesto 2
		Nombre: Caballero Moro4X2
		Vida: 13
		Fila: 8
		Columna: A
		Nivel de ataque: 13
		Nivel de Defensa: 7
		Nivel de vida: 13
		Velocidad: 5
		Actitud: Caballero Moro
		Estado: true
		------------------
		
		Puesto 3
		Nombre: Caballero1X2
		Vida: 12
		Fila: 10
		Columna: A
		Nivel de ataque: 13
		Nivel de Defensa: 7
		Nivel de vida: 12
		Velocidad: 3
		Actitud: Caballero
		Estado: true
		------------------
		
		Puesto 4
		Nombre: Espadachin2X2
		Vida: 10
		Fila: 6
		Columna: B
		Nivel de ataque: 10
		Nivel de Defensa: 8
		Nivel de vida: 10
		Velocidad: 2
		Actitud: Espadachin
		Estado: true
		------------------
		
		Puesto 5
		Nombre: Espadachin3X2
		Vida: 8
		Fila: 4
		Columna: J
		Nivel de ataque: 10
		Nivel de Defensa: 8
		Nivel de vida: 8
		Velocidad: 3
		Actitud: Espadachin
		Estado: true
		------------------
		
		Puesto 6
		Nombre: Arquero7X2
		Vida: 5
		Fila: 1
		Columna: H
		Nivel de ataque: 7
		Nivel de Defensa: 3
		Nivel de vida: 5
		Velocidad: 5
		Actitud: Arquero
		Estado: true
		------------------
		
		Puesto 7
		Nombre: Caballero0X2
		Vida: 2
		Fila: 9
		Columna: D
		Nivel de ataque: 13
		Nivel de Defensa: 7
		Nivel de vida: 12
		Velocidad: 1
		Actitud: Caballero
		Estado: true
		------------------
		*********************************
		
		*********************************
		El tipo de territorio es: playa
		
		*********************************
		
	\end{lstlisting}
	\begin{figure}[H]
		\centering
		\includegraphics[width=1.0\textwidth,keepaspectratio]{img/Commit11-3.png}
		%\includesvg{img/automata.svg}
		%\label{img:mot2}
		%\caption{Product backlog.}
	\end{figure}
	\begin{lstlisting}[language=bash,caption={Ejecucion:}][H]
		*********************************
		Ejercito 1 : Castilla
		Cantidad total de soldados creados: 8
		Espadachines: 0
		Arqueros: 3
		Caballeros: 4
		Lanceros: 0
		Espadachin Real: 0
		Caballero Moro: 0
		Caballero Franco: 0
		Espadachin Teutonico: 0
		Espadachin Conquistador: 1
		
		Ejercito 2 : Moros
		Cantidad total de soldados creados: 7
		Espadachines: 2
		Arqueros: 1
		Caballeros: 2
		Lanceros: 0
		Espadachin Real: 0
		Caballero Moro: 2
		Caballero Franco: 0
		Espadachin Teutonico: 0
		Espadachin Conquistador: 0
		
		Ejercito 1: Castilla: 71
		Ejercito 2: Moros: 73
		El ganador es el ejercito 2 de: Moros. Ya que al generar los porcentajes de probabilidad de victoria basada en los niveles de vida de sus soldados y aplicando un experimento aleatorio salio vencedor. (Aleatorio generado : 50.69)
		
		*********************************
		 DESEA VOLVER A JUGAR
		 1 : JUGAR
		 2 : NO JUGAR
		
	\end{lstlisting}
	\subsection{Ejercicio Mapa}
	\begin{itemize}	
		\item En este fragmento de código, solicito al usuario que ingrese su preferencia sobre jugar de nuevo mediante un cuadro de diálogo de entrada de JOptionPane. La respuesta se almacena en la variable respuesta. Luego, compruebo si su respuesta no es "si" (ignorando mayúsculas y minúsculas). Si la respuesta no es afirmativa, muestro un cuadro de diálogo informativo indicando "Fin del programa. ¡Hasta luego!" y establezco la variable play en false. Esto me permite controlar la ejecución del programa y salir de un posible bucle de juego en caso de que el usuario decida no jugar de nuevo.
		\item El codigo y el commit seria el siguiente:
	\end{itemize}	
	\begin{lstlisting}[language=bash,caption={Commit}][H]
		$ git commit -m "Agregando el menu de volver a jugar en JOption el cual me permitira jugar denuevo o no en caso que no mi imprime un mensage"
	\end{lstlisting}	
	\begin{lstlisting}[language=java,caption={Las lineas de codigos de la clase Mapa}][H]
		String respuesta = JOptionPane.showInputDialog("¿Quieres jugar de nuevo? (si/no): ");
		if (!respuesta.equalsIgnoreCase("si")) {
			JOptionPane.showMessageDialog(null, "Fin del programa. ¡Hasta luego!");
			play = false;
		}
	\end{lstlisting}
	\begin{figure}[H]
		\centering
		\includegraphics[width=1.0\textwidth,keepaspectratio]{img/Commit11-1.png}
		%\includesvg{img/automata.svg}
		%\label{img:mot2}
		%\caption{Product backlog.}
	\end{figure}
	\begin{figure}[H]
		\centering
		\includegraphics[width=1.0\textwidth,keepaspectratio]{img/Commit12-1.png}
		%\includesvg{img/automata.svg}
		%\label{img:mot2}
		%\caption{Product backlog.}
	\end{figure}
	\begin{lstlisting}[language=bash,caption={Ejecucion:}][H]
		El Ejercito Moros tiene 6 soldados : 

		Registrando al 1 soldado del Ejercito Moros
		
		Nombre: Espadachin0X1
		Vida: 8
		Fila: 7
		Columna: C
		Nivel de ataque: 10
		Nivel de Defensa: 8
		Nivel de vida: 8
		Velocidad: 4
		Actitud: Espadachin
		Estado: true
		---------------------------------
		Registrando al 2 soldado del Ejercito Moros
		
		Nombre: Caballero1X1
		Vida: 11
		Fila: 6
		Columna: F
		Nivel de ataque: 13
		Nivel de Defensa: 7
		Nivel de vida: 11
		Velocidad: 2
		Actitud: Caballero
		Estado: true
		---------------------------------
		Registrando al 3 soldado del Ejercito Moros
		
		Nombre: Lancero2X1
		Vida: 5
		Fila: 1
		Columna: G
		Nivel de ataque: 5
		Nivel de Defensa: 10
		Nivel de vida: 5
		Velocidad: 3
		Actitud: Lancero
		Estado: true
		---------------------------------
		Registrando al 4 soldado del Ejercito Moros
		
		Nombre: Lancero3X1
		Vida: 5
		Fila: 9
		Columna: J
		Nivel de ataque: 5
		Nivel de Defensa: 10
		Nivel de vida: 5
		Velocidad: 2
		Actitud: Lancero
		Estado: true
		---------------------------------
		Registrando al 5 soldado del Ejercito Moros
		
		Nombre: Espadachin4X1
		Vida: 10
		Fila: 3
		Columna: E
		Nivel de ataque: 10
		Nivel de Defensa: 8
		Nivel de vida: 10
		Velocidad: 2
		Actitud: Espadachin
		Estado: true
		---------------------------------
		Registrando al 6 soldado del Ejercito Moros
		
		Nombre: Espadachin5X1
		Vida: 9
		Fila: 7
		Columna: H
		Nivel de ataque: 10
		Nivel de Defensa: 8
		Nivel de vida: 9
		Velocidad: 5
		Actitud: Espadachin
		Estado: true
		---------------------------------
		*********************************
		El Ejercito Aragon tiene 3 soldados : 
		
		Registrando al 1 soldado del Ejercito Aragon
		
		Nombre: Espadachin0X2
		Vida: 8
		Fila: 2
		Columna: D
		Nivel de ataque: 10
		Nivel de Defensa: 8
		Nivel de vida: 8
		Velocidad: 5
		Actitud: Espadachin
		Estado: true
		---------------------------------
		Registrando al 2 soldado del Ejercito Aragon
		
		Nombre: Espadachin1X2
		Vida: 10
		Fila: 9
		Columna: I
		Nivel de ataque: 10
		Nivel de Defensa: 8
		Nivel de vida: 10
		Velocidad: 1
		Actitud: Espadachin
		Estado: true
		---------------------------------
		Registrando al 3 soldado del Ejercito Aragon
		
		Nombre: Lancero2X2
		Vida: 7
		Fila: 4
		Columna: F
		Nivel de ataque: 5
		Nivel de Defensa: 10
		Nivel de vida: 7
		Velocidad: 2
		Actitud: Lancero
		Estado: true
		---------------------------------
		*********************************
		
		*********************************
		El tipo de territorio es: montaña
		
		*********************************
		El Ejercito Moros del 1 ejercito sus soldados son :
		
		*********************************
		El 1 soldado es: 
		
		Nombre: Lancero2X1
		Vida: 5
		Fila: 1
		Columna: G
		Nivel de ataque: 5
		Nivel de Defensa: 10
		Nivel de vida: 5
		Velocidad: 3
		Actitud: Lancero
		Estado: true
		
		*********************************
		El 2 soldado es: 
		
		Nombre: Espadachin4X1
		Vida: 10
		Fila: 3
		Columna: E
		Nivel de ataque: 10
		Nivel de Defensa: 8
		Nivel de vida: 10
		Velocidad: 2
		Actitud: Espadachin
		Estado: true
		
		*********************************
		El 3 soldado es: 
		
		Nombre: Caballero1X1
		Vida: 11
		Fila: 6
		Columna: F
		Nivel de ataque: 13
		Nivel de Defensa: 7
		Nivel de vida: 11
		Velocidad: 2
		Actitud: Caballero
		Estado: true
		
		*********************************
		El 4 soldado es: 
		
		Nombre: Espadachin0X1
		Vida: 8
		Fila: 7
		Columna: C
		Nivel de ataque: 10
		Nivel de Defensa: 8
		Nivel de vida: 8
		Velocidad: 4
		Actitud: Espadachin
		Estado: true
		
		*********************************
		El 5 soldado es: 
		
		Nombre: Espadachin5X1
		Vida: 9
		Fila: 7
		Columna: H
		Nivel de ataque: 10
		Nivel de Defensa: 8
		Nivel de vida: 9
		Velocidad: 5
		Actitud: Espadachin
		Estado: true
		
		*********************************
		El 6 soldado es: 
		
		Nombre: Lancero3X1
		Vida: 5
		Fila: 9
		Columna: J
		Nivel de ataque: 5
		Nivel de Defensa: 10
		Nivel de vida: 5
		Velocidad: 2
		Actitud: Lancero
		Estado: true
		El Ejercito Aragon del 2 ejercito sus soldados son :
		
		*********************************
		El 1 soldado es: 
		
		Nombre: Espadachin0X2
		Vida: 8
		Fila: 2
		Columna: D
		Nivel de ataque: 10
		Nivel de Defensa: 8
		Nivel de vida: 9
		Velocidad: 5
		Actitud: Espadachin
		Estado: true
		
		*********************************
		El 2 soldado es: 
		
		Nombre: Lancero2X2
		Vida: 7
		Fila: 4
		Columna: F
		Nivel de ataque: 5
		Nivel de Defensa: 10
		Nivel de vida: 8
		Velocidad: 2
		Actitud: Lancero
		Estado: true
		
		*********************************
		El 3 soldado es: 
		
		Nombre: Espadachin1X2
		Vida: 10
		Fila: 9
		Columna: I
		Nivel de ataque: 10
		Nivel de Defensa: 8
		Nivel de vida: 11
		Velocidad: 1
		Actitud: Espadachin
		Estado: true
		
		*********************************
		El tipo de territorio es: montaña
		
		*********************************
		
	\end{lstlisting}
	\begin{figure}[H]
		\centering
		\includegraphics[width=1.0\textwidth,keepaspectratio]{img/Commit12-5.png}
		%\includesvg{img/automata.svg}
		%\label{img:mot2}
		%\caption{Product backlog.}
	\end{figure}
	\begin{lstlisting}[language=bash,caption={Ejecucion:}][H]
		*********************************
		El soldado con mayor vida del Ejercito Moros es: 
		
		Nombre: Caballero1X1
		Vida: 11
		Fila: 6
		Columna: F
		Nivel de ataque: 13
		Nivel de Defensa: 7
		Nivel de vida: 11
		Velocidad: 2
		Actitud: Caballero
		Estado: true
		*********************************
		El soldado con mayor vida del Ejercito Aragon es: 
		
		Nombre: Espadachin1X2
		Vida: 10
		Fila: 9
		Columna: I
		Nivel de ataque: 10
		Nivel de Defensa: 8
		Nivel de vida: 11
		Velocidad: 1
		Actitud: Espadachin
		Estado: true
		*********************************
		El promedio de puntos de vida del Ejercito Moros es: 
		8.0
		*********************************
		El promedio de puntos de vida del Ejercito Aragon es: 
		8.333333333333334
		*********************************
		
		El Ejercito Moros ordenando por metodo insercion: 
		------------------------------------------
		Mostrando Ranking del Ejercito Moros ..... ////// --->
		
		Puesto 1
		Nombre: Caballero1X1
		Vida: 11
		Fila: 6
		Columna: F
		Nivel de ataque: 13
		Nivel de Defensa: 7
		Nivel de vida: 11
		Velocidad: 2
		Actitud: Caballero
		Estado: true
		------------------
		
		Puesto 2
		Nombre: Espadachin4X1
		Vida: 10
		Fila: 3
		Columna: E
		Nivel de ataque: 10
		Nivel de Defensa: 8
		Nivel de vida: 10
		Velocidad: 2
		Actitud: Espadachin
		Estado: true
		------------------
		
		Puesto 3
		Nombre: Espadachin5X1
		Vida: 9
		Fila: 7
		Columna: H
		Nivel de ataque: 10
		Nivel de Defensa: 8
		Nivel de vida: 9
		Velocidad: 5
		Actitud: Espadachin
		Estado: true
		------------------
		
		Puesto 4
		Nombre: Espadachin0X1
		Vida: 8
		Fila: 7
		Columna: C
		Nivel de ataque: 10
		Nivel de Defensa: 8
		Nivel de vida: 8
		Velocidad: 4
		Actitud: Espadachin
		Estado: true
		------------------
		
		Puesto 5
		Nombre: Lancero2X1
		Vida: 5
		Fila: 1
		Columna: G
		Nivel de ataque: 5
		Nivel de Defensa: 10
		Nivel de vida: 5
		Velocidad: 3
		Actitud: Lancero
		Estado: true
		------------------
		
		Puesto 6
		Nombre: Lancero3X1
		Vida: 5
		Fila: 9
		Columna: J
		Nivel de ataque: 5
		Nivel de Defensa: 10
		Nivel de vida: 5
		Velocidad: 2
		Actitud: Lancero
		Estado: true
		------------------
		*********************************
		
		El Ejercito Aragon ordenando por metodo insercion: 
		------------------------------------------
		Mostrando Ranking del Ejercito Aragon ..... ////// --->
		
		Puesto 1
		Nombre: Espadachin1X2
		Vida: 10
		Fila: 9
		Columna: I
		Nivel de ataque: 10
		Nivel de Defensa: 8
		Nivel de vida: 11
		Velocidad: 1
		Actitud: Espadachin
		Estado: true
		------------------
		
		Puesto 2
		Nombre: Espadachin0X2
		Vida: 8
		Fila: 2
		Columna: D
		Nivel de ataque: 10
		Nivel de Defensa: 8
		Nivel de vida: 9
		Velocidad: 5
		Actitud: Espadachin
		Estado: true
		------------------
		
		Puesto 3
		Nombre: Lancero2X2
		Vida: 7
		Fila: 4
		Columna: F
		Nivel de ataque: 5
		Nivel de Defensa: 10
		Nivel de vida: 8
		Velocidad: 2
		Actitud: Lancero
		Estado: true
		------------------
		*********************************
		
		*********************************
		El tipo de territorio es: montaña
		
		*********************************
		
	\end{lstlisting}
	\begin{figure}[H]
		\centering
		\includegraphics[width=1.0\textwidth,keepaspectratio]{img/Commit12-5.png}
		%\includesvg{img/automata.svg}
		%\label{img:mot2}
		%\caption{Product backlog.}
	\end{figure}
	\begin{lstlisting}[language=bash,caption={Ejecucion:}][H]
		*********************************
		Ejercito 1 : Moros
		Cantidad total de soldados creados: 6
		Espadachines: 3
		Arqueros: 0
		Caballeros: 1
		Lanceros: 2
		Espadachin Real: 0
		Caballero Moro: 0
		Caballero Franco: 0
		Espadachin Teutonico: 0
		Espadachin Conquistador: 0
		
		Ejercito 2 : Aragon
		Cantidad total de soldados creados: 3
		Espadachines: 2
		Arqueros: 0
		Caballeros: 0
		Lanceros: 1
		Espadachin Real: 0
		Caballero Moro: 0
		Caballero Franco: 0
		Espadachin Teutonico: 0
		Espadachin Conquistador: 0
		
		Ejercito 1: Moros: 48
		Ejercito 2: Aragon: 28
		El ganador es el ejercito 1 de: Moros. Ya que al generar los porcentajes de probabilidad de victoria basada en los niveles de vida de sus soldados y aplicando un experimento aleatorio salió vencedor. (Aleatorio generado : 63.16)
		
	\end{lstlisting}
	\begin{figure}[H]
		\centering
		\includegraphics[width=1.0\textwidth,keepaspectratio]{img/Commit12-2.png}
		%\includesvg{img/automata.svg}
		%\label{img:mot2}
		%\caption{Product backlog.}
	\end{figure}
	\begin{figure}[H]
		\centering
		\includegraphics[width=1.0\textwidth,keepaspectratio]{img/Commit12-3.png}
		%\includesvg{img/automata.svg}
		%\label{img:mot2}
		%\caption{Product backlog.}
	\end{figure}
	\begin{figure}[H]
		\centering
		\includegraphics[width=1.0\textwidth,keepaspectratio]{img/Commit12-4.png}
		%\includesvg{img/automata.svg}
		%\label{img:mot2}
		%\caption{Product backlog.}
	\end{figure}
	\subsection{Ejercicio BoardView}
	\begin{itemize}	
		\item En este fragmento de código, En este bloque de código, he implementado la clase BoardView que extiende JFrame, con el propósito de mostrar gráficamente el tablero de batalla. El tablero se compone de un panel principal (mainPanel) organizado con un diseño de cuadrícula (GridLayout) de 12x12. Se han añadido elementos de interfaz de usuario, como la barra de menú (JMenuBar), con un menú de archivo que incluye una opción para salir del programa.
		\item La ventana se configura con un tamaño predeterminado y se llena con etiquetas (JLabel) que representan las letras de las columnas y los números de las filas del tablero. Además, he incorporado la lógica para mostrar la información de los soldados de los dos ejércitos en el tablero. Cada casilla muestra el número del ejército (1 o 2) y la inicial del tipo de soldado junto con su vida actual.
		\item La función obtenerInicial() determina la inicial del tipo de soldado en función del tipo de soldado proporcionado como argumento. La lógica del combate se implementa para actualizar la información de los soldados después de cada enfrentamiento, ajustando sus vidas y eliminándolos según sea necesario.
		\item El codigo y el commit seria el siguiente:
	\end{itemize}	
	\begin{lstlisting}[language=bash,caption={Commit}][H]
		$ git commit -m "Anadiendo la clase BoardView la cual me permite ver la tabla de forma grafica nos ayudamos de las java awt y java swing para lograr esto"
	\end{lstlisting}	
	\begin{lstlisting}[language=java,caption={Las lineas de codigos de la clase BoardView}][H]
		import javax.swing.*;
		import java.awt.*;
		import java.awt.event.ActionEvent;
		import java.awt.event.ActionListener;
		import java.util.ArrayList;
		public class BoardView extends JFrame {
		
			private static final String[] COLUMN_LETTERS = {"A", "B", "C", "D", "E", "F", "G", "H", "I", "J"};
		
			public BoardView(ArrayList<ArrayList<Soldado>> army1, ArrayList<ArrayList<Soldado>> army2) {
				setTitle("Tablero de Batalla");
				setDefaultCloseOperation(JFrame.EXIT_ON_CLOSE);
		
				JPanel mainPanel = new JPanel(new GridLayout(12, 12));
		
				JMenuBar menuBar = new JMenuBar();
				JMenu fileMenu = new JMenu("Archivo");
				JMenuItem exitMenuItem = new JMenuItem("Salir");
				exitMenuItem.addActionListener(new ActionListener() {
					@Override
					public void actionPerformed(ActionEvent e) {
						System.exit(0);
					}
				});
				fileMenu.add(exitMenuItem);
				menuBar.add(fileMenu);
				setJMenuBar(menuBar);
		
				// Configurar el tamano predeterminado de la ventana
				setSize(100, 800);
		
				mainPanel.add(new JLabel(" "));
				for (String letter : COLUMN_LETTERS) {
					mainPanel.add(new JLabel("   " + letter));
				}
		
				mainPanel.add(new JLabel("  "));
				for (String letter : COLUMN_LETTERS) {
					mainPanel.add(new JLabel("   " + letter));
				}
		
				for (int i = 0; i < 10; i++) {
					mainPanel.add(new JLabel(String.valueOf(i + 1)));
					for (int j = 0; j < 10; j++) {
						JLabel label = new JLabel();
						label.setBorder(BorderFactory.createLineBorder(Color.BLACK));
						label.setHorizontalAlignment(JLabel.CENTER);
		
						if (army1.get(i).get(j) != null && army2.get(i).get(j) != null) {
							if(army1.get(i).get(j).getLifeActual() > army2.get(i).get(j).getLifeActual()){
								army1.get(i).get(j).setLifeActual(army1.get(i).get(j).getLifeActual() - army2.get(i).get(j).getLifeActual()); //Cambiamos 
								army2.get(i).set(j, null); 
								label.setText("1" + obtenerInicial(army1.get(i).get(j)) + army1.get(i).get(j).getLifeActual());
							}else if(army2.get(i).get(j).getLifeActual() > army1.get(i).get(j).getLifeActual()){
								army2.get(i).get(j).setLifeActual(army2.get(i).get(j).getLifeActual() - army1.get(i).get(j).getLifeActual());
								army1.get(i).set(j, null);;
								label.setText("2" + obtenerInicial(army2.get(i).get(j)) + army2.get(i).get(j).getLifeActual());
							}else{
								army2.get(i).set(j, null);
								army1.get(i).set(j, null);
							}
						} else if (army1.get(i).get(j) != null) {
							label.setText("1" + obtenerInicial(army1.get(i).get(j)) + army1.get(i).get(j).getLifeActual());
						} else if (army2.get(i).get(j) != null) {
							label.setText("2" + obtenerInicial(army2.get(i).get(j)) + army2.get(i).get(j).getLifeActual());
						}
						mainPanel.add(label);
					}
				}
		
				add(mainPanel);
				pack();
				setLocationRelativeTo(null);
				setVisible(true);
			}
			public static String obtenerInicial(Soldado soldado) {
				if (soldado instanceof Espadachin) {
					return "E";
				} else if (soldado instanceof Arquero) {
					return "A";
				} else if (soldado instanceof Caballero) {
					return "C";
				} else if (soldado instanceof Lancero) {
					return "L";
				} else if (soldado instanceof EspadachinReal){
					return "ER";
				} else if (soldado instanceof CaballeroFranco){
					return "CF";
				} else if (soldado instanceof EspadachinTeutonico){
					return "ET";
				} else if (soldado instanceof EspadachinConquistador){
					return "EC";
				} else if (soldado instanceof CaballeroMoro){
					return "CM";
				}else{
					return " ";
				}
			}
		}
	\end{lstlisting}
	\begin{figure}[H]
		\centering
		\includegraphics[width=1.0\textwidth,keepaspectratio]{img/Commit11-1.png}
		%\includesvg{img/automata.svg}
		%\label{img:mot2}
		%\caption{Product backlog.}
	\end{figure}
	\begin{figure}[H]
		\centering
		\includegraphics[width=1.0\textwidth,keepaspectratio]{img/Commit12-1.png}
		%\includesvg{img/automata.svg}
		%\label{img:mot2}
		%\caption{Product backlog.}
	\end{figure}
	\begin{lstlisting}[language=bash,caption={Ejecucion:}][H]
		El Ejercito Castilla tiene 9 soldados : 

		Registrando al 1 soldado del Ejercito Castilla
		
		Nombre: Espadachin0X1
		Vida: 10
		Fila: 7
		Columna: C
		Nivel de ataque: 10
		Nivel de Defensa: 8
		Nivel de vida: 10
		Velocidad: 2
		Actitud: Espadachin
		Estado: true
		---------------------------------
		Registrando al 2 soldado del Ejercito Castilla
		
		Nombre: Arquero1X1
		Vida: 4
		Fila: 8
		Columna: J
		Nivel de ataque: 7
		Nivel de Defensa: 3
		Nivel de vida: 4
		Velocidad: 1
		Actitud: Arquero
		Estado: true
		---------------------------------
		Registrando al 3 soldado del Ejercito Castilla
		
		Nombre: Espadachin Conquistador2X1
		Vida: 14
		Fila: 1
		Columna: A
		Nivel de ataque: 10
		Nivel de Defensa: 8
		Nivel de vida: 14
		Velocidad: 1
		Actitud: Espadachin Conquistador
		Estado: true
		---------------------------------
		Registrando al 4 soldado del Ejercito Castilla
		
		Nombre: Lancero3X1
		Vida: 7
		Fila: 8
		Columna: I
		Nivel de ataque: 5
		Nivel de Defensa: 10
		Nivel de vida: 7
		Velocidad: 5
		Actitud: Lancero
		Estado: true
		---------------------------------
		Registrando al 5 soldado del Ejercito Castilla
		
		Nombre: Espadachin4X1
		Vida: 8
		Fila: 5
		Columna: E
		Nivel de ataque: 10
		Nivel de Defensa: 8
		Nivel de vida: 8
		Velocidad: 2
		Actitud: Espadachin
		Estado: true
		---------------------------------
		Registrando al 6 soldado del Ejercito Castilla
		
		Nombre: Caballero5X1
		Vida: 11
		Fila: 1
		Columna: G
		Nivel de ataque: 13
		Nivel de Defensa: 7
		Nivel de vida: 11
		Velocidad: 2
		Actitud: Caballero
		Estado: true
		---------------------------------
		Registrando al 7 soldado del Ejercito Castilla
		
		Nombre: Espadachin Conquistador6X1
		Vida: 14
		Fila: 9
		Columna: D
		Nivel de ataque: 10
		Nivel de Defensa: 8
		Nivel de vida: 14
		Velocidad: 4
		Actitud: Espadachin Conquistador
		Estado: true
		---------------------------------
		Registrando al 8 soldado del Ejercito Castilla
		
		Nombre: Espadachin Conquistador7X1
		Vida: 14
		Fila: 10
		Columna: F
		Nivel de ataque: 10
		Nivel de Defensa: 8
		Nivel de vida: 14
		Velocidad: 1
		Actitud: Espadachin Conquistador
		Estado: true
		---------------------------------
		Registrando al 9 soldado del Ejercito Castilla
		
		Nombre: Caballero8X1
		Vida: 12
		Fila: 5
		Columna: C
		Nivel de ataque: 13
		Nivel de Defensa: 7
		Nivel de vida: 12
		Velocidad: 5
		Actitud: Caballero
		Estado: true
		---------------------------------
		*********************************
		El Ejercito Moros tiene 5 soldados : 
		
		Registrando al 1 soldado del Ejercito Moros
		
		Nombre: Lancero0X2
		Vida: 7
		Fila: 9
		Columna: G
		Nivel de ataque: 5
		Nivel de Defensa: 10
		Nivel de vida: 7
		Velocidad: 2
		Actitud: Lancero
		Estado: true
		---------------------------------
		Registrando al 2 soldado del Ejercito Moros
		
		Nombre: Lancero1X2
		Vida: 6
		Fila: 6
		Columna: I
		Nivel de ataque: 5
		Nivel de Defensa: 10
		Nivel de vida: 6
		Velocidad: 2
		Actitud: Lancero
		Estado: true
		---------------------------------
		Registrando al 3 soldado del Ejercito Moros
		
		Nombre: Espadachin2X2
		Vida: 10
		Fila: 6
		Columna: F
		Nivel de ataque: 10
		Nivel de Defensa: 8
		Nivel de vida: 10
		Velocidad: 1
		Actitud: Espadachin
		Estado: true
		---------------------------------
		Registrando al 4 soldado del Ejercito Moros
		
		Nombre: Arquero3X2
		Vida: 5
		Fila: 8
		Columna: I
		Nivel de ataque: 7
		Nivel de Defensa: 3
		Nivel de vida: 5
		Velocidad: 5
		Actitud: Arquero
		Estado: true
		---------------------------------
		Registrando al 5 soldado del Ejercito Moros
		
		Nombre: Lancero4X2
		Vida: 6
		Fila: 4
		Columna: C
		Nivel de ataque: 5
		Nivel de Defensa: 10
		Nivel de vida: 6
		Velocidad: 2
		Actitud: Lancero
		Estado: true
		---------------------------------
		*********************************
		
		*********************************
		El tipo de territorio es: montaña
		
		*********************************
		El Ejercito Castilla del 1 ejercito sus soldados son :
		
		*********************************
		El 1 soldado es: 
		
		Nombre: Espadachin Conquistador2X1
		Vida: 14
		Fila: 1
		Columna: A
		Nivel de ataque: 10
		Nivel de Defensa: 8
		Nivel de vida: 15
		Velocidad: 1
		Actitud: Espadachin Conquistador
		Estado: true
		
		*********************************
		El 2 soldado es: 
		
		Nombre: Caballero5X1
		Vida: 11
		Fila: 1
		Columna: G
		Nivel de ataque: 13
		Nivel de Defensa: 7
		Nivel de vida: 12
		Velocidad: 2
		Actitud: Caballero
		Estado: true
		
		*********************************
		El 3 soldado es: 
		
		Nombre: Caballero8X1
		Vida: 12
		Fila: 5
		Columna: C
		Nivel de ataque: 13
		Nivel de Defensa: 7
		Nivel de vida: 13
		Velocidad: 5
		Actitud: Caballero
		Estado: true
		
		*********************************
		El 4 soldado es: 
		
		Nombre: Espadachin4X1
		Vida: 8
		Fila: 5
		Columna: E
		Nivel de ataque: 10
		Nivel de Defensa: 8
		Nivel de vida: 9
		Velocidad: 2
		Actitud: Espadachin
		Estado: true
		
		*********************************
		El 5 soldado es: 
		
		Nombre: Espadachin0X1
		Vida: 10
		Fila: 7
		Columna: C
		Nivel de ataque: 10
		Nivel de Defensa: 8
		Nivel de vida: 11
		Velocidad: 2
		Actitud: Espadachin
		Estado: true
		
		*********************************
		El 6 soldado es: 
		
		Nombre: Lancero3X1
		Vida: 7
		Fila: 8
		Columna: I
		Nivel de ataque: 5
		Nivel de Defensa: 10
		Nivel de vida: 8
		Velocidad: 5
		Actitud: Lancero
		Estado: true
		
		*********************************
		El 7 soldado es: 
		
		Nombre: Arquero1X1
		Vida: 4
		Fila: 8
		Columna: J
		Nivel de ataque: 7
		Nivel de Defensa: 3
		Nivel de vida: 5
		Velocidad: 1
		Actitud: Arquero
		Estado: true
		
		*********************************
		El 8 soldado es: 
		
		Nombre: Espadachin Conquistador6X1
		Vida: 14
		Fila: 9
		Columna: D
		Nivel de ataque: 10
		Nivel de Defensa: 8
		Nivel de vida: 15
		Velocidad: 4
		Actitud: Espadachin Conquistador
		Estado: true
		
		*********************************
		El 9 soldado es: 
		
		Nombre: Espadachin Conquistador7X1
		Vida: 14
		Fila: 10
		Columna: F
		Nivel de ataque: 10
		Nivel de Defensa: 8
		Nivel de vida: 15
		Velocidad: 1
		Actitud: Espadachin Conquistador
		Estado: true
		El Ejercito Moros del 2 ejercito sus soldados son :
		
		*********************************
		El 1 soldado es: 
		
		Nombre: Lancero4X2
		Vida: 6
		Fila: 4
		Columna: C
		Nivel de ataque: 5
		Nivel de Defensa: 10
		Nivel de vida: 6
		Velocidad: 2
		Actitud: Lancero
		Estado: true
		
		*********************************
		El 2 soldado es: 
		
		Nombre: Espadachin2X2
		Vida: 10
		Fila: 6
		Columna: F
		Nivel de ataque: 10
		Nivel de Defensa: 8
		Nivel de vida: 10
		Velocidad: 1
		Actitud: Espadachin
		Estado: true
		
		*********************************
		El 3 soldado es: 
		
		Nombre: Lancero1X2
		Vida: 6
		Fila: 6
		Columna: I
		Nivel de ataque: 5
		Nivel de Defensa: 10
		Nivel de vida: 6
		Velocidad: 2
		Actitud: Lancero
		Estado: true
		
		*********************************
		El 4 soldado es: 
		
		Nombre: Arquero3X2
		Vida: 5
		Fila: 8
		Columna: I
		Nivel de ataque: 7
		Nivel de Defensa: 3
		Nivel de vida: 5
		Velocidad: 5
		Actitud: Arquero
		Estado: true
		
		*********************************
		El 5 soldado es: 
		
		Nombre: Lancero0X2
		Vida: 7
		Fila: 9
		Columna: G
		Nivel de ataque: 5
		Nivel de Defensa: 10
		Nivel de vida: 7
		Velocidad: 2
		Actitud: Lancero
		Estado: true
		
		*********************************
		El tipo de territorio es: montaña
		
		*********************************
				
	\end{lstlisting}
	\begin{figure}[H]
		\centering
		\includegraphics[width=1.0\textwidth,keepaspectratio]{img/Commit13-1.png}
		%\includesvg{img/automata.svg}
		%\label{img:mot2}
		%\caption{Product backlog.}
	\end{figure}
	\begin{lstlisting}[language=bash,caption={Ejecucion:}][H]
		*********************************
		El soldado con mayor vida del Ejercito Castilla es: 
		
		Nombre: Espadachin Conquistador2X1
		Vida: 14
		Fila: 1
		Columna: A
		Nivel de ataque: 10
		Nivel de Defensa: 8
		Nivel de vida: 15
		Velocidad: 1
		Actitud: Espadachin Conquistador
		Estado: true
		*********************************
		El soldado con mayor vida del Ejercito Moros es: 
		
		Nombre: Espadachin2X2
		Vida: 10
		Fila: 6
		Columna: F
		Nivel de ataque: 10
		Nivel de Defensa: 8
		Nivel de vida: 10
		Velocidad: 1
		Actitud: Espadachin
		Estado: true
		*********************************
		El promedio de puntos de vida del Ejercito Castilla es: 
		9.88888888888889
		*********************************
		El promedio de puntos de vida del Ejercito Moros es: 
		7.25
		*********************************
		
		El Ejercito Castilla ordenando por metodo insercion: 
		------------------------------------------
		Mostrando Ranking del Ejercito Castilla ..... ////// --->
		
		Puesto 1
		Nombre: Espadachin Conquistador2X1
		Vida: 14
		Fila: 1
		Columna: A
		Nivel de ataque: 10
		Nivel de Defensa: 8
		Nivel de vida: 15
		Velocidad: 1
		Actitud: Espadachin Conquistador
		Estado: true
		------------------
		
		Puesto 2
		Nombre: Espadachin Conquistador6X1
		Vida: 14
		Fila: 9
		Columna: D
		Nivel de ataque: 10
		Nivel de Defensa: 8
		Nivel de vida: 15
		Velocidad: 4
		Actitud: Espadachin Conquistador
		Estado: true
		------------------
		
		Puesto 3
		Nombre: Espadachin Conquistador7X1
		Vida: 14
		Fila: 10
		Columna: F
		Nivel de ataque: 10
		Nivel de Defensa: 8
		Nivel de vida: 15
		Velocidad: 1
		Actitud: Espadachin Conquistador
		Estado: true
		------------------
		
		Puesto 4
		Nombre: Caballero8X1
		Vida: 12
		Fila: 5
		Columna: C
		Nivel de ataque: 13
		Nivel de Defensa: 7
		Nivel de vida: 13
		Velocidad: 5
		Actitud: Caballero
		Estado: true
		------------------
		
		Puesto 5
		Nombre: Caballero5X1
		Vida: 11
		Fila: 1
		Columna: G
		Nivel de ataque: 13
		Nivel de Defensa: 7
		Nivel de vida: 12
		Velocidad: 2
		Actitud: Caballero
		Estado: true
		------------------
		
		Puesto 6
		Nombre: Espadachin0X1
		Vida: 10
		Fila: 7
		Columna: C
		Nivel de ataque: 10
		Nivel de Defensa: 8
		Nivel de vida: 11
		Velocidad: 2
		Actitud: Espadachin
		Estado: true
		------------------
		
		Puesto 7
		Nombre: Espadachin4X1
		Vida: 8
		Fila: 5
		Columna: E
		Nivel de ataque: 10
		Nivel de Defensa: 8
		Nivel de vida: 9
		Velocidad: 2
		Actitud: Espadachin
		Estado: true
		------------------
		
		Puesto 8
		Nombre: Arquero1X1
		Vida: 4
		Fila: 8
		Columna: J
		Nivel de ataque: 7
		Nivel de Defensa: 3
		Nivel de vida: 5
		Velocidad: 1
		Actitud: Arquero
		Estado: true
		------------------
		
		Puesto 9
		Nombre: Lancero3X1
		Vida: 2
		Fila: 8
		Columna: I
		Nivel de ataque: 5
		Nivel de Defensa: 10
		Nivel de vida: 8
		Velocidad: 5
		Actitud: Lancero
		Estado: true
		------------------
		*********************************
		
		El Ejercito Moros ordenando por metodo insercion: 
		------------------------------------------
		Mostrando Ranking del Ejercito Moros ..... ////// --->
		
		Puesto 1
		Nombre: Espadachin2X2
		Vida: 10
		Fila: 6
		Columna: F
		Nivel de ataque: 10
		Nivel de Defensa: 8
		Nivel de vida: 10
		Velocidad: 1
		Actitud: Espadachin
		Estado: true
		------------------
		
		Puesto 2
		Nombre: Lancero0X2
		Vida: 7
		Fila: 9
		Columna: G
		Nivel de ataque: 5
		Nivel de Defensa: 10
		Nivel de vida: 7
		Velocidad: 2
		Actitud: Lancero
		Estado: true
		------------------
		
		Puesto 3
		Nombre: Lancero4X2
		Vida: 6
		Fila: 4
		Columna: C
		Nivel de ataque: 5
		Nivel de Defensa: 10
		Nivel de vida: 6
		Velocidad: 2
		Actitud: Lancero
		Estado: true
		------------------
		
		Puesto 4
		Nombre: Lancero1X2
		Vida: 6
		Fila: 6
		Columna: I
		Nivel de ataque: 5
		Nivel de Defensa: 10
		Nivel de vida: 6
		Velocidad: 2
		Actitud: Lancero
		Estado: true
		------------------
		*********************************
		
		*********************************
		El tipo de territorio es: montaña
		
		*********************************
				
	\end{lstlisting}
	\begin{figure}[H]
		\centering
		\includegraphics[width=1.0\textwidth,keepaspectratio]{img/Commit13-1.png}
		%\includesvg{img/automata.svg}
		%\label{img:mot2}
		%\caption{Product backlog.}
	\end{figure}
	\begin{figure}[H]
		\centering
		\includegraphics[width=1.0\textwidth,keepaspectratio]{img/Commit13-2.png}
		%\includesvg{img/automata.svg}
		%\label{img:mot2}
		%\caption{Product backlog.}
	\end{figure}
	\begin{lstlisting}[language=bash,caption={Ejecucion:}][H]
		*********************************
		Ejercito 1 : Castilla
		Cantidad total de soldados creados: 9
		Espadachines: 2
		Arqueros: 1
		Caballeros: 2
		Lanceros: 1
		Espadachin Real: 0
		Caballero Moro: 0
		Caballero Franco: 0
		Espadachin Teutonico: 0
		Espadachin Conquistador: 3
		
		Ejercito 2 : Moros
		Cantidad total de soldados creados: 4
		Espadachines: 1
		Arqueros: 0
		Caballeros: 0
		Lanceros: 3
		Espadachin Real: 0
		Caballero Moro: 0
		Caballero Franco: 0
		Espadachin Teutonico: 0
		Espadachin Conquistador: 0
		
		Ejercito 1: Castilla: 103
		Ejercito 2: Moros: 29
		El ganador es el ejercito 1 de: Castilla. Ya que al generar los porcentajes de probabilidad de victoria basada en los niveles de vida de sus soldados y aplicando un experimento aleatorio salió vencedor. (Aleatorio generado : 78.03)
	\end{lstlisting}
	\begin{figure}[H]
		\centering
		\includegraphics[width=1.0\textwidth,keepaspectratio]{img/Commit12-2.png}
		%\includesvg{img/automata.svg}
		%\label{img:mot2}
		%\caption{Product backlog.}
	\end{figure}
	\begin{figure}[H]
		\centering
		\includegraphics[width=1.0\textwidth,keepaspectratio]{img/Commit12-3.png}
		%\includesvg{img/automata.svg}
		%\label{img:mot2}
		%\caption{Product backlog.}
	\end{figure}
	\begin{figure}[H]
		\centering
		\includegraphics[width=1.0\textwidth,keepaspectratio]{img/Commit12-4.png}
		%\includesvg{img/automata.svg}
		%\label{img:mot2}
		%\caption{Product backlog.}
	\end{figure}
	\subsection{Estructura de laboratorio 22}
	\begin{itemize}	
		\item El contenido que se entrega en este laboratorio22 es el siguiente:
	\end{itemize}
	\begin{lstlisting}[style=ascii-tree]
	\Lab22
	├── Arquero.class
	├── Arquero.java
	├── BoardView.class
	├── BoardView.java
	├── Caballero.class
	├── CaballeroFranco.class
	├── CaballeroFranco.java
	├── Caballero.java
	├── CaballeroMoro.class
	├── CaballeroMoro.java
	├── Ejercito.class
	├── Ejercito.java
	├── Espadachin.class
	├── EspadachinConquistador.class
	├── EspadachinConquistador.java
	├── Espadachin.java
	├── EspadachinReal.class
	├── EspadachinReal.java
	├── EspadachinTeutonico.class
	├── EspadachinTeutonico.java
	├── Juegoprincipal$1.class
	├── Juegoprincipal$2.class
	├── Juegoprincipal.class
	├── Juegoprincipal.java
	├── Lancero.class
	├── Lancero.java
	├── Latex
	│   ├── img
	│   │   ├── Commit10.png
	│   │   ├── Commit11-1.png
	│   │   ├── Commit11-2.png
	│   │   ├── Commit11-3.png
	│   │   ├── Commit12-1.png
	│   │   ├── Commit12-2.png
	│   │   ├── Commit12-3.png
	│   │   ├── Commit12-4.png
	│   │   ├── Commit12-5.png
	│   │   ├── Commit13-1.png
	│   │   ├── Commit13-2.png
	│   │   ├── Commit3.png
	│   │   ├── Commit9.png
	│   │   ├── logo_abet.png
	│   │   ├── logo_episunsa.png
	│   │   └── logo_unsa.jpg
	│   ├── Informe22.aux
	│   ├── Informe22.fdb_latexmk
	│   ├── Informe22.fls
	│   ├── Informe22.log
	│   ├── Informe22.out
	│   ├── Informe22.pdf
	│   ├── Informe22.synctex.gz
	│   └── Informe22.tex
	├── Mapa.class
	├── Mapa.java
	├── Soldado.class
	└── Soldado.java
	
	\end{lstlisting}    
	\section{\textcolor{red}{Rúbricas}}
	
	\subsection{\textcolor{red}{Entregable Informe}}
	\begin{table}[H]
		\caption{Tipo de Informe}
		\setlength{\tabcolsep}{0.5em} % for the horizontal padding
		{\renewcommand{\arraystretch}{1.5}% for the vertical padding
		\begin{tabular}{|p{3cm}|p{12cm}|}
			\hline
			\multicolumn{2}{|c|}{\textbf{\textcolor{red}{Informe}}}  \\
			\hline 
			\textbf{\textcolor{red}{Latex}} & \textcolor{blue}{El informe está en formato PDF desde Latex,  con un formato limpio (buena presentación) y facil de leer.}   \\ 
			\hline 
			
			
		\end{tabular}
	}
	\end{table}
	
	\clearpage
	
	\subsection{\textcolor{red}{Rúbrica para el contenido del Informe y demostración}}
	\begin{itemize}			
		\item El alumno debe marcar o dejar en blanco en celdas de la columna \textbf{Checklist} si cumplio con el ítem correspondiente.
		\item Si un alumno supera la fecha de entrega,  su calificación será sobre la nota mínima aprobada, siempre y cuando cumpla con todos lo items.
		\item El alumno debe autocalificarse en la columna \textbf{Estudiante} de acuerdo a la siguiente tabla:
	
		\begin{table}[ht]
			\caption{Niveles de desempeño}
			\begin{center}
			\begin{tabular}{ccccc}
    			\hline
    			 & \multicolumn{4}{c}{Nivel}\\
    			\cline{1-5}
    			\textbf{Puntos} & Insatisfactorio 25\%& En Proceso 50\% & Satisfactorio 75\% & Sobresaliente 100\%\\
    			\textbf{2.0}&0.5&1.0&1.5&2.0\\
    			\textbf{4.0}&1.0&2.0&3.0&4.0\\
    		\hline
			\end{tabular}
		\end{center}
	\end{table}	
	
	\end{itemize}
	
	\begin{table}[H]
		\caption{Rúbrica para contenido del Informe y demostración}
		\setlength{\tabcolsep}{0.5em} % for the horizontal padding
		{\renewcommand{\arraystretch}{1.5}% for the vertical padding
		%\begin{center}
		\begin{tabular}{|p{2.7cm}|p{7cm}|x{1.3cm}|p{1.2cm}|p{1.5cm}|p{1.1cm}|}
			\hline
    		\multicolumn{2}{|c|}{Contenido y demostración} & Puntos & Checklist & Estudiante & Profesor\\
			\hline
			\textbf{1. GitHub} & Hay enlace URL activo del directorio para el  laboratorio hacia su repositorio GitHub con código fuente terminado y fácil de revisar. &2 &X &2 & \\ 
			\hline
			\textbf{2. Commits} &  Hay capturas de pantalla de los commits más importantes con sus explicaciones detalladas. (El profesor puede preguntar para refrendar calificación). &4 &X &4 & \\ 
			\hline 
			\textbf{3. Código fuente} &  Hay porciones de código fuente importantes con numeración y explicaciones detalladas de sus funciones. &2 &X &2 & \\ 
			\hline 
			\textbf{4. Ejecución} & Se incluyen ejecuciones/pruebas del código fuente  explicadas gradualmente. &2 &X &2 & \\ 
			\hline			
			\textbf{5. Pregunta} & Se responde con completitud a la pregunta formulada en la tarea.  (El profesor puede preguntar para refrendar calificación).  &2 &X &2 & \\ 
			\hline	
			\textbf{6. Fechas} & Las fechas de modificación del código fuente estan dentro de los plazos de fecha de entrega establecidos. &2 &X &2 & \\ 
			\hline 
			\textbf{7. Ortografía} & El documento no muestra errores ortográficos. &2 &X &1.5 & \\ 
			\hline 
			\textbf{8. Madurez} & El Informe muestra de manera general una evolución de la madurez del código fuente,  explicaciones puntuales pero precisas y un acabado impecable.   (El profesor puede preguntar para refrendar calificación).  &4 &X &2 & \\ 
			\hline
			\multicolumn{2}{|c|}{\textbf{Total}} &20 & &17.5 & \\ 
			\hline
		\end{tabular}
		%\end{center}
		%\label{tab:multicol}
		}
	\end{table}
	
\clearpage

\section{Referencias}
\begin{itemize}			
	\item \url{https://drive.google.com/drive/folders/1_yO46UOaxs7uKVK7nrrkcNwybnk_ZJXJ}
\end{itemize}	
	
%\clearpage
%\bibliographystyle{apalike}
%\bibliographystyle{IEEEtranN}
%\bibliography{bibliography}
			
\end{document}