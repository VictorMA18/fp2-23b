%package list
\documentclass{article}
\usepackage[top=3cm, bottom=3cm, outer=3cm, inner=3cm]{geometry}
\usepackage{multicol}
\usepackage{graphicx}
\usepackage{url}
%\usepackage{cite}
\usepackage{hyperref}
\usepackage{array}
%\usepackage{multicol}
\newcolumntype{x}[1]{>{\centering\arraybackslash\hspace{0pt}}p{#1}}
\usepackage{natbib}
\usepackage{pdfpages}
\usepackage{multirow}
\usepackage[normalem]{ulem}
\useunder{\uline}{\ul}{}
\usepackage{svg}
\usepackage{xcolor}
\usepackage{listings}
\lstdefinestyle{ascii-tree}{
    literate={├}{|}1 {─}{--}1 {└}{+}1 
  }
\lstset{basicstyle=\ttfamily,
  showstringspaces=false,
  commentstyle=\color{red},
  keywordstyle=\color{blue}
}
%\usepackage{booktabs}
\usepackage{caption}
\usepackage{subcaption}
\usepackage{float}
\usepackage{array}

\newcolumntype{M}[1]{>{\centering\arraybackslash}m{#1}}
\newcolumntype{N}{@{}m{0pt}@{}}


%%%%%%%%%%%%%%%%%%%%%%%%%%%%%%%%%%%%%%%%%%%%%%%%%%%%%%%%%%%%%%%%%%%%%%%%%%%%
%%%%%%%%%%%%%%%%%%%%%%%%%%%%%%%%%%%%%%%%%%%%%%%%%%%%%%%%%%%%%%%%%%%%%%%%%%%%
\newcommand{\itemEmail}{vmamanian@unsa.edu.pe}
\newcommand{\itemStudent}{Victor Mamani Anahua}
\newcommand{\itemCourse}{Fundamentos de la Programación II}
\newcommand{\itemCourseCode}{20230489}
\newcommand{\itemSemester}{II}
\newcommand{\itemUniversity}{Universidad Nacional de San Agustín de Arequipa}
\newcommand{\itemFaculty}{Facultad de Ingeniería de Producción y Servicios}
\newcommand{\itemDepartment}{Departamento Académico de Ingeniería de Sistemas e Informática}
\newcommand{\itemSchool}{Escuela Profesional de Ingeniería de Sistemas}
\newcommand{\itemAcademic}{2023 - B}
\newcommand{\itemInput}{Del 10 Setiembre 2023}
\newcommand{\itemOutput}{Al 17 Setiembre 2023}
\newcommand{\itemPracticeNumber}{01}
\newcommand{\itemTheme}{Laboratorio 01}
%%%%%%%%%%%%%%%%%%%%%%%%%%%%%%%%%%%%%%%%%%%%%%%%%%%%%%%%%%%%%%%%%%%%%%%%%%%%
%%%%%%%%%%%%%%%%%%%%%%%%%%%%%%%%%%%%%%%%%%%%%%%%%%%%%%%%%%%%%%%%%%%%%%%%%%%%

\usepackage[english,spanish]{babel}
\usepackage[utf8]{inputenc}
\AtBeginDocument{\selectlanguage{spanish}}
\renewcommand{\figurename}{Figura}
\renewcommand{\refname}{Referencias}
\renewcommand{\tablename}{Tabla} %esto no funciona cuando se usa babel
\AtBeginDocument{%
	\renewcommand\tablename{Tabla}
}

\usepackage{fancyhdr}
\pagestyle{fancy}
\fancyhf{}
\setlength{\headheight}{30pt}
\renewcommand{\headrulewidth}{1pt}
\renewcommand{\footrulewidth}{1pt}
\fancyhead[L]{\raisebox{-0.2\height}{\includegraphics[width=3cm]{img/logo_episunsa.png}}}
\fancyhead[C]{\fontsize{7}{7}\selectfont	\itemUniversity \\ \itemFaculty \\ \itemDepartment \\ \itemSchool \\ \textbf{\itemCourse}}
\fancyhead[R]{\raisebox{-0.2\height}{\includegraphics[width=1.2cm]{img/logo_abet}}}
\fancyfoot[L]{Estudiante Victor Mamani A.}
\fancyfoot[C]{\itemCourse}
\fancyfoot[R]{Página \thepage}

% para el codigo fuente
\usepackage{listings}
\usepackage{color, colortbl}
\definecolor{dkgreen}{rgb}{0,0.6,0}
\definecolor{gray}{rgb}{0.5,0.5,0.5}
\definecolor{mauve}{rgb}{0.58,0,0.82}
\definecolor{codebackground}{rgb}{0.95, 0.95, 0.92}
\definecolor{tablebackground}{rgb}{0.8, 0, 0}

\lstset{frame=tb,
	language=bash,
	aboveskip=3mm,
	belowskip=3mm,
	showstringspaces=false,
	columns=flexible,
	basicstyle={\small\ttfamily},
	numbers=none,
	numberstyle=\tiny\color{gray},
	keywordstyle=\color{blue},
	commentstyle=\color{dkgreen},
	stringstyle=\color{mauve},
	breaklines=true,
	breakatwhitespace=true,
	tabsize=3,
	backgroundcolor= \color{codebackground},
}

\begin{document}
	
	\vspace*{10px}
	
	\begin{center}	
		\fontsize{17}{17} \textbf{ Informe de Laboratorio \itemPracticeNumber}
	\end{center}
	\centerline{\textbf{\Large Tema: \itemTheme}}
	%\vspace*{0.5cm}	

	\begin{flushright}
		\begin{tabular}{|M{2.5cm}|N|}
			\hline 
			\rowcolor{tablebackground}
			\color{white} \textbf{Nota}  \\
			\hline 
			     \\[30pt]
			\hline 			
		\end{tabular}
	\end{flushright}	

	\begin{table}[H]
		\begin{tabular}{|x{4.7cm}|x{4.8cm}|x{4.8cm}|}
			\hline 
			\rowcolor{tablebackground}
			\color{white} \textbf{Estudiante} & \color{white}\textbf{Escuela}  & \color{white}\textbf{Asignatura}   \\
			\hline 
			{\itemStudent \par \itemEmail} & \itemSchool & {\itemCourse \par Semestre: \itemSemester \par Código: \itemCourseCode}     \\
			\hline 			
		\end{tabular}
	\end{table}		
	
	\begin{table}[H]
		\begin{tabular}{|x{4.7cm}|x{4.8cm}|x{4.8cm}|}
			\hline 
			\rowcolor{tablebackground}
			\color{white}\textbf{Laboratorio} & \color{white}\textbf{Tema}  & \color{white}\textbf{Duración}   \\
			\hline 
			\itemPracticeNumber & \itemTheme & 04 horas   \\
			\hline 
		\end{tabular}
	\end{table}
	
	\begin{table}[H]
		\begin{tabular}{|x{4.7cm}|x{4.8cm}|x{4.8cm}|}
			\hline 
			\rowcolor{tablebackground}
			\color{white}\textbf{Semestre académico} & \color{white}\textbf{Fecha de inicio}  & \color{white}\textbf{Fecha de entrega}   \\
			\hline 
			\itemAcademic & \itemInput &  \itemOutput  \\
			\hline 
		\end{tabular}
	\end{table}
	
	\section{Tarea}
	\begin{itemize}		
		\item Escribir un programa donde se creen 5 soldados considerando sólo su nombre. Ingresar sus datos y
		después mostrarlos.
		Restricción: se realizará considerando sólo los conocimientos que se tienen de FP1 y sin utilizar arreglos estándar,
		sólo usar variables simples.
		\item Escribir un programa donde se creen 5 soldados considerando su nombre y nivel de vida. Ingresar sus
		datos y después mostrarlos.
		Restricción: se realizará considerando sólo los conocimientos que se tienen de FP1 y sin utilizar arreglos estándar,
		sólo usar variables simples.
		\item Escribir un programa donde se creen 5 soldados considerando sólo su nombre. Ingresar sus datos y
		después mostrarlos.
		Restricción: aplicar arreglos estándar.
		\item Escribir un programa donde se creen 5 soldados considerando su nombre y nivel de vida. Ingresar sus
		datos y después mostrarlos.
		Restricción: aplicar arreglos estándar. (Todavía no aplicar arreglo de objetos).
		\item Escribir un programa donde se creen 2 ejércitos, cada uno con un número aleatorio de soldados entre
		1 y 5, considerando sólo su nombre. Sus datos se inicializan automáticamente con nombres tales como “Soldado0”,
		“Soldado1”, etc. Luego de crear los 2 ejércitos se deben mostrar los datos de todos los soldados de ambos ejércitos
		e indicar qué ejército fue el ganador.
		Restricción: aplicar arreglos estándar y métodos para inicializar los ejércitos, mostrar ejército y mostrar ejército
		ganador. La métrica a aplicar para indicar el ganador es el mayor número de soldados de cada ejército, puede
		haber empates. (Todavía no aplicar arreglo de objetos)
	\end{itemize}
		
	\section{Equipos, materiales y temas utilizados}
	\begin{itemize}
		\item Sistema Operativo Ubuntu GNU Linux 23 lunar 64 bits Kernell 6.2.v
		\item Sistema Operativo Windows 11.
		\item Visual Studio Code.
		\item OpenJDK 64-Bits 19.0.7.
		\item Git 2.39.2.
		\item Cuenta en GitHub con el correo institucional.
		\item Programación Orientada a Objetos.
		\item Actividades del Laboratorio 01	
	\end{itemize}
	
	\section{URL de Repositorio Github}
	\begin{itemize}
		\item URL del Repositorio GitHub para clonar o recuperar.
		\item \url{https://github.com/VictorMA18/fp2-23b.git}
		\item URL para el laboratorio 01 en el Repositorio GitHub.
		\item \url{https://github.com/VictorMA18/fp2-23b/tree/main/Fase01/Lab01}
	\end{itemize}
	
	\section{Actividades del Laboratorio 01}
	
	\subsection{Ejercicio 01}
	\begin{itemize}	
		\item En el primer ejercicio en el comit especificamos y aplicamos un for para el ingreso de datos que seria el nombre de los soldados.
		\item El codigo , la ejecucion y el commit seria el siguiente:
	\end{itemize}	
	\begin{lstlisting}[language=bash,caption={Commit}][H]
		$ git commit -m "Creamos el 1er ejercicio donde creamos 5 soldados y imprimimos sus datos"
	\end{lstlisting}	
	\begin{lstlisting}[language=java,caption={Las lineas de codigos del Ejercicio01 serian}][H]
		//  Laboratorio Nro 1 - Ejercicio1
		//  Autor: Mamani Anahua Victor Narciso
		//  Colaboro:
		//  Tiempo:
		import java.util.*;
		public class Videojuego{
			public static void main(String args[]){
				//1. DECLARAMOS LAS VARIABLES
				Scanner sc = new Scanner(System.in);
				String name;
				//2. LE PEDIMOS A LOS 5 USUARIOS ESCRIBIR SUS NOMBRES
				for(int i = 0; i < 5; i++){
					System.out.println("Ingrese su nombre: ");
					name = sc.nextLine();
					System.out.println("El nombre del soldado " + (i + 1) + " es: " + name);
				}
			}
		}
	\end{lstlisting}
	\begin{lstlisting}[language=bash,caption={La ejecucion dada:}][H]
		Ingrese su nombre: 
		JUAN
		El nombre del soldado 1 es: JUAN
		Ingrese su nombre: 
		BETO
		El nombre del soldado 2 es: BETO
		Ingrese su nombre: 
		TITO
		El nombre del soldado 3 es: TITO
		Ingrese su nombre: 
		HERNAN
		El nombre del soldado 4 es: HERNAN
		Ingrese su nombre: 
		HECTOR
		El nombre del soldado 5 es: HECTOR
	\end{lstlisting}
	\subsection{Ejercicio 02}
	\begin{itemize}	
		\item En el segundo ejercicio en el comit especificamos y aplicamos dos for para el ingreso de datos que seria el nombre de los soldados y vida de los soldados.
		\item El codigo , la ejecucion y el commit seria el siguiente:
	\end{itemize}
	\begin{lstlisting}[language=bash,caption={Commit}][H]
		$ git commit -m "Creamos el 2do ejericio donde creamos 5 soldados y imprimimos sus nombres y su vida y tambien borramos el archivo Sumadig.java ya que no lo necesitamos"
	\end{lstlisting}	
	\begin{lstlisting}[language=java,caption={Las lineas de codigo del Ejercicio02 serian}][H]
		//  Laboratorio Nro 1 - Ejercicio2
		//  Autor: Mamani Anahua Victor Narciso
		//  Colaboro:
		//  Tiempo: 
		import java.util.*;
			public class Videojuego{
				public static void main(String args[]){
					//1. DECLARAMOS LAS VARIABLES
					Scanner sc = new Scanner(System.in);
					String name;
					int health;
					//2. LE PEDIMOS A LOS 5 USUARIOS ESCRIBIR SUS NOMBRES Y SU NIVEL DE VIDA	
					for(int i = 0; i < 5; i++){
						System.out.println("Ingrese su nombre: ");
						name = sc.next();
						System.out.print("El nombre del soldado " + (i + 1) + " es: " + name);
						System.out.println("");
						for(int x = 0; x < 1; x++){
							System.out.println("Ingrese su niveldevida: ");
							health = sc.nextInt();
							System.out.println("El nivel de vida del soldado " + (i + 1) + " es: " + health);
						}
					}
				}
			}
	\end{lstlisting}
	\begin{lstlisting}[language=bash,caption={La ejecucion dada:}][H]
			Ingrese su nombre: 
			JUAN
			El nombre del soldado 1 es: JUAN
			Ingrese su nombre: 
			BETO
			El nombre del soldado 2 es: BETO
			Ingrese su nombre: 
			TITO
			El nombre del soldado 3 es: TITO
			Ingrese su nombre: 
			HERNAN
			El nombre del soldado 4 es: HERNAN
			Ingrese su nombre: 
			HECTOR
			El nombre del soldado 5 es: HECTOR
	\end{lstlisting}
	\subsection{Ejercicio 03}
	\begin{itemize}	
		\item En el tercer ejercicio en el comit especificamos y aplicamos dos for para el ingreso de datos que seria el nombre de los soldados usando arreglos.
		\item El codigo , la ejecucion y el commit seria el siguiente:
	\end{itemize}
	\begin{lstlisting}[language=bash,caption={Commit}][H]
		$ git commit -m "Creamos el 3er ejercicio donde creamos 5 soldados y imprimimos sus nombres usando arreglos"
	\end{lstlisting}
	\begin{lstlisting}[language=java,caption={Las lineas de codigo del Ejercicio03 serian}][H]
		//  Laboratorio Nro 1 - Ejercicio3
		//  Autor: Mamani Anahua Victor Narciso
		//  Colaboro:
		//  Tiempo: 
		import java.util.*;
			public class Videojuego {
				public static void main(String args[]){
					//1. DECLARAMOS NUESTRO ARREGLO Y TAMBIEN LO INSTANCIAMOS
					Scanner sc = new Scanner(System.in);
					String[] names = new String[5];
					//2. LLENAMOS CADA ELEMENTO DEL ARREGLO
					for(int i = 0; i < 5; i++){
						names[i] = sc.nextLine();
					}	
					//3. MOSTRAMOS EL CONTENIDO DE CADA ELEMENTO DEL ARREGLO
					for(int x = 0; x < 5 ; x++){
						System.out.println("El nombre del soldado numero " + (x + 1) + " es: " + names[x]);
					}
				}
			}
	\end{lstlisting}
	\begin{lstlisting}[language=bash,caption={La ejecucuion dada:}][H]
			Ingrese su nombre: 
			JUAN
			El nombre del soldado 1 es: JUAN
			Ingrese su nombre: 
			BETO
			El nombre del soldado 2 es: BETO
			Ingrese su nombre: 
			TITO
			El nombre del soldado 3 es: TITO
			Ingrese su nombre: 
			HERNAN
			El nombre del soldado 4 es: HERNAN
			Ingrese su nombre: 
			HECTOR
			El nombre del soldado 5 es: HECTOR
	\end{lstlisting}
	\subsection{Ejercicio 04}
	\begin{itemize}	
		\item En el cuarto ejercicio en el comit especificamos y aplicamos dos for para el ingreso de datos que seria el nombre de los soldados y la vida de los soldados usando arreglos.
		\item El codigo , la ejecucion y el commit seria el siguiente:
	\end{itemize}
	\begin{lstlisting}[language=bash,caption={Commit}][H]
		$ git commit -m "Creamos el 4to ejercicio donde creamos 5 soldados y imprimimos sus nombres y su vida usando arreglos"
	\end{lstlisting}
	\begin{lstlisting}[language=java,caption={Las lineas de codigo del Ejercicio04 serian}][H]
		//  Laboratorio Nro 1 - Ejercicio4
		//  Autor: Mamani Anahua Victor Narciso
		//  Colaboro:
		//  Tiempo: 
		import java.util.*;
			public class Videojuego {
				public static void main(String args[]){
					Scanner sc = new Scanner(System.in);
					//DECLARAMOS NUESTROS ARREGLOS
					String[] name = new String[5];
					int[] health = new int[5];
					//CREAMOS UN CICLO PARA LA INFORMACION DE LOS 5 SOLDADOS
					for(int i = 0; i < 5; i++){
						System.out.println("Ingresen sus nombres: ");
						name[i] = sc.next();
						System.out.println("Ingresen sus vidas:");
						health[i] = sc.nextInt();
					}
					//CREAMOS UN CICLO PARA DESPUES MOSTRARLOS
					for(int x = 0; x < 5; x++){
						System.out.println("El nombre del soldado " + (x + 1) + " es: " + name[x]);
						System.out.println("La vida del soldado " + (x + 1) + " es: " + health[x]);
					}
				}
			}
	\end{lstlisting}
	\begin{lstlisting}[language=bash,caption={La ejecucion dada:}][H]
		Ingresen sus nombres: 
		victor
		Ingresen sus vidas:
		45
		Ingresen sus nombres: 
		hector
		Ingresen sus vidas:
		34
		Ingresen sus nombres: 
		hernan
		Ingresen sus vidas:
		65
		Ingresen sus nombres: 
		elsa
		Ingresen sus vidas:
		34
		Ingresen sus nombres: 
		nar
		Ingresen sus vidas:
		23
		El nombre del soldado 1 es: victor
		La vida del soldado 1 es: 45
		El nombre del soldado 2 es: hector
		La vida del soldado 2 es: 34
		El nombre del soldado 3 es: hernan
		La vida del soldado 3 es: 65
		El nombre del soldado 4 es: elsa
		La vida del soldado 4 es: 34
		El nombre del soldado 5 es: nar
		La vida del soldado 5 es: 23
	\end{lstlisting}
	\subsection{Ejercicio 05}
	\begin{itemize}	
		\item En el quinto ejercicio en el comit especificamos y aplicamos dos arrays de tipo string para despues usarlos en el codigo debido a su numero de soldados y aplicamos la estructura de control for para sus nombres de los soldados y su impresion y tambien el uso de de un if para mostrar el resultado.
		\item El codigo , la ejecucion y el commit seria el siguiente:
	\end{itemize}
	\begin{lstlisting}[language=bash,caption={Commit}][H]
		$ git commit -m "En el 5toejercicio creamos 2 ejercitos y definimos un ganador o un empate dependiendo de su numero de soldados de cada ejercito"
	\end{lstlisting}
	\begin{lstlisting}[language=java,caption={Las lineas de codigo del Ejercicio04 serian}][H]
		//  Laboratorio Nro 1 - Ejercicio5
		//  Autor: Mamani Anahua Victor Narciso
		//  Colaboro:
		//  Tiempo:   
		import java.util.*;
		public class Videojuego{
			public static void main(String args[]){
				//DECLARAMOS Y INSTACIAMOS NUESTROS ARREGLOS
				Random random = new Random();
				String[] Army1 = new String[random.nextInt(5) + 1];
				String[] Army2 = new String[random.nextInt(5) + 1]; 
				//LLENAMOS CADA ARREGLO CON LO PEDIDO
				for(int x = 0; x < Army1.length; x++){
					Army1[x] = "Soldado" + x;
				}
				for(int y = 0; y < Army2.length; y++){
					Army2[y] = "Soldado" + y;
				}
				//MOSTRAMOS LOS DATOS DE LOS SOLDADOS DE AMBOS EJERCITOS
				System.out.println("LOS SOLDADOS DEL EJERCITO 1 SON: ");
				for(int z = 0; z < Army1.length; z++){
					System.out.println(Army1[z]);
				}
				System.out.println("LOS SOLDADOS DEL EJERCITO 2 SON: ");
				for(int i = 0; i < Army2.length; i++){
					System.out.println(Army2[i]);
				}
				//MOSTRAMOS EL GANADOR DE LA BATALLA
				System.out.println("EL RESULTADO DE LA BATALLA ES :");
				if(Army1.length > Army2.length){
					System.out.println("EL EJERCITO 1 ES EL GANADOR");
				}else{
					if(Army1.length < Army2.length){
						System.out.println("EL EJERCITO 2 ES EL GANADOR");
					}else{
						System.out.println("EMPATE");
					}
				}
			}
		}
	\end{lstlisting}	
	\begin{lstlisting}[language=bash,caption={La ejecucion dada:}][H]
		LOS SOLDADOS DEL EJERCITO 1 SON: 
		Soldado0
		Soldado1
		Soldado2
		Soldado3
		LOS SOLDADOS DEL EJERCITO 2 SON: 
		Soldado0
		Soldado1
		EL RESULTADO DE LA BATALLA ES :
		EL EJERCITO 1 ES EL GANADOR
	\end{lstlisting}

	\subsection{Estructura de laboratorio 01}
	\begin{itemize}	
		\item El contenido que se entrega en este laboratorio es el siguiente:
	\end{itemize}
	
\begin{lstlisting}[style=ascii-tree]
/Lab01
├── Latex
│   ├── img
│   │   ├── logo_abet.png
│   │   ├── logo_episunsa.png
│   │   ├── logo_unsa.jpg
│   │   └── pseudocodigo_insercion.png
│   ├── InformeLab01.aux
│   ├── InformeLab01.fdb_latexmk
│   ├── InformeLab01.fls
│   ├── InformeLab01.log
│   ├── InformeLab01.out
│   ├── InformeLab01.pdf
│   ├── InformeLab01.synctex.gz
│   ├── InformeLab01.tex
│   └── src
│       └── Videojuego01.java
└── Videojuego.java
	
\end{lstlisting}    

\section{Pregunta: antes de simular una batalla entre dos ejércitos, debemos considerar que cada ejército está
compuesto por soldados. Dada su experiencia con videojuegos de estrategia, ¿qué datos de los soldados son
importantes?}
	\begin{itemize}
		\item Salud (Health o HP).
		\item Munición (Ammunition).
		\item Armadura (Armor).
		\item Nivel de experiencia (XP).
		\item Puntuación/Kill Count. 
	\end{itemize}		

	\section{\textcolor{red}{Rúbricas}}
	
	\subsection{\textcolor{red}{Entregable Informe}}
	\begin{table}[H]
		\caption{Tipo de Informe}
		\setlength{\tabcolsep}{0.5em} % for the horizontal padding
		{\renewcommand{\arraystretch}{1.5}% for the vertical padding
		\begin{tabular}{|p{3cm}|p{12cm}|}
			\hline
			\multicolumn{2}{|c|}{\textbf{\textcolor{red}{Informe}}}  \\
			\hline 
			\textbf{\textcolor{red}{Latex}} & \textcolor{blue}{El informe está en formato PDF desde Latex,  con un formato limpio (buena presentación) y facil de leer.}   \\ 
			\hline 
			
			
		\end{tabular}
	}
	\end{table}
	
	\clearpage
	
	\subsection{\textcolor{red}{Rúbrica para el contenido del Informe y demostración}}
	\begin{itemize}			
		\item El alumno debe marcar o dejar en blanco en celdas de la columna \textbf{Checklist} si cumplio con el ítem correspondiente.
		\item Si un alumno supera la fecha de entrega,  su calificación será sobre la nota mínima aprobada, siempre y cuando cumpla con todos lo items.
		\item El alumno debe autocalificarse en la columna \textbf{Estudiante} de acuerdo a la siguiente tabla:
	
		\begin{table}[ht]
			\caption{Niveles de desempeño}
			\begin{center}
			\begin{tabular}{ccccc}
    			\hline
    			 & \multicolumn{4}{c}{Nivel}\\
    			\cline{1-5}
    			\textbf{Puntos} & Insatisfactorio 25\%& En Proceso 50\% & Satisfactorio 75\% & Sobresaliente 100\%\\
    			\textbf{2.0}&0.5&1.0&1.5&2.0\\
    			\textbf{4.0}&1.0&2.0&3.0&4.0\\
    		\hline
			\end{tabular}
		\end{center}
	\end{table}	
	
	\end{itemize}
	
	\begin{table}[H]
		\caption{Rúbrica para contenido del Informe y demostración}
		\setlength{\tabcolsep}{0.5em} % for the horizontal padding
		{\renewcommand{\arraystretch}{1.5}% for the vertical padding
		%\begin{center}
		\begin{tabular}{|p{2.7cm}|p{7cm}|x{1.3cm}|p{1.2cm}|p{1.5cm}|p{1.1cm}|}
			\hline
    		\multicolumn{2}{|c|}{Contenido y demostración} & Puntos & Checklist & Estudiante & Profesor\\
			\hline
			\textbf{1. GitHub} & Hay enlace URL activo del directorio para el  laboratorio hacia su repositorio GitHub con código fuente terminado y fácil de revisar. &2 &X &2 & \\ 
			\hline
			\textbf{2. Commits} &  Hay capturas de pantalla de los commits más importantes con sus explicaciones detalladas. (El profesor puede preguntar para refrendar calificación). &4 &X &1 & \\ 
			\hline 
			\textbf{3. Código fuente} &  Hay porciones de código fuente importantes con numeración y explicaciones detalladas de sus funciones. &2 &X &1 & \\ 
			\hline 
			\textbf{4. Ejecución} & Se incluyen ejecuciones/pruebas del código fuente  explicadas gradualmente. &2 &X &2 & \\ 
			\hline			
			\textbf{5. Pregunta} & Se responde con completitud a la pregunta formulada en la tarea.  (El profesor puede preguntar para refrendar calificación).  &2 &X &2 & \\ 
			\hline	
			\textbf{6. Fechas} & Las fechas de modificación del código fuente estan dentro de los plazos de fecha de entrega establecidos. &2 &X &0.5 & \\ 
			\hline 
			\textbf{7. Ortografía} & El documento no muestra errores ortográficos. &2 &X &2 & \\ 
			\hline 
			\textbf{8. Madurez} & El Informe muestra de manera general una evolución de la madurez del código fuente,  explicaciones puntuales pero precisas y un acabado impecable.   (El profesor puede preguntar para refrendar calificación).  &4 &X &2 & \\ 
			\hline
			\multicolumn{2}{|c|}{\textbf{Total}} &20 & &12.5 & \\ 
			\hline
		\end{tabular}
		%\end{center}
		%\label{tab:multicol}
		}
	\end{table}
	
\clearpage

\section{Referencias}
\begin{itemize}			
	\item \url{https://drive.google.com/file/d/1KAJ3-N5uiKSOJaW0TpOVycdIsxCYEXo3/view}
\end{itemize}	
	
%\clearpage
%\bibliographystyle{apalike}
%\bibliographystyle{IEEEtranN}
%\bibliography{bibliography}
			
\end{document}