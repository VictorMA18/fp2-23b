%package list
\documentclass{article}
\usepackage[top=3cm, bottom=3cm, outer=3cm, inner=3cm]{geometry}
\usepackage{multicol}
\usepackage{graphicx}
\usepackage{url}
%\usepackage{cite}
\usepackage{hyperref}
\usepackage{array}
%\usepackage{multicol}
\newcolumntype{x}[1]{>{\centering\arraybackslash\hspace{0pt}}p{#1}}
\usepackage{natbib}
\usepackage{pdfpages}
\usepackage{multirow}
\usepackage[normalem]{ulem}
\useunder{\uline}{\ul}{}
\usepackage{svg}
\usepackage{xcolor}
\usepackage{listings}
\lstdefinestyle{ascii-tree}{
    literate={├}{|}1 {─}{--}1 {└}{+}1 
  }
\lstset{basicstyle=\ttfamily,
  showstringspaces=false,
  commentstyle=\color{red},
  keywordstyle=\color{blue}
}
%\usepackage{booktabs}
\usepackage{caption}
\usepackage{subcaption}
\usepackage{float}
\usepackage{array}

\newcolumntype{M}[1]{>{\centering\arraybackslash}m{#1}}
\newcolumntype{N}{@{}m{0pt}@{}}


%%%%%%%%%%%%%%%%%%%%%%%%%%%%%%%%%%%%%%%%%%%%%%%%%%%%%%%%%%%%%%%%%%%%%%%%%%%%
%%%%%%%%%%%%%%%%%%%%%%%%%%%%%%%%%%%%%%%%%%%%%%%%%%%%%%%%%%%%%%%%%%%%%%%%%%%%
\newcommand{\itemEmail}{vmamanian@unsa.edu.pe}
\newcommand{\itemStudent}{Victor Mamani Anahua}
\newcommand{\itemCourse}{Fundamentos de la Programación II}
\newcommand{\itemCourseCode}{20230489}
\newcommand{\itemSemester}{II}
\newcommand{\itemUniversity}{Universidad Nacional de San Agustín de Arequipa}
\newcommand{\itemFaculty}{Facultad de Ingeniería de Producción y Servicios}
\newcommand{\itemDepartment}{Departamento Académico de Ingeniería de Sistemas e Informática}
\newcommand{\itemSchool}{Escuela Profesional de Ingeniería de Sistemas}
\newcommand{\itemAcademic}{2023 - B}
\newcommand{\itemInput}{Del 17 Setiembre 2023}
\newcommand{\itemOutput}{Al 24 Setiembre 2023}
\newcommand{\itemPracticeNumber}{04}
\newcommand{\itemTheme}{Laboratorio 04}
%%%%%%%%%%%%%%%%%%%%%%%%%%%%%%%%%%%%%%%%%%%%%%%%%%%%%%%%%%%%%%%%%%%%%%%%%%%%
%%%%%%%%%%%%%%%%%%%%%%%%%%%%%%%%%%%%%%%%%%%%%%%%%%%%%%%%%%%%%%%%%%%%%%%%%%%%

\usepackage[english,spanish]{babel}
\usepackage[utf8]{inputenc}
\AtBeginDocument{\selectlanguage{spanish}}
\renewcommand{\figurename}{Figura}
\renewcommand{\refname}{Referencias}
\renewcommand{\tablename}{Tabla} %esto no funciona cuando se usa babel
\AtBeginDocument{%
	\renewcommand\tablename{Tabla}
}

\usepackage{fancyhdr}
\pagestyle{fancy}
\fancyhf{}
\setlength{\headheight}{30pt}
\renewcommand{\headrulewidth}{1pt}
\renewcommand{\footrulewidth}{1pt}
\fancyhead[L]{\raisebox{-0.2\height}{\includegraphics[width=3cm]{img/logo_episunsa.png}}}
\fancyhead[C]{\fontsize{7}{7}\selectfont	\itemUniversity \\ \itemFaculty \\ \itemDepartment \\ \itemSchool \\ \textbf{\itemCourse}}
\fancyhead[R]{\raisebox{-0.2\height}{\includegraphics[width=1.2cm]{img/logo_abet}}}
\fancyfoot[L]{Estudiante Victor Mamani A.}
\fancyfoot[C]{\itemCourse}
\fancyfoot[R]{Página \thepage}

% para el codigo fuente
\usepackage{listings}
\usepackage{color, colortbl}
\definecolor{dkgreen}{rgb}{0,0.6,0}
\definecolor{gray}{rgb}{0.5,0.5,0.5}
\definecolor{mauve}{rgb}{0.58,0,0.82}
\definecolor{codebackground}{rgb}{0.95, 0.95, 0.92}
\definecolor{tablebackground}{rgb}{0.8, 0, 0}
\definecolor[codebackgroundCode]{RGB}{10, 10 , 20}

\lstdefinestyle{java}{frame=tb,
	language=Java,
	showstringspaces=false,
	columns=flexible,
	basicstyle={\footnotesize\ttfamily\color[RGB]{255,255,255}},
	numberstyle=\color{mygray},
	numbers=left, 
	keywordstyle=\color{myblue},
	morekeywords={String, System},
	commentstyle=\color{mygray},
	stringstyle=\color{mygreen},
	breaklines=true,
	breakatwhitespace=true,
	tabsize=2,
	backgroundcolor= \color{codebackgroundCode},
	showspaces=false,
	showtabs=false,
	showlines=false,
}

\lstset{frame=tb,
	language=bash,
	aboveskip=3mm,
	belowskip=3mm,
	showstringspaces=false,
	columns=flexible,
	basicstyle={\small\ttfamily},
	numbers=none,
	numberstyle=\tiny\color{gray},
	keywordstyle=\color{blue},
	commentstyle=\color{dkgreen},
	stringstyle=\color{mauve},
	breaklines=true,
	breakatwhitespace=true,
	tabsize=3,
	backgroundcolor= \color{codebackground},
}

\begin{document}
	
	\vspace*{10px}
	
	\begin{center}	
		\fontsize{17}{17} \textbf{ Informe de Laboratorio \itemPracticeNumber}
	\end{center}
	\centerline{\textbf{\Large Tema: \itemTheme}}
	%\vspace*{0.5cm}	

	\begin{flushright}
		\begin{tabular}{|M{2.5cm}|N|}
			\hline 
			\rowcolor{tablebackground}
			\color{white} \textbf{Nota}  \\
			\hline 
			     \\[30pt]
			\hline 			
		\end{tabular}
	\end{flushright}	

	\begin{table}[H]
		\begin{tabular}{|x{4.7cm}|x{4.8cm}|x{4.8cm}|}
			\hline 
			\rowcolor{tablebackground}
			\color{white} \textbf{Estudiante} & \color{white}\textbf{Escuela}  & \color{white}\textbf{Asignatura}   \\
			\hline 
			{\itemStudent \par \itemEmail} & \itemSchool & {\itemCourse \par Semestre: \itemSemester \par Código: \itemCourseCode}     \\
			\hline 			
		\end{tabular}
	\end{table}		
	
	\begin{table}[H]
		\begin{tabular}{|x{4.7cm}|x{4.8cm}|x{4.8cm}|}
			\hline 
			\rowcolor{tablebackground}
			\color{white}\textbf{Laboratorio} & \color{white}\textbf{Tema}  & \color{white}\textbf{Duración}   \\
			\hline 
			\itemPracticeNumber & \itemTheme & 04 horas   \\
			\hline 
		\end{tabular}
	\end{table}
	
	\begin{table}[H]
		\begin{tabular}{|x{4.7cm}|x{4.8cm}|x{4.8cm}|}
			\hline 
			\rowcolor{tablebackground}
			\color{white}\textbf{Semestre académico} & \color{white}\textbf{Fecha de inicio}  & \color{white}\textbf{Fecha de entrega}   \\
			\hline 
			\itemAcademic & \itemInput &  \itemOutput  \\
			\hline 
		\end{tabular}
	\end{table}
	
	\section{Tarea}
	\begin{itemize}		
        \item Cree un Proyecto llamado Laboratorio4
		\item Usted podrá reutilizar las dos clases Nave.java y DemoBatalla.java. creadas en Laboratorio
		\item Completar el Código de la clase DemoBatalla
	\end{itemize}

	\section{Equipos, materiales y temas utilizados}
	\begin{itemize}
		\item Sistema Operativo Ubuntu GNU Linux 23 lunar 64 bits Kernell 6.2.v
		\item Visual Studio Code.
		\item OpenJDK 64-Bits 19.0.7.
		\item Git 2.39.2.
		\item Cuenta en GitHub con el correo institucional.
		\item Programación Orientada a Objetos.
		\item Actividades del Laboratorio 04.	
	\end{itemize}
	
	\section{URL de Repositorio Github}
	\begin{itemize}
		\item URL del Repositorio GitHub para clonar o recuperar.
		\item \url{https://github.com/VictorMA18/fp2-23b.git}
		\item URL para el laboratorio 01 en el Repositorio GitHub.
		\item \url{https://github.com/VictorMA18/fp2-23b/tree/main/Fase01/Lab03}
	\end{itemize}
	
	\section{Actividades del Laboratorio 04}
	
	\subsection{Ejercicio DemoBatalla01}
	\begin{itemize}	
		\item En el primer commit agregamos un metodo busquedaLinealNombre() CREADO PARA LA BUSQUEDA DE LA PRIMERA NAVE QUE ES IGUAL AL NOMBRE QUE ESCRIBIMOS ESTA VA PASANDO POR TODAS LAS NAVES Y COMPARANDO SI ALGUNA DE ESTAS TIENE ESE NOMBRE Y SI HAY PONDRA LA PRIMERA CONCURRENCIA. 
		\item El codigo , el commit y la ejecucion seria el siguiente:
	\end{itemize}	
	\begin{lstlisting}[language=bash,caption={Commit}][H]
		$ git commit -m "Completamos el metodo busquedaLinealNombre() donde el nombre ingresado por el usuario buscara la primera igualidad a esta y tambien devolvera su posicion que sera un int"
	\end{lstlisting}	
	\begin{lstlisting}[language=java,caption={Las lineas de codigos del metodo creado:}][H]
		public static int busquedaLinealNombre(Nave[] flota, String s){ //METODO CREADO PARA LA BUSQUEDA DE LA PRIMERA NAVE QUE ES IGUAL AL NOMBRE QUE ESCRIBIMOS
        System.out.println("***********************");
        int count = 0;
        for(int i = 0; i < flota.length; i++){
            if(flota[i].getNombre().equals(s)){
                count++;
                if(count > 0){
                    System.out.println(flota[i].toString());
                    break;
                }else{
                    System.out.println("Nave no encontrada");
                }
                return i;
            }
        }
        return -1;
    }
	\end{lstlisting}
	\begin{lstlisting}[language=bash,caption={La ejecucion dada:}][H]
		Nave 1 : 
		Nombre: Victor
		Fila: 2
		Columna: C
		Estado: true
		Puntos: 45
		Nave 2 : 
		Nombre: Victor
		Fila: 3
		Columna: A
		Estado: true
		Puntos: 653
		Nave 3 : 
		Nombre: Victor
		Fila: 4
		Columna: A
		Estado: true
		Puntos: 9865

		Ingrese el nombre para buscar a la primera nave: 
		Victor
		***********************
		Nombre: Victor
		Columna: C
		Fila: 2
		Estado: true
		Puntos: 45
		--------------------------------------

    \end{lstlisting}
	\subsection{Ejercicio DemoBatalla01}
	\begin{itemize}	
		\item En el segundo commit tambien completamos el metodo ordenarPorPuntosBurbuja() en la cual NOS PERMITE ORDENAR DE MENOR A MAYOR DE LA MANERA DE ORDENAR BURBUJA QUE SERIA CAMBIAR POSCIONES SI TU ELEMENTO DE ADELANTE ES MENOR AL ACTUAL Y ASI INTERCAMBIANDO CON LOS SIGUIENTES ELEMENTOS.
		\item El codigo , el commit y la ejecucion seria el siguiente:
	\end{itemize}	
	\begin{lstlisting}[language=bash,caption={Commit}][H]
		$ git commit -m "COMPLETAMOS ESTE METODO QUE NOS PERMITE ORDENAR DE MENOR A MAYOR DE LA MANERA DE ORDENAR BURBUJA QUE SERIA CAMBIAR POSCIONES SI TU ELEMENTO DE ADELANTE ES MENOR AL ACTUAL"
	\end{lstlisting}	
	\begin{lstlisting}[language=java,caption={Las lineas de codigo del metodo completado:}][H]
		public static void ordenarPorPuntosBurbuja(Nave[] flota){ //COMPLETAMOS ESTE METODO QUE NOS PERMITE ORDENAR DE MENOR A MAYOR DE LA MANERA DE ORDENAR BURBUJA QUE SERIA CAMBIAR POSCIONES SI TU ELEMENTO DE ADELANTE ES MENOR AL ACTUAL
			for(int i = 0; i < flota.length - 1;i++){
				if(flota[i].getPuntos() > flota[i + 1].getPuntos()){
					Nave temp = flota[i];
					flota[i] = flota[i + 1];
					flota[i + 1] = temp;
				}
			}
		}
	\end{lstlisting}
    \begin{lstlisting}[language=bash,caption={La ejecucion dada:}][H]
		Nave 1 : 
		Nombre: Victor 
		Fila: 2
		Columna: A
		Estado: true
		Puntos: 34
		Nave 2 : 
		Nombre: Pepe
		Fila: 3
		Columna: E
		Estado: true
		Puntos: 76
		Nave 3 : 
		Nombre: Lalo
		Fila: 6
		Columna: G
		Estado: true
		Puntos: 874

		Ordenado por la cantidad de puntos del menor al mayor mediante el metodo burbuja: 
		Nombre: Victor
		Columna: A
		Fila: 2
		Estado: true
		Puntos: 34
		***********************
		Nombre: Pepe
		Columna: E
		Fila: 3
		Estado: true
		Puntos: 76
		***********************
		Nombre: Lalo
		Columna: G
		Fila: 6
		Estado: true
		Puntos: 874
		***********************

    \end{lstlisting}
	\subsection{Ejercicio DemoBatalla01}
	\begin{itemize}	
		\item En el tercer commit completamos este metodo mostrarPorNombreBurbuja() donde aplicamos un for para ir por cada elemento comparando si este su primer caracter es mayor a la siguiente este cambiara su poscicion haciendo que se ordene de una manera que sea de la A hasta la Z
		\item El codigo , el commit , la ejecucion seria el siguiente:
	\end{itemize}
	\begin{lstlisting}[language=bash,caption={Commit}][H]
		$ git commit -m "COMPLETAMOS ESTE METODO QUE NOS PERMITE ORDENAR DE A HASTA Z PARA ESTO COMPARAMOS EL PRIMER CARACTER DE CADA UNA Y SI ESTA ACTUAL ES MAYOR A LA SIGUIENTE ESTA CAMBIARA DE POSICION"
	\end{lstlisting}
	\begin{lstlisting}[language=java,caption={Las lineas de codigo del metodo completado:}][H]
		public static void ordenarPorNombreBurbuja(Nave[] flota){ //COMPLETAMOS ESTE METODO QUE NOS PERMITE ORDENAR DE A HASTA Z PARA ESTO COMPARAMOS EL PRIMER CARACTER DE CADA UNA Y SI ESTA ACTUAL ES MAYOR A LA SIGUIENTE ESTA CAMBIARA DE POSICION
			for(int i = 0; i < flota.length - 1;i++){
				if(flota[i].getNombre().charAt(0) > flota[i + 1].getNombre().charAt(0)){
					Nave temp = flota[i];
					flota[i] = flota[i + 1];
					flota[i + 1] = temp;
				}
			}
		}
	\end{lstlisting}
    \begin{lstlisting}[language=bash,caption={La ejecucion dada:}][H]
		Nave 1 : 
		Nombre: Victor 
		Fila: 2
		Columna: A
		Estado: true
		Puntos: 34
		Nave 2 : 
		Nombre: Pepe
		Fila: 3
		Columna: E
		Estado: true
		Puntos: 76
		Nave 3 : 
		Nombre: Lalo
		Fila: 6
		Columna: G
		Estado: true
		Puntos: 874
            
		Ordenado por las iniciales de cada nombre de A a la Z mediante el metodo burbuja: 
		Nombre: Pepe
		Columna: E
		Fila: 3
		Estado: true
		Puntos: 76
		***********************
		Nombre: Lalo
		Columna: G
		Fila: 6
		Estado: true
		Puntos: 874
		***********************
		Nombre: Victor
		Columna: A
		Fila: 2
		Estado: true
		Puntos: 34
		***********************
		--------------------------------------

    \end{lstlisting}
	\subsection{Ejercicio DemoBatalla01}
	\begin{itemize}	
		\item En el cuarto commit completamos el metodo ordenarPorPuntosSeleccion donde usamos un for para pasar por todos los elementos y asi poder cambiar con el indice del proximo menor y en el otro metodo creado nos permite sacar el indice de este proximo menor de puntos usando un for pasando por todos elementos posteriores al actual y asi siguiendo. 
		\item El codigo , el commit y la ejecucion seria el siguiente:
	\end{itemize}
	\begin{lstlisting}[language=bash,caption={Commit}][H]
		$ git commit -m "METODO COMPLETADO QUE NOS PERMITE CAMBIAR LAS POSICIONES DE CADA ARREGLO DEPENDIENDO DE LO QUE RETORNE EL METODO indProxMin que sera el indice cual debemos cambiar con el actual y tambien METODO CREADO QUE NOS AYUDA A BUSCAR EL INDICE DEL OBJETO Y NOS DICE CUAL ES EL PROXIMO MENOR APARTIR DEL QUE ESTAMOS Y VA PASANDO POR TODOS LOS ELEMENTOS ASI QUE VA ACTUALIZANDOSE LA VARIABLE MINDEX"
	\end{lstlisting}
	\begin{lstlisting}[language=java,caption={Las lineas de codigo del metodo completado:}][H]
		public static void ordenarPorPuntosSeleccion(Nave[] flota){ //METODO COMPLETADO QUE NOS PERMITE CAMBIAR LAS POSICIONES DE CADA ARREGLO DEPENDIENDO DE LO QUE RETORNE EL METODO indProxMin que sera el indice cual debemos cambiar con el actual
			for(int i = 0; i < flota.length - 1; i++){
				int j = indProxMin(flota, i);
				Nave temp = flota[i];
				flota[i] = flota[j];
				flota[j] = temp;
			}
		}
		public static int indProxMin(Nave[] flota, int index){ //METODO CREADO QUE NOS AYUDA A BUSCAR EL INDICE DEL OBJETO Y NOS DICE CUAL ES EL PROXIMO MENOR  APARTIR DEL QUE ESTAMOS Y VA PASANDO POR TODOS LOS ELEMENTOS ASI QUE VA ACTUALIZANDOSE LA VARIABLE MINDEX
			int minindex = index;
			for(int i = index + 1; i < flota.length; i++){
				if(flota[i].getPuntos() < flota[minindex].getPuntos()){
					minindex = i;
				}
			}
			return minindex;
		}
	\end{lstlisting}
    \begin{lstlisting}[language=bash,caption={La ejecucion dada:}][H]
		Nave 1 : 
		Nombre: Victor
		Fila: 2
		Columna: D
		Estado: true
		Puntos: 753
		Nave 2 : 
		Nombre: Cristian
		Fila: 4
		Columna: B
		Estado: true
		Puntos: 234
		Nave 3 : 
		Nombre: Lalo
		Fila: 
		2
		Columna: E
		Estado: true
		Puntos: 9866
            
		Ordenado por la cantidad de puntos del menor al mayor mediante el metodo seleccion: 
		Nombre: Cristian
		Columna: B
		Fila: 4
		Estado: true
		Puntos: 234
		***********************
		Nombre: Victor
		Columna: D
		Fila: 2
		Estado: true
		Puntos: 753
		***********************
		Nombre: Lalo
		Columna: E
		Fila: 2
		Estado: true
		Puntos: 9866
		***********************

    \end{lstlisting}
	\subsection{Ejercicio DemoBatalla01}
	\begin{itemize}	
		\item En el quinto commit completamos el metodo ordenarPorNombreSeleccion() donde usamos un for para pasar por todos los elementos y asi poder cambiar con el indice del proximo menor de las inciales de los nombres y en el otro metodo creado nos permite sacar el indice de este proximo menor de los indices de los nombres usando un for pasando por todos elementos posteriores al actual y asi siguiendo. 
		\item El codigo , el commit y la ejecucion seria el siguiente:
	\end{itemize}
	\begin{lstlisting}[language=bash,caption={Commit}][H]
		$ git commit -m "METODO COMPLETADO QUE NOS PERMITE CAMBIAR LAS POSICIONES DE CADA ARREGLO DEPENDIENDO DE LO QUE RETORNE EL METODO indProxMin que sera el indice cual debemos cambiar con el actual y tambien METODO CREADO QUE NOS AYUDA A BUSCAR EL INDICE DEL OBJETO Y NOS DICE CUAL ES EL PROXIMO MENOR EN TERMINOS DE A HASTA Z APARTIR DEL QUE ESTAMOS Y VA PASANDO POR TODOS LOS ELEMENTOS ASI QUE VA ACTUALIZANDOSE LA VARIABLE MINDEX"
	\end{lstlisting}
	\begin{lstlisting}[language=java,caption={Las lineas de codigo del metodo completado:}][H]
		public static void ordenarPorNombreSeleccion(Nave[] flota){ //METODO COMPLETADO QUE NOS PERMITE CAMBIAR LAS POSICIONES DE CADA ARREGLO DEPENDIENDO DE LO QUE RETORNE EL METODO indProxMin que sera el indice cual debemos cambiar con el actual
			for(int i = 0; i < flota.length - 1; i++){
				int j = indProxMin02(flota, i);
				Nave temp = flota[i];
				flota[i] = flota[j];
				flota[j] = temp;
			}
		}
		public static int indProxMin02(Nave[] flota, int index){ //METODO CREADO QUE NOS AYUDA A BUSCAR EL INDICE DEL OBJETO Y NOS DICE CUAL ES EL PROXIMO MENOR EN TERMINOS DE A HASTA Z APARTIR DEL QUE ESTAMOS Y VA PASANDO POR TODOS LOS ELEMENTOS ASI QUE VA ACTUALIZANDOSE LA VARIABLE MINDEX
			int minindex = index;
			for(int i = index + 1; i < flota.length; i++){
				if(flota[i].getNombre().charAt(0) < flota[minindex].getNombre().charAt(0)){
					minindex = i;
				}
			}
			return minindex;
		}
	\end{lstlisting}
    \begin{lstlisting}[language=bash,caption={La ejecucion dada:}][H]
		Nave 1 : 
		Nombre: Victor
		Fila: 2
		Columna: D
		Estado: true
		Puntos: 753
		Nave 2 : 
		Nombre: Cristian
		Fila: 4
		Columna: B
		Estado: true
		Puntos: 234
		Nave 3 : 
		Nombre: Lalo
		Fila: 
		2
		Columna: E
		Estado: true
		Puntos: 9866
            
		Ordenado por las iniciales de cada nombre de A a la Z mediante el metodo seleccion: 
		Nombre: Cristian
		Columna: B
		Fila: 4
		Estado: true
		Puntos: 234
		***********************
		Nombre: Lalo
		Columna: E
		Fila: 2
		Estado: true
		Puntos: 9866
		***********************
		Nombre: Victor
		Columna: D
		Fila: 2
		Estado: true
		Puntos: 753
		***********************

    \end{lstlisting}	
	\subsection{Ejercicio DemoBatalla01}
	\begin{itemize}	
		\item En el sexto commit completamos este metodo ordenarPorPuntosInsercion() onsiste en recorrer todo el array comenzando desde el segundo elemento hasta el final. Para cada elemento, se trata de colocarlo en el lugar correcto entre todos los elementos con menor punto anteriores a él y asi por cada uno completando del mayor al menor
		\item El codigo , el commit y la ejecucion seria el siguiente:
	\end{itemize}
	\begin{lstlisting}[language=bash,caption={Commit}][H]
		$ git commit -m "Este metodo ordenarPorPuntosInsercion() consiste en recorrer todo el array comenzando desde el segundo elemento hasta el final. Para cada elemento, se trata de colocarlo en el lugar correcto entre todos los elementos con menor punto anteriores a el y asi por cada uno completando del mayor al menor"
	\end{lstlisting}
	\begin{lstlisting}[language=java,caption={Las lineas de codigo del metodo creado:}][H]
		public static void ordenarPorPuntosInsercion(Nave[] flota){ //Este metodo ordenarPorPuntosInsercion() consiste en recorrer todo el array comenzando desde el segundo elemento hasta el final. Para cada elemento, se trata de colocarlo en el lugar correcto entre todos los elementos con menor punto anteriores a el y asi por cada uno completando del mayor al menor 
			for(int i = 1; i < flota.length; i++){
				Nave temp = flota[i];
				int j = i - 1;
				while(j >= 0 && (temp.getPuntos() > flota[j].getPuntos())){
					flota[j + 1] = flota[j];
					j--;
				}
				flota[j + 1] = temp;
			}
		}
	\end{lstlisting}
	\begin{lstlisting}[language=bash,caption={La ejecucion dada:}][H]
		Nave 1 : 
		Nombre: Victor
		Fila: 2
		Columna: D
		Estado: true
		Puntos: 753
		Nave 2 : 
		Nombre: Cristian
		Fila: 4
		Columna: B
		Estado: true
		Puntos: 234
		Nave 3 : 
		Nombre: Lalo
		Fila: 
		2
		Columna: E
		Estado: true
		Puntos: 9866
            
		Ordenado por la cantidad de puntos del mayor al menor mediante el metodo insercion: 
		Nombre: Lalo
		Columna: E
		Fila: 2
		Estado: true
		Puntos: 9866
		***********************
		Nombre: Victor
		Columna: D
		Fila: 2
		Estado: true
		Puntos: 753
		***********************
		Nombre: Cristian
		Columna: B
		Fila: 4
		Estado: true
		Puntos: 234
		***********************

    \end{lstlisting}
    \subsection{Ejercicio DemoBatalla01}
	\begin{itemize}	
		\item En el septimo commit el metodo ordenarPorNombreInsercion() consiste en recorrer todo el array comenzando desde el segundo elemento hasta el final. Para cada elemento, se trata de colocarlo en el lugar correcto entre todos los elementos con que sean menores a z anteriores a él y asi por cada uno completando por cada incial de cada nombre de la Z hasta A
		\item El codigo , el commit y La ejecucion completa del codigo seria el siguiente:
	\end{itemize}
	\begin{lstlisting}[language=bash,caption={Commit}][H]
		$ git commit -m "Este metodo ordenarPorNombreInsercion() consiste en recorrer todo el array comenzando desde el segundo elemento hasta el final. Para cada elemento, se trata de colocarlo en el lugar correcto entre todos los elementos con que sean menores a z anteriores a el y asi por cada uno completando por cada incial de cada nombre de la Z hasta A"
	\end{lstlisting}
	\begin{lstlisting}[language=java,caption={Las lineas de codigo del metodo creado:}][H]
		public static void ordenarPorNombreInsercion(Nave[] flota){ //Este metodo ordenarPorNombreInsercion() consiste en recorrer todo el array comenzando desde el segundo elemento hasta el final. Para cada elemento, se trata de colocarlo en el lugar correcto entre todos los elementos con que sean menores a z anteriores a el y asi por cada uno completando por cada incial de cada nombre de la Z hasta A
			for(int i = 1; i < flota.length; i++){
				Nave temp = flota[i];
				int j = i - 1;
				while(j >= 0 && (temp.getNombre().charAt(0) > flota[j].getNombre().charAt(0))){
					flota[j + 1] = flota[j];
					j--;
				}
				flota[j + 1] = temp;
			}  
		}
	\end{lstlisting}
    \begin{lstlisting}[language=bash,caption={La ejecucion del codigo completo:}][H]
		Nave 1 : 
		Nombre: Victor
		Fila: 2
		Columna: D
		Estado: true
		Puntos: 753
		Nave 2 : 
		Nombre: Cristian
		Fila: 4
		Columna: B
		Estado: true
		Puntos: 234
		Nave 3 : 
		Nombre: Lalo
		Fila: 
		2
		Columna: E
		Estado: true
		Puntos: 9866
        
		Ordenado por las iniciales de cada nombre de Z a la A mediante el metodo insercion: 
		Nombre: Victor
		Columna: D
		Fila: 2
		Estado: true
		Puntos: 753
		***********************
		Nombre: Lalo
		Columna: E
		Fila: 2
		Estado: true
		Puntos: 9866
		***********************
		Nombre: Cristian
		Columna: B
		Fila: 4
		Estado: true
		Puntos: 234
		***********************

    \end{lstlisting}
	\subsection{Ejercicio DemoBatalla01}
	\begin{itemize}	
		\item En el octavo commit creamos Estructura de control creado para mostrar el mensaje de (Nave no encontrada) debido a que esta comparando con los demas nombres de las otras naves
		\item El codigo y el commit seria el siguiente:
	\end{itemize}
	\begin{lstlisting}[language=bash,caption={Commit}][H]
		$ git commit -m "Estructura de control creado para mostrar el mensaje de (Nave no encontrada) debido a que esta comparando con los demas nombres de las otras naves"
	\end{lstlisting}
	\begin{lstlisting}[language=java,caption={Las lineas de codigo de lo creado:}][H]
        if(pos == -1){ //Estructura de control creado para mostrar el mensaje de "Nave no encontrada" debido a que esta comparando con los demas nombres de las otras naves
            System.out.println("Nave no encontrada");
        }
	\end{lstlisting}
	\subsection{Ejercicio DemoBatalla01}
	\begin{itemize}	
		\item En el noveno commit  Metodo completado que nos permite buscar la nave con el nombre ingresado este funciona en la que ponemos de limites una izquierda y una derecha que son los limites del arreglo y despues este se va reduciendo su limte dependiendo en donde se encuentre nuestra nave desada capaz si esta atras de nuestra mitad esta va reducir su derecha poniendo a esta como la mitad - 1 y si esta adelante de la mitad esta reducira su izquierda poniendo a esta como la mitad + 1 y asi sucesivamente y va a parar cuando si la derecha sea menor a la izquierda y nos dara como resultado que la nave no fue encontrada y tambien antes de usar este metodo usamos el metodo ordenarPorNombreSeleccion(misNaves) usado para poder ordenar las palabras y asi que esta pueda ser buscada por la busqueda binaria ya para que esta sirva todos los nombres deben estar ordenados de la A hasta la Z y tambien una estructura de control if
		\item El codigo , el commit y la ejecucion seria el siguiente:
	\end{itemize}
	\begin{lstlisting}[language=bash,caption={Commit}][H]
		$ git commit -m " Metodo completado que nos permite buscar la nave con el nombre ingresado este funciona en la que ponemos de limites una izquierda y una derecha que son los limites del arreglo y despues este se va reduciendo su limte dependiendo en donde se encuentre nuestra nave desada capaz si esta atras de nuestra mitad esta va reducir su derecha poniendo a esta como la mitad - 1 y si esta adelante de la mitad esta reducira su izquierda poniendo a esta como la mitad + 1 y asi sucesivamente y va a parar cuando si la derecha sea menor a la izquierda y nos dara como resultado que la nave no fue encontrada y tambien antes de usar este metodo usamos el metodo ordenarPorNombreSeleccion(misNaves) usado para poder ordenar las palabras y asi que esta pueda ser buscada por la busqueda binaria ya para que esta sirva todos los nombres deben estar ordenados de la A hasta la Z y tambien una estructura de control if"
	\end{lstlisting}
	\begin{lstlisting}[language=java,caption={Las lineas de codigo de lo creado:}][H]
		public static int busquedaBinariaNombre(Nave[] flota, String s){
			return 0;
			int left = 0;
			int right = flota.length - 1;
			while(left <= right) {
				int mid = left + (right - left) / 2;
				if(s.equals(flota[mid].getNombre())){
					return mid;
				}else if(s.charAt(0) > flota[mid].getNombre().charAt(0)){
					left = mid + 1;
				}else{
					right = mid - 1;
				}
			}
			return -1;
		}
	\end{lstlisting}
	\begin{lstlisting}[language=java,caption={Las lineas de codigo de lo creado:}][H]
        ordenarPorNombreSeleccion(misNaves);
        pos=busquedaBinariaNombre(misNaves,searchedname01);
        if(pos == -1){ //Estructura de control creado para mostrar el mensaje de "Nave no encontrada" debido a que esta buscando y comparando con los demas nombres de las otras naves
            System.out.println("Nave no encontrada");
        }else{
            System.out.println(misNaves[pos].toString());
        }
        System.out.println("--------------------------------------");
	\end{lstlisting}
    \begin{lstlisting}[language=bash,caption={La ejecucion del codigo completo:}][H]
		Nave 1 : 
		Nombre: Victor
		Fila: 2
		Columna: C
		Estado: true
		Puntos: 7654
		Nave 2 : 
		Nombre: Lalo
		Fila: 2
		Columna: E
		Estado: true
		Puntos: 7443
		Nave 3 : 
		Nombre: Tato
		Fila: 2
		Columna: G
		Estado: true
		Puntos: 9765
		
		Ingrese el nombre para buscar a la nave: 
		Victor
		***********************
		Nombre: Victor
		Columna: C
		Fila: 2
		Estado: true
		Puntos: 7654

    \end{lstlisting}
	\subsection{Ejercicio DemoBatalla01}
	\begin{itemize}	
		\item En el decimo commit  Arreglando algunos errores y añadiendo comentarios a los metodos que faltaban y mostramos el codigo completo y su ejecucion
		\item El codigo y el commit seria el siguiente:
	\end{itemize}
	\begin{lstlisting}[language=bash,caption={Commit}][H]
		$ git commit -m " Arreglando algunos errores y añadiendo comentarios a los metodos que faltaban"
	\end{lstlisting}
	\begin{lstlisting}[language=java,caption={Las lineas de codigo de lo creado:}][H]
		import java.util.*;
		public class DemoBatalla01{
			public static void main(String [] args){
				Nave [] misNaves = new Nave [3]; // LE PONEMOS AL ARREGLO UN TAMAÑO DE 2 PARA SU POSTERIOR PRUEBA 
				Scanner sc = new Scanner(System.in);
				String nomb, col;
				int fil, punt;
				boolean est;
				for (int i = 0; i < misNaves.length; i++) {
					System.out.println("Nave " + (i+1) + " : ");
					System.out.print("Nombre: ");
					nomb = sc.next();
					System.out.print("Fila: ");
					fil = sc.nextInt();
					System.out.print("Columna: ");
					col = sc.next();
					System.out.print("Estado: ");
					est = sc.nextBoolean();
					System.out.print("Puntos: ");
					punt = sc.nextInt();
					misNaves[i] = new Nave(); //Se crea un objeto Nave y se asigna su referencia a misNaves
					misNaves[i].setNombre(nomb);
					misNaves[i].setFila(fil);
					misNaves[i].setColumna(col);
					misNaves[i].setEstado(est);
					misNaves[i].setPuntos(punt);
				}
				System.out.println("\nNaves creadas:");
				System.out.println("--------------------------------------");
				mostrarNaves(misNaves);
				System.out.println("--------------------------------------");
				mostrarPorNombre(misNaves);
				System.out.println("--------------------------------------");        
				mostrarPorPuntos(misNaves);
				System.out.println("--------------------------------------");
				System.out.println("Nave con mayor numero de puntos: \n" + mostrarMayorPuntos(misNaves));
				System.out.println("***********************");
				System.out.println("--------------------------------------");
				//leer un nombre
				//mostrar los datos de la nave con dicho nombre, mensaje de “no encontrado” en caso contrario
				System.out.println("Ingrese el nombre para buscar a la primera nave: ");
				String searchedname = sc.next();
				int pos = busquedaLinealNombre(misNaves, searchedname);
				if(pos == -1){ //Estructura de control creado para mostrar el mensaje de "Nave no encontrada" debido a que esta comparando con los demas nombres de las otras naves
					System.out.println("Nave no encontrada");
				}else{
					System.out.println(misNaves[pos].toString());// En caso de encontrarlo imprimira sus datos de esta nave en caso de no dara mensaja de (Nave no encontrada)
				}
				System.out.println("--------------------------------------");
				System.out.println("Ordenado por la cantidad de puntos del menor al mayor mediante el metodo burbuja: ");
				ordenarPorPuntosBurbuja(misNaves);
				mostrarNaves(misNaves);
				System.out.println("--------------------------------------");
				System.out.println("Ordenado por las iniciales de cada nombre de A a la Z mediante el metodo burbuja: ");
				ordenarPorNombreBurbuja(misNaves);
				mostrarNaves(misNaves);
				System.out.println("--------------------------------------");
				//mostrar los datos de la nave con dicho nombre, mensaje de “no encontrado” en caso contrario
				System.out.println("Ingrese el nombre para buscar a la nave: ");
				String searchedname01 = sc.next();
				System.out.println("***********************");
				ordenarPorNombreSeleccion(misNaves); //Metodo usado para poder ordenar las palabras y asi que esta pueda ser buscada por la busqueda binaria ya para que esta sirva todos los nombres deben estar ordenados de la A hasta la Z 
				pos=busquedaBinariaNombre(misNaves,searchedname01);
				if(pos == -1){ //Estructura de control creado para mostrar el mensaje de "Nave no encontrada" debido a que esta buscando y comparando con los demas nombres de las otras naves
					System.out.println("Nave no encontrada");
				}else{
					System.out.println(misNaves[pos].toString());// En caso de encontrarlo imprimira sus datos de esta nave en caso de no dara mensaja de (Nave no encontrada)
				}
				System.out.println("--------------------------------------");
				System.out.println("Ordenado por la cantidad de puntos del menor al mayor mediante el metodo seleccion: ");
				ordenarPorPuntosSeleccion(misNaves);
				mostrarNaves(misNaves);
				System.out.println("--------------------------------------");
				System.out.println("Ordenado por la cantidad de puntos del mayor al menor mediante el metodo insercion: ");
				ordenarPorPuntosInsercion(misNaves);
				mostrarNaves(misNaves);
				System.out.println("--------------------------------------");
				System.out.println("Ordenado por las iniciales de cada nombre de A a la Z mediante el metodo seleccion: ");
				ordenarPorNombreSeleccion(misNaves);
				mostrarNaves(misNaves);
				System.out.println("--------------------------------------");
				System.out.println("Ordenado por las iniciales de cada nombre de Z a la A mediante el metodo insercion: ");
				ordenarPorNombreInsercion(misNaves);
				mostrarNaves(misNaves);
			}
			//Método para mostrar todas las naves
			public static void mostrarNaves(Nave [] flota){ //COMPLETAMOS EL METODO mostrarNaves Y NOS AYUDAMOS DE UN FOR EACH PARA EL MUESTREO DE LOS DATOS DE CADA OBJETO Y TAMBIEN USAMOS EL METODO toString()
				for(Nave ship: flota){
					System.out.println(ship.toString());
					System.out.println("***********************");
				}
			}
			//Método para mostrar todas las naves de un nombre que se pide por teclado
			public static void mostrarPorNombre(Nave [] flota){ //COMPLETAMOS EL METODO mostrarPorNombre Y NOS AYUDAMOS DE UN FOR EACH CON TAL QUE SI EL NOMBRE INGRESADO ERA IGUAL AL OBJETO CREADO MOSTRABA LOS DATOS DEL OBJETO Y TAMBIEN USAMOS EL METODO toString()
				Scanner sc = new Scanner(System.in);
				System.out.println("Ingrese el nombre para buscar a las naves: ");
				String name = sc.next();
				System.out.println("***********************");
				for(Nave ship: flota){
					if(ship.getNombre().equals(name)){
						System.out.println(ship.toString());
						System.out.println("***********************");
					}
				}
			}
			//Método para mostrar todas las naves con un número de puntos inferior o igual
			//al número de puntos que se pide por teclado
			public static void mostrarPorPuntos(Nave [] flota){ // Completamos el metodo mostrarPorPuntos donde ingresamos un numero de puntos en esta usamos un for each que pase por todos los elementos donde si su numero de puntos es menor o igual a este numero ingresado se imprimira sus datos
				Scanner sc = new Scanner(System.in);
				System.out.println("Ingrese un número puntos para buscar a las naves que son menor o igual a esta: ");
				int point = sc.nextInt();
				System.out.println("***********************");
				for(Nave ship: flota){
					if(ship.getPuntos() <= point){
						System.out.println(ship.toString());
						System.out.println("***********************");
					}
				}
			}
			//Método que devuelve la Nave con mayor número de Puntos
			public static Nave mostrarMayorPuntos(Nave [] flota){ //COMPLETAMOS ESTE METODO mostrarMayorPuntos DONDE CREAMOS UN OBJETO DE LA CLASE NAVE QUE ES SHIP EN ESTE PODREMOS GUARDAR LOS DATOS DE LA NAVE CON LA MAYOR CANTIDAD DE PUNTOS Y DESPUES RETONARNLO
				Nave ship = new Nave();
				for(int i = 0; i < flota.length - 1; i++){
					if(flota[i].getPuntos() < flota[i + 1].getPuntos()){
						ship = flota[i + 1];
					}else{
						ship = flota[i];
					}
				}
				return ship;
			}
			//Método para buscar la primera nave con un nombre que se pidió por teclado
			public static int busquedaLinealNombre(Nave[] flota, String s){ //METODO CREADO PARA LA BUSQUEDA DE LA PRIMERA NAVE QUE ES IGUAL AL NOMBRE QUE ESCRIBIMOS
				System.out.println("***********************");
				for(int i = 0; i < flota.length; i++){
					if(flota[i].getNombre().equals(s)){
						return i;
					}
				}
				return -1;
			}
			//Método que ordena por número de puntos de menor a mayor
			public static void ordenarPorPuntosBurbuja(Nave[] flota){ //COMPLETAMOS ESTE METODO QUE NOS PERMITE ORDENAR DE MENOR A MAYOR DE LA MANERA DE ORDENAR BURBUJA QUE SERIA CAMBIAR POSCIONES SI TU ELEMENTO DE ADELANTE ES MENOR AL ACTUAL
				for(int i = 0; i < flota.length - 1;i++){
					if(flota[i].getPuntos() > flota[i + 1].getPuntos()){
						Nave temp = flota[i];
						flota[i] = flota[i + 1];
						flota[i + 1] = temp;
					}
				}
			}
			//Método que ordena por nombre de A a Z
			public static void ordenarPorNombreBurbuja(Nave[] flota){ //COMPLETAMOS ESTE METODO QUE NOS PERMITE ORDENAR DE A HASTA Z PARA ESTO COMPARAMOS EL PRIMER CARACTER DE CADA UNA Y SI ESTA ACTUAL ES MAYOR A LA SIGUIENTE ESTA CAMBIARA DE POSICION
				for(int i = 0; i < flota.length - 1;i++){
					if(flota[i].getNombre().charAt(0) > flota[i + 1].getNombre().charAt(0)){
						Nave temp = flota[i];
						flota[i] = flota[i + 1];
						flota[i + 1] = temp;
					}
				}
			}
			//Método para buscar la primera nave con un nombre que se pidió por teclado
			public static int busquedaBinariaNombre(Nave[] flota, String s){ // Metodo completado que nos permite buscar la nave con el nombre ingresado  este funciona en la que ponemos de limites una izquierda y una derecha que son los limites del arreglo y despues este se va reduciendo su limte dependiendo en donde se encuentre nuestra nave desada capaz si esta atras de nuestra mitad esta va reducir su derecha poniendo a esta como la mitad - 1 y si esta adelante de la mitad esta reducira su izquierda poniendo a esta como la mitad + 1 y asi sucesivamente y va a parar cuando si la derecha sea menor a la izquierda y nos dara como resultado que la nave no fue encontrada
				int left = 0;
				int right = flota.length - 1;
				while(left <= right) {
					int mid = left + (right - left) / 2;
					if(s.equals(flota[mid].getNombre())){
						return mid;
					}else if(s.charAt(0) > flota[mid].getNombre().charAt(0)){
						left = mid + 1;
					}else{
						right = mid - 1;
					}
				}
				return -1;
			}
			//Método que ordena por número de puntos de menor a mayor
			public static void ordenarPorPuntosSeleccion(Nave[] flota){ //METODO COMPLETADO QUE NOS PERMITE CAMBIAR LAS POSICIONES DE CADA ARREGLO DEPENDIENDO DE LO QUE RETORNE EL METODO indProxMin que sera el indice cual debemos cambiar con el actual
				for(int i = 0; i < flota.length - 1; i++){
					int j = indProxMin(flota, i);
					Nave temp = flota[i];
					flota[i] = flota[j];
					flota[j] = temp;
				}
			}
			public static int indProxMin(Nave[] flota, int index){ //METODO CREADO QUE NOS AYUDA A BUSCAR EL INDICE DEL OBJETO Y NOS DICE CUAL ES EL PROXIMO MENOR  APARTIR DEL QUE ESTAMOS Y VA PASANDO POR TODOS LOS ELEMENTOS ASI QUE VA ACTUALIZANDOSE LA VARIABLE MINDEX
				int minindex = index;
				for(int i = index + 1; i < flota.length; i++){
					if(flota[i].getPuntos() < flota[minindex].getPuntos()){
						minindex = i;
					}
				}
				return minindex;
			}
			//Método que ordena por nombre de A a Z
			public static void ordenarPorNombreSeleccion(Nave[] flota){ //METODO COMPLETADO QUE NOS PERMITE CAMBIAR LAS POSICIONES DE CADA ARREGLO DEPENDIENDO DE LO QUE RETORNE EL METODO indProxMin que sera el indice cual debemos cambiar con el actual
				for(int i = 0; i < flota.length - 1; i++){
					int j = indProxMin02(flota, i);
					Nave temp = flota[i];
					flota[i] = flota[j];
					flota[j] = temp;
				}
			}
			public static int indProxMin02(Nave[] flota, int index){ //METODO CREADO QUE NOS AYUDA A BUSCAR EL INDICE DEL OBJETO Y NOS DICE CUAL ES EL PROXIMO MENOR EN TERMINOS DE A HASTA Z APARTIR DEL QUE ESTAMOS Y VA PASANDO POR TODOS LOS ELEMENTOS ASI QUE VA ACTUALIZANDOSE LA VARIABLE MINDEX
				int minindex = index;
				for(int i = index + 1; i < flota.length; i++){
					if(flota[i].getNombre().charAt(0) < flota[minindex].getNombre().charAt(0)){
						minindex = i;
					}
				}
				return minindex;
			}
			//Método que muestra las naves ordenadas por número de puntos de mayor a menor
			public static void ordenarPorPuntosInsercion(Nave[] flota){ //Este metodo ordenarPorPuntosInsercion() consiste en recorrer todo el array comenzando desde el segundo elemento hasta el final. Para cada elemento, se trata de colocarlo en el lugar correcto entre todos los elementos con menor punto anteriores a él y asi por cada uno completando del mayor al menor 
				for(int i = 1; i < flota.length; i++){
					Nave temp = flota[i];
					int j = i - 1;
					while(j >= 0 && (temp.getPuntos() > flota[j].getPuntos())){
						flota[j + 1] = flota[j];
						j--;
					}
					flota[j + 1] = temp;
				}
			}
			//Método que muestra las naves ordenadas por nombre de Z a A
			public static void ordenarPorNombreInsercion(Nave[] flota){ //Este metodo ordenarPorNombreInsercion() consiste en recorrer todo el array comenzando desde el segundo elemento hasta el final. Para cada elemento, se trata de colocarlo en el lugar correcto entre todos los elementos con que sean menores a z anteriores a él y asi por cada uno completando por cada incial de cada nombre de la Z hasta A
				for(int i = 1; i < flota.length; i++){
					Nave temp = flota[i];
					int j = i - 1;
					while(j >= 0 && (temp.getNombre().charAt(0) > flota[j].getNombre().charAt(0))){
						flota[j + 1] = flota[j];
						j--;
					}
					flota[j + 1] = temp;
				}  
			}
		}
	\end{lstlisting}
    \begin{lstlisting}[language=bash,caption={La ejecucion del codigo completo:}][H]
		Nave 1 : 
		Nombre: Victor
		Fila: 2
		Columna: A
		Estado: true
		Puntos: 
		5674
		Nave 2 : 
		Nombre: Lalo
		Fila: 3
		Columna: E
		Estado: true
		Puntos: 23432
		Nave 3 : 
		Nombre: Pepe 
		Fila: 3
		Columna: F
		Estado: true
		Puntos: 456456

		Naves creadas:
		--------------------------------------
		Nombre: Victor
		Columna: A
		Fila: 2
		Estado: true
		Puntos: 5674
		***********************
		Nombre: Lalo
		Columna: E
		Fila: 3
		Estado: true
		Puntos: 23432
		***********************
		Nombre: Pepe
		Columna: F
		Fila: 3
		Estado: true
		Puntos: 456456
		***********************
		--------------------------------------
		Ingrese el nombre para buscar a las naves: 
		Victor
		***********************
		Nombre: Victor
		Columna: A
		Fila: 2
		Estado: true
		Puntos: 5674
		***********************
		--------------------------------------
		Ingrese un número puntos para buscar a las naves que son menor o igual a esta: 
		756758
		***********************
		Nombre: Victor
		Columna: A
		Fila: 2
		Estado: true
		Puntos: 5674
		***********************
		Nombre: Lalo
		Columna: E
		Fila: 3
		Estado: true
		Puntos: 23432
		***********************
		Nombre: Pepe
		Columna: F
		Fila: 3
		Estado: true
		Puntos: 456456
		***********************
		--------------------------------------
		Nave con mayor numero de puntos: 
		Nombre: Pepe
		Columna: F
		Fila: 3
		Estado: true
		Puntos: 456456
		***********************
		--------------------------------------
		Ingrese el nombre para buscar a la primera nave: 
		Victor
		***********************
		Nombre: Victor
		Columna: A
		Fila: 2
		Estado: true
		Puntos: 5674
		--------------------------------------
		Ordenado por la cantidad de puntos del menor al mayor mediante el metodo burbuja: 
		Nombre: Victor
		Columna: A
		Fila: 2
		Estado: true
		Puntos: 5674
		***********************
		Nombre: Lalo
		Columna: E
		Fila: 3
		Estado: true
		Puntos: 23432
		***********************
		Nombre: Pepe
		Columna: F
		Fila: 3
		Estado: true
		Puntos: 456456
		***********************
		--------------------------------------
		Ordenado por las iniciales de cada nombre de A a la Z mediante el metodo burbuja: 
		Nombre: Lalo
		Columna: E
		Fila: 3
		Estado: true
		Puntos: 23432
		***********************
		Nombre: Pepe
		Columna: F
		Fila: 3
		Estado: true
		Puntos: 456456
		***********************
		Nombre: Victor
		Columna: A
		Fila: 2
		Estado: true
		Puntos: 5674
		***********************
		--------------------------------------
		Ingrese el nombre para buscar a la nave: 
		Lalo
		***********************
		Nombre: Lalo
		Columna: E
		Fila: 3
		Estado: true
		Puntos: 23432
		--------------------------------------
		Ordenado por la cantidad de puntos del menor al mayor mediante el metodo seleccion: 
		Nombre: Victor
		Columna: A
		Fila: 2
		Estado: true
		Puntos: 5674
		***********************
		Nombre: Lalo
		Columna: E
		Fila: 3
		Estado: true
		Puntos: 23432
		***********************
		Nombre: Pepe
		Columna: F
		Fila: 3
		Estado: true
		Puntos: 456456
		***********************
		--------------------------------------
		Ordenado por la cantidad de puntos del mayor al menor mediante el metodo insercion: 
		Nombre: Pepe
		Columna: F
		Fila: 3
		Estado: true
		Puntos: 456456
		***********************
		Nombre: Lalo
		Columna: E
		Fila: 3
		Estado: true
		Puntos: 23432
		***********************
		Nombre: Victor
		Columna: A
		Fila: 2
		Estado: true
		Puntos: 5674
		***********************
		--------------------------------------
		Ordenado por las iniciales de cada nombre de A a la Z mediante el metodo seleccion: 
		Nombre: Lalo
		Columna: E
		Fila: 3
		Estado: true
		Puntos: 23432
		***********************
		Nombre: Pepe
		Columna: F
		Fila: 3
		Estado: true
		Puntos: 456456
		***********************
		Nombre: Victor
		Columna: A
		Fila: 2
		Estado: true
		Puntos: 5674
		***********************
		--------------------------------------
		Ordenado por las iniciales de cada nombre de Z a la A mediante el metodo insercion: 
		Nombre: Victor
		Columna: A
		Fila: 2
		Estado: true
		Puntos: 5674
		***********************
		Nombre: Pepe
		Columna: F
		Fila: 3
		Estado: true
		Puntos: 456456
		***********************
		Nombre: Lalo
		Columna: E
		Fila: 3
		Estado: true
		Puntos: 23432
		***********************
    \end{lstlisting}
	\subsection{Estructura de laboratorio 04}
	\begin{itemize}	
		\item El contenido que se entrega en este laboratorio04 es el siguiente:
	\end{itemize}
	\begin{lstlisting}[style=ascii-tree]
	/Lab04
	├── DemoBatalla01.java
	├── Latex
	│   ├── img
	│   │   ├── logo_abet.png
	│   │   ├── logo_episunsa.png
	│   │   ├── logo_unsa.jpg
	│   │   └── pseudocodigo_insercion.png
	│   ├── Informe04.aux
	│   ├── Informe04.fdb_latexmk
	│   ├── Informe04.fls
	│   ├── Informe04.log
	│   ├── Informe04.out
	│   ├── Informe04.pdf
	│   ├── Informe04.synctex.gz
	│   └── Informe04.tex
	└── Nave.java

	\end{lstlisting}    
	\section{\textcolor{red}{Rúbricas}}
	
	\subsection{\textcolor{red}{Entregable Informe}}
	\begin{table}[H]
		\caption{Tipo de Informe}
		\setlength{\tabcolsep}{0.5em} % for the horizontal padding
		{\renewcommand{\arraystretch}{1.5}% for the vertical padding
		\begin{tabular}{|p{3cm}|p{12cm}|}
			\hline
			\multicolumn{2}{|c|}{\textbf{\textcolor{red}{Informe}}}  \\
			\hline 
			\textbf{\textcolor{red}{Latex}} & \textcolor{blue}{El informe está en formato PDF desde Latex,  con un formato limpio (buena presentación) y facil de leer.}   \\ 
			\hline 
			
			
		\end{tabular}
	}
	\end{table}
	
	\clearpage
	
	\subsection{\textcolor{red}{Rúbrica para el contenido del Informe y demostración}}
	\begin{itemize}			
		\item El alumno debe marcar o dejar en blanco en celdas de la columna \textbf{Checklist} si cumplio con el ítem correspondiente.
		\item Si un alumno supera la fecha de entrega,  su calificación será sobre la nota mínima aprobada, siempre y cuando cumpla con todos lo items.
		\item El alumno debe autocalificarse en la columna \textbf{Estudiante} de acuerdo a la siguiente tabla:
	
		\begin{table}[ht]
			\caption{Niveles de desempeño}
			\begin{center}
			\begin{tabular}{ccccc}
    			\hline
    			 & \multicolumn{4}{c}{Nivel}\\
    			\cline{1-5}
    			\textbf{Puntos} & Insatisfactorio 25\%& En Proceso 50\% & Satisfactorio 75\% & Sobresaliente 100\%\\
    			\textbf{2.0}&0.5&1.0&1.5&2.0\\
    			\textbf{4.0}&1.0&2.0&3.0&4.0\\
    		\hline
			\end{tabular}
		\end{center}
	\end{table}	
	
	\end{itemize}
	
	\begin{table}[H]
		\caption{Rúbrica para contenido del Informe y demostración}
		\setlength{\tabcolsep}{0.5em} % for the horizontal padding
		{\renewcommand{\arraystretch}{1.5}% for the vertical padding
		%\begin{center}
		\begin{tabular}{|p{2.7cm}|p{7cm}|x{1.3cm}|p{1.2cm}|p{1.5cm}|p{1.1cm}|}
			\hline
    		\multicolumn{2}{|c|}{Contenido y demostración} & Puntos & Checklist & Estudiante & Profesor\\
			\hline
			\textbf{1. GitHub} & Hay enlace URL activo del directorio para el  laboratorio hacia su repositorio GitHub con código fuente terminado y fácil de revisar. &2 &X &2 & \\ 
			\hline
			\textbf{2. Commits} &  Hay capturas de pantalla de los commits más importantes con sus explicaciones detalladas. (El profesor puede preguntar para refrendar calificación). &4 &X &1 & \\ 
			\hline 
			\textbf{3. Código fuente} &  Hay porciones de código fuente importantes con numeración y explicaciones detalladas de sus funciones. &2 &X &1 & \\ 
			\hline 
			\textbf{4. Ejecución} & Se incluyen ejecuciones/pruebas del código fuente  explicadas gradualmente. &2 &X &2 & \\ 
			\hline			
			\textbf{5. Pregunta} & Se responde con completitud a la pregunta formulada en la tarea.  (El profesor puede preguntar para refrendar calificación).  &2 &X &2 & \\ 
			\hline	
			\textbf{6. Fechas} & Las fechas de modificación del código fuente estan dentro de los plazos de fecha de entrega establecidos. &2 &X &0.5 & \\ 
			\hline 
			\textbf{7. Ortografía} & El documento no muestra errores ortográficos. &2 &X &2 & \\ 
			\hline 
			\textbf{8. Madurez} & El Informe muestra de manera general una evolución de la madurez del código fuente,  explicaciones puntuales pero precisas y un acabado impecable.   (El profesor puede preguntar para refrendar calificación).  &4 &X &2 & \\ 
			\hline
			\multicolumn{2}{|c|}{\textbf{Total}} &20 & &12.5 & \\ 
			\hline
		\end{tabular}
		%\end{center}
		%\label{tab:multicol}
		}
	\end{table}
	
\clearpage

\section{Referencias}
\begin{itemize}			
	\item \url{https://drive.google.com/file/d/1CoQAKeKW-QDYRmHLrBdbSopFB1Z_Qmk3/view}
\end{itemize}	
	
%\clearpage
%\bibliographystyle{apalike}
%\bibliographystyle{IEEEtranN}
%\bibliography{bibliography}
			
\end{document}