%package list
\documentclass{article}
\usepackage[top=3cm, bottom=3cm, outer=3cm, inner=3cm]{geometry}
\usepackage{multicol}
\usepackage{graphicx}
\usepackage{url}
%\usepackage{cite}
\usepackage{hyperref}
\usepackage{array}
%\usepackage{multicol}
\newcolumntype{x}[1]{>{\centering\arraybackslash\hspace{0pt}}p{#1}}
\usepackage{natbib}
\usepackage{pdfpages}
\usepackage{multirow}
\usepackage[normalem]{ulem}
\useunder{\uline}{\ul}{}
\usepackage{svg}
\usepackage{xcolor}
\usepackage{listings}
\lstdefinestyle{ascii-tree}{
    literate={├}{|}1 {─}{--}1 {└}{+}1 
  }
\lstset{basicstyle=\ttfamily,
  showstringspaces=false,
  commentstyle=\color{red},
  keywordstyle=\color{blue}
}
%\usepackage{booktabs}
\usepackage{caption}
\usepackage{subcaption}
\usepackage{float}
\usepackage{array}

\newcolumntype{M}[1]{>{\centering\arraybackslash}m{#1}}
\newcolumntype{N}{@{}m{0pt}@{}}


%%%%%%%%%%%%%%%%%%%%%%%%%%%%%%%%%%%%%%%%%%%%%%%%%%%%%%%%%%%%%%%%%%%%%%%%%%%%
%%%%%%%%%%%%%%%%%%%%%%%%%%%%%%%%%%%%%%%%%%%%%%%%%%%%%%%%%%%%%%%%%%%%%%%%%%%%
\newcommand{\itemEmail}{vmamanian@unsa.edu.pe}
\newcommand{\itemStudent}{Victor Mamani Anahua}
\newcommand{\itemCourse}{Fundamentos de la Programación II}
\newcommand{\itemCourseCode}{20230489}
\newcommand{\itemSemester}{II}
\newcommand{\itemUniversity}{Universidad Nacional de San Agustín de Arequipa}
\newcommand{\itemFaculty}{Facultad de Ingeniería de Producción y Servicios}
\newcommand{\itemDepartment}{Departamento Académico de Ingeniería de Sistemas e Informática}
\newcommand{\itemSchool}{Escuela Profesional de Ingeniería de Sistemas}
\newcommand{\itemAcademic}{2023 - B}
\newcommand{\itemInput}{Del 25 Octubre 2023}
\newcommand{\itemOutput}{Al 1 Noviembre 2023}
\newcommand{\itemPracticeNumber}{10}
\newcommand{\itemTheme}{Laboratorio 10}
%%%%%%%%%%%%%%%%%%%%%%%%%%%%%%%%%%%%%%%%%%%%%%%%%%%%%%%%%%%%%%%%%%%%%%%%%%%%
%%%%%%%%%%%%%%%%%%%%%%%%%%%%%%%%%%%%%%%%%%%%%%%%%%%%%%%%%%%%%%%%%%%%%%%%%%%%

\usepackage[english,spanish]{babel}
\usepackage[utf8]{inputenc}
\AtBeginDocument{\selectlanguage{spanish}}
\renewcommand{\figurename}{Figura}
\renewcommand{\refname}{Referencias}
\renewcommand{\tablename}{Tabla} %esto no funciona cuando se usa babel
\AtBeginDocument{%
	\renewcommand\tablename{Tabla}
}

\usepackage{fancyhdr}
\pagestyle{fancy}
\fancyhf{}
\setlength{\headheight}{30pt}
\renewcommand{\headrulewidth}{1pt}
\renewcommand{\footrulewidth}{1pt}
\fancyhead[L]{\raisebox{-0.2\height}{\includegraphics[width=3cm]{img/logo_episunsa.png}}}
\fancyhead[C]{\fontsize{7}{7}\selectfont	\itemUniversity \\ \itemFaculty \\ \itemDepartment \\ \itemSchool \\ \textbf{\itemCourse}}
\fancyhead[R]{\raisebox{-0.2\height}{\includegraphics[width=1.2cm]{img/logo_abet}}}
\fancyfoot[L]{Estudiante Victor Mamani A.}
\fancyfoot[C]{\itemCourse}
\fancyfoot[R]{Página \thepage}

% para el codigo fuente
\usepackage{listings}
\usepackage{color, colortbl}
\definecolor{dkgreen}{rgb}{0,0.6,0}
\definecolor{gray}{rgb}{0.5,0.5,0.5}
\definecolor{mauve}{rgb}{0.58,0,0.82}
\definecolor{codebackground}{rgb}{0.95, 0.95, 0.92}
\definecolor{tablebackground}{rgb}{0.8, 0, 0}

\lstdefinestyle{java}{frame=tb,
	language=Java,
	showstringspaces=false,
	columns=flexible,
	basicstyle={\footnotesize\ttfamily\color[RGB]{255,255,255}},
	numberstyle=\color{mygray},
	numbers=left, 
	keywordstyle=\color{myblue},
	morekeywords={String, System},
	commentstyle=\color{mygray},
	stringstyle=\color{mygreen},
	breaklines=true,
	breakatwhitespace=true,
	tabsize=2,
	backgroundcolor= \color{codebackgroundCode},
	showspaces=false,
	showtabs=false,
	showlines=false,
}

\lstset{frame=tb,
	language=bash,
	aboveskip=3mm,
	belowskip=3mm,
	showstringspaces=false,
	columns=flexible,
	basicstyle={\small\ttfamily},
	numbers=none,
	numberstyle=\tiny\color{gray},
	keywordstyle=\color{blue},
	commentstyle=\color{dkgreen},
	stringstyle=\color{mauve},
	breaklines=true,
	breakatwhitespace=true,
	tabsize=3,
	backgroundcolor= \color{codebackground},
}

\begin{document}
	
	\vspace*{10px}
	
	\begin{center}	
		\fontsize{17}{17} \textbf{ Informe de Laboratorio \itemPracticeNumber}
	\end{center}
	\centerline{\textbf{\Large Tema: \itemTheme}}
	%\vspace*{0.5cm}	

	\begin{flushright}
		\begin{tabular}{|M{2.5cm}|N|}
			\hline 
			\rowcolor{tablebackground}
			\color{white} \textbf{Nota}  \\
			\hline 
			     \\[30pt]
			\hline 			
		\end{tabular}
	\end{flushright}	

	\begin{table}[H]
		\begin{tabular}{|x{4.7cm}|x{4.8cm}|x{4.8cm}|}
			\hline 
			\rowcolor{tablebackground}
			\color{white} \textbf{Estudiante} & \color{white}\textbf{Escuela}  & \color{white}\textbf{Asignatura}   \\
			\hline 
			{\itemStudent \par \itemEmail} & \itemSchool & {\itemCourse \par Semestre: \itemSemester \par Código: \itemCourseCode}     \\
			\hline 			
		\end{tabular}
	\end{table}		
	
	\begin{table}[H]
		\begin{tabular}{|x{4.7cm}|x{4.8cm}|x{4.8cm}|}
			\hline 
			\rowcolor{tablebackground}
			\color{white}\textbf{Laboratorio} & \color{white}\textbf{Tema}  & \color{white}\textbf{Duración}   \\
			\hline 
			\itemPracticeNumber & \itemTheme & 04 horas   \\
			\hline 
		\end{tabular}
	\end{table}
	
	\begin{table}[H]
		\begin{tabular}{|x{4.7cm}|x{4.8cm}|x{4.8cm}|}
			\hline 
			\rowcolor{tablebackground}
			\color{white}\textbf{Semestre académico} & \color{white}\textbf{Fecha de inicio}  & \color{white}\textbf{Fecha de entrega}   \\
			\hline 
			\itemAcademic & \itemInput &  \itemOutput  \\
			\hline 
		\end{tabular}
	\end{table}
	
	\section{Tarea}
	\begin{itemize}		
        \item Cree un Proyecto llamado Laboratorio10
        \item Crear 3 constructores sobrecargados.
        \item La actitud puede ser defensiva, ofensiva, fuga. Dicha actitud varía cuando el soldado defiende, ataca o huye respectivamente.
        \item Al atacar el soldado avanza, al avanzar aumenta su velocidad en 1. Al defender el soldado se para. Al huir aumenta su velocidad en 2. Al retroceder, si su velocidad es mayor que 0, entonces primero para y su actitud es defensiva, y si su velocidad es 0 entonces disminuirá a valores negativos. Al ser atacado su vida actual disminuye y puede llegar incluso a morir.
        \item Crear los atributos y métodos extra que considere necesarios.
		\item Usted deberá crear las dos clases Soldado.java y VideoJuego5.java. Puede reutilizar lo desarrollado en Laboratorios anteriores.
		\item Del Soldado nos importa el nombre, puntos de vida, fila y columna (posición en el tablero).
		\item El juego se desarrollará en el mismo tablero de los laboratorios anteriores. Para el tablero utilizar la estructura de datos más adecuada.
		\item Tendrá 2 Ejércitos. Inicializar el tablero con n soldados aleatorios entre 1 y 10 para cada Ejército. Cada soldado tendrá un nombre autogenerado: Soldado0X1, Soldado1X1, etc., un valor de puntos de vida autogenerado aleatoriamente [1..5], la fila y columna también autogenerados aleatoriamente (no puede haber 2 soldados en el mismo cuadrado). 
		\item Además de los datos del Soldado con mayor vida de cada ejército, el promedio de puntos de vida de todos los soldados creados por ejército, los datos de todos los soldados porejército en el orden que fueron creados y un ranking de poder de todos los soldados creados por ejército (del que tiene más nivel de vida al que tiene menos) usando 2 diferentes algoritmos de ordenamiento (indicar conclusiones respecto a este ordenamiento de HashMaps).
		\item Finalmente, que muestre qué ejército ganará la batalla (indicar la métrica usada para decidir al ganador de la batalla).
		\item Crear el diagrama de clases UML.
		\item El juego es humano contra humano y consistirá en mover un soldado por cada turno de cada jugador. Se puede mover en cualquier dirección, Ud. deberá darle la coordenada del soldado a mover y la dirección de movimiento, el programa deberá verificar que hay un soldado del ejército que corresponda en dicha posición y que el movimiento es válido (no puede haber 2 soldados del mismo ejército en el cuadrado y no se puede ordenar moverse a una posición fuera del tablero), pidiendo ingresar nuevos datos si no es así.
		\item Cuando un soldado se mueve a una posición donde hay un soldado rival, se produce una batalla y gana el soldado que tenga mayor nivel de vida actual. Gana el juego quien deje al otro ejército vacío. Después de cada movida se deberá mostrar el tablero con su estado actual. Hacer un programa iterativo.
	\end{itemize}

	\section{Equipos, materiales y temas utilizados}
	\begin{itemize}
		\item Sistema Operativo Ubuntu GNU Linux 23 lunar 64 bits Kernell 6.2.v
		\item Visual Studio Code.
		\item VIM 9.0.
		\item OpenJDK 64-Bits 19.0.7.
		\item Git 2.39.2.
		\item Cuenta en GitHub con el correo institucional.
		\item Programación Orientada a Objetos.
		\item Actividades del Laboratorio 10.	
	\end{itemize}
	
	\section{URL de Repositorio Github}
	\begin{itemize}
		\item URL del Repositorio GitHub para clonar o recuperar.
		\item \url{https://github.com/VictorMA18/fp2-23b.git}
		\item URL para el laboratorio 10 en el Repositorio GitHub.
		\item \url{https://github.com/VictorMA18/fp2-23b/tree/main/Fase02/Lab10}
	\end{itemize}
	
	\section{Actividades del Laboratorio 10}

	\subsection{Ejercicio Soldado}
	\begin{itemize}	
		\item En el primer commit agregando la clase soldado que esta siendo reutilizada para podef reslizar el siguiente ejercicio.
		\item El codigo y el commit seria el siguiente:
	\end{itemize}	
	\begin{lstlisting}[language=bash,caption={Commit}][H]
		$ git commit -m "Agregando la clase soldado para podef reslizar el siguiente ejercicio "
	\end{lstlisting}	
	\begin{lstlisting}[language=java,caption={Las lineas de codigos del metodo creado:}][H]
		// Laboratorio Nro 10 - Ejercicio Soldado
		// Autor: Mamani Anahua Victor Narciso
		// Colaboro:
		// Tiempo:
		import java.util.*;
		public class Soldado { //CREAMOS LA CLASE SOLDODADO PARA PODER USAR UN ARREGLO BIDIMENSIONAL DONDE NECESITAMOS LA VIDA , EL NOMBRE DEL SOLDADO Y TAMBIEN SU POSICION COMO LA FILA Y LA COLUMNA   
		
			private String name;
			private int lifeactual;
			private int row;
			private String column;
			private int attacklevel;
			private int defenselevel;
			private int lifelevel;
			private int speed;
			private String attitude;
			private boolean lives;
		
			Random rdm = new Random();
		
			//Anadiendo metodo que nos permita que un arreglo tenga datos nulos si este esta vacio
			public Soldado(){
				this.name = "";
				this.row = 0;
				this.column  = "";
				this.attacklevel = 0;
				this.defenselevel = 0;
				this.lifelevel = 0
				this.lifeactual = 0;
				this.speed = 0;
				this.attitude = "";
				this.lives = false;
			}
		
			//Constructor
			public Soldado(String name, int health, int row, String column){
				this.name = name;
				this.lifeactual = health;
				this.lifelevel = health;
				this.lifeactual = health;
				this.row = row;
				this.column = column;
				this.lives = true;
				
				//YA QUE ESTOS DATOS SERIAN ALEATORIOS YA QUE SE ESTARIA CREANDO EL SOLDADO TENDRIAMOS DATOS QUE SERIAN COMO ATTACKLEVEL DEFENSELEVEL EL CUAL TENDRIAN QUE SER ALEATORIOS    
				this.attacklevel = rdm.nextInt(5) + 1;
				this.defenselevel = rdm.nextInt(5) + 1;
		
			}
			
			//Constructor para los diferentes niveles como de vidad defensa ataque velocidad
			public Soldado(String name , int attacklevel, int defenselevel, int lifelevel, int speed, boolean lives, int row, String column) {
				this.name = name;
				this.attacklevel = attacklevel;
				this.defenselevel = defenselevel;
				this.lifelevel = lifelevel;
				this.speed = speed;
				this.lives = lives;
				this.row = row;
				this.column = column;
			}
		
			//Metodos necesarios como avanzar defender huir al seratacado al retroceder
			public void advance(){
				this.speed = getSpeed() + 1;
				System.out.println("El soldado " + this.name + "avanzo");
			}
			public void defense(){
				this.speed = 0;
				this.attitude = "DEFENSIVA";
				System.out.println("El soldado " + this.name + "esta defendiendo");
			}
			public void flee(){
				this.speed = getSpeed() + 2;
				this.attitude = "HUYE";
				System.out.println("El soldado " + this.name + "esta huyendo");
			}
			public void back(){
				System.out.println("El soldado " + this.name + "esta retrocediendo");
				if(this.speed == 0){
					this.speed = rdm.nextInt(5) - 5;
				}else{
					if(this.speed > 0){
						this.speed = 0;
						this.attitude = "DEFENSIVA";
					}
				}
			}
			public void attack(Soldado soldier){
				if(this.getLifeActual() > soldier.getLifeActual()){
					int life = this.getLifeActual() - soldier.getLifeActual();
					this.setLifeActual(life);
					this.setLifeLevel(life);
					soldier.lives = false;
					System.out.println(this.name + " asesino al soldado " + soldier.name);
				}else if(soldier.getLifeActual() > this.getLifeActual()){
					int life = soldier.getLifeActual() - this.getLifeActual();
					this.lives = false;
					soldier.setLifeActual(life);
					soldier.setLifeLevel(life);
					System.out.println(soldier.name + " asesino al soldado " + this.name);
				}else{
					this.lives = false;
					soldier.lives = false;
					System.out.println("los 2 soldados se asesinaron");
				}
			}
			public void morir(){
				this.lives = false;
				this.attitude = "SOLDADO MUERTO";
			}
		
			// Metodos mutadores
			public void setName(String n){
				name = n;
			}
			public void setLifeActual(int p){
				lifeactual = p;
			}
			public void setRow(int b){
				row = b;
			}
			public void setColumn(String c){
				column = c; 
			}
			public void setAttackLevel(int attacklevel) {
				this.attacklevel = attacklevel;
			}
			public void setDefenseLevel(int defenselevel) {
				this.defenselevel = defenselevel;
			}
			public void setLifeLevel(int lifelevel){
				this.lifelevel = lifelevel;
			}
			public void setSpeed(int speed) {
				this.speed = speed;
			}
			public void setAttitude(String attitude) {
				this.attitude = attitude;
			}
			public void setLives(boolean lives) {
				this.lives = lives;
			}
		
			// Metodos accesores
			public String getName(){
				return name;
			}
			public int getLifeActual(){
				return lifeactual;
			}
			public int getRow(){
				return row;
			}
			public String getColumn(){
				return column;
			}
			public int getAttackLevel() {
				return attacklevel;
			}
			public int getDefenseLevel() {
				return defenselevel;
			}
			public int getLifeLevel(){
				return lifelevel;
			}
			public int getSpeed() {
				return speed;
			}
			public String getAttitude() {
				return attitude;
			}
			public boolean getLives() {
				return lives;
			}
		
			// Completar con otros metodos necesarios
			public String toString(){ //CREAMOS ESTE METODO PARA IMPRIMIR LOS DATOS DEl OBJETO
				String join = "\nNombre: " + getName() + "\nVida: " + getLifeActual() + "\nFila: " + getRow() + "\nColumna: " + getColumn() + "\nNivel de ataque: " + getAttackLevel() + "\nNivel de Defensa: " + getDefenseLevel() + "\nNivel de vida: " + getLifeLevel() + "\nVelocidad: " + getSpeed() + "\nActitud: " + getAttitude() + "\nEstado: " + getLives(); //Agregamos un espaciador para poder separar
				return join;
			}
			
		}

	\end{lstlisting}
	\subsection{Ejercicio Soldado}
	\begin{itemize}	
		\item En el segundo commit creamos el metodo fillRegister() se utiliza para simular la creación de un ejército de soldados en un tablero bidimensional. Comienza inicializando una matriz bidimensional llamada army que representará el ejército. Luego, se generan aleatoriamente entre 1 y 10 soldados y se los asigna a casillas en el tablero. El código evita que múltiples soldados ocupen la misma casilla. Cada soldado recibe un nombre único, puntos de salud, posición (fila y columna), y velocidad. La información de cada soldado se imprime en la consola, lo que permite rastrear su registro. Al final, la función devuelve la matriz army, que contiene la disposición aleatoria de los soldados en el tablero. Este código podría formar parte de un juego o simulación militar donde se requiere gestionar y rastrear un ejército de soldados.
		\item El codigo , el commit y la ejecucion seria el siguiente:
	\end{itemize}	
	\begin{lstlisting}[language=bash,caption={Commit}][H]
		$ git commit -m "Este metodo fillRegister() se utiliza para simular la creacion de un ejercito de soldados en un tablero bidimensional Comienza inicializando una matriz bidimensional llamada army que representara el ejercito. Luego, se generan aleatoriamente entre 1 y 10 soldados y se los asigna a casillas en el tablero. El codigo evita que multiples soldados ocupen la misma casilla. Cada soldado recibe un nombre unico, puntos de salud, posicion (fila y columna), y velocidad. La informacion de cada soldado se imprime en la consola, lo que permite rastrear su registro. Al final, la funcion devuelve la matriz army, que contiene la disposicion aleatoria de los soldados en el tablero. Este codigo podria formar parte de un juego o simulacion militar donde se requiere gestionar y rastrear un ejercito de soldados."
	\end{lstlisting}	
	\begin{lstlisting}[language=java,caption={Las lineas de codigos del metodo creado:}][H]
		// Laboratorio Nro 10 - Ejercicio Videojuego1
		// Autor: Mamani Anahua Victor Narciso
		// Colaboro:
		// Tiempo:
		import java.util.*;
		class Videojuego1 {
			public static ArrayList<ArrayList<Soldado>> fillRegister(int num){
				Random rdm = new Random();
				ArrayList<ArrayList<Soldado>> army = new ArrayList<ArrayList<Soldado>>();
				int numbersoldiers = rdm.nextInt(10) + 1; //NUMERO DE SOLDADOS ALEATORIOS ENTRE 1 A 10 SOLDADOS 
				for(int i = 0; i < 10; i++){ //ITERACION
					army.add(new ArrayList<Soldado>()); //LLENAMOS NUESTROS ARRAYLIST BIDIMENSIONAL CON CADA FILA PARA QUE CUMPLAN CON ESTRUCTURA DEL TABLERO
					for(int j = 0; j < 10 ; j++){//ITERACION
						army.get(i).add(null); // LLENAMOS CADA FILA DEL ARRAYLIST CON UN OBJETO SOLDADO CON TAL QUE ESTE SEA NULL PARA QUE SEPA QUE ESTE TIENE UNA CASILLA PERO NO HAY NADIE TODAVIA SE PUEDE LLENAR 
					}
				}
				System.out.println("El Ejercito " + num + " tiene " + numbersoldiers + " soldados : " ); 
				System.out.println("");
				for(int i = 0; i < numbersoldiers; i++){ //LLENAMOS CASILLAS CON CADA SOLDADO CREADO ALEATORIAMENTE
					String name = "Soldado" + i + "X" + num;
					//System.out.println(name); PRUEBA QUE SE HIZO PARA VER LOS NOMBRES
					int health = rdm.nextInt(5) + 1;
					int row = rdm.nextInt(10) + 1;
					int speed = rdm.nextInt(5) + 1;
					String column = String.valueOf((char)(rdm.nextInt(10) + 65)); //REUTILIZAMOS CODIGO DEL ANTERIOR ARCHIVO VIDEOJUEGO2.JAVA YA QUE TENDRIAN LA MISMA FUNCIONALIDAD
					//System.out.println(army.get(row - 1).get((int)column.charAt(0) - 65)); PRUEBA QUE SE HIZO PARA COMPROBAR SI EL OBJETO SE ESTABA DANDO O NO CAPAZ NI EXISTIA  
					if(army.get(row - 1).get((int)column.charAt(0) - 65) == null){
						System.out.println("Registrando al " + (i + 1) + " soldado del Ejercito " + num + "");
						army.get(row - 1).set((int)column.charAt(0) - 65, new Soldado(name, health, row, column));
						army.get(row - 1).get((int)column.charAt(0) - 65).setSpeed(speed);
						System.out.println(army.get(row - 1).get((int)column.charAt(0) - 65).toString());
						System.out.println("---------------------------------");
					}else{
						i -= 1; //NOS AYUDARIA CON LOS SOLDADOS QUE SE REPITEN EN EL MISMO CASILLERO CON TAL QUE NO DEBERIA CONTAR 
					}
				}
				System.out.println("*********************************");
				return army;
			}   
			public static void main(String[] args) {
				ArrayList<ArrayList<Soldado>> army1 = fillRegister(1);
				ArrayList<ArrayList<Soldado>> army2 = fillRegister(2);
		
			}
		}
	\end{lstlisting}

	\subsection{Estructura de laboratorio 10}
	\begin{itemize}	
		\item El contenido que se entrega en este laboratorio10 es el siguiente:
	\end{itemize}
	\begin{lstlisting}[style=ascii-tree]
	/Lab10	
		"PONER RAMA"

	\end{lstlisting}    
	\section{\textcolor{red}{Rúbricas}}
	
	\subsection{\textcolor{red}{Entregable Informe}}
	\begin{table}[H]
		\caption{Tipo de Informe}
		\setlength{\tabcolsep}{0.5em} % for the horizontal padding
		{\renewcommand{\arraystretch}{1.5}% for the vertical padding
		\begin{tabular}{|p{3cm}|p{12cm}|}
			\hline
			\multicolumn{2}{|c|}{\textbf{\textcolor{red}{Informe}}}  \\
			\hline 
			\textbf{\textcolor{red}{Latex}} & \textcolor{blue}{El informe está en formato PDF desde Latex,  con un formato limpio (buena presentación) y facil de leer.}   \\ 
			\hline 
			
			
		\end{tabular}
	}
	\end{table}
	
	\clearpage
	
	\subsection{\textcolor{red}{Rúbrica para el contenido del Informe y demostración}}
	\begin{itemize}			
		\item El alumno debe marcar o dejar en blanco en celdas de la columna \textbf{Checklist} si cumplio con el ítem correspondiente.
		\item Si un alumno supera la fecha de entrega,  su calificación será sobre la nota mínima aprobada, siempre y cuando cumpla con todos lo items.
		\item El alumno debe autocalificarse en la columna \textbf{Estudiante} de acuerdo a la siguiente tabla:
	
		\begin{table}[ht]
			\caption{Niveles de desempeño}
			\begin{center}
			\begin{tabular}{ccccc}
    			\hline
    			 & \multicolumn{4}{c}{Nivel}\\
    			\cline{1-5}
    			\textbf{Puntos} & Insatisfactorio 25\%& En Proceso 50\% & Satisfactorio 75\% & Sobresaliente 100\%\\
    			\textbf{2.0}&0.5&1.0&1.5&2.0\\
    			\textbf{4.0}&1.0&2.0&3.0&4.0\\
    		\hline
			\end{tabular}
		\end{center}
	\end{table}	
	
	\end{itemize}
	
	\begin{table}[H]
		\caption{Rúbrica para contenido del Informe y demostración}
		\setlength{\tabcolsep}{0.5em} % for the horizontal padding
		{\renewcommand{\arraystretch}{1.5}% for the vertical padding
		%\begin{center}
		\begin{tabular}{|p{2.7cm}|p{7cm}|x{1.3cm}|p{1.2cm}|p{1.5cm}|p{1.1cm}|}
			\hline
    		\multicolumn{2}{|c|}{Contenido y demostración} & Puntos & Checklist & Estudiante & Profesor\\
			\hline
			\textbf{1. GitHub} & Hay enlace URL activo del directorio para el  laboratorio hacia su repositorio GitHub con código fuente terminado y fácil de revisar. &2 &X &2 & \\ 
			\hline
			\textbf{2. Commits} &  Hay capturas de pantalla de los commits más importantes con sus explicaciones detalladas. (El profesor puede preguntar para refrendar calificación). &4 &X &4 & \\ 
			\hline 
			\textbf{3. Código fuente} &  Hay porciones de código fuente importantes con numeración y explicaciones detalladas de sus funciones. &2 &X &2 & \\ 
			\hline 
			\textbf{4. Ejecución} & Se incluyen ejecuciones/pruebas del código fuente  explicadas gradualmente. &2 &X &2 & \\ 
			\hline			
			\textbf{5. Pregunta} & Se responde con completitud a la pregunta formulada en la tarea.  (El profesor puede preguntar para refrendar calificación).  &2 &X &2 & \\ 
			\hline	
			\textbf{6. Fechas} & Las fechas de modificación del código fuente estan dentro de los plazos de fecha de entrega establecidos. &2 &X &2 & \\ 
			\hline 
			\textbf{7. Ortografía} & El documento no muestra errores ortográficos. &2 &X &2 & \\ 
			\hline 
			\textbf{8. Madurez} & El Informe muestra de manera general una evolución de la madurez del código fuente,  explicaciones puntuales pero precisas y un acabado impecable.   (El profesor puede preguntar para refrendar calificación).  &4 &X &2 & \\ 
			\hline
			\multicolumn{2}{|c|}{\textbf{Total}} &20 & &18 & \\ 
			\hline
		\end{tabular}
		%\end{center}
		%\label{tab:multicol}
		}
	\end{table}
	
\clearpage

\section{Referencias}
\begin{itemize}			
	\item \url{https://drive.google.com/drive/u/1/folders/19TzLFO-T77qG7bOWmg5OH7FXAMD2CrJL}
\end{itemize}	
	
%\clearpage
%\bibliographystyle{apalike}
%\bibliographystyle{IEEEtranN}
%\bibliography{bibliography}
			
\end{document}