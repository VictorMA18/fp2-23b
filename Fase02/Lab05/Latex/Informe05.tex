%package list
\documentclass{article}
\usepackage[top=3cm, bottom=3cm, outer=3cm, inner=3cm]{geometry}
\usepackage{multicol}
\usepackage{graphicx}
\usepackage{url}
%\usepackage{cite}
\usepackage{hyperref}
\usepackage{array}
%\usepackage{multicol}
\newcolumntype{x}[1]{>{\centering\arraybackslash\hspace{0pt}}p{#1}}
\usepackage{natbib}
\usepackage{pdfpages}
\usepackage{multirow}
\usepackage[normalem]{ulem}
\useunder{\uline}{\ul}{}
\usepackage{svg}
\usepackage{xcolor}
\usepackage{listings}
\lstdefinestyle{ascii-tree}{
    literate={├}{|}1 {─}{--}1 {└}{+}1 
  }
\lstset{basicstyle=\ttfamily,
  showstringspaces=false,
  commentstyle=\color{red},
  keywordstyle=\color{blue}
}
%\usepackage{booktabs}
\usepackage{caption}
\usepackage{subcaption}
\usepackage{float}
\usepackage{array}

\newcolumntype{M}[1]{>{\centering\arraybackslash}m{#1}}
\newcolumntype{N}{@{}m{0pt}@{}}


%%%%%%%%%%%%%%%%%%%%%%%%%%%%%%%%%%%%%%%%%%%%%%%%%%%%%%%%%%%%%%%%%%%%%%%%%%%%
%%%%%%%%%%%%%%%%%%%%%%%%%%%%%%%%%%%%%%%%%%%%%%%%%%%%%%%%%%%%%%%%%%%%%%%%%%%%
\newcommand{\itemEmail}{vmamanian@unsa.edu.pe}
\newcommand{\itemStudent}{Victor Mamani Anahua}
\newcommand{\itemCourse}{Fundamentos de la Programación II}
\newcommand{\itemCourseCode}{20230489}
\newcommand{\itemSemester}{II}
\newcommand{\itemUniversity}{Universidad Nacional de San Agustín de Arequipa}
\newcommand{\itemFaculty}{Facultad de Ingeniería de Producción y Servicios}
\newcommand{\itemDepartment}{Departamento Académico de Ingeniería de Sistemas e Informática}
\newcommand{\itemSchool}{Escuela Profesional de Ingeniería de Sistemas}
\newcommand{\itemAcademic}{2023 - B}
\newcommand{\itemInput}{Del 09 Octubre 2023}
\newcommand{\itemOutput}{Al 16 Octubre 2023}
\newcommand{\itemPracticeNumber}{05}
\newcommand{\itemTheme}{Laboratorio 05}
%%%%%%%%%%%%%%%%%%%%%%%%%%%%%%%%%%%%%%%%%%%%%%%%%%%%%%%%%%%%%%%%%%%%%%%%%%%%
%%%%%%%%%%%%%%%%%%%%%%%%%%%%%%%%%%%%%%%%%%%%%%%%%%%%%%%%%%%%%%%%%%%%%%%%%%%%

\usepackage[english,spanish]{babel}
\usepackage[utf8]{inputenc}
\AtBeginDocument{\selectlanguage{spanish}}
\renewcommand{\figurename}{Figura}
\renewcommand{\refname}{Referencias}
\renewcommand{\tablename}{Tabla} %esto no funciona cuando se usa babel
\AtBeginDocument{%
	\renewcommand\tablename{Tabla}
}

\usepackage{fancyhdr}
\pagestyle{fancy}
\fancyhf{}
\setlength{\headheight}{30pt}
\renewcommand{\headrulewidth}{1pt}
\renewcommand{\footrulewidth}{1pt}
\fancyhead[L]{\raisebox{-0.2\height}{\includegraphics[width=3cm]{img/logo_episunsa.png}}}
\fancyhead[C]{\fontsize{7}{7}\selectfont	\itemUniversity \\ \itemFaculty \\ \itemDepartment \\ \itemSchool \\ \textbf{\itemCourse}}
\fancyhead[R]{\raisebox{-0.2\height}{\includegraphics[width=1.2cm]{img/logo_abet}}}
\fancyfoot[L]{Estudiante Victor Mamani A.}
\fancyfoot[C]{\itemCourse}
\fancyfoot[R]{Página \thepage}

% para el codigo fuente
\usepackage{listings}
\usepackage{color, colortbl}
\definecolor{dkgreen}{rgb}{0,0.6,0}
\definecolor{gray}{rgb}{0.5,0.5,0.5}
\definecolor{mauve}{rgb}{0.58,0,0.82}
\definecolor{codebackground}{rgb}{0.95, 0.95, 0.92}
\definecolor{tablebackground}{rgb}{0.8, 0, 0}

\lstdefinestyle{java}{frame=tb,
	language=Java,
	showstringspaces=false,
	columns=flexible,
	basicstyle={\footnotesize\ttfamily\color[RGB]{255,255,255}},
	numberstyle=\color{mygray},
	numbers=left, 
	keywordstyle=\color{myblue},
	morekeywords={String, System},
	commentstyle=\color{mygray},
	stringstyle=\color{mygreen},
	breaklines=true,
	breakatwhitespace=true,
	tabsize=2,
	backgroundcolor= \color{codebackgroundCode},
	showspaces=false,
	showtabs=false,
	showlines=false,
}

\lstset{frame=tb,
	language=bash,
	aboveskip=3mm,
	belowskip=3mm,
	showstringspaces=false,
	columns=flexible,
	basicstyle={\small\ttfamily},
	numbers=none,
	numberstyle=\tiny\color{gray},
	keywordstyle=\color{blue},
	commentstyle=\color{dkgreen},
	stringstyle=\color{mauve},
	breaklines=true,
	breakatwhitespace=true,
	tabsize=3,
	backgroundcolor= \color{codebackground},
}

\begin{document}
	
	\vspace*{10px}
	
	\begin{center}	
		\fontsize{17}{17} \textbf{ Informe de Laboratorio \itemPracticeNumber}
	\end{center}
	\centerline{\textbf{\Large Tema: \itemTheme}}
	%\vspace*{0.5cm}	

	\begin{flushright}
		\begin{tabular}{|M{2.5cm}|N|}
			\hline 
			\rowcolor{tablebackground}
			\color{white} \textbf{Nota}  \\
			\hline 
			     \\[30pt]
			\hline 			
		\end{tabular}
	\end{flushright}	

	\begin{table}[H]
		\begin{tabular}{|x{4.7cm}|x{4.8cm}|x{4.8cm}|}
			\hline 
			\rowcolor{tablebackground}
			\color{white} \textbf{Estudiante} & \color{white}\textbf{Escuela}  & \color{white}\textbf{Asignatura}   \\
			\hline 
			{\itemStudent \par \itemEmail} & \itemSchool & {\itemCourse \par Semestre: \itemSemester \par Código: \itemCourseCode}     \\
			\hline 			
		\end{tabular}
	\end{table}		
	
	\begin{table}[H]
		\begin{tabular}{|x{4.7cm}|x{4.8cm}|x{4.8cm}|}
			\hline 
			\rowcolor{tablebackground}
			\color{white}\textbf{Laboratorio} & \color{white}\textbf{Tema}  & \color{white}\textbf{Duración}   \\
			\hline 
			\itemPracticeNumber & \itemTheme & 04 horas   \\
			\hline 
		\end{tabular}
	\end{table}
	
	\begin{table}[H]
		\begin{tabular}{|x{4.7cm}|x{4.8cm}|x{4.8cm}|}
			\hline 
			\rowcolor{tablebackground}
			\color{white}\textbf{Semestre académico} & \color{white}\textbf{Fecha de inicio}  & \color{white}\textbf{Fecha de entrega}   \\
			\hline 
			\itemAcademic & \itemInput &  \itemOutput  \\
			\hline 
		\end{tabular}
	\end{table}
	
	\section{Tarea}
	\begin{itemize}		
        \item Cree un Proyecto llamado Laboratorio5
		\item Usted deberá crear las dos clases Soldado.java y VideoJuego2.java. Puede reutilizar lo desarrollado en Laboratorio 3 y 4.
		\item Del Soldado nos importa el nombre, puntos de vida, fila y columna (posición en el tablero).
		\item El juego se desarrollará en el mismo tablero de los laboratorios anteriores. Pero ahora el tablero debe ser un arreglo bidimensional de objetos.
		\item Inicializar el tablero con n soldados aleatorios entre 1 y 10. Cada soldado tendrá un nombre autogenerado: Soldado0, Soldado1, etc., un valor de puntos de vida autogeneradoaleatoriamente [1..5], la fila y columna también autogenerados aleatoriamente (no puede haber 2 soldados en el mismo cuadrado). 
		\item Se debe mostrar el tablero con todos los soldados creados (usar caracteres y () otros).
		\item Además de los datos del Soldado con mayor vida el promedio de puntos de vida de todos los soldados creados, el nivel de vida de todo el ejército, los datos de todos los soldados en el orden que fueron creados y un ranking de poder de todos los soldados creados, del que tiene más nivel de vida al que tiene menos (usar al menos 2 algoritmos de ordenamiento).
	\end{itemize}

	\section{Equipos, materiales y temas utilizados}
	\begin{itemize}
		\item Sistema Operativo Ubuntu GNU Linux 23 lunar 64 bits Kernell 6.2.v
		\item VIM 9.0.
		\item OpenJDK 64-Bits 19.0.7.
		\item Git 2.39.2.
		\item Cuenta en GitHub con el correo institucional.
		\item Programación Orientada a Objetos.
		\item Actividades del Laboratorio 05.	
	\end{itemize}
	
	\section{URL de Repositorio Github}
	\begin{itemize}
		\item URL del Repositorio GitHub para clonar o recuperar.
		\item \url{https://github.com/VictorMA18/fp2-23b.git}
		\item URL para el laboratorio 01 en el Repositorio GitHub.
		\item \url{https://github.com/VictorMA18/fp2-23b/tree/main/Fase02/Lab05}
	\end{itemize}
	
	\section{Actividades del Laboratorio 05}
	
	\subsection{Ejercicio Soldado}
	\begin{itemize}	
		\item En el primer commit explicamos el uso que le vamos a dar a esta clase para el siguiente ejercicio para esto necesitamos mas atributos como row y column para su posterior ubicacion en el tablero
		\item El codigo y el commit seria el siguiente:
	\end{itemize}	
	\begin{lstlisting}[language=bash,caption={Commit}][H]
		$ git commit -m "Agregando la clase Soldado la cual reutilizamos donde agregamos un metodo constructor y tambien anadimos los atributos row y column con sus respectivos getters y setters"
	\end{lstlisting}	
	\begin{lstlisting}[language=java,caption={Las lineas de codigos del metodo creado:}][H]
		public class Soldado { //CREAMOS LA CLASE SOLDODADO PARA PODER USAR UN ARREGLO BIDIMENSIONAL DONDE NECESITAMOS LA VIDA , EL NOMBRE DEL SOLDADO Y TAMBIEN SU POSICION COMO LA FILA Y LA COLUMNA   

			private String name;
			private int heatlh; 
			private int row;
			private String column;

			//Constructor
			public Soldado(String name, int health, int row, String column){
			this.name = name;
			this.health = health;
			this.row = row;
			this.column = column;
			}

			// Metodos mutadores
			public void setName(String n){
				name = n;
			}
			public void setHealth(int p){
				heatlh = p;
			}
			public void setRow(int b){
				row = b;
			}
			public void setColumn(String c){
				column = c; 
			}

			// Metodos accesores
			public String getName(){
				return name;
			}
			public int getHealth(){
				return heatlh;

			}
			public int getRow(){
				return row;
			}
			public String getColumn(){
				return column;
			}

			// Completar con otros metodos necesarios
			public String toString(){ //CREAMOS ESTE METODO PARA IMPRIMIR LOS DATOS DEl OBJETO
				String join = "\nNombre: " + getName() + "\nVida: " + getHealth() + "\nFila: " + getRow() + "\nColumna: " + getColumn(); //Agregamos un espaciador para poder separar
				return join;
			}
		}
	\end{lstlisting}
	\subsection{Ejercicio Soldado}
	\begin{itemize}	
		\item En el segundo commit ponemos el nombre adecuado para el atributo salud el cual seria health y cambiamos todas sus concurrencias 
		\item El codigo y el commit seria el siguiente:
	\end{itemize}	
	\begin{lstlisting}[language=bash,caption={Commit}][H]
		$ git commit -m "Arreglando el nombre de la variable health en la clase soldado "
	\end{lstlisting}	
	\begin{lstlisting}[language=java,caption={Las lineas de codigos del metodo creado:}][H]
		private int health;
	\end{lstlisting}
	\subsection{Ejercicio VideoJuego2}
	\begin{itemize}	
		\item En el tercer commit creamos el metodo viewboard el cual nos permite dar con el tablero de manera grafica y si habria un soldado dentro de una de estas casillas se marcara con una X y si no dejara vacio 
		\item El codigo , el commit y la ejecucion seria el siguiente:
	\end{itemize}	
	\begin{lstlisting}[language=bash,caption={Commit}][H]
		$ git commit -m "Probando Metodo creado para poder ver el tablero para los soldados el cual si habria un soldado este se marcara con una X"
	\end{lstlisting}	
	\begin{lstlisting}[language=java,caption={Las lineas de codigos del metodo creado:}][H]
		// Laboratorio Nro 5  - Ejercicio Videojuego2
		// Autor: Mamani Anahua Victor Narciso
		// Colaboro:
		// Tiempo:
		import java.util.*;
		class VideoJuego2 {
			public static void viewboard(Soldado[][] army){
				System.out.println("Mostrando tabla de posicion ... --");
				System.out.println("_____________________________________________________");
				for(int i = 0; i < army.length; i++ ){
						for(int j = 0; j < army[i].length; j++){
								if(army[i][j].getHealth() == 0){
										System.out.print("|  " + "X" + "  ");
								}else{
										System.out.print("|\t");
								}
						}
						System.out.println("");
						System.out.println("|_______|_______|_______|_______|_______|_______|_______|_______|_______|_______|");
				}
			}
			public static void main (String args[]){
				Random rdm = new Random();
				System.out.println("Cuantos soldados? ");
				int numsoldiers = rdm.nextInt(10) + 1;
				Soldado[][] army = new Soldado[10][10];
				viewboard(army);
			}
		}
	\end{lstlisting}
	\begin{lstlisting}[language=bash,caption={Ejecucion:}][H]
		Mostrando tabla de posicion ... --
		_____________________________________________________
		Exception in thread "main" java.lang.NullPointerException: Cannot invoke "Soldado.getHealth()" because "<parameter1>[<local1>][<local2>]" is null
			at VideoJuego2.viewboard(VideoJuego2.java:12)
			at VideoJuego2.main(VideoJuego2.java:27)
	\end{lstlisting}
	\subsection{Ejercicio Soldado}
	\begin{itemize}	
		\item En el cuarto commit creamos este metodo que nos puede ayudar a identificar casillas nulas para nuestro ejercicio con Videojuego2.java y asi poder imprimirlas 
		\item El codigo y el commit seria el siguiente:
	\end{itemize}	
	\begin{lstlisting}[language=bash,caption={Commit}][H]
		$ git commit -m "Arreglando el nombre de la variable health en la clase soldado "
	\end{lstlisting}	
	\begin{lstlisting}[language=java,caption={Las lineas de codigos del metodo creado:}][H]
		//Anadiendo metodo que nos permita que un arreglo tenga datos nulos si este esta vacio
		public Soldado(){
			this.name = "";
			this.health = 0;
			this.row = 0;
			this.column  = "";
		}
	\end{lstlisting}
	\subsection{Ejercicio VideoJuego2}
	\begin{itemize}	
		\item En el quinto commit modificamos el metodo viewboard() el cual seria de lo que imprime y modificamos la sentencia del if para que cuando un soldado de las casillas no esta vacia el cual dice que tiene datos que verifican esto entonces este retornara un X pero si esta vacia este no imprimira nada y lo dejara vacio la casilla 
		\item El codigo , el commit y la ejecucion seria el siguiente:
	\end{itemize}	
	\begin{lstlisting}[language=bash,caption={Commit}][H]
		$ git commit -m "Arreglando cosas para que se de la grafica del tableroen esto esta cambiar la condicion para que cuando esta sea null 65265206"
	\end{lstlisting}	
	\begin{lstlisting}[language=java,caption={Las lineas de codigos del metodo creado:}][H]
		public static void viewboard(Soldado[][] army){
			System.out.println("Mostrando tabla de posicion ... --");
			System.out.println("_________________________________________________________________________________");
			for(int i = 0; i < army.length; i++ ){
				   for(int j = 0; j < army[i].length; j++){
						  if(army[i][j] != null){
								 System.out.print("|  " + "X" + "  ");
						  }else{
								 System.out.print("|\t");
						  }
				   }
				   System.out.println("|");
				   System.out.println("|_______|_______|_______|_______|_______|_______|_______|_______|_______|_______|");
			}
	   }
	\end{lstlisting}
	\begin{lstlisting}[language=java,caption={Las lineas de codigos del metodo creado: \textcolor{red}{VER EL TEXTO EN LATEX EN LA IMAGEN SE DEFORMA O EJECUTARLO}}][H]
		Cuantos soldados? 
		Mostrando tabla de posicion ... -- 
		_________________________________________________________________________________
		|       |       |       |       |       |       |       |       |       |       |
		|_______|_______|_______|_______|_______|_______|_______|_______|_______|_______|
		|       |       |       |       |       |       |       |       |       |       |
		|_______|_______|_______|_______|_______|_______|_______|_______|_______|_______|
		|       |       |       |       |       |       |       |       |       |       |
		|_______|_______|_______|_______|_______|_______|_______|_______|_______|_______|
		|       |       |       |       |       |       |       |       |       |       |
		|_______|_______|_______|_______|_______|_______|_______|_______|_______|_______|
		|       |       |       |       |       |       |       |       |       |       |
		|_______|_______|_______|_______|_______|_______|_______|_______|_______|_______|
		|       |       |       |       |       |       |       |       |       |       |
		|_______|_______|_______|_______|_______|_______|_______|_______|_______|_______|
		|       |       |       |       |       |       |       |       |       |       |
		|_______|_______|_______|_______|_______|_______|_______|_______|_______|_______|
		|       |       |       |       |       |       |       |       |       |       |
		|_______|_______|_______|_______|_______|_______|_______|_______|_______|_______|
		|       |       |       |       |       |       |       |       |       |       |
		|_______|_______|_______|_______|_______|_______|_______|_______|_______|_______|
		|       |       |       |       |       |       |       |       |       |       |
		|_______|_______|_______|_______|_______|_______|_______|_______|_______|_______|
		
	\end{lstlisting}
	\subsection{Ejercicio VideoJuego2}
	\begin{itemize}	
		\item En el SEXTO commit creamos el metodo fillarray() el cual queremos rellenar el array de soldados para poder verlos en el tablero a la ves su informacion en este commit se prodro un resultado el cual seria de la columna y el resultado de esta con una imprimicion ya que no estaria seguro de su resultado
		\item El codigo , el commit y la ejecucion seria el siguiente:
	\end{itemize}	
	\begin{lstlisting}[language=bash,caption={Commit}][H]
		$ git commit -m "Anadiendo el metodo fillarray() el cual va a llenar nuestro arreglo de soldados para eso necesitabamos su nombre su vida su fila y su columna la cual yo quiero probar la cilumna lo que imprime para que compruebe que este saliendo lo esperado un letra la cual nos de su posicion"
	\end{lstlisting}	
	\begin{lstlisting}[language=java,caption={Las lineas de codigos del metodo creado:}][H]
		public static Soldado[][] fillarray(int number){
			Random rdm = new Random();
			Soldado[][] army= new Soldado[10][10];
			for(int i = 0; i < number; i++){
				   String name = "Soldado" + (i + 1);
				   int health = rdm.nextInt(5) + 1;
				   int row = rdm.nextInt(10) + 1;
				   String column = String.valueOf((char)(rdm.nextInt(10) + 65));  
				   System.out.println(column);    
			}
			return null;
	   }
	\end{lstlisting}
	\begin{lstlisting}[language=java,caption={Las lineas de codigos del metodo creado: \textcolor{red}{VER EL TEXTO EN LATEX EN LA IMAGEN SE DEFORMA O EJECUTARLO}}][H]
		Cuantos soldados? 
		1
		Mostrando tabla de posicion ... --
		_________________________________________________________________________________
		|       |       |       |       |       |       |       |       |       |       |
		|_______|_______|_______|_______|_______|_______|_______|_______|_______|_______|
		|       |       |       |       |       |       |       |       |       |       |
		|_______|_______|_______|_______|_______|_______|_______|_______|_______|_______|
		|       |       |       |       |       |       |       |       |       |       |
		|_______|_______|_______|_______|_______|_______|_______|_______|_______|_______|
		|       |       |       |       |       |       |       |       |       |       |
		|_______|_______|_______|_______|_______|_______|_______|_______|_______|_______|
		|       |       |       |       |       |       |       |       |       |       |
		|_______|_______|_______|_______|_______|_______|_______|_______|_______|_______|
		|       |       |       |       |       |       |       |       |       |       |
		|_______|_______|_______|_______|_______|_______|_______|_______|_______|_______|
		|       |       |       |       |       |       |       |       |       |       |
		|_______|_______|_______|_______|_______|_______|_______|_______|_______|_______|
		|       |       |       |       |       |       |       |       |       |       |
		|_______|_______|_______|_______|_______|_______|_______|_______|_______|_______|
		|       |       |       |       |       |       |       |       |       |       |
		|_______|_______|_______|_______|_______|_______|_______|_______|_______|_______|
		|       |       |       |       |       |       |       |       |       |       |
		|_______|_______|_______|_______|_______|_______|_______|_______|_______|_______|
		F
	\end{lstlisting}
	\subsection{Ejercicio VideoJuego2}
	\begin{itemize}	
		\item En el septimo commit completamos el metodo fillarray() el cual ya usamos el constructor soldado ya que tendriamos los datos despues de esto comprobariamos si este casillero usado seria algo vacio y se podria llenarlo y si no este retrocederia una iteracion del ciclo ya que se estaria usando uno para preguntar a la ves tambien imprimimos los datos del soldado creado 
		\item El codigo , el commit y la ejecucion seria el siguiente:
	\end{itemize}	
	\begin{lstlisting}[language=bash,caption={Commit}][H]
		$ git commit -m "Completando el metodo fillarray() para que este pueda imprmir los datos de los soldados que estan en el tablero este te devuelve el array lleno con la cantidad de soldados y texto que dice su informacion de cada soldado Y tambien cumplimos con el cual no se repita un soldado en el mismo casillero ya que al ver que este lleno este retrocedera una repeticion ya que estaria tomando la repeticion en un casillero lleno"
	\end{lstlisting}	
	\begin{lstlisting}[language=java,caption={Las lineas de codigos del metodo creado:}][H]
		public static Soldado[][] fillarray(int number){
			Random rdm = new Random();
			Soldado[][] army= new Soldado[10][10];
			System.out.println("Registrando soldados .......");
			for(int i = 0; i < number; i++){
				   String name = "Soldado" + (i); //CORREGIMOS EN EL NOMBRE YA QUE ESTE COMNEZABA DESDE EL 0
				   int health = rdm.nextInt(5) + 1;
				   int row = rdm.nextInt(10) + 1;
				   String column = String.valueOf((char)(rdm.nextInt(10) + 65));  
				   if(army[row - 1][(int)column.charAt(0) - 65] == null){
						  System.out.print("*********************************");
						  army[row - 1][(int)column.charAt(0) - 65] = new Soldado(name, health, row, column);
						  System.out.println(army[row - 1][(int)column.charAt(0) - 65].toString());
				   }else{
						  i -= 1;
				   }
			}
			return army;
	   }
	\end{lstlisting}
	\begin{lstlisting}[language=java,caption={Las lineas de codigos del metodo creado: \textcolor{red}{VER EL TEXTO EN LATEX EN LA IMAGEN SE DEFORMA O EJECUTARLO}}][H]
		Cuantos soldados? 
		8
		Registrando soldados .......
		*********************************
		Nombre: Soldado1
		Vida: 2
		Fila: 10
		Columna: D
		*********************************
		Nombre: Soldado2
		Vida: 2
		Fila: 6
		Columna: E
		*********************************
		Nombre: Soldado3
		Vida: 5
		Fila: 5
		Columna: F
		*********************************
		Nombre: Soldado4
		Vida: 1
		Fila: 9
		Columna: I
		*********************************
		Nombre: Soldado5
		Vida: 2
		Fila: 9
		Columna: E
		*********************************
		Nombre: Soldado6
		Vida: 4
		Fila: 2
		Columna: F
		*********************************
		Nombre: Soldado7
		Vida: 1
		Fila: 4
		Columna: H
		*********************************
		Nombre: Soldado8
		Vida: 4
		Fila: 3
		Columna: E
		Mostrando tabla de posicion ... --
		_________________________________________________________________________________
		|       |       |       |       |       |       |       |       |       |       |
		|_______|_______|_______|_______|_______|_______|_______|_______|_______|_______|
		|       |       |       |       |       |   X   |       |       |       |       |
		|_______|_______|_______|_______|_______|_______|_______|_______|_______|_______|
		|       |       |       |       |   X   |       |       |       |       |       |
		|_______|_______|_______|_______|_______|_______|_______|_______|_______|_______|
		|       |       |       |       |       |       |       |   X   |       |       |
		|_______|_______|_______|_______|_______|_______|_______|_______|_______|_______|
		|       |       |       |       |       |   X   |       |       |       |       |
		|_______|_______|_______|_______|_______|_______|_______|_______|_______|_______|
		|       |       |       |       |   X   |       |       |       |       |       |
		|_______|_______|_______|_______|_______|_______|_______|_______|_______|_______|
		|       |       |       |       |       |       |       |       |       |       |
		|_______|_______|_______|_______|_______|_______|_______|_______|_______|_______|
		|       |       |       |       |       |       |       |       |       |       |
		|_______|_______|_______|_______|_______|_______|_______|_______|_______|_______|
		|       |       |       |       |   X   |       |       |       |   X   |       |
		|_______|_______|_______|_______|_______|_______|_______|_______|_______|_______|
		|       |       |       |   X   |       |       |       |       |       |       |
		|_______|_______|_______|_______|_______|_______|_______|_______|_______|_______|

	\end{lstlisting}
	\subsection{Ejercicio VideoJuego2}
	\begin{itemize}	
		\item En el octavo commit creamos el metodo longerLife() el cual este nos permite a dar la informacion del soldado con mayor vida del ejercito el cual comparamos con otros soldados que esten en este ejercito y dependiendo de eso va a recoger informacion del soldado con mayor vida y lo guarda en soldier el cual sera imprimido despues.
		\item El codigo , el commit y la ejecucion seria el siguiente:
	\end{itemize}	
	\begin{lstlisting}[language=bash,caption={Commit}][H]
		$ git commit -m "Creamos el metodo longerLife() el cual nos imprimira el dato del soldado con mayor vida dependiendo de los que esten en el ejercito que va comparando con los demas para ver quien es el mayor de todos"
	\end{lstlisting}	
	\begin{lstlisting}[language=java,caption={Las lineas de codigos del metodo creado:}][H]
		public static void longerLife(Soldado[][] army){
			int mayor = 0;
			Soldado soldier = null;
			for(int i = 0; i < army.length; i++){
				   for(int j = 0; j < army[i].length; j++){
						  if(army[i][j] != null){
								 if(mayor < army[i][j].getHealth()){
										mayor = army[i][j].getHealth();
										soldier = army[i][j];
								 }
						  }
				   }
			}
			System.out.println("");
			System.out.print("El primer soldado con mayor vida es: ");
			System.out.println(soldier.toString());
			System.out.println("*********************************");
	   	}
	\end{lstlisting}
	\begin{lstlisting}[language=java,caption={Las lineas de codigos del metodo creado: \textcolor{red}{VER EL TEXTO EN LATEX EN LA IMAGEN SE DEFORMA O EJECUTARLO}}][H]
		Cuantos soldados? 
		6
		Registrando soldados .......
		*********************************
		Nombre: Soldado0
		Vida: 1
		Fila: 9
		Columna: C
		*********************************
		Nombre: Soldado1
		Vida: 3
		Fila: 10
		Columna: E
		*********************************
		Nombre: Soldado2
		Vida: 2
		Fila: 4
		Columna: I
		*********************************
		Nombre: Soldado3
		Vida: 3
		Fila: 7
		Columna: F
		*********************************
		Nombre: Soldado4
		Vida: 3
		Fila: 8
		Columna: H
		*********************************
		Nombre: Soldado5
		Vida: 2
		Fila: 1
		Columna: G
		Mostrando tabla de posicion ... --
		_________________________________________________________________________________
		|       |       |       |       |       |       |   X   |       |       |       |
		|_______|_______|_______|_______|_______|_______|_______|_______|_______|_______|
		|       |       |       |       |       |       |       |       |       |       |
		|_______|_______|_______|_______|_______|_______|_______|_______|_______|_______|
		|       |       |       |       |       |       |       |       |       |       |
		|_______|_______|_______|_______|_______|_______|_______|_______|_______|_______|
		|       |       |       |       |       |       |       |       |   X   |       |
		|_______|_______|_______|_______|_______|_______|_______|_______|_______|_______|
		|       |       |       |       |       |       |       |       |       |       |
		|_______|_______|_______|_______|_______|_______|_______|_______|_______|_______|
		|       |       |       |       |       |       |       |       |       |       |
		|_______|_______|_______|_______|_______|_______|_______|_______|_______|_______|
		|       |       |       |       |       |   X   |       |       |       |       |
		|_______|_______|_______|_______|_______|_______|_______|_______|_______|_______|
		|       |       |       |       |       |       |       |   X   |       |       |
		|_______|_______|_______|_______|_______|_______|_______|_______|_______|_______|
		|       |       |   X   |       |       |       |       |       |       |       |
		|_______|_______|_______|_______|_______|_______|_______|_______|_______|_______|
		|       |       |       |       |   X   |       |       |       |       |       |
		|_______|_______|_______|_______|_______|_______|_______|_______|_______|_______|
		
		El primer soldado con mayor vida es: 
		Nombre: Soldado3
		Vida: 3
		Fila: 7
		Columna: F
		*********************************

	\end{lstlisting}
	\subsection{Ejercicio VideoJuego2}
	\begin{itemize}	
		\item En el noveno commit creamos el metodo averageLife() el cual nos retorna la imprimicion del promedio de vida de todos los soldados del ejercito el cual se va recolectando para despues dividirlo por la cantidad de soldados el cual va a ser decimal para eso necesitamos hacerlo double.
		\item El codigo , el commit y la ejecucion seria el siguiente:
	\end{itemize}	
	\begin{lstlisting}[language=bash,caption={Commit}][H]
		$ git commit -m "Agregamos el metodo averageLife() el cual nos dara el promedio de vida del ejercito el cual recolecteria la vida de todos y despues se dividiria en la cantidad de soldados el cual lo multiplicamos por 1.0 para que este sea double decimal y sea mas especifico y despues se imprimiria este resultado"
	\end{lstlisting}	
	\begin{lstlisting}[language=java,caption={Las lineas de codigos del metodo creado:}][H]
		public static void averageLife(Soldado[][] army){
			int sum = 0;
			int cont = 0;
			for(int i = 0; i < army.length; i++){
				   for(int j = 0; j < army.length; j++){
						  if(army[i][j] != null){
								 sum += army[i][j].getHealth();
								 cont++;
						  }
				   }
			}
			double avg = sum / (cont * 1.0);
			System.out.println("El promedio de vida del ejercito es : " + avg);
	   	}
	\end{lstlisting}
	\begin{lstlisting}[language=java,caption={Las lineas de codigos del metodo creado: \textcolor{red}{VER EL TEXTO EN LATEX EN LA IMAGEN SE DEFORMA O EJECUTARLO}}][H]
		Cuantos soldados? 
		10
		Registrando soldados .......
		*********************************
		Nombre: Soldado0
		Vida: 3
		Fila: 5
		Columna: E
		*********************************
		Nombre: Soldado1
		Vida: 3
		Fila: 4
		Columna: F
		*********************************
		Nombre: Soldado2
		Vida: 4
		Fila: 4
		Columna: A
		*********************************
		Nombre: Soldado3
		Vida: 3
		Fila: 2
		Columna: I
		*********************************
		Nombre: Soldado4
		Vida: 4
		Fila: 9
		Columna: A
		*********************************
		Nombre: Soldado5
		Vida: 1
		Fila: 1
		Columna: B
		*********************************
		Nombre: Soldado6
		Vida: 1
		Fila: 1
		Columna: C
		*********************************
		Nombre: Soldado7
		Vida: 4
		Fila: 4
		Columna: E
		*********************************
		Nombre: Soldado8
		Vida: 2
		Fila: 10
		Columna: A
		*********************************
		Nombre: Soldado9
		Vida: 4
		Fila: 5
		Columna: G
		Mostrando tabla de posicion ... --
		_________________________________________________________________________________
		|       |   X   |   X   |       |       |       |       |       |       |       |
		|_______|_______|_______|_______|_______|_______|_______|_______|_______|_______|
		|       |       |       |       |       |       |       |       |   X   |       |
		|_______|_______|_______|_______|_______|_______|_______|_______|_______|_______|
		|       |       |       |       |       |       |       |       |       |       |
		|_______|_______|_______|_______|_______|_______|_______|_______|_______|_______|
		|   X   |       |       |       |   X   |   X   |       |       |       |       |
		|_______|_______|_______|_______|_______|_______|_______|_______|_______|_______|
		|       |       |       |       |   X   |       |   X   |       |       |       |
		|_______|_______|_______|_______|_______|_______|_______|_______|_______|_______|
		|       |       |       |       |       |       |       |       |       |       |
		|_______|_______|_______|_______|_______|_______|_______|_______|_______|_______|
		|       |       |       |       |       |       |       |       |       |       |
		|_______|_______|_______|_______|_______|_______|_______|_______|_______|_______|
		|       |       |       |       |       |       |       |       |       |       |
		|_______|_______|_______|_______|_______|_______|_______|_______|_______|_______|
		|   X   |       |       |       |       |       |       |       |       |       |
		|_______|_______|_______|_______|_______|_______|_______|_______|_______|_______|
		|   X   |       |       |       |       |       |       |       |       |       |
		|_______|_______|_______|_______|_______|_______|_______|_______|_______|_______|

		El primer soldado con mayor vida es: 
		Nombre: Soldado2
		Vida: 4
		Fila: 4
		Columna: A
		*********************************
		El promedio de vida del ejercito es : 2.9

	\end{lstlisting}
	\subsection{Ejercicio VideoJuego2}
	\begin{itemize}	
		\item En el decimo commit creamos el metodo rankingBurbujaHealth() el cual aplicamos que en un arreglo unidimensional ponemos todos los soldados del arreglo bidimensional para asi poderaplicar efectivamente el metodo burbuja el cual aplicariamos con el atributo health el cual nos dara un ranking que despues se imprimira los resultados del ranking del mayor vida al menor que se imprimiran 
		\item El codigo , el commit y la ejecucion seria el siguiente:
	\end{itemize}	
	\begin{lstlisting}[language=bash,caption={Commit}][H]
		$ git commit -m "Creando el metodo rankingBurbujaHealth() el cual aplicamos que en un arreglo unidimensional ponemos todos los soldados del arreglo bidimensional para asi poderaplicar efectivamente el metodo burbuja el cual aplicariamos con el atributo health el cual nos dara un ranking que despues se imprimira los resultados del ranking del mayor vida al menor"
	\end{lstlisting}	
	\begin{lstlisting}[language=java,caption={Las lineas de codigos del metodo creado:}][H]
		public static void rankingBurbujaHealth(Soldado[][] army, int numsoldiers){
			Soldado[] soldiers = new Soldado[numsoldiers];
			int count = 0;
			Soldado soldier = null;
			for(int i = 0; i < army.length; i++){ //CREAMOS ARREGLO PARA QUE LOS SOLDADOS SE TRASLADEN DE UN ARREGLO BIDIMENSIONAL A UNO DIMENSIONAL PARA APLICAR EL METODO BURBUJA
				   for(int j = 0; j < army[i].length; j++){
						  if(army[i][j] != null){
								 soldiers[count] = army[i][j];
								 count++;
						  }
				   }
			}
			System.out.println("Ordenando a los soldados por el metodo burbuja: "); //APLICAMOS EL METODO BURBUJA CON LOS PUNTOS DE VIDA
			for(int i = 0; i < numsoldiers - 1; i++){
				   for(int j = 0; j < numsoldiers - i - 1; j++){
						  if(soldiers[j].getHealth() < soldiers[j + 1].getHealth()){
								 soldier = soldiers[j];
								 soldiers[j] = soldiers[j + 1];
								 soldiers[j + 1] = soldier;
						  }
				   }      
			}
			System.out.println("------------------------------------------");
			System.out.println("Mostrando Ranking...");
			for(int i = 0; i < soldiers.length; i++){
				   System.out.print("\n" + "Puesto " + (i + 1));
				   System.out.println(soldiers[i].toString());
				   System.out.println("------------------");
			}
			System.out.println("*********************************");
	   } 
	\end{lstlisting}
	\begin{lstlisting}[language=java,caption={Las lineas de codigos del metodo creado: \textcolor{red}{VER EL TEXTO EN LATEX EN LA IMAGEN SE DEFORMA O EJECUTARLO}}][H]
		Cuantos soldados? 
		6
		Registrando soldados .......
		*********************************
		Nombre: Soldado0
		Vida: 3
		Fila: 7
		Columna: I
		*********************************
		Nombre: Soldado1
		Vida: 1
		Fila: 5
		Columna: C
		*********************************
		Nombre: Soldado2
		Vida: 1
		Fila: 10
		Columna: E
		*********************************
		Nombre: Soldado3
		Vida: 2
		Fila: 3
		Columna: C
		*********************************
		Nombre: Soldado4
		Vida: 4
		Fila: 6
		Columna: D
		*********************************
		Nombre: Soldado5
		Vida: 1
		Fila: 1
		Columna: H
		Mostrando tabla de posicion ... --
		_________________________________________________________________________________
		|       |       |       |       |       |       |       |   X   |       |       |
		|_______|_______|_______|_______|_______|_______|_______|_______|_______|_______|
		|       |       |       |       |       |       |       |       |       |       |
		|_______|_______|_______|_______|_______|_______|_______|_______|_______|_______|
		|       |       |   X   |       |       |       |       |       |       |       |
		|_______|_______|_______|_______|_______|_______|_______|_______|_______|_______|
		|       |       |       |       |       |       |       |       |       |       |
		|_______|_______|_______|_______|_______|_______|_______|_______|_______|_______|
		|       |       |   X   |       |       |       |       |       |       |       |
		|_______|_______|_______|_______|_______|_______|_______|_______|_______|_______|
		|       |       |       |   X   |       |       |       |       |       |       |
		|_______|_______|_______|_______|_______|_______|_______|_______|_______|_______|
		|       |       |       |       |       |       |       |       |   X   |       |
		|_______|_______|_______|_______|_______|_______|_______|_______|_______|_______|
		|       |       |       |       |       |       |       |       |       |       |
		|_______|_______|_______|_______|_______|_______|_______|_______|_______|_______|
		|       |       |       |       |       |       |       |       |       |       |
		|_______|_______|_______|_______|_______|_______|_______|_______|_______|_______|
		|       |       |       |       |   X   |       |       |       |       |       |
		|_______|_______|_______|_______|_______|_______|_______|_______|_______|_______|

		El primer soldado con mayor vida es: 
		Nombre: Soldado4
		Vida: 4
		Fila: 6
		Columna: D
		*********************************
		El promedio de vida del ejercito es : 2.0
		*********************************
		Ordenando a los soldados por el metodo burbuja: 
		------------------------------------------
		Mostrando Ranking...

		Puesto 1
		Nombre: Soldado4
		Vida: 4
		Fila: 6
		Columna: D
		------------------

		Puesto 2
		Nombre: Soldado0
		Vida: 3
		Fila: 7
		Columna: I
		------------------

		Puesto 3
		Nombre: Soldado3
		Vida: 2
		Fila: 3
		Columna: C
		------------------

		Puesto 4
		Nombre: Soldado5
		Vida: 1
		Fila: 1
		Columna: H
		------------------

		Puesto 5
		Nombre: Soldado1
		Vida: 1
		Fila: 5
		Columna: C
		------------------

		Puesto 6
		Nombre: Soldado2
		Vida: 1
		Fila: 10
		Columna: E
		------------------
		*********************************
		
	\end{lstlisting}
	\subsection{Ejercicio VideoJuego2}
	\begin{itemize}	
		\item En el decimoprimer commit creamos el metodo rankingInsercionHealth() el cual aplicamos que en un arreglo unidimensional ponemos todos los soldados del arreglo bidimensional para asi poderaplicar efectivamente el metodo insercion el cual aplicariamos con el atributo health el cual nos dara un ranking debido a la logica usada en este metodo para poder ordenarlo y que despues se imprimira los resultados del ranking del mayor vida al menor que se imprimiran 
		\item El codigo , el commit y la ejecucion seria el siguiente:
	\end{itemize}	
	\begin{lstlisting}[language=bash,caption={Commit}][H]
		$ git commit -m "Creando el metodo rankingInsercionHealth() el cual tambien ordenaria nuestros soldados pero esto ya seria de manera de la forma insercion ya que el anterior fue burbuja en este aplicamos la logica para la insercion al igual que el otro tuvimos que poner a los soldados en un arreglo unidimensional para poder despues ordenarlos y imprimirlos en el debido puesto que deberian estar por su ranking de puntos"
	\end{lstlisting}	
	\begin{lstlisting}[language=java,caption={Las lineas de codigos del metodo creado:}][H]
		public static void rankingInsercionHealth(Soldado[][] army, int numsoldiers){
			Soldado[] soldiers = new Soldado[numsoldiers];
			int count = 0;
			for(int i = 0; i < army.length; i++){ //CREAMOS ARREGLO PARA QUE LOS SOLDADOS SE TRASLADEN DE UN ARREGLO BIDIMENSIONAL A UNO DIMENSIONAL PARA APLICAR EL METODO INSERCION
				   for(int j = 0; j < army[i].length; j++){
						  if(army[i][j] != null){
								 soldiers[count] = army[i][j];
								 count++;
						  }
				   }
			}
			System.out.println("Ordenando a los soldados por el metodo insercion: "); //APLICAMOS EL METODO INSERCION CON LOS PUNTOS DE VIDA
			for(int i = 1; i < soldiers.length; i++){
				   Soldado temp = soldiers[i];
				   int j = i - 1;
				   while(j >= 0 && (temp.getHealth() > soldiers[j].getHealth())){
					   soldiers[j + 1] = soldiers[j];
					   j--;
				   }
				   soldiers[j + 1] = temp;
			}
			System.out.println("------------------------------------------");
			System.out.println("Mostrando Ranking...");
			for(int i = 0; i < soldiers.length; i++){
				   System.out.print("\n" + "Puesto " + (i + 1));
				   System.out.println(soldiers[i].toString());
				   System.out.println("------------------");
			}
			System.out.println("*********************************");
	   	}
	\end{lstlisting}
	\begin{lstlisting}[language=java,caption={Las lineas de codigos del metodo creado: \textcolor{red}{VER EL TEXTO EN LATEX EN LA IMAGEN SE DEFORMA O EJECUTARLO}}][H]
		Cuantos soldados? 
		9
		Registrando soldados .......
		*********************************
		Nombre: Soldado0
		Vida: 3
		Fila: 1
		Columna: D
		*********************************
		Nombre: Soldado1
		Vida: 3
		Fila: 1
		Columna: B
		*********************************
		Nombre: Soldado2
		Vida: 5
		Fila: 7
		Columna: H
		*********************************
		Nombre: Soldado3
		Vida: 3
		Fila: 6
		Columna: D
		*********************************
		Nombre: Soldado4
		Vida: 1
		Fila: 8
		Columna: H
		*********************************
		Nombre: Soldado5
		Vida: 3
		Fila: 5
		Columna: E
		*********************************
		Nombre: Soldado6
		Vida: 1
		Fila: 1
		Columna: E
		*********************************
		Nombre: Soldado7
		Vida: 1
		Fila: 8
		Columna: G
		*********************************
		Nombre: Soldado8
		Vida: 2
		Fila: 2
		Columna: H
		Mostrando tabla de posicion ... --
		_________________________________________________________________________________
		|       |   X   |       |   X   |   X   |       |       |       |       |       |
		|_______|_______|_______|_______|_______|_______|_______|_______|_______|_______|
		|       |       |       |       |       |       |       |   X   |       |       |
		|_______|_______|_______|_______|_______|_______|_______|_______|_______|_______|
		|       |       |       |       |       |       |       |       |       |       |
		|_______|_______|_______|_______|_______|_______|_______|_______|_______|_______|
		|       |       |       |       |       |       |       |       |       |       |
		|_______|_______|_______|_______|_______|_______|_______|_______|_______|_______|
		|       |       |       |       |   X   |       |       |       |       |       |
		|_______|_______|_______|_______|_______|_______|_______|_______|_______|_______|
		|       |       |       |   X   |       |       |       |       |       |       |
		|_______|_______|_______|_______|_______|_______|_______|_______|_______|_______|
		|       |       |       |       |       |       |       |   X   |       |       |
		|_______|_______|_______|_______|_______|_______|_______|_______|_______|_______|
		|       |       |       |       |       |       |   X   |   X   |       |       |
		|_______|_______|_______|_______|_______|_______|_______|_______|_______|_______|
		|       |       |       |       |       |       |       |       |       |       |
		|_______|_______|_______|_______|_______|_______|_______|_______|_______|_______|
		|       |       |       |       |       |       |       |       |       |       |
		|_______|_______|_______|_______|_______|_______|_______|_______|_______|_______|

		El primer soldado con mayor vida es: 
		Nombre: Soldado2
		Vida: 5
		Fila: 7
		Columna: H
		*********************************
		El promedio de vida del ejercito es : 2.4444444444444446
		*********************************
		Ordenando a los soldados por el metodo burbuja: 
		------------------------------------------
		Mostrando Ranking...

		Puesto 1
		Nombre: Soldado2
		Vida: 5
		Fila: 7
		Columna: H
		------------------

		Puesto 2
		Nombre: Soldado1
		Vida: 3
		Fila: 1
		Columna: B
		------------------

		Puesto 3
		Nombre: Soldado0
		Vida: 3
		Fila: 1
		Columna: D
		------------------

		Puesto 4
		Nombre: Soldado5
		Vida: 3
		Fila: 5
		Columna: E
		------------------

		Puesto 5
		Nombre: Soldado3
		Vida: 3
		Fila: 6
		Columna: D
		------------------

		Puesto 6
		Nombre: Soldado8
		Vida: 2
		Fila: 2
		Columna: H
		------------------

		Puesto 7
		Nombre: Soldado6
		Vida: 1
		Fila: 1
		Columna: E
		------------------

		Puesto 8
		Nombre: Soldado7
		Vida: 1
		Fila: 8
		Columna: G
		------------------

		Puesto 9
		Nombre: Soldado4
		Vida: 1
		Fila: 8
		Columna: H
		------------------
		*********************************
		Ordenando a los soldados por el metodo insercion: 
		------------------------------------------
		Mostrando Ranking...

		Puesto 1
		Nombre: Soldado2
		Vida: 5
		Fila: 7
		Columna: H
		------------------

		Puesto 2
		Nombre: Soldado1
		Vida: 3
		Fila: 1
		Columna: B
		------------------

		Puesto 3
		Nombre: Soldado0
		Vida: 3
		Fila: 1
		Columna: D
		------------------

		Puesto 4
		Nombre: Soldado5
		Vida: 3
		Fila: 5
		Columna: E
		------------------

		Puesto 5
		Nombre: Soldado3
		Vida: 3
		Fila: 6
		Columna: D
		------------------

		Puesto 6
		Nombre: Soldado8
		Vida: 2
		Fila: 2
		Columna: H
		------------------

		Puesto 7
		Nombre: Soldado6
		Vida: 1
		Fila: 1
		Columna: E
		------------------

		Puesto 8
		Nombre: Soldado7
		Vida: 1
		Fila: 8
		Columna: G
		------------------

		Puesto 9
		Nombre: Soldado4
		Vida: 1
		Fila: 8
		Columna: H
		------------------
		*********************************
		
	\end{lstlisting}

	\subsection{Estructura de laboratorio 05}
	\begin{itemize}	
		\item El contenido que se entrega en este laboratorio05 es el siguiente:
	\end{itemize}
	\begin{lstlisting}[style=ascii-tree]
	/Lab05 		"Poner rama"

	\end{lstlisting}    
	\section{\textcolor{red}{Rúbricas}}
	
	\subsection{\textcolor{red}{Entregable Informe}}
	\begin{table}[H]
		\caption{Tipo de Informe}
		\setlength{\tabcolsep}{0.5em} % for the horizontal padding
		{\renewcommand{\arraystretch}{1.5}% for the vertical padding
		\begin{tabular}{|p{3cm}|p{12cm}|}
			\hline
			\multicolumn{2}{|c|}{\textbf{\textcolor{red}{Informe}}}  \\
			\hline 
			\textbf{\textcolor{red}{Latex}} & \textcolor{blue}{El informe está en formato PDF desde Latex,  con un formato limpio (buena presentación) y facil de leer.}   \\ 
			\hline 
			
			
		\end{tabular}
	}
	\end{table}
	
	\clearpage
	
	\subsection{\textcolor{red}{Rúbrica para el contenido del Informe y demostración}}
	\begin{itemize}			
		\item El alumno debe marcar o dejar en blanco en celdas de la columna \textbf{Checklist} si cumplio con el ítem correspondiente.
		\item Si un alumno supera la fecha de entrega,  su calificación será sobre la nota mínima aprobada, siempre y cuando cumpla con todos lo items.
		\item El alumno debe autocalificarse en la columna \textbf{Estudiante} de acuerdo a la siguiente tabla:
	
		\begin{table}[ht]
			\caption{Niveles de desempeño}
			\begin{center}
			\begin{tabular}{ccccc}
    			\hline
    			 & \multicolumn{4}{c}{Nivel}\\
    			\cline{1-5}
    			\textbf{Puntos} & Insatisfactorio 25\%& En Proceso 50\% & Satisfactorio 75\% & Sobresaliente 100\%\\
    			\textbf{2.0}&0.5&1.0&1.5&2.0\\
    			\textbf{4.0}&1.0&2.0&3.0&4.0\\
    		\hline
			\end{tabular}
		\end{center}
	\end{table}	
	
	\end{itemize}
	
	\begin{table}[H]
		\caption{Rúbrica para contenido del Informe y demostración}
		\setlength{\tabcolsep}{0.5em} % for the horizontal padding
		{\renewcommand{\arraystretch}{1.5}% for the vertical padding
		%\begin{center}
		\begin{tabular}{|p{2.7cm}|p{7cm}|x{1.3cm}|p{1.2cm}|p{1.5cm}|p{1.1cm}|}
			\hline
    		\multicolumn{2}{|c|}{Contenido y demostración} & Puntos & Checklist & Estudiante & Profesor\\
			\hline
			\textbf{1. GitHub} & Hay enlace URL activo del directorio para el  laboratorio hacia su repositorio GitHub con código fuente terminado y fácil de revisar. &2 &X &2 & \\ 
			\hline
			\textbf{2. Commits} &  Hay capturas de pantalla de los commits más importantes con sus explicaciones detalladas. (El profesor puede preguntar para refrendar calificación). &4 &X &1 & \\ 
			\hline 
			\textbf{3. Código fuente} &  Hay porciones de código fuente importantes con numeración y explicaciones detalladas de sus funciones. &2 &X &1 & \\ 
			\hline 
			\textbf{4. Ejecución} & Se incluyen ejecuciones/pruebas del código fuente  explicadas gradualmente. &2 &X &2 & \\ 
			\hline			
			\textbf{5. Pregunta} & Se responde con completitud a la pregunta formulada en la tarea.  (El profesor puede preguntar para refrendar calificación).  &2 &X &2 & \\ 
			\hline	
			\textbf{6. Fechas} & Las fechas de modificación del código fuente estan dentro de los plazos de fecha de entrega establecidos. &2 &X &0.5 & \\ 
			\hline 
			\textbf{7. Ortografía} & El documento no muestra errores ortográficos. &2 &X &2 & \\ 
			\hline 
			\textbf{8. Madurez} & El Informe muestra de manera general una evolución de la madurez del código fuente,  explicaciones puntuales pero precisas y un acabado impecable.   (El profesor puede preguntar para refrendar calificación).  &4 &X &2 & \\ 
			\hline
			\multicolumn{2}{|c|}{\textbf{Total}} &20 & &12.5 & \\ 
			\hline
		\end{tabular}
		%\end{center}
		%\label{tab:multicol}
		}
	\end{table}
	
\clearpage

\section{Referencias}
\begin{itemize}			
	\item \url{https://drive.google.com/file/d/1CoQAKeKW-QDYRmHLrBdbSopFB1Z_Qmk3/view}
\end{itemize}	
	
%\clearpage
%\bibliographystyle{apalike}
%\bibliographystyle{IEEEtranN}
%\bibliography{bibliography}
			
\end{document}