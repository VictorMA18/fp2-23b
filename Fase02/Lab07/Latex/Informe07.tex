%package list
\documentclass{article}
\usepackage[top=3cm, bottom=3cm, outer=3cm, inner=3cm]{geometry}
\usepackage{multicol}
\usepackage{graphicx}
\usepackage{url}
%\usepackage{cite}
\usepackage{hyperref}
\usepackage{array}
%\usepackage{multicol}
\newcolumntype{x}[1]{>{\centering\arraybackslash\hspace{0pt}}p{#1}}
\usepackage{natbib}
\usepackage{pdfpages}
\usepackage{multirow}
\usepackage[normalem]{ulem}
\useunder{\uline}{\ul}{}
\usepackage{svg}
\usepackage{xcolor}
\usepackage{listings}
\lstdefinestyle{ascii-tree}{
    literate={├}{|}1 {─}{--}1 {└}{+}1 
  }
\lstset{basicstyle=\ttfamily,
  showstringspaces=false,
  commentstyle=\color{red},
  keywordstyle=\color{blue}
}
%\usepackage{booktabs}
\usepackage{caption}
\usepackage{subcaption}
\usepackage{float}
\usepackage{array}

\newcolumntype{M}[1]{>{\centering\arraybackslash}m{#1}}
\newcolumntype{N}{@{}m{0pt}@{}}


%%%%%%%%%%%%%%%%%%%%%%%%%%%%%%%%%%%%%%%%%%%%%%%%%%%%%%%%%%%%%%%%%%%%%%%%%%%%
%%%%%%%%%%%%%%%%%%%%%%%%%%%%%%%%%%%%%%%%%%%%%%%%%%%%%%%%%%%%%%%%%%%%%%%%%%%%
\newcommand{\itemEmail}{vmamanian@unsa.edu.pe}
\newcommand{\itemStudent}{Victor Mamani Anahua}
\newcommand{\itemCourse}{Fundamentos de la Programación II}
\newcommand{\itemCourseCode}{20230489}
\newcommand{\itemSemester}{II}
\newcommand{\itemUniversity}{Universidad Nacional de San Agustín de Arequipa}
\newcommand{\itemFaculty}{Facultad de Ingeniería de Producción y Servicios}
\newcommand{\itemDepartment}{Departamento Académico de Ingeniería de Sistemas e Informática}
\newcommand{\itemSchool}{Escuela Profesional de Ingeniería de Sistemas}
\newcommand{\itemAcademic}{2023 - B}
\newcommand{\itemInput}{Del 17 Octubre 2023}
\newcommand{\itemOutput}{Al 24 Octubre 2023}
\newcommand{\itemPracticeNumber}{07}
\newcommand{\itemTheme}{Laboratorio 07}
%%%%%%%%%%%%%%%%%%%%%%%%%%%%%%%%%%%%%%%%%%%%%%%%%%%%%%%%%%%%%%%%%%%%%%%%%%%%
%%%%%%%%%%%%%%%%%%%%%%%%%%%%%%%%%%%%%%%%%%%%%%%%%%%%%%%%%%%%%%%%%%%%%%%%%%%%

\usepackage[english,spanish]{babel}
\usepackage[utf8]{inputenc}
\AtBeginDocument{\selectlanguage{spanish}}
\renewcommand{\figurename}{Figura}
\renewcommand{\refname}{Referencias}
\renewcommand{\tablename}{Tabla} %esto no funciona cuando se usa babel
\AtBeginDocument{%
	\renewcommand\tablename{Tabla}
}

\usepackage{fancyhdr}
\pagestyle{fancy}
\fancyhf{}
\setlength{\headheight}{30pt}
\renewcommand{\headrulewidth}{1pt}
\renewcommand{\footrulewidth}{1pt}
\fancyhead[L]{\raisebox{-0.2\height}{\includegraphics[width=3cm]{img/logo_episunsa.png}}}
\fancyhead[C]{\fontsize{7}{7}\selectfont	\itemUniversity \\ \itemFaculty \\ \itemDepartment \\ \itemSchool \\ \textbf{\itemCourse}}
\fancyhead[R]{\raisebox{-0.2\height}{\includegraphics[width=1.2cm]{img/logo_abet}}}
\fancyfoot[L]{Estudiante Victor Mamani A.}
\fancyfoot[C]{\itemCourse}
\fancyfoot[R]{Página \thepage}

% para el codigo fuente
\usepackage{listings}
\usepackage{color, colortbl}
\definecolor{dkgreen}{rgb}{0,0.6,0}
\definecolor{gray}{rgb}{0.5,0.5,0.5}
\definecolor{mauve}{rgb}{0.58,0,0.82}
\definecolor{codebackground}{rgb}{0.95, 0.95, 0.92}
\definecolor{tablebackground}{rgb}{0.8, 0, 0}

\lstdefinestyle{java}{frame=tb,
	language=Java,
	showstringspaces=false,
	columns=flexible,
	basicstyle={\footnotesize\ttfamily\color[RGB]{255,255,255}},
	numberstyle=\color{mygray},
	numbers=left, 
	keywordstyle=\color{myblue},
	morekeywords={String, System},
	commentstyle=\color{mygray},
	stringstyle=\color{mygreen},
	breaklines=true,
	breakatwhitespace=true,
	tabsize=2,
	backgroundcolor= \color{codebackgroundCode},
	showspaces=false,
	showtabs=false,
	showlines=false,
}

\lstset{frame=tb,
	language=bash,
	aboveskip=3mm,
	belowskip=3mm,
	showstringspaces=false,
	columns=flexible,
	basicstyle={\small\ttfamily},
	numbers=none,
	numberstyle=\tiny\color{gray},
	keywordstyle=\color{blue},
	commentstyle=\color{dkgreen},
	stringstyle=\color{mauve},
	breaklines=true,
	breakatwhitespace=true,
	tabsize=3,
	backgroundcolor= \color{codebackground},
}

\begin{document}
	
	\vspace*{10px}
	
	\begin{center}	
		\fontsize{17}{17} \textbf{ Informe de Laboratorio \itemPracticeNumber}
	\end{center}
	\centerline{\textbf{\Large Tema: \itemTheme}}
	%\vspace*{0.5cm}	

	\begin{flushright}
		\begin{tabular}{|M{2.5cm}|N|}
			\hline 
			\rowcolor{tablebackground}
			\color{white} \textbf{Nota}  \\
			\hline 
			     \\[30pt]
			\hline 			
		\end{tabular}
	\end{flushright}	

	\begin{table}[H]
		\begin{tabular}{|x{4.7cm}|x{4.8cm}|x{4.8cm}|}
			\hline 
			\rowcolor{tablebackground}
			\color{white} \textbf{Estudiante} & \color{white}\textbf{Escuela}  & \color{white}\textbf{Asignatura}   \\
			\hline 
			{\itemStudent \par \itemEmail} & \itemSchool & {\itemCourse \par Semestre: \itemSemester \par Código: \itemCourseCode}     \\
			\hline 			
		\end{tabular}
	\end{table}		
	
	\begin{table}[H]
		\begin{tabular}{|x{4.7cm}|x{4.8cm}|x{4.8cm}|}
			\hline 
			\rowcolor{tablebackground}
			\color{white}\textbf{Laboratorio} & \color{white}\textbf{Tema}  & \color{white}\textbf{Duración}   \\
			\hline 
			\itemPracticeNumber & \itemTheme & 04 horas   \\
			\hline 
		\end{tabular}
	\end{table}
	
	\begin{table}[H]
		\begin{tabular}{|x{4.7cm}|x{4.8cm}|x{4.8cm}|}
			\hline 
			\rowcolor{tablebackground}
			\color{white}\textbf{Semestre académico} & \color{white}\textbf{Fecha de inicio}  & \color{white}\textbf{Fecha de entrega}   \\
			\hline 
			\itemAcademic & \itemInput &  \itemOutput  \\
			\hline 
		\end{tabular}
	\end{table}
	
	\section{Tarea}
	\begin{itemize}		
        \item Cree un Proyecto llamado Laboratorio7
		\item Usted deberá crear las dos clases Soldado.java y VideoJuego4.java. Puede reutilizar lo desarrollado en Laboratorios anteriores.
		\item Del Soldado nos importa el nombre, puntos de vida, fila y columna (posición en el tablero).
		\item El juego se desarrollará en el mismo tablero de los laboratorios anteriores. Para el tablero utilizar la estructura de datos más adecuada.
		\item Tendrá 2 Ejércitos. Inicializar el tablero con n soldados aleatorios entre 1 y 10 para cada Ejército. Cada soldado tendrá un nombre autogenerado: Soldado0X1, Soldado1X1, etc., un valor de puntos de vida autogenerado aleatoriamente [1..5], la fila y columna también autogenerados aleatoriamente (no puede haber 2 soldados en el mismo cuadrado). 
		\item Además de los datos del Soldado con mayor vida de cada ejército, el promedio de puntos de vida de todos los soldados creados por ejército, los datos de todos los soldados porejército en el orden que fueron creados y un ranking de poder de todos los soldados creados por ejército (del que tiene más nivel de vida al que tiene menos) usando 2 diferentes algoritmos de ordenamiento.
		\item Finalmente, que muestre qué ejército ganará la batalla (indicar la métrica usada para decidir al ganador de la batalla).
	\end{itemize}

	\section{Equipos, materiales y temas utilizados}
	\begin{itemize}
		\item Sistema Operativo Ubuntu GNU Linux 23 lunar 64 bits Kernell 6.2.v
		\item Visual Studio Code.
		\item VIM 9.0.
		\item OpenJDK 64-Bits 19.0.7.
		\item Git 2.39.2.
		\item Cuenta en GitHub con el correo institucional.
		\item Programación Orientada a Objetos.
		\item Actividades del Laboratorio 07.	
	\end{itemize}
	
	\section{URL de Repositorio Github}
	\begin{itemize}
		\item URL del Repositorio GitHub para clonar o recuperar.
		\item \url{https://github.com/VictorMA18/fp2-23b.git}
		\item URL para el laboratorio 07 en el Repositorio GitHub.
		\item \url{https://github.com/VictorMA18/fp2-23b/tree/main/Fase02/Lab07}
	\end{itemize}
	
	\section{Actividades del Laboratorio 07}
	
	\subsection{Ejercicio Soldado}
	\begin{itemize}	
		\item En el primer commit bueno reutilizamos el archivo que seria nuestra clase Soldado el cual la utilizaremos para poder avanzar con el siguiente ejercicio que seria VideoJuego4.
		\item El codigo y el commit seria el siguiente:
	\end{itemize}	
	\begin{lstlisting}[language=bash,caption={Commit}][H]
		$ git commit -m "Publicando la clase soldado para el ejercicio7 la cual es la clase soldado donde estan los atributos mas importantes que nos serviran"
	\end{lstlisting}	
	\begin{lstlisting}[language=java,caption={Las lineas de codigos del metodo creado:}][H]
		// Laboratorio Nro 7  - Ejercicio Soldado
		// Autor: Mamani Anahua Victor Narciso
		// Colaboro:
		// Tiempo:
		public class Soldado { //CREAMOS LA CLASE SOLDODADO PARA PODER USAR UN ARREGLO BIDIMENSIONAL DONDE NECESITAMOS LA VIDA , EL NOMBRE DEL SOLDADO Y TAMBIEN SU POSICION COMO LA FILA Y LA COLUMNA   

			private String name;
			private int heatlh; 
			private int row;
			private String column;

			//Constructor
			public Soldado(String name, int health, int row, String column){
			this.name = name;
			this.health = health;
			this.row = row;
			this.column = column;
			}

			// Metodos mutadores
			public void setName(String n){
				name = n;
			}
			public void setHealth(int p){
				heatlh = p;
			}
			public void setRow(int b){
				row = b;
			}
			public void setColumn(String c){
				column = c; 
			}

			// Metodos accesores
			public String getName(){
				return name;
			}
			public int getHealth(){
				return heatlh;

			}
			public int getRow(){
				return row;
			}
			public String getColumn(){
				return column;
			}

			// Completar con otros metodos necesarios
			public String toString(){ //CREAMOS ESTE METODO PARA IMPRIMIR LOS DATOS DEl OBJETO
				String join = "\nNombre: " + getName() + "\nVida: " + getHealth() + "\nFila: " + getRow() + "\nColumna: " + getColumn(); //Agregamos un espaciador para poder separar
				return join;
			}
		}
	\end{lstlisting}
	\subsection{Ejercicio VideoJuego4}
	\begin{itemize}	
		\item En el segundo commit completamos el metodo arrayfillregister() el cual queremos rellenar el array de soldados del ejercito 2 para poder verlos en el tablero a la ves su informacion de cada soldado el cual va hacer por orden de creacion
		\item El codigo , el commit y la ejecucion seria el siguiente:
	\end{itemize}	
	\begin{lstlisting}[language=bash,caption={Commit}][H]
		$ git commit -m "Creando el metodo arrayfillregister() para que este pueda imprimir los datos de los soldados que estan en el tablero este te devuelve el array lleno con la cantidad de soldados y texto que dice su informacion de cada soldado Y tambien cumplimos con el cual no se repita un soldado en el mismo casillero ya que al ver que este lleno este retrocedera una repeticion ya que estaria tomando la repeticion en un casillero lleno"
	\end{lstlisting}	
	\begin{lstlisting}[language=java,caption={Las lineas de codigos del metodo creado:}][H]
		public static Soldado[][] arrayfillregister(int num){ //METODO CREADO PARA PODER CREAR AL EJERCITO 2 EL CUAL USAREMOS LA ESTRUCTURA DE DATO QUE ES EL ARRAY CON TAL QUE TAMBIEN REGISTRAMOS 
			Random rdm = new Random();
			int numsoldiers = rdm.nextInt(10) + 1;
			System.out.println("La Ejercito " + num + " tiene " + numsoldiers + " soldados:");  
			System.out.println("*********************************");
			Soldado[][] army = new Soldado[10][10];
			for(int i = 0; i < numsoldiers; i++){ //LOS REGISTRAMOS A CADA UNO POR EL ORDEN DE CREACION QUE FUERON CREADOS EL CUAL TAMBIEN COMPLETAMOS SUS DATOS Y LOS PUBLICAMOS POR ORDEN 
				System.out.println("Registrando al " + (i + 1) + " soldado del Ejercito " + num + "");            
				String name = "Soldado" + i + "X" + num;            
				int health = rdm.nextInt(5) + 1;
				int row = rdm.nextInt(10) + 1;
				String column = String.valueOf((char)(rdm.nextInt(10) + 65));  
				if(army[row - 1][(int)column.charAt(0) - 65] == null){ //VERIFICAMOS QUE NO SE REPITAN MISMOS SOLDADOS DE UN EJERCITO EN EL MISMO CUADRADO 
					System.out.print("------------------");
					army[row - 1][(int)column.charAt(0) - 65] = new Soldado(name, health, row, column);
					System.out.println(army[row - 1][(int)column.charAt(0) - 65].toString());
				}else{
					i -= 1;
				}
			}
			return army;
    	}
	\end{lstlisting}
	\begin{lstlisting}[language=bash,caption={Ejecucion:}][H]
		La Ejercito 2 tiene 6 soldados:
		*********************************
		Registrando al 1 soldado del Ejercito 2
		------------------
		Nombre: Soldado0X2
		Vida: 5
		Fila: 8
		Columna: C
		Registrando al 2 soldado del Ejercito 2
		------------------
		Nombre: Soldado1X2
		Vida: 5
		Fila: 2
		Columna: I
		Registrando al 3 soldado del Ejercito 2
		------------------
		Nombre: Soldado2X2
		Vida: 5
		Fila: 5
		Columna: F
		Registrando al 4 soldado del Ejercito 2
		------------------
		Nombre: Soldado3X2
		Vida: 5
		Fila: 2
		Columna: G
		Registrando al 5 soldado del Ejercito 2
		------------------
		Nombre: Soldado4X2
		Vida: 2
		Fila: 10
		Columna: G
		Registrando al 6 soldado del Ejercito 2
		------------------
		Nombre: Soldado5X2
		Vida: 4
		Fila: 9
		Columna: F
	\end{lstlisting}
	\subsection{Ejercicio VideoJuego4}
	\begin{itemize}	
		\item En el tercer commit creamos el metodo arrayListFillRegister() para que este pueda llenar los ArrayList que creamos para el ejercito 1 en este se creara un arraylist con casillas de soldados con datos nulos el cual se va ir llenando aleatoriamente con soldados y a la vez de esto nos mostrara por orden de creacion la informacion de los soldados a la vez tambien permitiriamos que cada casilla no se repita un mismo soldado ya que este sera verificado mediante si este casilla sea diferente a un soldado nulo
		\item El codigo , el commit y la ejecucion seria el siguiente:
	\end{itemize}	
	\begin{lstlisting}[language=bash,caption={Commit}][H]
		$ git commit -m "Creamos el metodo arrayListFillRegister() para que este pueda llenar los ArrayList que creamos para el ejercito 1 en este se creara un arraylist con casillas de soldados con datos nulos el cual se va ir llenando aleatoriamente con soldados y a la vez de esto nos mostrara por orden de creacion la informacion de los soldados a la vez tambien permitiriamos que cada casilla no se repita un mismo soldado ya que este sera verificado mediante si este casilla sea diferente a un soldado nulo"
	\end{lstlisting}	
	\begin{lstlisting}[language=java,caption={Las lineas de codigos del metodo creado:}][H]
		public  static ArrayList<ArrayList<Soldado>> arrayListFillRegister(int num){
			Random rdm = new Random();
			ArrayList<ArrayList<Soldado>> army = new ArrayList<ArrayList<Soldado>>();
			int numbersoldiers = rdm.nextInt(10) + 1;
			for(int i = 0; i < 10; i++){
				army.add(new ArrayList<Soldado>()); //LLENAMOS NUESTROS ARRAYLIST BIDIMENSIONAL CON CADA FILA PARA QUE CUMPLAN CON ESTRUCTURA DEL TABLERO
				for(int j = 0; j < 10 ; j++){
					army.get(i).add(null); // LLENAMOS CADA FILA DEL ARRAYLIST CON UN OBJETO SOLDADO CON TAL QUE ESTE SEA NULL PARA QUE SEPA QUE ESTE TIENE UNA CASILLA PERO NO HAY NADIE TODAVIA SE PUEDE LLENAR 
				}
			}
			System.out.println("El Ejercito " + num + " tiene " + numbersoldiers + " soldados : " ); 
			System.out.println("*********************************");
			for(int i = 0; i < numbersoldiers; i++){ //LLENAMOS CASILLAS CON CADA SOLDADO CREADO ALEATORIAMENTE
				String name = "Soldado" + i + "X" + num;
				int health = rdm.nextInt(5) + 1;
				int row = rdm.nextInt(10) + 1;
				String column = String.valueOf((char)(rdm.nextInt(10) + 65)); //REUTILIZAMOS CODIGO DEL ANTERIOR ARCHIVO VIDEOJUEGO3.JAVA YA QUE TENDRIAN LA MISMA FUNCIONALIDAD
				if(army.get(row - 1).get((int)column.charAt(0) - 65) == null){
					System.out.println("Registrando al " + (i + 1) + " soldado del Ejercito " + num + "");
					System.out.print("------------------");
					army.get(row - 1).set((int)column.charAt(0) - 65, new Soldado(name, health, row, column));
					System.out.println(army.get(row - 1).get((int)column.charAt(0) - 65).toString());
				}else{
					i -= 1; //NOS AYUDARIA CON LOS SOLDADOS QUE SE REPITEN EN EL MISMO CASILLERO CON TAL QUE NO DEBERIA CONTAR 
				}
			}
			System.out.println("*********************************");
			return army;
		}
	\end{lstlisting}
	\begin{lstlisting}[language=bash,caption={Ejecucion:}][H]
		El Ejercito 1 tiene 2 soldados : 
		*********************************
		Registrando al 1 soldado del Ejercito 1
		------------------
		Nombre: Soldado0X1
		Vida: 4
		Fila: 2
		Columna: F
		Registrando al 2 soldado del Ejercito 1
		------------------
		Nombre: Soldado1X1
		Vida: 2
		Fila: 8
		Columna: I
		*********************************

		El Ejercito 2 tiene 4 soldados:
		*********************************
		Registrando al 1 soldado del Ejercito 2
		------------------
		Nombre: Soldado0X2
		Vida: 1
		Fila: 4
		Columna: D
		Registrando al 2 soldado del Ejercito 2
		------------------
		Nombre: Soldado1X2
		Vida: 4
		Fila: 4
		Columna: H
		Registrando al 3 soldado del Ejercito 2
		------------------
		Nombre: Soldado2X2
		Vida: 2
		Fila: 7
		Columna: F
		Registrando al 4 soldado del Ejercito 2
		------------------
		Nombre: Soldado3X2
		Vida: 2
		Fila: 8
		Columna: I
	\end{lstlisting}
	\subsection{Ejercicio VideoJuego4}
	\begin{itemize}	
		\item En el cuarto commit creamos el metodo viewBoard() el cual nos dejaria permitir visualizar la tabla junto a su leyenda , con los soldados de cada ejercito su posicion la cual del ejercito 1 seria una x y el ejercito 2 seria una y en este tambien aplicamos para cada casilla la cual no sea nula poner una x para el ejercito 1 y en caso contrario debria ser del ejercito 2 el cual seria una y y si tambien seria en caso contrario seria nulo
		\item El codigo , el commit y la ejecucion seria el siguiente:
	\end{itemize}	
	\begin{lstlisting}[language=bash,caption={Commit}][H]
		$ git commit -m "Creamos el metodo arrayListFillRegister() para que este pueda llenar los ArrayList que creamos para el ejercito 1 en este se creara un arraylist con casillas de soldados con datos nulos el cual se va ir llenando aleatoriamente con soldados y a la vez de esto nos mostrara por orden de creacion la informacion de los soldados a la vez tambien permitiriamos que cada casilla no se repita un mismo soldado ya que este sera verificado mediante si este casilla sea diferente a un soldado nulo"
	\end{lstlisting}	
	\begin{lstlisting}[language=java,caption={Las lineas de codigos del metodo creado:}][H]
		public  static ArrayList<ArrayList<Soldado>> arrayListFillRegister(int num){
			Random rdm = new Random();
			ArrayList<ArrayList<Soldado>> army = new ArrayList<ArrayList<Soldado>>();
			int numbersoldiers = rdm.nextInt(10) + 1;
			for(int i = 0; i < 10; i++){
				army.add(new ArrayList<Soldado>()); //LLENAMOS NUESTROS ARRAYLIST BIDIMENSIONAL CON CADA FILA PARA QUE CUMPLAN CON ESTRUCTURA DEL TABLERO
				for(int j = 0; j < 10 ; j++){
					army.get(i).add(null); // LLENAMOS CADA FILA DEL ARRAYLIST CON UN OBJETO SOLDADO CON TAL QUE ESTE SEA NULL PARA QUE SEPA QUE ESTE TIENE UNA CASILLA PERO NO HAY NADIE TODAVIA SE PUEDE LLENAR 
				}
			}
			System.out.println("El Ejercito " + num + " tiene " + numbersoldiers + " soldados : " ); 
			System.out.println("*********************************");
			for(int i = 0; i < numbersoldiers; i++){ //LLENAMOS CASILLAS CON CADA SOLDADO CREADO ALEATORIAMENTE
				String name = "Soldado" + i + "X" + num;
				int health = rdm.nextInt(5) + 1;
				int row = rdm.nextInt(10) + 1;
				String column = String.valueOf((char)(rdm.nextInt(10) + 65)); //REUTILIZAMOS CODIGO DEL ANTERIOR ARCHIVO VIDEOJUEGO3.JAVA YA QUE TENDRIAN LA MISMA FUNCIONALIDAD
				if(army.get(row - 1).get((int)column.charAt(0) - 65) == null){
					System.out.println("Registrando al " + (i + 1) + " soldado del Ejercito " + num + "");
					System.out.print("------------------");
					army.get(row - 1).set((int)column.charAt(0) - 65, new Soldado(name, health, row, column));
					System.out.println(army.get(row - 1).get((int)column.charAt(0) - 65).toString());
				}else{
					i -= 1; //NOS AYUDARIA CON LOS SOLDADOS QUE SE REPITEN EN EL MISMO CASILLERO CON TAL QUE NO DEBERIA CONTAR 
				}
			}
			System.out.println("*********************************");
			return army;
		}
	\end{lstlisting}
	\begin{lstlisting}[language=bash,caption={Ejecucion:}][H]
		El Ejercito 1 tiene 5 soldados : 
		*********************************
		Registrando al 1 soldado del Ejercito 1
		------------------
		Nombre: Soldado0X1
		Vida: 4
		Fila: 9
		Columna: C
		Registrando al 2 soldado del Ejercito 1
		------------------
		Nombre: Soldado1X1
		Vida: 2
		Fila: 2
		Columna: E
		Registrando al 3 soldado del Ejercito 1
		------------------
		Nombre: Soldado2X1
		Vida: 5
		Fila: 8
		Columna: I
		Registrando al 4 soldado del Ejercito 1
		------------------
		Nombre: Soldado3X1
		Vida: 3
		Fila: 6
		Columna: G
		Registrando al 5 soldado del Ejercito 1
		------------------
		Nombre: Soldado4X1
		Vida: 1
		Fila: 10
		Columna: C
		*********************************
		
		El Ejercito 2 tiene 10 soldados:
		*********************************
		Registrando al 1 soldado del Ejercito 2
		------------------
		Nombre: Soldado0X2
		Vida: 5
		Fila: 6
		Columna: E
		Registrando al 2 soldado del Ejercito 2
		------------------
		Nombre: Soldado1X2
		Vida: 4
		Fila: 7
		Columna: H
		Registrando al 3 soldado del Ejercito 2
		------------------
		Nombre: Soldado2X2
		Vida: 3
		Fila: 10
		Columna: D
		Registrando al 4 soldado del Ejercito 2
		------------------
		Nombre: Soldado3X2
		Vida: 5
		Fila: 9
		Columna: A
		Registrando al 5 soldado del Ejercito 2
		------------------
		Nombre: Soldado4X2
		Vida: 1
		Fila: 2
		Columna: G
		Registrando al 6 soldado del Ejercito 2
		------------------
		Nombre: Soldado5X2
		Vida: 5
		Fila: 8
		Columna: F
		Registrando al 7 soldado del Ejercito 2
		------------------
		Nombre: Soldado6X2
		Vida: 1
		Fila: 5
		Columna: D
		Registrando al 8 soldado del Ejercito 2
		------------------
		Nombre: Soldado7X2
		Vida: 3
		Fila: 9
		Columna: B
		Registrando al 9 soldado del Ejercito 2
		------------------
		Nombre: Soldado8X2
		Vida: 5
		Fila: 7
		Columna: G
		Registrando al 10 soldado del Ejercito 2
		------------------
		Nombre: Soldado9X2
		Vida: 1
		Fila: 9
		Columna: E
		*********************************
		
		Mostrando tabla de posicion ... --
		Leyenda: Ejercito1 --> X | Ejercito2 --> Y
	\end{lstlisting}
	\begin{figure}[H]
		\centering
		\includegraphics[width=1.0\textwidth,keepaspectratio]{img/Commit4.png}
		%\includesvg{img/automata.svg}
		%\label{img:mot2}
		%\caption{Product backlog.}
	\end{figure}
	\subsection{Ejercicio VideoJuego4}
	\begin{itemize}	
		\item En el quinto commit creamos el metodo arrayListLongerLife() el cual nos permitira conocer del soldado con mayor puntos de vida para esto hacemos una comprobacion con cada uno de estos para poder despues compararlos gracias a una iteracion sobre todos los soldados de este ejercito y despues de tener el soldado con mayor puntos de vida se imprimira sus datos para ver de quien se trata 
		\item El codigo , el commit y la ejecucion seria el siguiente:
	\end{itemize}	
	\begin{lstlisting}[language=bash,caption={Commit}][H]
		$ git commit -m "Creamos el metodo arrayListLongerLife() el cual nos permitira conocer del soldado con mas vida del ejercito 1 para esto hacemos una comprobacion con cada uno de estos soldados el cual vamos iterando por cada uno de estos para poder despues compararlos y despues de tener el soldado con mayor puntos de vida se imprimira sus datos para ver de quien se trata"
	\end{lstlisting}	
	\begin{lstlisting}[language=java,caption={Las lineas de codigos del metodo creado:}][H]
		public static void arrayListLongerLife(ArrayList<ArrayList<Soldado>> army, int num){
			System.out.println("El soldado con mayor vida del Ejercito " + num + " es: "); //METODO CREADO PARA PODER PERMITIRNOS A CONOCER EL SOLDADO CON MAYOR VIDA DE CADA EJERCITO 
			int mayor = 0;
			Soldado soldier = null;
			for(int i = 0; i < army.size(); i++){
				for(int j = 0; j < army.get(i).size(); j++){
					if(army.get(i).get(j) != null){ //COMPROBACION QUE HACEMOS PARA PODER DECIR QUE EL CASILLERO DONDE ESTAMOS ES UN SOLDADO QUE EXISTE
						if(army.get(i).get(j).getHealth() > mayor){ //COMPARAMOS PUNTOS DE VIDA DE CADA SOLDADO PARA VER QUIEN ES EL MAYOR 
							mayor = army.get(i).get(j).getHealth();
							soldier = army.get(i).get(j);
						}
					}
				}
			}
			System.out.println(soldier.toString());//IMPRIMIMOS SUS DATOS PARA PODER VER DE QUE SOLDADO SE TRATA 
			System.out.println("*********************************");
		}
	\end{lstlisting}
	\begin{lstlisting}[language=bash,caption={Ejecucion:}][H]
		El Ejercito 1 tiene 1 soldados : 
		*********************************
		Registrando al 1 soldado del Ejercito 1
		------------------
		Nombre: Soldado0X1
		Vida: 1
		Fila: 10
		Columna: F
		*********************************

		El Ejercito 2 tiene 5 soldados:
		*********************************
		Registrando al 1 soldado del Ejercito 2
		------------------
		Nombre: Soldado0X2
		Vida: 4
		Fila: 2
		Columna: E
		Registrando al 2 soldado del Ejercito 2
		------------------
		Nombre: Soldado1X2
		Vida: 1
		Fila: 1
		Columna: H
		Registrando al 3 soldado del Ejercito 2
		------------------
		Nombre: Soldado2X2
		Vida: 1
		Fila: 9
		Columna: B
		Registrando al 4 soldado del Ejercito 2
		------------------
		Nombre: Soldado3X2
		Vida: 3
		Fila: 9
		Columna: D
		Registrando al 5 soldado del Ejercito 2
		------------------
		Nombre: Soldado4X2
		Vida: 5
		Fila: 4
		Columna: I
		*********************************

		Mostrando tabla de posicion ... --
		Leyenda: Ejercito1 --> X | Ejercito2 --> Y

	\end{lstlisting}
	\begin{figure}[H]
		\centering
		\includegraphics[width=1.0\textwidth,keepaspectratio]{img/Commit5.png}
		%\includesvg{img/automata.svg}
		%\label{img:mot2}
		%\caption{Product backlog.}
	\end{figure}
	\begin{lstlisting}[language=bash,caption={Ejecucion:}][H]
		*********************************
		El soldado con mayor vida del Ejercito 1 es: 
		
		Nombre: Soldado0X1
		Vida: 1
		Fila: 10
		Columna: F
		*********************************
		
	\end{lstlisting}
	\subsection{Ejercicio VideoJuego4}
	\begin{itemize}	
		\item En el sexto commit creamos el metodo arrayLongerLife() el cual este nos permite a dar la informacion del soldado con mayor vida del ejercito 2 el cual comparamos con otros soldados que esten en este ejercito y dependiendo de eso va a recoger informacion del soldado con mayor vida y lo guarda en soldier el cual sera imprimido despues.
		\item El codigo , el commit y la ejecucion seria el siguiente:
	\end{itemize}	
	\begin{lstlisting}[language=bash,caption={Commit}][H]
		$ git commit -m "Creamos el metodo arrayLongerLife() el cual nos imprimira el dato del soldado con mayor vida dependiendo de los que esten en el ejercito que va comparando con los demas para ver quien es el mayor de todos aplicariamos la misma logica que con el metodo arrayListLongerLife() pero en este caso seria para los propios array"
	\end{lstlisting}	
	\begin{lstlisting}[language=java,caption={Las lineas de codigos del metodo creado:}][H]
		public static void arrayLongerLife(Soldado[][] army, int num){
			int mayor = 0;
			Soldado soldier = null; //METODO CREADO QUE NOS VA AYUDAR A DECIRNOS SOBRE LA INFORMACION DEL SOLDADO CON MAYOR VIDAD DEL EJERCITO2 EL CUAL TENDREMOS QUE ITERAR
			for(int i = 0; i < army.length; i++){
				   for(int j = 0; j < army[i].length; j++){//ITERAMOS SOBRE CADA SOLDADO EL CUAL COMPARAMOS CON SI ESTE ES EL MAYOR EN CUESTION DE VIDA VAMOS PASANDO POR CADA SOLDADO
						  if(army[i][j] != null){
								 if(mayor < army[i][j].getHealth()){
										mayor = army[i][j].getHealth();
										soldier = army[i][j];//ACTUALIZAMOS A ESTE SOLDADO CON EL QUE TIENE MAS VIDA DESPUES LO IMPRIMIMOS PARA VER DE QUE SOLDADO SE TRATA 
								 }
						  }
				   }
			}
			System.out.println("");
			System.out.println("El soldado con mayor vida del Ejercito " + num + " es: ");
			System.out.println(soldier.toString());
			System.out.println("*********************************");
		}
	\end{lstlisting}
	\begin{lstlisting}[language=bash,caption={Ejecucion:}][H]
		El Ejercito 1 tiene 9 soldados : 
		*********************************
		Registrando al 1 soldado del Ejercito 1
		------------------
		Nombre: Soldado0X1
		Vida: 3
		Fila: 1
		Columna: E
		Registrando al 2 soldado del Ejercito 1
		------------------
		Nombre: Soldado1X1
		Vida: 3
		Fila: 7
		Columna: J
		Registrando al 3 soldado del Ejercito 1
		------------------
		Nombre: Soldado2X1
		Vida: 1
		Fila: 5
		Columna: D
		Registrando al 4 soldado del Ejercito 1
		------------------
		Nombre: Soldado3X1
		Vida: 4
		Fila: 1
		Columna: D
		Registrando al 5 soldado del Ejercito 1
		------------------
		Nombre: Soldado4X1
		Vida: 1
		Fila: 8
		Columna: E
		Registrando al 6 soldado del Ejercito 1
		------------------
		Nombre: Soldado5X1
		Vida: 3
		Fila: 2
		Columna: E
		Registrando al 7 soldado del Ejercito 1
		------------------
		Nombre: Soldado6X1
		Vida: 2
		Fila: 5
		Columna: A
		Registrando al 8 soldado del Ejercito 1
		------------------
		Nombre: Soldado7X1
		Vida: 4
		Fila: 2
		Columna: H
		Registrando al 9 soldado del Ejercito 1
		------------------
		Nombre: Soldado8X1
		Vida: 1
		Fila: 4
		Columna: C
		*********************************

		El Ejercito 2 tiene 10 soldados:
		*********************************
		Registrando al 1 soldado del Ejercito 2
		------------------
		Nombre: Soldado0X2
		Vida: 4
		Fila: 3
		Columna: H
		Registrando al 2 soldado del Ejercito 2
		------------------
		Nombre: Soldado1X2
		Vida: 2
		Fila: 10
		Columna: G
		Registrando al 3 soldado del Ejercito 2
		------------------
		Nombre: Soldado2X2
		Vida: 4
		Fila: 3
		Columna: C
		Registrando al 4 soldado del Ejercito 2
		------------------
		Nombre: Soldado3X2
		Vida: 1
		Fila: 4
		Columna: J
		Registrando al 5 soldado del Ejercito 2
		------------------
		Nombre: Soldado4X2
		Vida: 2
		Fila: 3
		Columna: E
		Registrando al 6 soldado del Ejercito 2
		------------------
		Nombre: Soldado5X2
		Vida: 1
		Fila: 9
		Columna: H
		Registrando al 7 soldado del Ejercito 2
		------------------
		Nombre: Soldado6X2
		Vida: 2
		Fila: 5
		Columna: I
		Registrando al 8 soldado del Ejercito 2
		------------------
		Nombre: Soldado7X2
		Vida: 2
		Fila: 1
		Columna: H
		Registrando al 9 soldado del Ejercito 2
		------------------
		Nombre: Soldado8X2
		Vida: 5
		Fila: 2
		Columna: D
		Registrando al 10 soldado del Ejercito 2
		------------------
		Nombre: Soldado9X2
		Vida: 2
		Fila: 3
		Columna: J
		*********************************

		Mostrando tabla de posicion ... --
		Leyenda: Ejercito1 --> X | Ejercito2 --> Y

	\end{lstlisting}
	\begin{figure}[H]
		\centering
		\includegraphics[width=1.0\textwidth,keepaspectratio]{img/Commit6.png}
		%\includesvg{img/automata.svg}
		%\label{img:mot2}
		%\caption{Product backlog.}
	\end{figure}
	\begin{lstlisting}[language=bash,caption={Ejecucion:}][H]
		*********************************
		El soldado con mayor vida del Ejercito 1 es: 
		
		Nombre: Soldado3X1
		Vida: 4
		Fila: 1
		Columna: D
		*********************************
		
		El soldado con mayor vida del Ejercito 2 es: 
		
		Nombre: Soldado8X2
		Vida: 5
		Fila: 2
		Columna: D
		*********************************
		
	\end{lstlisting}
	\subsection{Ejercicio VideoJuego4}
	\begin{itemize}	
		\item En el septimo commit creamos el metodo arrayListAverageLife() el cual nos permite dar a conocer el promedio de vida del ejercito 1 para esto debemos contar la cantidad de soldados de cada ejercito y juntar la vida de cada soldado para despues poder dividirlo con la cantidad de soldados el cual seria el promedio de vida para eso vemos la verificacion de cada casilla sea un soldado no nulo para poder recolectando sus puntos de vida y la cantidad de soldados que hay en este ejercito
		\item El codigo , el commit y la ejecucion seria el siguiente:
	\end{itemize}	
	\begin{lstlisting}[language=bash,caption={Commit}][H]
		$ git commit -m "Creamos el metodo arrayListAverageLife() el cual nos permite dar a conocer el promedio de vida del ejercito 1 para esto debemos contar la cantidad de soldados de cada ejercito y juntar la vida de cada soldado para despues poder dividirlo con la cantidad de soldados el cual seria el promedio de vida para eso vemos la verificacion de cada casilla sea un soldado no nulo para poder recolectando sus puntos de vida y la cantidad de soldados que hay en el ejercito 1 el cual los vamos contando dependiendo de cuantos existan debido a una comprobacion"
	\end{lstlisting}	
	\begin{lstlisting}[language=java,caption={Las lineas de codigos del metodo creado:}][H]
		public static void arrayListAverageLife(ArrayList<ArrayList<Soldado>> army, int num){
			int sum = 0;
			int count = 0;
			System.out.println("El promedio de puntos de vida del Ejercito " + num + " es: "); //METODO CREADO QUE NOS PERMITE DAR A CONOCER EL PROMEDIO DE VIDA DE CADA EJERCITO
			for(int i = 0; i < army.size(); i++){
				for(int j = 0; j < army.get(i).size(); j++){
					if(army.get(i).get(j) != null){ //VERIFICAMOS QUE EL SOLDADO DE CADA CASILLA SEA NO NULO 
						sum += army.get(i).get(j).getHealth(); //JUNTAMOS LOS VALORES DE VIDA DE CADA SOLDADO DE CADA EJERCITO 
						count++; //CONTAMOS CANTIDAD DE SOLDADOS DE CADA EJERCITO PARA DESPUES PODER DIVIDIRLO CON LA SUMA DE VIDA DE CADA EJERCITO
					}
				}
			}
			double avg = sum / (count * 1.0);
			System.out.println(avg); // DAMOS A CONOCER EL PROMEDIO DE VIDA DE CADA EJERCITO 
			System.out.println("*********************************");
		}
	\end{lstlisting}
	\begin{lstlisting}[language=bash,caption={Ejecucion:}][H]
		El Ejercito 1 tiene 6 soldados : 
		*********************************
		Registrando al 1 soldado del Ejercito 1
		------------------
		Nombre: Soldado0X1
		Vida: 3
		Fila: 1
		Columna: G
		Registrando al 2 soldado del Ejercito 1
		------------------
		Nombre: Soldado1X1
		Vida: 1
		Fila: 9
		Columna: H
		Registrando al 3 soldado del Ejercito 1
		------------------
		Nombre: Soldado2X1
		Vida: 1
		Fila: 4
		Columna: B
		Registrando al 4 soldado del Ejercito 1
		------------------
		Nombre: Soldado3X1
		Vida: 5
		Fila: 10
		Columna: D
		Registrando al 5 soldado del Ejercito 1
		------------------
		Nombre: Soldado4X1
		Vida: 3
		Fila: 4
		Columna: C
		Registrando al 6 soldado del Ejercito 1
		------------------
		Nombre: Soldado5X1
		Vida: 4
		Fila: 6
		Columna: J
		*********************************
		
		El Ejercito 2 tiene 2 soldados:
		*********************************
		Registrando al 1 soldado del Ejercito 2
		------------------
		Nombre: Soldado0X2
		Vida: 3
		Fila: 9
		Columna: G
		Registrando al 2 soldado del Ejercito 2
		------------------
		Nombre: Soldado1X2
		Vida: 4
		Fila: 5
		Columna: D
		*********************************
		
		Mostrando tabla de posicion ... --
		Leyenda: Ejercito1 --> X | Ejercito2 --> Y		

	\end{lstlisting}
	\begin{figure}[H]
		\centering
		\includegraphics[width=1.0\textwidth,keepaspectratio]{img/Commit7.png}
		%\includesvg{img/automata.svg}
		%\label{img:mot2}
		%\caption{Product backlog.}
	\end{figure}
	\begin{lstlisting}[language=bash,caption={Ejecucion:}][H]
		*********************************
		El soldado con mayor vida del Ejercito 1 es: 
		
		Nombre: Soldado3X1
		Vida: 5
		Fila: 10
		Columna: D
		*********************************
		El soldado con mayor vida del Ejercito 2 es: 
		
		Nombre: Soldado1X2
		Vida: 4
		Fila: 5
		Columna: D
		*********************************
		El promedio de puntos de vida del Ejercito 1 es: 
		2.8333333333333335
		*********************************
		
	\end{lstlisting}
	\subsection{Ejercicio VideoJuego4}
	\begin{itemize}	
		\item En el octavo commit creamos el metodo arrayAverageLife() el cual nos retorna la imprimicion del promedio de vida de todos los soldados del ejercito el cual se va recolectando para despues dividirlo por la cantidad de soldados el cual va a ser decimal para eso necesitamos hacerlo double.
		\item El codigo , el commit y la ejecucion seria el siguiente:
	\end{itemize}	
	\begin{lstlisting}[language=bash,caption={Commit}][H]
		$ git commit -m "Agregamos el metodo arrayAverageLife() el cual nos dara el promedio de vida del ejercito el cual recolecteria la vida de todos y despues se dividiria en la cantidad de soldados el cual lo multiplicamos por 1.0 para que este sea double decimal y sea mas especifico y despues se imprimiria este resultadon"
	\end{lstlisting}	
	\begin{lstlisting}[language=java,caption={Las lineas de codigos del metodo creado:}][H]
		public static void arrayAverageLife(Soldado[][] army, int num){
			int sum = 0;
			int cont = 0;
			for(int i = 0; i < army.length; i++){
				   for(int j = 0; j < army.length; j++){
						  if(army[i][j] != null){
								 sum += army[i][j].getHealth(); //SUMAMOS LA VIDA DE LOS SOLDADOS DEL EJERCITO 2 
								 cont++;//CONTADOR PARA VER CUANTOS SOLDADOS EXISTEN EN ESTE EJERCITO PARA DESPUES PODER DIVIDIRLO CON SU SUMA
						  }
				   }
			}
			double avg = sum / (cont * 1.0);
			System.out.println("El promedio de puntos de vida del Ejercito " + num + " es: " + "\n" + avg);
			System.out.println("*********************************"); // AGREGANDOLO PARA HACER EL SIGUIENTE METODO Y SEPARARLOS
	   	}
	\end{lstlisting}
	\begin{lstlisting}[language=bash,caption={Ejecucion:}][H]
		El Ejercito 1 tiene 2 soldados : 
		*********************************
		Registrando al 1 soldado del Ejercito 1
		------------------
		Nombre: Soldado0X1
		Vida: 1
		Fila: 7
		Columna: D
		Registrando al 2 soldado del Ejercito 1
		------------------
		Nombre: Soldado1X1
		Vida: 5
		Fila: 7
		Columna: F
		*********************************
		
		El Ejercito 2 tiene 3 soldados:
		*********************************
		Registrando al 1 soldado del Ejercito 2
		------------------
		Nombre: Soldado0X2
		Vida: 5
		Fila: 9
		Columna: F
		Registrando al 2 soldado del Ejercito 2
		------------------
		Nombre: Soldado1X2
		Vida: 5
		Fila: 8
		Columna: G
		Registrando al 3 soldado del Ejercito 2
		------------------
		Nombre: Soldado2X2
		Vida: 2
		Fila: 6
		Columna: C
		*********************************
		
		Mostrando tabla de posicion ... --
		Leyenda: Ejercito1 --> X | Ejercito2 --> Y
				

	\end{lstlisting}
	\begin{figure}[H]
		\centering
		\includegraphics[width=1.0\textwidth,keepaspectratio]{img/Commit8.png}
		%\includesvg{img/automata.svg}
		%\label{img:mot2}
		%\caption{Product backlog.}
	\end{figure}
	\begin{lstlisting}[language=bash,caption={Ejecucion:}][H]
		*********************************
		El soldado con mayor vida del Ejercito 1 es: 
		
		Nombre: Soldado1X1
		Vida: 5
		Fila: 7
		Columna: F
		*********************************
		El soldado con mayor vida del Ejercito 2 es: 
		
		Nombre: Soldado1X2
		Vida: 5
		Fila: 8
		Columna: G
		*********************************
		El promedio de puntos de vida del Ejercito 1 es: 
		3.0
		*********************************
		El promedio de puntos de vida del Ejercito 2 es: 
		4.0
		*********************************
		
	\end{lstlisting}
	\subsection{Ejercicio VideoJuego4}
	\begin{itemize}	
		\item En el noveno commit creamos el metodo arrayListRankingBurbujaLife() el cual nos ayudara para poder rankear a nuestros soldados con mas vida con los de menor vida para esto aplicamos el uso de crear un nuevo arraylist para guardar a los soldados el cual recolectara del ejercito 1 y despues intercambiarlos en el metodo burbuja y despues mostrar los resultados de este intercambio con los mensajes de su ranking .
		\item El codigo , el commit y la ejecucion seria el siguiente:
	\end{itemize}	
	\begin{lstlisting}[language=bash,caption={Commit}][H]
		$ git commit -m "Creamos el metodo arrayListRankingBurbujaLife() el cual nos ayudara para poder rankear a nuestros soldados con mas vida con los de menor vida para esto aplicamos el uso de crear un nuevo arraylist para guardar a los soldados el cual recolectara del ejercito 1 y despues intercambiarlos en el metodo burbuja y despues mostrar los resultados de este intercambio con los mensajes de su ranking"
	\end{lstlisting}	
	\begin{lstlisting}[language=java,caption={Las lineas de codigos del metodo creado:}][H]
		public static void arrayListRankingBurbujaLife(ArrayList<ArrayList<Soldado>> army, int num){
			ArrayList<Soldado> fillList = new ArrayList<Soldado>(); //CREAMOS ESTE ARRAYLIST PARA PODER GUARDAR A LOS SOLDADOS EN UN SOLO ARRAYLIST EL CUAL SEA EFECTIVO EL METODO BURBUJA 
			Soldado soldier = null; //SOLDADO CREADO PARA PODER CONTENER EL INTERCAMBIO ENTRE SOLDADOS EN EL METODO BURBUJA
			for(int i = 0; i < army.size(); i++){ //CREAMOS ESTAS SENTENCIAS PARA PODER VERIFICAR EL NUMERO DE SOLDADOS PARA DESPUES PONER EL RANKING DE PUESTOS DE CADA UNO DE ESTOS SOLDADOS
				for(int j = 0; j < army.get(i).size(); j++){
						if(army.get(i).get(j) != null){
								fillList.add(army.get(i).get(j));
						}
				}
			}
			System.out.println("Ordenando a los soldados del Ejercito " + num + " por el metodo burbuja: "); //APLICAMOS EL METODO BURBUJA CON LOS PUNTOS DE VIDA
			for(int i = 0; i < fillList.size() - 1; i++){
				for(int j = 0; j < fillList.size() - i - 1; j++){
					if(fillList.get(j).getHealth() < fillList.get(j + 1).getHealth()){
							soldier = fillList.get(j); //INTERCAMBIO
							fillList.set(j , fillList.get(j + 1));
							fillList.set(j + 1, soldier);
					}
				}      
			}
			System.out.println("------------------------------------------");
			System.out.println("Mostrando Ranking del Ejercito " + num + " ..... ////// --->"); //MOSTRADOR DE RANKING DE LOS SOLDADOS
			for(int i = 0; i < fillList.size(); i++){
				System.out.print("\n" + "Puesto " + (i + 1));
				System.out.println(fillList.get(i).toString());
				System.out.println("------------------");
			}
			System.out.println("*********************************");
	   	}
	\end{lstlisting}
	\begin{lstlisting}[language=bash,caption={Ejecucion:}][H]
		El Ejercito 1 tiene 9 soldados : 
		*********************************
		Registrando al 1 soldado del Ejercito 1
		------------------
		Nombre: Soldado0X1
		Vida: 1
		Fila: 4
		Columna: F
		Registrando al 2 soldado del Ejercito 1
		------------------
		Nombre: Soldado1X1
		Vida: 5
		Fila: 8
		Columna: E
		Registrando al 3 soldado del Ejercito 1
		------------------
		Nombre: Soldado2X1
		Vida: 3
		Fila: 10
		Columna: G
		Registrando al 4 soldado del Ejercito 1
		------------------
		Nombre: Soldado3X1
		Vida: 2
		Fila: 5
		Columna: J
		Registrando al 5 soldado del Ejercito 1
		------------------
		Nombre: Soldado4X1
		Vida: 3
		Fila: 3
		Columna: I
		Registrando al 6 soldado del Ejercito 1
		------------------
		Nombre: Soldado5X1
		Vida: 4
		Fila: 4
		Columna: A
		Registrando al 7 soldado del Ejercito 1
		------------------
		Nombre: Soldado6X1
		Vida: 1
		Fila: 4
		Columna: B
		Registrando al 8 soldado del Ejercito 1
		------------------
		Nombre: Soldado7X1
		Vida: 5
		Fila: 1
		Columna: J
		Registrando al 9 soldado del Ejercito 1
		------------------
		Nombre: Soldado8X1
		Vida: 2
		Fila: 3
		Columna: H
		*********************************

		El Ejercito 2 tiene 10 soldados:
		*********************************
		Registrando al 1 soldado del Ejercito 2
		------------------
		Nombre: Soldado0X2
		Vida: 3
		Fila: 7
		Columna: F
		Registrando al 2 soldado del Ejercito 2
		------------------
		Nombre: Soldado1X2
		Vida: 5
		Fila: 1
		Columna: I
		Registrando al 3 soldado del Ejercito 2
		------------------
		Nombre: Soldado2X2
		Vida: 5
		Fila: 1
		Columna: E
		Registrando al 4 soldado del Ejercito 2
		------------------
		Nombre: Soldado3X2
		Vida: 5
		Fila: 10
		Columna: F
		Registrando al 5 soldado del Ejercito 2
		------------------
		Nombre: Soldado4X2
		Vida: 1
		Fila: 1
		Columna: G
		Registrando al 6 soldado del Ejercito 2
		------------------
		Nombre: Soldado5X2
		Vida: 4
		Fila: 9
		Columna: H
		Registrando al 7 soldado del Ejercito 2
		------------------
		Nombre: Soldado6X2
		Vida: 4
		Fila: 4
		Columna: C
		Registrando al 8 soldado del Ejercito 2
		------------------
		Nombre: Soldado7X2
		Vida: 5
		Fila: 7
		Columna: H
		Registrando al 9 soldado del Ejercito 2
		------------------
		Nombre: Soldado8X2
		Vida: 2
		Fila: 3
		Columna: B
		Registrando al 10 soldado del Ejercito 2
		------------------
		Nombre: Soldado9X2
		Vida: 4
		Fila: 3
		Columna: E
		*********************************

		Mostrando tabla de posicion ... --
		Leyenda: Ejercito1 --> X | Ejercito2 --> Y

	\end{lstlisting}
	\begin{figure}[H]
		\centering
		\includegraphics[width=1.0\textwidth,keepaspectratio]{img/Commit9.png}
		%\includesvg{img/automata.svg}
		%\label{img:mot2}
		%\caption{Product backlog.}
	\end{figure}
	\begin{lstlisting}[language=bash,caption={Ejecucion:}][H]
		*********************************
		El soldado con mayor vida del Ejercito 1 es: 
		
		Nombre: Soldado7X1
		Vida: 5
		Fila: 1
		Columna: J
		*********************************
		El soldado con mayor vida del Ejercito 2 es: 
		
		Nombre: Soldado2X2
		Vida: 5
		Fila: 1
		Columna: E
		*********************************
		El promedio de puntos de vida del Ejercito 1 es: 
		2.888888888888889
		*********************************
		El promedio de puntos de vida del Ejercito 2 es: 
		3.8
		*********************************
		Ordenando a los soldados del Ejercito 1 por el metodo burbuja: 
		------------------------------------------
		Mostrando Ranking del Ejercito 1 ..... ////// --->
		
		Puesto 1
		Nombre: Soldado7X1
		Vida: 5
		Fila: 1
		Columna: J
		------------------
		
		Puesto 2
		Nombre: Soldado1X1
		Vida: 5
		Fila: 8
		Columna: E
		------------------
		
		Puesto 3
		Nombre: Soldado5X1
		Vida: 4
		Fila: 4
		Columna: A
		------------------
		
		Puesto 4
		Nombre: Soldado4X1
		Vida: 3
		Fila: 3
		Columna: I
		------------------
		
		Puesto 5
		Nombre: Soldado2X1
		Vida: 3
		Fila: 10
		Columna: G
		------------------
		
		Puesto 6
		Nombre: Soldado8X1
		Vida: 2
		Fila: 3
		Columna: H
		------------------
		
		Puesto 7
		Nombre: Soldado3X1
		Vida: 2
		Fila: 5
		Columna: J
		------------------
		
		Puesto 8
		Nombre: Soldado6X1
		Vida: 1
		Fila: 4
		Columna: B
		------------------
		
		Puesto 9
		Nombre: Soldado0X1
		Vida: 1
		Fila: 4
		Columna: F
		------------------
		*********************************		
		
	\end{lstlisting}
	\subsection{Ejercicio VideoJuego4}
	\begin{itemize}	
		\item En el decimo commit creamos el metodo arrayRankingBurbujaLife() el cual aplicamos que en un arreglo unidimensional ponemos todos los soldados del arreglo bidimensional para asi poderaplicar efectivamente el metodo burbuja el cual aplicariamos con el atributo health el cual nos dara un ranking que despues se imprimira los resultados del ranking del mayor vida al menor que se imprimiran .
		\item El codigo , el commit y la ejecucion seria el siguiente:
	\end{itemize}	
	\begin{lstlisting}[language=bash,caption={Commit}][H]
		$ git commit -m "Creamos el metodo arrayRankingBurbujaLife() el cual aplicamos que en un arreglo unidimensional ponemos todos los soldados del arreglo bidimensional para asi poder aplicar efectivamente el metodo burbuja el cual aplicariamos con el atributo health el cual nos dara un ranking que despues se imprimira los resultados del ranking del mayor vida al menor que se imprimiran"
	\end{lstlisting}	
	\begin{lstlisting}[language=java,caption={Las lineas de codigos del metodo creado:}][H]
		public static void arrayRankingBurbujaLife(Soldado[][] army, int num){
			int numsoldiers = 0;
			int count = 0;
			Soldado soldier = null;
			for(int i = 0; i < army.length; i++){ //CREAMOS UN CONTADOR PARA SABER EL NUMERO DE SOLDADOS DE ESTE EJERCITO Y DESPUES CREAR UN ARREGLO EL CUAL PODEAMOS DARLE ESTE TAMANO Y LO PODAMOS USAR PARA EL METODO BURBUJA
				for(int j = 0; j < army[i].length; j++){
						if(army[i][j] != null){
							numsoldiers++;
						}
				}
			}
			Soldado[] soldiers = new Soldado[numsoldiers];
			for(int i = 0; i < army.length; i++){ //CREAMOS ARREGLO PARA QUE LOS SOLDADOS SE TRASLADEN DE UN ARREGLO BIDIMENSIONAL A UNO UNIDIMENSIONAL PARA APLICAR EL METODO BURBUJA
				for(int j = 0; j < army[i].length; j++){
						if(army[i][j] != null){
							soldiers[count] = army[i][j];
							count++;
						}
				}
			}
			System.out.println("Ordenando a los soldados del Ejercito " + num + " por el metodo burbuja: ");//APLICAMOS EL METODO BURBUJA CON LOS PUNTOS DE VIDA
			for(int i = 0; i < numsoldiers - 1; i++){
				for(int j = 0; j < numsoldiers - i - 1; j++){
						if(soldiers[j].getHealth() < soldiers[j + 1].getHealth()){
								soldier = soldiers[j];
								soldiers[j] = soldiers[j + 1];
								soldiers[j + 1] = soldier;
						}
				}      
			}
			System.out.println("------------------------------------------");
			System.out.println("Mostrando Ranking del Ejercito " + num + " ..... ////// --->"); //DAMOS A CONOCER EL RANKING DE ESTE EJERCITO  EL CUAL LO PUBLICAMOS MEDIANTE PUESTOS 
			for(int i = 0; i < soldiers.length; i++){
				System.out.print("\n" + "Puesto " + (i + 1));
				System.out.println(soldiers[i].toString()); //PUBLICAMOS INFORMACION DE CADA SOLDADO
				System.out.println("------------------");
			}
			System.out.println("*********************************");
		}
	\end{lstlisting}
	\begin{lstlisting}[language=bash,caption={Ejecucion:}][H]
		El Ejercito 1 tiene 6 soldados : 
		*********************************
		Registrando al 1 soldado del Ejercito 1
		------------------
		Nombre: Soldado0X1
		Vida: 5
		Fila: 3
		Columna: B
		Registrando al 2 soldado del Ejercito 1
		------------------
		Nombre: Soldado1X1
		Vida: 4
		Fila: 7
		Columna: A
		Registrando al 3 soldado del Ejercito 1
		------------------
		Nombre: Soldado2X1
		Vida: 5
		Fila: 8
		Columna: D
		Registrando al 4 soldado del Ejercito 1
		------------------
		Nombre: Soldado3X1
		Vida: 1
		Fila: 9
		Columna: B
		Registrando al 5 soldado del Ejercito 1
		------------------
		Nombre: Soldado4X1
		Vida: 2
		Fila: 8
		Columna: G
		Registrando al 6 soldado del Ejercito 1
		------------------
		Nombre: Soldado5X1
		Vida: 4
		Fila: 2
		Columna: A
		*********************************
		
		El Ejercito 2 tiene 3 soldados:
		*********************************
		Registrando al 1 soldado del Ejercito 2
		------------------
		Nombre: Soldado0X2
		Vida: 3
		Fila: 1
		Columna: E
		Registrando al 2 soldado del Ejercito 2
		------------------
		Nombre: Soldado1X2
		Vida: 4
		Fila: 10
		Columna: B
		Registrando al 3 soldado del Ejercito 2
		------------------
		Nombre: Soldado2X2
		Vida: 1
		Fila: 9
		Columna: H
		*********************************
		
		Mostrando tabla de posicion ... --
		Leyenda: Ejercito1 --> X | Ejercito2 --> Y
		
	\end{lstlisting}
	\begin{figure}[H]
		\centering
		\includegraphics[width=1.0\textwidth,keepaspectratio]{img/Commit10.png}
		%\includesvg{img/automata.svg}
		%\label{img:mot2}
		%\caption{Product backlog.}
	\end{figure}
	\begin{lstlisting}[language=bash,caption={Ejecucion:}][H]
		*********************************
		El soldado con mayor vida del Ejercito 1 es: 
		
		Nombre: Soldado0X1
		Vida: 5
		Fila: 3
		Columna: B
		*********************************
		El soldado con mayor vida del Ejercito 2 es: 
		
		Nombre: Soldado1X2
		Vida: 4
		Fila: 10
		Columna: B
		*********************************
		El promedio de puntos de vida del Ejercito 1 es: 
		3.5
		*********************************
		El promedio de puntos de vida del Ejercito 2 es: 
		2.6666666666666665
		*********************************
		Ordenando a los soldados del Ejercito 1 por el metodo burbuja: 
		------------------------------------------
		Mostrando Ranking del Ejercito 1 ..... ////// --->
		
		Puesto 1
		Nombre: Soldado0X1
		Vida: 5
		Fila: 3
		Columna: B
		------------------
		
		Puesto 2
		Nombre: Soldado2X1
		Vida: 5
		Fila: 8
		Columna: D
		------------------
		
		Puesto 3
		Nombre: Soldado5X1
		Vida: 4
		Fila: 2
		Columna: A
		------------------
		
		Puesto 4
		Nombre: Soldado1X1
		Vida: 4
		Fila: 7
		Columna: A
		------------------
		
		Puesto 5
		Nombre: Soldado4X1
		Vida: 2
		Fila: 8
		Columna: G
		------------------
		
		Puesto 6
		Nombre: Soldado3X1
		Vida: 1
		Fila: 9
		Columna: B
		------------------
		*********************************
		Ordenando a los soldados del Ejercito 2 por el metodo burbuja: 
		------------------------------------------
		Mostrando Ranking del Ejercito 2 ..... ////// --->
		
		Puesto 1
		Nombre: Soldado1X2
		Vida: 4
		Fila: 10
		Columna: B
		------------------
		
		Puesto 2
		Nombre: Soldado0X2
		Vida: 3
		Fila: 1
		Columna: E
		------------------
		
		Puesto 3
		Nombre: Soldado2X2
		Vida: 1
		Fila: 9
		Columna: H
		------------------
		*********************************
		
	\end{lstlisting}
	\subsection{Ejercicio VideoJuego4}
	\begin{itemize}	
		\item En el decimoprimero commit creamos el metodo arrayListRankingInsercionLife() el cual aplicariamos la logica del ordenamiento en insercion de soldados debida para aplicar efectivamente este metodo.
		\item El codigo , el commit y la ejecucion seria el siguiente:
	\end{itemize}	
	\begin{lstlisting}[language=bash,caption={Commit}][H]
		$ git commit -m "En el metodo arrayListRankingInsercionLife() el cual aplicariamos la logica del ordenamiento en insercion de soldados debida para aplicar efectivamente este metodo"
	\end{lstlisting}	
	\begin{lstlisting}[language=java,caption={Las lineas de codigos del metodo creado:}][H]
		public static void arrayListRankingInsercionLife(ArrayList<ArrayList<Soldado>> army, int num){
			ArrayList<Soldado> fillList = new ArrayList<Soldado>(); //CREAMOS ESTE ARRAYLIST PARA PODER GUARDAR A LOS SOLDADOS EN UN SOLO ARRAYLIST EL CUAL SEA EFECTIVO EL METODO INSERCION 
			for(int i = 0; i < army.size(); i++){ //CREAMOS ESTAS SENTENCIAS PARA PODER VERIFICAR EL NUMERO DE SOLDADOS Y TAMBIEN ANADIRLO EN EL ARRAYLIST CREADO PARA DESPUES PONER EL RANKING DE PUESTOS DE CADA UNO DE ESTOS SOLDADOS
				   for(int j = 0; j < army.get(i).size(); j++){
						  if(army.get(i).get(j) != null){
								fillList.add(army.get(i).get(j));
						  }
				   }
			}
			System.out.println("Ordenando a los soldados del Ejercito " + num + " por el insercion: "); //APLICAMOS EL METODO INSERCION CON LOS PUNTOS DE VIDA
			for(int i = 1; i < fillList.size(); i++){
				Soldado soldier = fillList.get(i);
				int j = i - 1;
				while(j >= 0 && (soldier.getHealth() > fillList.get(j).getHealth())){ //APLICAMOS EL METODO INSERCION
					fillList.set(j + 1, fillList.get(j));
					j--;
				}
				fillList.set(j + 1, soldier);
			}
			System.out.println("Mostrando Ranking del Ejercito " + num + "....."); //MOSTRADOR DE RANKING DE LOS SOLDADOS
			for(int i = 0; i < fillList.size(); i++){
				   System.out.print("\n" + "Puesto " + (i + 1));
				   System.out.println(fillList.get(i).toString()); //DAMOS A CONOCER SUS DATOS Y EL PUESTO EN EL RANKING
				   System.out.println("------------------");
			}
			System.out.println("*********************************");
		}
	\end{lstlisting}
	\begin{lstlisting}[language=bash,caption={Ejecucion:}][H]
		El Ejercito 1 tiene 1 soldados : 
		*********************************
		Registrando al 1 soldado del Ejercito 1
		------------------
		Nombre: Soldado0X1
		Vida: 2
		Fila: 1
		Columna: J
		*********************************

		El Ejercito 2 tiene 10 soldados:
		*********************************
		Registrando al 1 soldado del Ejercito 2
		------------------
		Nombre: Soldado0X2
		Vida: 4
		Fila: 5
		Columna: F
		Registrando al 2 soldado del Ejercito 2
		------------------
		Nombre: Soldado1X2
		Vida: 1
		Fila: 5
		Columna: E
		Registrando al 3 soldado del Ejercito 2
		------------------
		Nombre: Soldado2X2
		Vida: 4
		Fila: 1
		Columna: C
		Registrando al 4 soldado del Ejercito 2
		------------------
		Nombre: Soldado3X2
		Vida: 2
		Fila: 1
		Columna: B
		Registrando al 5 soldado del Ejercito 2
		------------------
		Nombre: Soldado4X2
		Vida: 2
		Fila: 3
		Columna: G
		Registrando al 6 soldado del Ejercito 2
		------------------
		Nombre: Soldado5X2
		Vida: 1
		Fila: 2
		Columna: A
		Registrando al 7 soldado del Ejercito 2
		------------------
		Nombre: Soldado6X2
		Vida: 5
		Fila: 4
		Columna: D
		Registrando al 8 soldado del Ejercito 2
		------------------
		Nombre: Soldado7X2
		Vida: 5
		Fila: 4
		Columna: H
		Registrando al 9 soldado del Ejercito 2
		------------------
		Nombre: Soldado8X2
		Vida: 2
		Fila: 3
		Columna: E
		Registrando al 10 soldado del Ejercito 2
		------------------
		Nombre: Soldado9X2
		Vida: 4
		Fila: 10
		Columna: I
		*********************************

		Mostrando tabla de posicion ... --
		Leyenda: Ejercito1 --> X | Ejercito2 --> Y

	\end{lstlisting}
	\begin{figure}[H]
		\centering
		\includegraphics[width=1.0\textwidth,keepaspectratio]{img/Commit11.png}
		%\includesvg{img/automata.svg}
		%\label{img:mot2}
		%\caption{Product backlog.}
	\end{figure}
	\begin{lstlisting}[language=bash,caption={Ejecucion:}][H]
		*********************************
		El soldado con mayor vida del Ejercito 1 es: 

		Nombre: Soldado0X1
		Vida: 2
		Fila: 1
		Columna: J
		*********************************
		El soldado con mayor vida del Ejercito 2 es: 

		Nombre: Soldado6X2
		Vida: 5
		Fila: 4
		Columna: D
		*********************************
		El promedio de puntos de vida del Ejercito 1 es: 
		2.0
		*********************************
		El promedio de puntos de vida del Ejercito 2 es: 
		3.0
		*********************************
		Ordenando a los soldados del Ejercito 1 por el metodo burbuja: 
		------------------------------------------
		Mostrando Ranking del Ejercito 1 ..... ////// --->

		Puesto 1
		Nombre: Soldado0X1
		Vida: 2
		Fila: 1
		Columna: J
		------------------
		*********************************
		Ordenando a los soldados del Ejercito 2 por el metodo burbuja: 
		------------------------------------------
		Mostrando Ranking del Ejercito 2 ..... ////// --->

		Puesto 1
		Nombre: Soldado6X2
		Vida: 5
		Fila: 4
		Columna: D
		------------------

		Puesto 2
		Nombre: Soldado7X2
		Vida: 5
		Fila: 4
		Columna: H
		------------------

		Puesto 3
		Nombre: Soldado2X2
		Vida: 4
		Fila: 1
		Columna: C
		------------------

		Puesto 4
		Nombre: Soldado0X2
		Vida: 4
		Fila: 5
		Columna: F
		------------------

		Puesto 5
		Nombre: Soldado9X2
		Vida: 4
		Fila: 10
		Columna: I
		------------------

		Puesto 6
		Nombre: Soldado3X2
		Vida: 2
		Fila: 1
		Columna: B
		------------------

		Puesto 7
		Nombre: Soldado8X2
		Vida: 2
		Fila: 3
		Columna: E
		------------------

		Puesto 8
		Nombre: Soldado4X2
		Vida: 2
		Fila: 3
		Columna: G
		------------------

		Puesto 9
		Nombre: Soldado5X2
		Vida: 1
		Fila: 2
		Columna: A
		------------------

		Puesto 10
		Nombre: Soldado1X2
		Vida: 1
		Fila: 5
		Columna: E
		------------------
		*********************************
		Ordenando a los soldados del Ejercito 1 por el insercion: 
		Mostrando Ranking del Ejercito 1.....

		Puesto 1
		Nombre: Soldado0X1
		Vida: 2
		Fila: 1
		Columna: J
		------------------
		*********************************

	\end{lstlisting}
	\subsection{Ejercicio VideoJuego4}
	\begin{itemize}	
		\item En el decimosegundo commit Creamos el metodo arrayRankingInsercionLife() el cual aplicamos que en un arreglo unidimensional ponemos todos los soldados del arreglo bidimensional para asi poderaplicar efectivamente el metodo insercion el cual aplicariamos con el atributo health el cual nos dara un ranking debido a la logica usada en este metodo para poder ordenarlo y que despues se imprimira los resultados del ranking del mayor vida al menor que se imprimiran
		\item El codigo , el commit y la ejecucion seria el siguiente:
	\end{itemize}	
	\begin{lstlisting}[language=bash,caption={Commit}][H]
		$ git commit -m "creamos el metodo arrayRankingInsercionLife() el cual aplicamos que en un arreglo unidimensional ponemos todos los soldados del arreglo bidimensional para asi poderaplicar efectivamente el metodo insercion el cual aplicariamos con el atributo health el cual nos dara un ranking debido a la logica usada en este metodo para poder ordenarlo y que despues se imprimira los resultados del ranking del mayor vida al menor que se imprimiran"
	\end{lstlisting}	
	\begin{lstlisting}[language=java,caption={Las lineas de codigos del metodo creado:}][H]
		public static void arrayRankingInsercionLife(Soldado[][] army, int num){
			int numsoldiers = 0;
			int count = 0;
			for(int i = 0; i < army.length; i++){ //CREAMOS UN CONTADOR PARA SABER EL NUMERO DE SOLDADOS DE ESTE EJERCITO Y DESPUES CREAR UN ARREGLO EL CUAL PODEAMOS DARLE ESTE TAMANO Y LO PODAMOS USAR PARA EL METODO BURBUJA
				for(int j = 0; j < army[i].length; j++){
						if(army[i][j] != null){
							numsoldiers++;
						}
				}
			}
			Soldado[] soldiers = new Soldado[numsoldiers];
			for(int i = 0; i < army.length; i++){ //CREAMOS ARREGLO PARA QUE LOS SOLDADOS SE TRASLADEN DE UN ARREGLO BIDIMENSIONAL A UNO DIMENSIONAL PARA APLICAR EL METODO INSERCION
				   for(int j = 0; j < army[i].length; j++){
						  if(army[i][j] != null){
								 soldiers[count] = army[i][j];
								 count++;
						  }
				   }
			}
			System.out.println("Ordenando a los soldados del Ejercito " + num + " por el insercion: ");  //APLICAMOS EL METODO INSERCION CON LOS PUNTOS DE VIDA
			for(int i = 1; i < soldiers.length; i++){
				   Soldado temp = soldiers[i];
				   int j = i - 1;
				   while(j >= 0 && (temp.getHealth() > soldiers[j].getHealth())){
					   soldiers[j + 1] = soldiers[j];
					   j--;
				   }
				   soldiers[j + 1] = temp;
			}
			System.out.println("------------------------------------------");
			System.out.println("Mostrando Ranking del Ejercito " + num + ".....");
			for(int i = 0; i < soldiers.length; i++){
				   System.out.print("\n" + "Puesto " + (i + 1));
				   System.out.println(soldiers[i].toString());
				   System.out.println("------------------");
			}
			System.out.println("*********************************");
	   }
	\end{lstlisting}
	\begin{lstlisting}[language=bash,caption={Ejecucion:}][H]
		El Ejercito 1 tiene 1 soldados : 
		*********************************
		Registrando al 1 soldado del Ejercito 1
		------------------
		Nombre: Soldado0X1
		Vida: 1
		Fila: 9
		Columna: D
		*********************************

		El Ejercito 2 tiene 4 soldados:
		*********************************
		Registrando al 1 soldado del Ejercito 2
		------------------
		Nombre: Soldado0X2
		Vida: 2
		Fila: 5
		Columna: B
		Registrando al 2 soldado del Ejercito 2
		------------------
		Nombre: Soldado1X2
		Vida: 2
		Fila: 7
		Columna: G
		Registrando al 3 soldado del Ejercito 2
		------------------
		Nombre: Soldado2X2
		Vida: 2
		Fila: 3
		Columna: J
		Registrando al 4 soldado del Ejercito 2
		------------------
		Nombre: Soldado3X2
		Vida: 1
		Fila: 10
		Columna: I
		*********************************

		Mostrando tabla de posicion ... --
		Leyenda: Ejercito1 --> X | Ejercito2 --> Y


	\end{lstlisting}
	\begin{figure}[H]
		\centering
		\includegraphics[width=1.0\textwidth,keepaspectratio]{img/Commit12.png}
		%\includesvg{img/automata.svg}
		%\label{img:mot2}
		%\caption{Product backlog.}
	\end{figure}
	\begin{lstlisting}[language=bash,caption={Ejecucion:}][H]
		*********************************
		El soldado con mayor vida del Ejercito 1 es: 
		
		Nombre: Soldado0X1
		Vida: 1
		Fila: 9
		Columna: D
		*********************************
		El soldado con mayor vida del Ejercito 2 es: 
		
		Nombre: Soldado2X2
		Vida: 2
		Fila: 3
		Columna: J
		*********************************
		El promedio de puntos de vida del Ejercito 1 es: 
		1.0
		*********************************
		El promedio de puntos de vida del Ejercito 2 es: 
		1.75
		*********************************
		Ordenando a los soldados del Ejercito 1 por el metodo burbuja: 
		------------------------------------------
		Mostrando Ranking del Ejercito 1 ..... ////// --->
		
		Puesto 1
		Nombre: Soldado0X1
		Vida: 1
		Fila: 9
		Columna: D
		------------------
		*********************************
		Ordenando a los soldados del Ejercito 2 por el metodo burbuja: 
		------------------------------------------
		Mostrando Ranking del Ejercito 2 ..... ////// --->
		
		Puesto 1
		Nombre: Soldado2X2
		Vida: 2
		Fila: 3
		Columna: J
		------------------
		
		Puesto 2
		Nombre: Soldado0X2
		Vida: 2
		Fila: 5
		Columna: B
		------------------
		
		Puesto 3
		Nombre: Soldado1X2
		Vida: 2
		Fila: 7
		Columna: G
		------------------
		
		Puesto 4
		Nombre: Soldado3X2
		Vida: 1
		Fila: 10
		Columna: I
		------------------
		*********************************
		Ordenando a los soldados del Ejercito 1 por el insercion: 
		Mostrando Ranking del Ejercito 1.....
		
		Puesto 1
		Nombre: Soldado0X1
		Vida: 1
		Fila: 9
		Columna: D
		------------------
		*********************************
		Ordenando a los soldados del Ejercito 2 por el insercion: 
		------------------------------------------
		Mostrando Ranking del Ejercito 2.....
		
		Puesto 1
		Nombre: Soldado2X2
		Vida: 2
		Fila: 3
		Columna: J
		------------------
		
		Puesto 2
		Nombre: Soldado0X2
		Vida: 2
		Fila: 5
		Columna: B
		------------------
		
		Puesto 3
		Nombre: Soldado1X2
		Vida: 2
		Fila: 7
		Columna: G
		------------------
		
		Puesto 4
		Nombre: Soldado3X2
		Vida: 1
		Fila: 10
		Columna: I
		------------------
		*********************************
		
	\end{lstlisting}
	\subsection{Ejercicio VideoJuego4}
	\begin{itemize}	
		\item En el decimotercero commit creamos el metodo resultBattle() el cual nos dara el resultado de la batalla entre estos 2 ejercitos por lo que nuestra decision fue poner a comparacion el nivel de fuerza de cada ejercito el cual se va sumando para despues dar el resultado de la batalla
		\item El codigo , el commit y la ejecucion seria el siguiente:
	\end{itemize}	
	\begin{lstlisting}[language=bash,caption={Commit}][H]
		$ git commit -m "el metodo resultBattle() el cual nos dara el resultado de la batalla entre estos 2 ejercitos por lo que nuestra decision fue poner a comparacion el nivel de fuerza de cada ejercito el cual se va sumando para despues dar el resultado de la batalla"
	\end{lstlisting}	
	\begin{lstlisting}[language=java,caption={Las lineas de codigos del metodo creado:}][H]
		public static void resultBattle(ArrayList<ArrayList<Soldado>> army1, Soldado[][] army2, int num, int num2){
			int sumarmy1 = 0;
			int sumarmy2 = 0;
			System.out.println("El resultado de esta Batalla se decidio por el nivel de fuerza de cada ejercito por lo que el resultado es: ...");
			for(int i = 0; i < army1.size(); i++){ //METODO CREADO QUE NOS PERMITE DAR CON UN GANADOR ESTO GRACIAS AL NIVEL DE PUNTOS DE VIDA O FUERZA DE CADA EJERCITO EL CUAL VAMOS SUMANDO DE CADA EJERCITO PARA DESPUES COMPARARLOS Y DECIDIR EL RESULTADO DE ESTA BATALLA
				for(int j = 0 ; j < army1.get(i).size(); j++){
					if(army1.get(i).get(j) != null){
						sumarmy1 += army1.get(i).get(j).getHealth(); //SUMA DE PUNTOS DEL EJERCITO 1
					}
				}
			}
			for(int i = 0; i < army2.length; i++){
				for(int j = 0 ; j < army2[i].length; j++){
					if(army2[i][j] != null){
						sumarmy2 += army2[i][j].getHealth(); //SUMA DE PUNTOS DEL EJERCITO 2
					}
				}
			}
			if(sumarmy1 > sumarmy2){ //PUBLICACION DE LOS RESULTADOS
				System.out.println("El Ejercito " + num + " es el GANADOR con " + sumarmy1 + " puntos");
			}else if(sumarmy2 > sumarmy1){
				System.out.println("El Ejercito " + num2 + " es el GANADOR con " + sumarmy2 + " puntos");
			}else{
				System.out.println("EMPATE con " + sumarmy1 + " puntos");
			}
		}
	\end{lstlisting}
	\begin{lstlisting}[language=bash,caption={Ejecucion:}][H]
		El Ejercito 1 tiene 8 soldados : 
		*********************************
		Registrando al 1 soldado del Ejercito 1
		------------------
		Nombre: Soldado0X1
		Vida: 3
		Fila: 8
		Columna: B
		Registrando al 2 soldado del Ejercito 1
		------------------
		Nombre: Soldado1X1
		Vida: 5
		Fila: 1
		Columna: C
		Registrando al 3 soldado del Ejercito 1
		------------------
		Nombre: Soldado2X1
		Vida: 1
		Fila: 3
		Columna: E
		Registrando al 4 soldado del Ejercito 1
		------------------
		Nombre: Soldado3X1
		Vida: 3
		Fila: 8
		Columna: G
		Registrando al 5 soldado del Ejercito 1
		------------------
		Nombre: Soldado4X1
		Vida: 1
		Fila: 7
		Columna: I
		Registrando al 6 soldado del Ejercito 1
		------------------
		Nombre: Soldado5X1
		Vida: 4
		Fila: 6
		Columna: F
		Registrando al 7 soldado del Ejercito 1
		------------------
		Nombre: Soldado6X1
		Vida: 1
		Fila: 4
		Columna: C
		Registrando al 8 soldado del Ejercito 1
		------------------
		Nombre: Soldado7X1
		Vida: 3
		Fila: 3
		Columna: F
		*********************************
		
		El Ejercito 2 tiene 1 soldados:
		*********************************
		Registrando al 1 soldado del Ejercito 2
		------------------
		Nombre: Soldado0X2
		Vida: 5
		Fila: 6
		Columna: A
		*********************************
		
		Mostrando tabla de posicion ... --
		Leyenda: Ejercito1 --> X | Ejercito2 --> Y		

	\end{lstlisting}
	\begin{figure}[H]
		\centering
		\includegraphics[width=1.0\textwidth,keepaspectratio]{img/Commit13.png}
		%\includesvg{img/automata.svg}
		%\label{img:mot2}
		%\caption{Product backlog.}
	\end{figure}
	\begin{lstlisting}[language=bash,caption={Ejecucion:}][H]
		*********************************
		El soldado con mayor vida del Ejercito 1 es: 
		
		Nombre: Soldado1X1
		Vida: 5
		Fila: 1
		Columna: C
		*********************************
		El soldado con mayor vida del Ejercito 2 es: 
		
		Nombre: Soldado0X2
		Vida: 5
		Fila: 6
		Columna: A
		*********************************
		El promedio de puntos de vida del Ejercito 1 es: 
		2.625
		*********************************
		El promedio de puntos de vida del Ejercito 2 es: 
		5.0
		*********************************
		Ordenando a los soldados del Ejercito 1 por el metodo burbuja: 
		------------------------------------------
		Mostrando Ranking del Ejercito 1 ..... ////// --->
		
		Puesto 1
		Nombre: Soldado1X1
		Vida: 5
		Fila: 1
		Columna: C
		------------------
		
		Puesto 2
		Nombre: Soldado5X1
		Vida: 4
		Fila: 6
		Columna: F
		------------------
		
		Puesto 3
		Nombre: Soldado7X1
		Vida: 3
		Fila: 3
		Columna: F
		------------------
		
		Puesto 4
		Nombre: Soldado0X1
		Vida: 3
		Fila: 8
		Columna: B
		------------------
		
		Puesto 5
		Nombre: Soldado3X1
		Vida: 3
		Fila: 8
		Columna: G
		------------------
		
		Puesto 6
		Nombre: Soldado2X1
		Vida: 1
		Fila: 3
		Columna: E
		------------------
		
		Puesto 7
		Nombre: Soldado6X1
		Vida: 1
		Fila: 4
		Columna: C
		------------------
		
		Puesto 8
		Nombre: Soldado4X1
		Vida: 1
		Fila: 7
		Columna: I
		------------------
		*********************************
		Ordenando a los soldados del Ejercito 2 por el metodo burbuja: 
		------------------------------------------
		Mostrando Ranking del Ejercito 2 ..... ////// --->
		
		Puesto 1
		Nombre: Soldado0X2
		Vida: 5
		Fila: 6
		Columna: A
		------------------
		*********************************
		Ordenando a los soldados del Ejercito 1 por el insercion: 
		Mostrando Ranking del Ejercito 1.....
		
		Puesto 1
		Nombre: Soldado1X1
		Vida: 5
		Fila: 1
		Columna: C
		------------------
		
		Puesto 2
		Nombre: Soldado5X1
		Vida: 4
		Fila: 6
		Columna: F
		------------------
		
		Puesto 3
		Nombre: Soldado7X1
		Vida: 3
		Fila: 3
		Columna: F
		------------------
		
		Puesto 4
		Nombre: Soldado0X1
		Vida: 3
		Fila: 8
		Columna: B
		------------------
		
		Puesto 5
		Nombre: Soldado3X1
		Vida: 3
		Fila: 8
		Columna: G
		------------------
		
		Puesto 6
		Nombre: Soldado2X1
		Vida: 1
		Fila: 3
		Columna: E
		------------------
		
		Puesto 7
		Nombre: Soldado6X1
		Vida: 1
		Fila: 4
		Columna: C
		------------------
		
		Puesto 8
		Nombre: Soldado4X1
		Vida: 1
		Fila: 7
		Columna: I
		------------------
		*********************************
		Ordenando a los soldados del Ejercito 2 por el insercion: 
		------------------------------------------
		Mostrando Ranking del Ejercito 2.....
		
		Puesto 1
		Nombre: Soldado0X2
		Vida: 5
		Fila: 6
		Columna: A
		------------------
		*********************************
		El resultado de esta Batalla se decidio por el nivel de fuerza de cada ejercito por lo que el resultado es: ...
		El Ejercito 1 es el GANADOR con 21 puntos
		
	\end{lstlisting}
	\subsection{Estructura de laboratorio 07}
	\begin{itemize}	
		\item El contenido que se entrega en este laboratorio07 es el siguiente:
	\end{itemize}
	\begin{lstlisting}[style=ascii-tree]
	/Lab07	
		"PONER RAMA"
	\end{lstlisting}    
	\section{\textcolor{red}{Rúbricas}}
	
	\subsection{\textcolor{red}{Entregable Informe}}
	\begin{table}[H]
		\caption{Tipo de Informe}
		\setlength{\tabcolsep}{0.5em} % for the horizontal padding
		{\renewcommand{\arraystretch}{1.5}% for the vertical padding
		\begin{tabular}{|p{3cm}|p{12cm}|}
			\hline
			\multicolumn{2}{|c|}{\textbf{\textcolor{red}{Informe}}}  \\
			\hline 
			\textbf{\textcolor{red}{Latex}} & \textcolor{blue}{El informe está en formato PDF desde Latex,  con un formato limpio (buena presentación) y facil de leer.}   \\ 
			\hline 
			
			
		\end{tabular}
	}
	\end{table}
	
	\clearpage
	
	\subsection{\textcolor{red}{Rúbrica para el contenido del Informe y demostración}}
	\begin{itemize}			
		\item El alumno debe marcar o dejar en blanco en celdas de la columna \textbf{Checklist} si cumplio con el ítem correspondiente.
		\item Si un alumno supera la fecha de entrega,  su calificación será sobre la nota mínima aprobada, siempre y cuando cumpla con todos lo items.
		\item El alumno debe autocalificarse en la columna \textbf{Estudiante} de acuerdo a la siguiente tabla:
	
		\begin{table}[ht]
			\caption{Niveles de desempeño}
			\begin{center}
			\begin{tabular}{ccccc}
    			\hline
    			 & \multicolumn{4}{c}{Nivel}\\
    			\cline{1-5}
    			\textbf{Puntos} & Insatisfactorio 25\%& En Proceso 50\% & Satisfactorio 75\% & Sobresaliente 100\%\\
    			\textbf{2.0}&0.5&1.0&1.5&2.0\\
    			\textbf{4.0}&1.0&2.0&3.0&4.0\\
    		\hline
			\end{tabular}
		\end{center}
	\end{table}	
	
	\end{itemize}
	
	\begin{table}[H]
		\caption{Rúbrica para contenido del Informe y demostración}
		\setlength{\tabcolsep}{0.5em} % for the horizontal padding
		{\renewcommand{\arraystretch}{1.5}% for the vertical padding
		%\begin{center}
		\begin{tabular}{|p{2.7cm}|p{7cm}|x{1.3cm}|p{1.2cm}|p{1.5cm}|p{1.1cm}|}
			\hline
    		\multicolumn{2}{|c|}{Contenido y demostración} & Puntos & Checklist & Estudiante & Profesor\\
			\hline
			\textbf{1. GitHub} & Hay enlace URL activo del directorio para el  laboratorio hacia su repositorio GitHub con código fuente terminado y fácil de revisar. &2 &X &2 & \\ 
			\hline
			\textbf{2. Commits} &  Hay capturas de pantalla de los commits más importantes con sus explicaciones detalladas. (El profesor puede preguntar para refrendar calificación). &4 &X &4 & \\ 
			\hline 
			\textbf{3. Código fuente} &  Hay porciones de código fuente importantes con numeración y explicaciones detalladas de sus funciones. &2 &X &2 & \\ 
			\hline 
			\textbf{4. Ejecución} & Se incluyen ejecuciones/pruebas del código fuente  explicadas gradualmente. &2 &X &2 & \\ 
			\hline			
			\textbf{5. Pregunta} & Se responde con completitud a la pregunta formulada en la tarea.  (El profesor puede preguntar para refrendar calificación).  &2 &X &2 & \\ 
			\hline	
			\textbf{6. Fechas} & Las fechas de modificación del código fuente estan dentro de los plazos de fecha de entrega establecidos. &2 &X &2 & \\ 
			\hline 
			\textbf{7. Ortografía} & El documento no muestra errores ortográficos. &2 &X &2 & \\ 
			\hline 
			\textbf{8. Madurez} & El Informe muestra de manera general una evolución de la madurez del código fuente,  explicaciones puntuales pero precisas y un acabado impecable.   (El profesor puede preguntar para refrendar calificación).  &4 &X &2 & \\ 
			\hline
			\multicolumn{2}{|c|}{\textbf{Total}} &20 & &18 & \\ 
			\hline
		\end{tabular}
		%\end{center}
		%\label{tab:multicol}
		}
	\end{table}
	
\clearpage

\section{Referencias}
\begin{itemize}			
	\item \url{https://drive.google.com/file/d/128O7v3wmrnl9g0a6BhMTNMnGIL3jTUDz/view}
\end{itemize}	
	
%\clearpage
%\bibliographystyle{apalike}
%\bibliographystyle{IEEEtranN}
%\bibliography{bibliography}
			
\end{document}